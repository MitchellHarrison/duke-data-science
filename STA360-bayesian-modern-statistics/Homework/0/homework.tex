\documentclass[titlepage, 12pt, leqno]{article}

% -------------------------------------------------- %
% -------------------- PACKAGES -------------------- %
% -------------------------------------------------- %
\usepackage{import}
\usepackage{pdfpages}
\usepackage{mathtools}
\usepackage{transparent}
\usepackage{xcolor}
\usepackage{tcolorbox}
\usepackage{amsmath}
\usepackage{amssymb}
\usepackage{parskip}
\usepackage{bbm}
\usepackage[margin = 1in]{geometry}
\tcbuselibrary{breakable}
\tcbset{breakable = true}


% -------------------------------------------------- %
% -------------- CUSTOM ENVIRONMENTS --------------- %
% -------------------------------------------------- %
\newtcolorbox{note}{colback=black!5!white,
                          colframe=black!55!white,
                          fonttitle=\bfseries,title=Note}

\newtcolorbox{ex}{colback=blue!5!white,
                          colframe=blue!55!white,
                          fonttitle=\bfseries,title=Example}

\newtcolorbox{definition}{colback=red!5!white,
                          colframe=red!55!white,
                          fonttitle=\bfseries,title=Definition}


% -------------------------------------------------- %
% ------------------- COMMANDS --------------------- %
% -------------------------------------------------- %
% Brackets, braces, etc. 
\newcommand{\abs}[1]{\lvert #1 \rvert}
\newcommand{\bigabs}[1]{\Bigl \lvert #1 \Bigr \rvert}
\newcommand{\bigbracket}[1]{\Bigl [ #1 \Bigr ]}
\newcommand{\bigparen}[1]{\Bigl ( #1 \Bigr )}
\newcommand{\ceil}[1]{\lceil #1 \rceil}
\newcommand{\floor}[1]{\lfloor #1 \rfloor}
\newcommand{\norm}[1]{\| #1 \|}
\newcommand{\bignorm}[1]{\Bigl \| #1 \Bigr \| #1}
\newcommand{\inner}[1]{\langle #1 \rangle}
\newcommand{\set}[1]{{ #1 }}


% -------------------------------------------------- %
% -------------------- SETUP ----------------------- %
% -------------------------------------------------- %
\title{\Huge{Homework 0}}
\author{\large{Mitch Harrison}}
\date{\today}   
\begin{document}
\setlength{\parskip}{1\baselineskip}
\setlength{\parindent}{15pt}
\maketitle
\newpage


% -------------------------------------------------- %
% --------------------- BODY ----------------------- %
% -------------------------------------------------- %
\section{Question 1}
\begin{ex}
    \textbf{Simplify}:
    \[
        log(e^{a_1}e^{a_2}e^{a_3} \cdots e^{a_n}).
    \]
\end{ex}

\begin{align*}
    log(e^{a_1}e^{a_2}e^{a_3} \cdots e^{a_n}) &= log(e^{a_1+a_2 \cdots +a_n}) \\
                                              &= a_1+a_2 \cdots +a_n \\
    \Aboxed{log(e^{a_1}e^{a_2}e^{a_3} \cdots e^{a_n}) &= \sum_{i=1}^{n}a_n}
\end{align*}

\section{Question 2}
\begin{ex}
    \textbf{Find the derivative}:
    \[
        \frac{d}{dx}\left( \frac{x}{log(x)}\right)
    \]
\end{ex}

Per the quotient rule, we know:
\[
    \frac{d}{dx}\left(\frac{u(x)}{v(x)}\right) = \frac{u'(x)v(x) - 
    u(x)v'(x)}{v(x)^{2}}
\]
\begin{align*}
    u(x) &= x & u'(x) &= 1 \\
    v(x) &= log(x) & v'(x) &= \frac{1}{x}
\end{align*}

Thus,

\begin{align*}
    \frac{d}{dx}\left(\frac{x}{log(x)}\right) &= \frac{
        log(x) - x\left(\frac{1}{x}\right)}{log(x)^{2}} \\
        &= \frac{log(x) - \frac{x}{x}}{log(x)^{2}} \\
        \Aboxed{\frac{d}{dx}\left(\frac{x}{log(x)}\right) &=
        \frac{log(x) - 1}{log(x)^{2}}} 
\end{align*}

\pagebreak
\section{Question 3}
\begin{ex}
    \textbf{What is the OLS estimator} of $\beta$ (1-dimensional) in the linear
    regression $y = x\beta + \epsilon$ with iid errors?
\end{ex}

The formula for OLS estimation of simple linear regression is as follows:
\[
    \boxed{\hat\beta = \frac{\sum_{i=1}^{n}(x_i - \bar x)(y_i - \bar y)}
    {\sum_{j=1}^{n}(x_i - \bar x)^{2}}} 
\]

\section{Question 4}
\begin{ex}
    \textbf{What is the OLS estimator} of $\beta$ ($p$-dimensional) in the linear
    regression $y = X\beta + \epsilon$ with iid errors?
\end{ex}

The formula for OLS estimation of $p$-dimensional regression is follows, where
$y$ is a vector of response variable values, $X$ is a matrix of observed data,
$\beta$ is a coefficient vector:
\[
    \boxed{\hat\beta = \left(X^{T}X\right)^{-1}X^{T}y} 
\]
\section{Question 5}
\begin{ex}
    In linear regression with $p$-dimensional $\beta$, \textbf{what is the 
    interpretation} of the estimate for the $j$th coefficient?
\end{ex}

Given that the $\beta$ vector is a coefficient vector for all predictors in
the regression, $\beta_j$ represents the predicted increase in $y$ after a
one-unit increase in $x_j$ (the $j$th predictor), while holding all other 
predictors constant.

\pagebreak
\section{Question 6}
\begin{ex}
    \textbf{Compute the integral}:
    \[
        \int_{-\infty}^{\infty}e^{-x^{2}}dx
    \]
\end{ex}

This is the Gaussian integral, the solution for which is known to be
$\boxed{\approx \sqrt{\pi }}$.

\section{Question 7}
\begin{ex}
    $X \sim N(\mu, \sigma^{2}$ reads "$X$ is a normally distributed random
    variable with mean $\mu$ and variance $\sigma^{2}$. Let
    \begin{align*}
        & X \sim N(0,1) \\
        & Y \sim N(3,2) \\
        & Z = X + Y
    \end{align*}
    \textbf{What is the distribution of $Z$? What is $ \mathbb{E}(Z)$ and
    $Var(Z)$?}
\end{ex}

Since both $X$ and $Y$ are normally distributed, the mean of $Z$ 
(i.e. $ \mathbb{E}(Z)$) will be $\mu_X + \mu_Y$, and the variance of $Z$ will be
$Var(X) + Var(Y)$. Thus,
\begin{align*}
    0+3 = \Aboxed{\mathbb{E}(Z) &= 3} \\
    1+2 = \Aboxed{Var(Z) &= 3}
\end{align*}

$Z$ will therefore be $\boxed{ \text{normally distributed}}$ such that
$Z \sim N(3,3)$.

\section{Question 8}
\begin{ex}
    In your own words, the \textbf{support} of a random variable is...
\end{ex}

The support of a random variable is the set of all possible values for a random
variable for which the propability of occurence is non-zero. For example, 7 is
not contained in the support for a random variable representing a standard 
6-sided die roll, which is $\{1,2,3,4,5,6\}$.

\pagebreak
\section{Question 9}
\begin{ex}
    \textbf{TRUE/FALSE}: The product of two uniform[0, 1] random variables is
    uniform[0, 1].
\end{ex}

\textbf{False}. Multiplying two standard uniform random variables results in a 
triangular distribution, which is non-uniform.

\end{document}

\documentclass[titlepage, 12pt, leqno]{article}

% -------------------------------------------------- %
% -------------------- PACKAGES -------------------- %
% -------------------------------------------------- %
\usepackage{import}
\usepackage{pdfpages}
\usepackage{mathtools}
\usepackage{transparent}
\usepackage{enumitem}
\usepackage{xcolor}
\usepackage{tcolorbox}
\usepackage{amsmath}
\usepackage{amssymb}
\usepackage{parskip}
\usepackage{bbm}
\usepackage[margin = 1in]{geometry}
\tcbuselibrary{breakable}
\tcbset{breakable = true}


% -------------------------------------------------- %
% -------------- CUSTOM ENVIRONMENTS --------------- %
% -------------------------------------------------- %
\newtcolorbox{note}{colback=black!5!white,
                          colframe=black!55!white,
                          fonttitle=\bfseries,title=Note}

\newtcolorbox{ex}{colback=blue!5!white,
                          colframe=blue!55!white,
                          fonttitle=\bfseries,title=Example}

\newtcolorbox{definition}{colback=red!5!white,
                          colframe=red!55!white,
                          fonttitle=\bfseries,title=Definition}


% -------------------------------------------------- %
% ------------------- COMMANDS --------------------- %
% -------------------------------------------------- %
% Brackets, braces, etc. 
\newcommand{\abs}[1]{\lvert #1 \rvert}
\newcommand{\bigabs}[1]{\Bigl \lvert #1 \Bigr \rvert}
\newcommand{\bigbracket}[1]{\Bigl [ #1 \Bigr ]}
\newcommand{\bigparen}[1]{\Bigl ( #1 \Bigr )}
\newcommand{\ceil}[1]{\lceil #1 \rceil}
\newcommand{\floor}[1]{\lfloor #1 \rfloor}
\newcommand{\norm}[1]{\| #1 \|}
\newcommand{\bignorm}[1]{\Bigl \| #1 \Bigr \| #1}
\newcommand{\inner}[1]{\langle #1 \rangle}
\newcommand{\set}[1]{{ #1 }}


% -------------------------------------------------- %
% -------------------- SETUP ----------------------- %
% -------------------------------------------------- %
\title{\Huge{Lecture 12 - Non-conjugate Priors and Gibbs Sampler}}
\author{\large{Mitch Harrison}}
\date{\today}   
\begin{document}
\setlength{\parskip}{1\baselineskip}
\setlength{\parindent}{15pt}
\maketitle
\tableofcontents
\newpage


% -------------------------------------------------- %
% --------------------- BODY ----------------------- %
% -------------------------------------------------- %
\section{Non-Conjugate Priors}

\begin{definition}
    A \textbf{semiconjugate} or \textbf{conditionally conjugate} prior is a
    prior that is conjugate to the full conditional posterior.
\end{definition}

Note that the idea of a semiconjugate prior only makes sense when making
inferences about two or more parameters. For example:
\begin{align*}
    Y|\theta,\sigma^{2} & \sim N(\theta,\sigma^{2})\\
    \theta & \sim N(\mu_{0}, \tau_{0}^{2}) \\
    \frac{1}{\sigma^{2}} & \sim gamma\left(\frac{\nu}{2},
        \frac{\nu_{0}\sigma_{0}^{2}}{2}\right)
\end{align*}
In this case, $\tau_{0}^{2}$ is \textit{not} a function of $\sigma^{2}$, and
thus $p(\theta, \sigma^{2}) = p(\theta)p(\sigma^{2})$.
Each prior is "semiconjugate" since $p(\theta|\sigma^{2}, \vec y)$ is normal
and $p(\frac{1}{\sigma^{2}} | \theta, \vec y)$ is gamma but $p(\theta,
\sigma^{2})$ is not conjugate to $p(\theta,\sigma^{2}|\vec y)$.

\begin{definition}
    A \textbf{proper prior} is a density function that does not depend on data
    and integrates to 1. If a prior integrates to a positive finite value, it
    is an un-normalized density that can be renormalized by being multiplies by
    a constant to integrate to 1. If a prior is \textit{not} proper, we call it
    an \textbf{improper prior}.
\end{definition}
One example of an improper prior is
\begin{align*}
    Y|\theta,\sigma^{2} & \sim N(\theta, \sigma^{2})\\
    p(\theta,\sigma^{2}) &= \frac{1}{\sigma^{2}}
\end{align*}
where $p(\theta, \sigma^{2})$ is an improper prior. Here, $p(\theta, 
\sigma^{2}$ does not integrate to a finite value and thus cannot be 
re-normalized. It is not a probability density. However, it yields a tractable
posterior for $\theta$ and $\sigma^{2}$ (see p.79 of Hoff).

\pagebreak
\section{Non-Informative Priors}
Priors are meant to describe our state of knowledge before examining data. In
some cases, we may wish to describe our ignorance a priori using a vague prior
that playes a minimal role in the posterior distribution.

A common trap is to imagine that a flat, uniform prior is uninformative. 
Previously (on homework 3), we showed a uniform prior on binary probability of
success $\theta$ is informative on the log-odds. Additionally, an improper flat
prior may carry a lot of information since most of the mass in infinitely far
away.
\begin{definition}
    The \textbf{Jeffreys prior} is
    \[
    J(\theta) \propto \sqrt{I(\theta)},
    \]
    where $I(\theta) = - \mathbb{E}[\frac{\partial}{\partial \theta^{2}}
    log(p(Y|\theta)) | \theta]$. The defining feature of Jeffreys' prior is
    tha tit will yield an equivalent result if applied to a transformed 
    parameter. This principle of invariance is one approach to non-informative
    priors that works well for single parameter priors. Multiparameter 
    extensions are often less useful.
\end{definition}

\pagebreak
\section{Gibbs Sampler}
What if we have a non-conjugate prior? How can we look at $p(\theta,\sigma^{2}|
\vec y)$? In general, suppose we don't know
\[
p(\theta,\sigma^{2}|\vec y),
\]
but we do know the full conditional posteriors
\begin{align*}
    &p(\theta|\sigma^{2},\vec y)\\
    &p(\sigma^{2}|\theta, \vec y),
\end{align*}
we can generate sample $\theta^{(s)}$, $\sigma^{2(s)}$ from the joint
posterior by the following algorithm:
\begin{enumerate}
    \item sample $\theta^{(s+1)}$ from $p(\theta|\sigma^{2(s)},\vec y)$
    \item sample $\sigma^{2(s+1)}$ from $p(\sigma^{2}|\theta^{(2+1)}, \vec y)$
    \item let $\phi^{(s+1)} = \{\theta^{(s+1)}, \sigma^{2(s+1)}\}$
\end{enumerate}
and iterate these steps $S$ times. This algorithm is called the \textbf{Gibbs
sampler}. The Gibbs sampler creates a dependent set of values $\phi^{(1)},
\cdots , \phi^{(S)}$. This sequence is called a \textbf{Markov Chain}, and it
lets us approximate the posterior. That is, the histogram of $\{\psi^{(1)},
\cdots \psi^{(S)}\}$ is a Markov Chain Monte Carlo approximation to
$p(\psi|\vec y)$.

\subsection{Example}
In the semiconjugate normal model described above, the resulting posteriors are
\[
\theta|\sigma^{2},\vec y \sim N(\mu_{n}, \tau_{n}^{2}),
\]
where 
\begin{align*}
    &\mu_{n} = \frac{\frac{\mu_{0}}{\tau_{0}^{2}} + \frac{n\bar y}{\sigma^{2}}}
    {\frac{1}{\tau_{0}^{2}} + \frac{n}{\sigma^{2}}}
    & \tau_{n}^{2} = \left(\frac{1}{\tau_{0}^{2}} + \frac{n}{\sigma^{2}}
        \right)^{-1}
\end{align*}
and
\[
\sigma|\theta, \vec y \sim  invgamma\left(\frac{\nu_{n}}{2}, 
    \frac{\nu_{n}\sigma^{2}_{n}}{2}\right)
\]
where
\begin{align*}
    &\nu_{n} = \nu_{0} + n \\
    &\sigma_{n}^{2} = \frac{1}{\nu_{n}}[\nu_{0}
    \sigma^{2}_{0} + ns_{n}^{2}(\theta)]\\
    &s_{n}^{2}(\theta) = \frac{1}{n}\sum (y_{i}-\theta)^{2}
\end{align*}

\begin{definition}
    Since the sequence $\{\phi^{s}$ depends on $\phi^{(0)}, \cdots ,
    \phi^{(s-1)}\}$ only through $\phi^{(s-1)}$, we say that the sequence is
    \textit{memoryless}. This is called the \textbf{Markov property}, and so 
    the sequence is a \textbf{Markov chain}. In plain english, we can say that
    "what happens next only depends on the state of affairs now."
\end{definition}

Under some conditions,
\[
p(\psi^{(s)}\in A) \rightarrow \int_{A}p(\phi)d \phi \text{ as }s\rightarrow 0.
\]
That is, the \textit{sampling distribution} of $\phi^{(s)}$ approaches the
\textit{target distribution} as $s \rightarrow \infty$ regardless of 
$\psi^{0}$. Furthermore, 
\[
    \frac{1}{S}\sum_{s=1}^{S}g(\psi^{(s)}) \rightarrow \mathbb{E}[g(\psi)]
\]
\begin{note}
    \textbf{Big take-away:} If we can sample from the full conditional
    posteriors, we can construct a Markov chain with samples from the joint
    posterior! We can then use the Monte Carlo approximation to use the samples
    to summarize aspects of the posterior.
\end{note}

\pagebreak
\section{Example Code}
This example code is from Hoff chapter 6.
\begin{verbatim}
# data
y = c(1.64, 1.70, 1.72, 1.74, 1.82, 1.82, 1.82, 1.90, 2.08)
mean.y = mean(y) ; var.y = var(y) ; n = length(y)

# priors
mu0 = 0
t20 = 100
nu0 = 1
s20 = 2

# starting point
S = 1000
PHI = matrix(nrow = S, ncol = 2)
phi = c(mean.y, var(y))
PHI[1, ] = phi

# Gibbs sampling
set.seed(360)
for(s in 2:S) { 
    ## generate theta from sigma2
    mun = (mu0 / t20 + n * mean.y * phi[2]) / (1 / t20 + n * phi[2])
    t2n = 1 / (1 / t20 + n * phi[2])
    phi[1] = rnorm(1, mun, sqrt(t2n))

    ## generate 1/sigma2 from theta
    nun = nu0 + n
    s2n = (nu0 * s20 + (n - 1) * var.y + n * (mean.y - phi[1])^2 ) / nun
    phi[2] = rgamma(1, nun/2, nun * s2n / 2)

    ## update chain
    PHI[s,] = phi
}
\end{verbatim}

\end{document}

\documentclass[titlepage, 12pt, leqno]{article}

% -------------------------------------------------- %
% -------------------- PACKAGES -------------------- %
% -------------------------------------------------- %
\usepackage{import}
\usepackage{pdfpages}
\usepackage{mathtools}
\usepackage{transparent}
\usepackage{xcolor}
\usepackage{tcolorbox}
\usepackage{amsmath}
\usepackage{amssymb}
\usepackage{parskip}
\usepackage{algpseudocode}
\usepackage{algorithm}
\usepackage{bbm}
\usepackage[margin = 1in]{geometry}
\tcbuselibrary{breakable}
\tcbset{breakable = true}


% -------------------------------------------------- %
% -------------- CUSTOM ENVIRONMENTS --------------- %
% -------------------------------------------------- %
\newtcolorbox{note}{colback=black!5!white,
                          colframe=black!55!white,
                          fonttitle=\bfseries,title=Note}

\newtcolorbox{ex}{colback=blue!5!white,
                          colframe=blue!55!white,
                          fonttitle=\bfseries,title=Example}

\newtcolorbox{definition}{colback=red!5!white,
                          colframe=red!55!white,
                          fonttitle=\bfseries,title=Definition}


% -------------------------------------------------- %
% ------------------- COMMANDS --------------------- %
% -------------------------------------------------- %
% Brackets, braces, etc. 
\newcommand{\abs}[1]{\lvert #1 \rvert}
\newcommand{\bigabs}[1]{\Bigl \lvert #1 \Bigr \rvert}
\newcommand{\bigbracket}[1]{\Bigl [ #1 \Bigr ]}
\newcommand{\bigparen}[1]{\Bigl ( #1 \Bigr )}
\newcommand{\ceil}[1]{\lceil #1 \rceil}
\newcommand{\floor}[1]{\lfloor #1 \rfloor}
\newcommand{\norm}[1]{\| #1 \|}
\newcommand{\bignorm}[1]{\Bigl \| #1 \Bigr \| #1}
\newcommand{\inner}[1]{\langle #1 \rangle}
\newcommand{\set}[1]{{ #1 }}


% -------------------------------------------------- %
% -------------------- SETUP ----------------------- %
% -------------------------------------------------- %
\title{\Huge{Lecture 1 - Intro, History, Notation}}
\author{\large{Mitch Harrison}}
\date{\today}   
\begin{document}
\setlength{\parskip}{1\baselineskip}
\setlength{\parindent}{15pt}
\maketitle
\tableofcontents
\newpage


% -------------------------------------------------- %
% --------------------- BODY ----------------------- %
% -------------------------------------------------- %
\section{Overview}
\subsection{Our goal as statisticians}
\begin{definition}
    Suppose we flip a fair coin 10 times and get $h$ heads. Previous courses have 
    covered the calculation of probabilities like $\mathbb{P}(h=8)$. This is
    \textbf{deductive reasoning}.

    Usually, scientists and statisticians are more interested in the opposite 
    problem form the one above. For example, \textit{given some data, is the coin 
    fair?} This is \textbf{inductive reasoning}.
\end{definition}

\subsection{Why Bayesian statistics?}

\begin{definition}
    \textbf{Bayesian statistics} is in contrast with more classical (or
    \textit{frequentist} statistics). Frequentists define probability as the
    long-run frequency of an event (e.g. $\mathbb{P}( \text{heads})$ in 
    infinitely many coin flips.

    In Bayesian statistics, probability is a numeric quantity that describes
    one's belief that an event will occur.
\end{definition}

Bayesian statistics provides a unified theory for inductive reasoning and
applying it to any number of problems. Second, Bayesian statistics provides a
more interpretable uncertainty quantification. Classical statistics are often more
challenging to interpret.
    
\pagebreak
\section{History and Notation}
\subsection{History}

In 1713, James Bernoulli posed statistical questions that were answered by 
reverand Thoman Bayes in 1763. Most of these solutions were formalized by 
Laplace in 1812, who used these solutions to solve new statistical questions 
(e.g. the mass of Saturn). Not until the 1990s did Bayesian statistics explode
in popularity, largely due to the increase in computing power around (and since)
that time.

\subsection{Notation}

\begin{itemize}
    \item We use $\mathbb{P}(x)$ to mean both the $pmf(x)$ if $x$ is discrete and
        $pdf(x)$ if $x$ is continuous. It is shorthand for "the probability of 
        $x$". 
        This is an abuse of notation since, strictly speaking:
        \[
            pdf(x) = \lim_{\Delta x \to 0} 
            \frac{\mathbb{P}(x<X<x+\Delta x)}{\Delta x}
        \]
        and
        \[
            \mathbb{P}(x_1<X<x_2) = \int_{x_1}^{x_2}pdf(x)dx
        \]
    \item $log(n)$ will mean the \textit{natural log} (i.e. $ln(x)$).
    \item $\mathbb{P}(\neg A)$ is the probability of "not $A$"
    \item $\mathbb{P}(A,B)$ is the probability of "$A$ and $B$"
    \item $ \mathbb{P}\left(A \;\middle|\; B\right) $ is "probability of $A$ given
        $B$."
\end{itemize}

\pagebreak
\section{Review of Textbook Chapter 2}

For all sets $F$, $G$, and $H$:
\begin{itemize}
    \item $0 = \mathbb{P}(\neg A) \le \mathbb{P}\left(F \;\middle|\; H\right) 
        \le \mathbb{P}\left(H \;\middle|\; H\right) = 1$
    \item $\mathbb{P}(F \cup G | H) = \mathbb{P}\left(F \;\middle|\; H\right) 
        + \mathbb{P}\left(G \;\middle|\; H\right) \text{ if } F \cap G = 
        \emptyset$
    \item $\mathbb{P}(F \cap G | H) = \mathbb{P}\left(G \;\middle|\; H\right) 
        \mathbb{P}\left(F \;\middle|\; G \cap H\right) $
\end{itemize}

\end{document}

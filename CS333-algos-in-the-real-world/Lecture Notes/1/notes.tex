\documentclass[titlepage, 12pt, leqno]{article}

% -------------------------------------------------- %
% -------------------- PACKAGES -------------------- %
% -------------------------------------------------- %
\usepackage{import}
\usepackage{pdfpages}
\usepackage{mathtools}
\usepackage{transparent}
\usepackage{xcolor}
\usepackage{tcolorbox}
\usepackage{amsmath}
\usepackage{amssymb}
\usepackage{parskip}
\usepackage{bbm}
\usepackage[margin = 1in]{geometry}
\tcbuselibrary{breakable}
\tcbset{breakable = true}


% -------------------------------------------------- %
% -------------- CUSTOM ENVIRONMENTS --------------- %
% -------------------------------------------------- %
\newtcolorbox{note}{colback=black!5!white,
                          colframe=black!55!white,
                          fonttitle=\bfseries,title=Note}

\newtcolorbox{ex}{colback=blue!5!white,
                          colframe=blue!55!white,
                          fonttitle=\bfseries,title=Example}

\newtcolorbox{definition}{colback=red!5!white,
                          colframe=red!55!white,
                          fonttitle=\bfseries,title=Definition}


% -------------------------------------------------- %
% ------------------- COMMANDS --------------------- %
% -------------------------------------------------- %
% Brackets, braces, etc. 
\newcommand{\abs}[1]{\lvert #1 \rvert}
\newcommand{\bigabs}[1]{\Bigl \lvert #1 \Bigr \rvert}
\newcommand{\bigbracket}[1]{\Bigl [ #1 \Bigr ]}
\newcommand{\bigparen}[1]{\Bigl ( #1 \Bigr )}
\newcommand{\ceil}[1]{\lceil #1 \rceil}
\newcommand{\floor}[1]{\lfloor #1 \rfloor}
\newcommand{\norm}[1]{\| #1 \|}
\newcommand{\bignorm}[1]{\Bigl \| #1 \Bigr \| #1}
\newcommand{\inner}[1]{\langle #1 \rangle}
\newcommand{\set}[1]{{ #1 }}


% -------------------------------------------------- %
% -------------------- SETUP ----------------------- %
% -------------------------------------------------- %
\title{\Huge{Lecture 1 - Algorithmic Realism, Stable Matching, and School Choice}}
\author{\large{Mitch Harrison}}
\date{\today}   
\begin{document}
\setlength{\parskip}{1\baselineskip}
\setlength{\parindent}{15pt}
\maketitle
\tableofcontents
\newpage


% -------------------------------------------------- %
% --------------------- BODY ----------------------- %
% -------------------------------------------------- %
\section{Algorithm Examples}

\begin{definition}
    \textbf{Desiderata} (singular \textit{desideratum}) are desirable properties
    of an algorithmic systems. For example, deciding what the algorithm should be
    optimizing or what would make one want to use it.
\end{definition}

\subsection{Dorm assignment}
Suppose there are $m$ dorms $\{1,2, \cdots ,m\}$. Dorm $j$ has capacity 
$c_j$. There are $n$ students. Student $i$ has preference ordering over each 
dorm (e.g. $\{5, 3, 4, 2, 1\}$). \textbf{How should you assign students to 
dorms?}

\subsection{Recent graduate hiring}
Hiring recent college graduates has two-sided preferences. That is, both
employers and candidates have preferences for the other. The market is
\textit{decentralized} on an annual cycle (e.g. applications in the winter,
negotiations in the spring, etc.). There are also timing effects where a
candidate could say yes to a sub-optimal employer while waiting for a response
from a more optimal one.

\subsection{National Resident Matching Program (NRMP)}
This system has been used to assign recent medical school graduates to medical
residencies in the United States since 1953. It is a matching algorithm in the
vein of recent graduate hiring.

\pagebreak
\section{Algorithms and Algorithmic Systems}
"Algorithms in the real world" refers primarily to algorithmic systems that play
a role in laypeoples' lives, regardless of their understanding of the 
underlying algorithms. Most of these algorithms take advantage of the flood of
data produced by modern living (e.g. smartphones).

\begin{note}
    Book recommendation: \textit{Weapons of Math Destruction}
\end{note}

Algorithmic systems have deep impacts on the lives of individuals, from deciding
who goes to jail to which medical student gets his or her dream residency. An
algorithm exists, for example, to decide whether or not to grant bail to people
accused of crimes. It was found to potentially be biased against people of color,
especially African Americans.

\begin{note}
    Book recommendations: \textit{Algorithms of Opression} and \textit{System 
    Error: Where Big Tech Went Wrong and How We Can Reboot}
\end{note}

\subsection{Algorithmic Formalism, Algorithmic Realism}
Algorithms are developed inside of a strict formalism of math and engineering, but
then deployed into the real world where such formalisms are less rigid. This 
often leads to designers with good intentions building systems that go against
their own goals.

\pagebreak
\section{Course Information}

This course is both technical \textit{and} societal. It will be a collaborative,
project-driven, research-oriented course. Our work will (hopefully) be
public-facing, with projects being shown to friends, family, or the public at 
large.

This course is \textit{not} a software engineering course or job interview 
preparation. It is also not a "problem-solving" course in the abstract.

\subsection{What kinds of systems will we study?}
More established algorithms that we will study will include stable matching, web
search, sorting and ranking, and others. Later in the course, we will study more
modern algorithmic systems like machine learning, generative AI, and language
models.

\end{document}

\documentclass[titlepage, 12pt, leqno]{article}

% -------------------------------------------------- %
% -------------------- PACKAGES -------------------- %
% -------------------------------------------------- %
\usepackage{import}
\usepackage{pdfpages}
\usepackage{mathtools}
\usepackage{transparent}
\usepackage{xcolor}
\usepackage{tcolorbox}
\usepackage{amsmath}
\usepackage{amssymb}
\usepackage{parskip}
\usepackage{bbm}
\usepackage[margin = 1in]{geometry}
\tcbuselibrary{breakable}
\tcbset{breakable = true}


% -------------------------------------------------- %
% -------------- CUSTOM ENVIRONMENTS --------------- %
% -------------------------------------------------- %
\newtcolorbox{note}{colback=black!5!white,
                          colframe=black!55!white,
                          fonttitle=\bfseries,title=Note}

\newtcolorbox{ex}{colback=blue!5!white,
                          colframe=blue!55!white,
                          fonttitle=\bfseries,title=Example}

\newtcolorbox{definition}{colback=red!5!white,
                          colframe=red!55!white,
                          fonttitle=\bfseries,title=Definition}


% -------------------------------------------------- %
% ------------------- COMMANDS --------------------- %
% -------------------------------------------------- %
% Brackets, braces, etc. 
\newcommand{\abs}[1]{\lvert #1 \rvert}
\newcommand{\bigabs}[1]{\Bigl \lvert #1 \Bigr \rvert}
\newcommand{\bigbracket}[1]{\Bigl [ #1 \Bigr ]}
\newcommand{\bigparen}[1]{\Bigl ( #1 \Bigr )}
\newcommand{\ceil}[1]{\lceil #1 \rceil}
\newcommand{\floor}[1]{\lfloor #1 \rfloor}
\newcommand{\norm}[1]{\| #1 \|}
\newcommand{\bignorm}[1]{\Bigl \| #1 \Bigr \| #1}
\newcommand{\inner}[1]{\langle #1 \rangle}
\newcommand{\set}[1]{{ #1 }}


% -------------------------------------------------- %
% -------------------- SETUP ----------------------- %
% -------------------------------------------------- %
\title{\Huge{Lecture 2 - Matching Algorithms and Implementation}}
\author{\large{Mitch Harrison}}
\date{\today}   
\begin{document}
\setlength{\parskip}{1\baselineskip}
\setlength{\parindent}{15pt}
\maketitle
\tableofcontents
\newpage


% -------------------------------------------------- %
% --------------------- BODY ----------------------- %
% -------------------------------------------------- %
\section{Matching Algorithms in Practice}
\subsection{School choice in NYC}
For the context of this example, \textbf{school choice} refers to choices of 
public schools, not the choice between public, private, and charter. Since 2003,
NYC has used a matching algorithm similar to the algorithm used to place medical
students into residency programs.

Before 2003, school placement was decided either by lottery or by geography (with
some exceptions made for schools that require auditions for example). First,
students/parents apply to five schools. For three rounds, the algorithm gets 
decisions from schools and the families accept or reject offers. After those three
rounds, students were assigned geographically. With this program, roughly 30\% of 
students were not matched to anywhere they applied, and families needed to be
strategic about where they applied. Schools also concealed their capacity for
students, and would fill slots themselves after the final cycle of matching.

\subsection{Preference formalism}
Ideally, every student (or employee, etc.) will rank every school (or employer,
etc.) and each school will rank each student. Mathematically, we say that there 
are $n$ applicants $A = \{1,2 \cdots ,n\}$ and $m$ schools $B = \{1,2,\cdots,m\}$.
Applicant $i$ has a preference ordering over employers. We call each ranking 
$\pi_A$. School $j$ has capacity $c_j$ for students.

\subsection{Boston matching algorithm}
In the Boston matching algorithm, there are a series of rounds. In the first 
round, each applicant "applies" to their most preferred school and each school
"accepts" their most preferred applicant, up to capacity. In subsequent rounds, 
unmatched applicants apply to their next most preferred school and each school
accepts applicants up to their capacity. This repeats until all applicants are
matched.

This algorithm matches the number of applicants that get their top choice, but 
schools that applicants don't prefer get the least "competitive" students, which
could keep lower-preference schools

\begin{definition}
    \textbf{Stability}: $f$ is \textit{stable} if it has no blocking pairs. A
    \textbf{blocking pair} is an applicant $i \in A$ and an employer $j \in B$
    such that:
    \begin{enumerate}
        \item $j \succ_i f(i)$, and
        \item There exists $i' \in A$ auch that $f(i')=j$ and $i\succ_ji'$
    \end{enumerate}
\end{definition}

\pagebreak
\section{Gale-Shapely Deferred Acceptance Algorithm}
The \textbf{Gale-Shapely Deferred Acceptance Algorithm} works as follows:
\begin{enumerate}
    \item In the first round, each applicant applies to their most preferred
        school and each school \textit{tentatively} accepts their most
        preferred applicants (up to capacity).
    \item In subsequent rounds, each applicant who is not tentatively matched
        applies to their next most preferred school. Then, each school 
        \textit{tentatively} accepts their most preferred applicants from
        among the new applicants.
\end{enumerate}

\subsection{Proof sketch of stability}
Our \textbf{claim }is that the matching produced by the deferred acceptance
algorithm is stable. Recall that the following describes a blocking pair:
\begin{enumerate}
    \item $j \succ_i f(i)$
    \item There exists $i'\in A$ such that $f(i') = j$ and $i \succ_j i'$.
\end{enumerate}
Since $j \succ_i f(i)$, $i$ must have applied to $j$ before $f(i)$. Since 
$f(i) \ne j$, $j$ must have been tentatively matched to $c_j$ other applicants,
all of whom prefers $j$ over $i$. However, emplyers only ever get more
preferred applicants as time goes on, so $j$ must prefer all of the final $c_j$
applicants mathed to $j$ over $i$. Thus, from the employer's perspective, as time
goes on, applicants become increasingly preferred.

\subsection{Another desideratum: strategy-proofness}
\begin{definition}
    $M$ is a \textbf{strategy-proof} for applicants if, for all $i \in A$ and
    preferences $\succ_i$, for all $\pi_A$ with $i$'s true preferences and 
    $\pi_A$ with any other reported preferences of $i$, the following holds:
    \[
    M(\pi_A, \pi_B, c_1, \cdots ,c_m)(i) \succeq M(\pi_A', \pi_B, c_1, \cdots 
    ,c_m)(i)
    \]
    where $j \succeq_i j'$ means $j \succ j'$ or $j = j'$.
\end{definition}
The intuitive "proof" of the strategy-proofness of deferred acceptance is that
applicants do not need to worry about theie less competitive middle choices
"filling up" while they apply to their more competitive top choices.

\subsection{Optimality}
\begin{definition}
    \textbf{Proposer-side optimality}: For every applicant $i$, let $h(i)$ be the
    most-preferred employer to whom $i$ is matched in \textit{any} stable
    matching. A stable matching $f$ is \textbf{applicant-optimal} if for every
    applicant $i$, $f(i) \succeq_i h(i)$.
\end{definition}

Deferred-acceptance is applicant-optimal, but \textit{not} employer-optimal. No
matching algorithm can be both applicant- \textit{and} employer-optimal.

\begin{definition}
    A matching $f$ is \textbf{Pareto Optimal} if there does not exist another
    matching $f'$ such that $f'(i) \succeq_i f(i)$ for all $i$ with at least one
    preference strict.
\end{definition}

Deferred acceptance is both strategy proof \textit{and} stable, but \textbf{not
Pareto Optimal}. There is another algorithm, the \textbf{Top Trading Cycles}
algorithm that is strategy-proof \textit{and} Pareto optimal, but not
stable.

\pagebreak
\section{Practical Problems}
\subsection{Preference generation}
Both schools and student must submit a preference ranking of the other.
The act of an individual or school generating a list of preferences is
non-trivial, since many students may not be able to visit or research every
school to generate an accurate preference list. Schools/employers have a
similar issue. In NYC, lower-performing students often list worse schools
higher on their preferences lists than higher-achieving student do. NYC also saw
roughly 300 students appeal placement decisions, even though they were given their
first choice. Clearly, educating applicants about the programs to which they are
applying is necessary.

\subsection{Affirmative action}
Some schools may have an incentive to accept minoritized or economically 
disadvantaged students into their programs. One way to apply affirmative action is
to add a coice function to the algorithm that generates preferences for schools.
Another, perhaps superior, implementation is as follows:
\begin{enumerate}
    \item Each school splits their capacity $c_j$ into two components: $q_j$ and
        $c_j-q_j$.
    \item The employer appears as two \textit{different} employers in the
        algorithm: one with capacity $q_j$ that ranks underrepresented applicants
        before any others, and another with capacity $c_j-q_j$ that has 
        preferences for the employer disregarding diversity.
    \item The applicant preferences are simply duplicated for both entities.
\end{enumerate}

\subsection{Coupled preferences}
\begin{definition}
    \textbf{Coupled preferences} occur when two applicants only want to be matched
    together. For example, if romantic partners want to be matched together.
    \begin{note}
        Note that there may not be a stable matching with coupled preferences.
    \end{note}
\end{definition}

\end{document}

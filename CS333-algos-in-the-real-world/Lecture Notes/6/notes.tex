\documentclass[titlepage, 12pt, leqno]{article}

% -------------------------------------------------- %
% -------------------- PACKAGES -------------------- %
% -------------------------------------------------- %
\usepackage{import}
\usepackage{pdfpages}
\usepackage{mathtools}
\usepackage{transparent}
\usepackage{enumitem}
\usepackage{xcolor}
\usepackage{tcolorbox}
\usepackage{amsmath}
\usepackage{amssymb}
\usepackage{parskip}
\usepackage{bbm}
\usepackage[margin = 1in]{geometry}
\tcbuselibrary{breakable}
\tcbset{breakable = true}


% -------------------------------------------------- %
% -------------- CUSTOM ENVIRONMENTS --------------- %
% -------------------------------------------------- %
\newtcolorbox{note}{colback=black!5!white,
                          colframe=black!55!white,
                          fonttitle=\bfseries,title=Note}

\newtcolorbox{ex}{colback=blue!5!white,
                          colframe=blue!55!white,
                          fonttitle=\bfseries,title=Example}

\newtcolorbox{definition}{colback=red!5!white,
                          colframe=red!55!white,
                          fonttitle=\bfseries,title=Definition}


% -------------------------------------------------- %
% ------------------- COMMANDS --------------------- %
% -------------------------------------------------- %
% Brackets, braces, etc. 
\newcommand{\abs}[1]{\lvert #1 \rvert}
\newcommand{\bigabs}[1]{\Bigl \lvert #1 \Bigr \rvert}
\newcommand{\bigbracket}[1]{\Bigl [ #1 \Bigr ]}
\newcommand{\bigparen}[1]{\Bigl ( #1 \Bigr )}
\newcommand{\ceil}[1]{\lceil #1 \rceil}
\newcommand{\floor}[1]{\lfloor #1 \rfloor}
\newcommand{\norm}[1]{\| #1 \|}
\newcommand{\bignorm}[1]{\Bigl \| #1 \Bigr \| #1}
\newcommand{\inner}[1]{\langle #1 \rangle}
\newcommand{\set}[1]{{ #1 }}


% -------------------------------------------------- %
% -------------------- SETUP ----------------------- %
% -------------------------------------------------- %
\title{\Huge{Lecture 6 - Web Search (cont'd) and Cryptography}}
\author{\large{Mitch Harrison}}
\date{\today}   
\begin{document}
\setlength{\parskip}{1\baselineskip}
\setlength{\parindent}{15pt}
\maketitle
\tableofcontents
\newpage


% -------------------------------------------------- %
% --------------------- BODY ----------------------- %
% -------------------------------------------------- %
\section{Web Search}

Recall that we introduced the fact that web search is crucial to modern life: it
effectively selects the information to which the public is exposed. It is likely
too simplistic to say that the prejudices that search results show is because of
the inherit bias in individuals. In fact, the vast majority of people surveyed
claim to trust web search results for information, but much of the top results are
sponsored content.

\begin{ex}
    On your own, \textbf{articulate a societal critique} of web search engines. Do
    you agree or disagree with this critique. Additionally, \textbf{how} could
    information access to the web be improved from the perspective of social
    good and justice?
\end{ex}

\pagebreak
\section{E-Commerce and Cryptography}
We generally feel comfortable inputting our credit card information onto a 
website in order to buy things. This information passes through several nodes on
the way to the website in which you are making a purchase. This feeling of
security is not a given. We take it for granted, but it is made possible by 
\textbf{cryptography}.

\begin{definition}
    A \textbf{man-in-the-middle attack} involves a malicious actor that intercepts
    data on its way to a final destination (e.g. Amazon), records it, and passes
    it on to its desired destination.
\end{definition}

Given a message $M$, but \textbf{not} the key $S$, if the key is a truly random
bit sequence,
\[
\mathbb{P}(M|C) = \left(\frac{1}{2}\right)^{|M|}
\]
for all possible plain text messages $M$. This method of encryption is
\textbf{information-} \textbf{theoretically} secure, which means it is robust 
even against an adversary with infinite computing power. This method's runtime is 
good (specifically $O(|M|)$) to encrypt or decrypt a message $M$ with $|M|$ bits,
and XOR is close to the hardware. However, we need $|M|$ random bits to send a 
message of length $|M|$. Additionally, random bits can only be used onces and 
\textit{must} be exchanged in advance.

\begin{definition}
    A \textbf{pseudorandom number} is generated by a \textit{deterministic
    algorithm} that is (1) hard to predict without knowing the state of the
    algorithm, and (2) initialized (or seeded) by some real randomness.

    For example,
    \[
    PNSG(x) = (ax+b) \% m
    \]
    where $a$, $b$, and $m$ are secret and $x$ is the "seeded" random number.
\end{definition}

The quality of a pseudorandom number generator comes from the sophistication of
the pseudorandom number generating algorithm (periodicity, autocorrelation, etc.)
and the quality of the randomness of the seed.

\begin{note}
    Standard random nubers from library calls in most programming languages are
    \textit{not} sufficient for cryptographic use.
\end{note}

\end{document}

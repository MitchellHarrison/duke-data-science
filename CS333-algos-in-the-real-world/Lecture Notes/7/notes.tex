\documentclass[titlepage, 12pt, leqno]{article}

% -------------------------------------------------- %
% -------------------- PACKAGES -------------------- %
% -------------------------------------------------- %
\usepackage{import}
\usepackage{pdfpages}
\usepackage{mathtools}
\usepackage{transparent}
\usepackage{enumitem}
\usepackage{xcolor}
\usepackage{tcolorbox}
\usepackage{amsmath}
\usepackage{amssymb}
\usepackage{parskip}
\usepackage{bbm}
\usepackage[margin = 1in]{geometry}
\tcbuselibrary{breakable}
\tcbset{breakable = true}


% -------------------------------------------------- %
% -------------- CUSTOM ENVIRONMENTS --------------- %
% -------------------------------------------------- %
\newtcolorbox{note}{colback=black!5!white,
                          colframe=black!55!white,
                          fonttitle=\bfseries,title=Note}

\newtcolorbox{ex}{colback=blue!5!white,
                          colframe=blue!55!white,
                          fonttitle=\bfseries,title=Example}

\newtcolorbox{definition}{colback=red!5!white,
                          colframe=red!55!white,
                          fonttitle=\bfseries,title=Definition}


% -------------------------------------------------- %
% ------------------- COMMANDS --------------------- %
% -------------------------------------------------- %
% Brackets, braces, etc. 
\newcommand{\abs}[1]{\lvert #1 \rvert}
\newcommand{\bigabs}[1]{\Bigl \lvert #1 \Bigr \rvert}
\newcommand{\bigbracket}[1]{\Bigl [ #1 \Bigr ]}
\newcommand{\bigparen}[1]{\Bigl ( #1 \Bigr )}
\newcommand{\ceil}[1]{\lceil #1 \rceil}
\newcommand{\floor}[1]{\lfloor #1 \rfloor}
\newcommand{\norm}[1]{\| #1 \|}
\newcommand{\bignorm}[1]{\Bigl \| #1 \Bigr \| #1}
\newcommand{\inner}[1]{\langle #1 \rangle}
\newcommand{\set}[1]{{ #1 }}


% -------------------------------------------------- %
% -------------------- SETUP ----------------------- %
% -------------------------------------------------- %
\title{\Huge{Lecture 7 - E-Commerce and Cryptography (cont'd)}}
\author{\large{Mitch Harrison}}
\date{\today}   
\begin{document}
\setlength{\parskip}{1\baselineskip}
\setlength{\parindent}{15pt}
\maketitle
\tableofcontents
\newpage


% -------------------------------------------------- %
% --------------------- BODY ----------------------- %
% -------------------------------------------------- %
\section{Cryptography}

\subsection{Simple Stream Ciphers}
\textbf{Stream ciphers} exchange a small (e.g. 128-bit) truly random secret
private key $S$ in advance. Parties agree on a common pseudorandom number 
generator $PRNG$. To decrypt an encrypted message $M$, they generate $|M|$
pseudorandom bits with $PRNG$ seeded by seed $S$. They then take the bitwise
XOR to arrive at cipher text that can then be decrypted by the distant end.

\subsection{Asymmetric/Public Key Cryptography}
In \textit{private key cryptography}, parties must share their key in advance,
sometimes by meeting physically to exchange them (e.g. paper tape). This isn't
ideal for establishing internet communications.

\begin{definition}
    \textbf{Asymmetric/Public Key cryptography} uses different keys for
    encryption and decryption. Each party has a secret private and public key,
    so no keys need be exchanged.
\end{definition}

Public key cryptography is only used to establish a secret key with a distant end.
That secret key is then used for \textit{symmetric cryptography} that is used for
the remainder of the session established asymmetrically. The most common 
asymmetric cryptography algorithm is \textit{RSA}.

As a general sketch of asymmetric cryptography, let Alice and Bob have public keys
$P_{A}$, $P_{B}$ and private keys $S_{A}$, $S_{B}$, which you should think of as
sequences of 2,000 to 4,000 bits. \textit{Everyone} knows the public keys (just
post them "in the clear"), but only Alice knows $S_{A}$, and only Bob knows
$S_{B}$. We want to know two functions, both of which are easy to compute:
$Encrypt(M,P)$ and $Decrypt(M,S)$.

such that $Decrypt(Encrypt(M,P_{B}), S_{B}) == M$. If this formula is true, this
established the \textit{correctness} of a public key cryptographic operation, but
not necessarilt its security. To establish \textit{security} of public key
encryption, a message must be impossible to recover a message without $S_{B}$.

\begin{note}
    When you want to excrypt a message, you use the public key \textit{of the
    recipient} to do so. This means that a sender cannot decrypt the message that
    they encrypt.

    Decryption is performed with the \textit{private key} of the recipient.
\end{note}

\subsection{Digital signatures}

Say Alice wants to prove to Bob that \textit{she} sent a message. We want two 
functions, both of which are easy to compute: $Sign(M,S)$ and $Verify(M,P)$
such that
\[
Verify(Sign(M,S_{A}), P_{A}) == M,
\]
where $M$ is the plain text of a sent message. 

\subsection{Outline for RSA encryption/decryption}
\begin{definition}
    Two numbers are \textbf{relatively prime} if they share no common factors
    between each other, even if neither are actually prime.
\end{definition}

The following is how Bob creates his cryptographic keys:
\begin{enumerate}
    \item Choose 2 large random prime numbers $p$ and $q$
    \item The public key is $(N=pq,e)$ where $e$ is \textit{relatively prime} to
        $(p-1)(q-1)$.
    \item The private key is the inverse of $(e \text{ mod } (p-1)(q-1))$. Call it
        $d$.
\end{enumerate}
Here, both $N$ and $e$ are public, which means they are available for anyone
looking for them to see. Now, suppose Alice wants to send the message $M$ to Bob:

\begin{enumerate}
    \item She looks up his public key $(N,e)$ and sends $y = M^{e} \% N$.
    \item Bob decodes the message by computing $(y^{d} \text{ mod } N)$.
\end{enumerate}

To see that this is correct, observe that the final result Bob sees is 
\[
    y^{d} \text{ mod } N = (M^{e} \text{ mod } N)^{d}.
\]
\begin{note}
    Note that the last step relies on how the keys are generated.
\end{note}

This method (RSA) is \textit{not} information-theoretically secure. That is, it
can be computed given infinite computing power. But it is \textbf{computationally
secure}.

\begin{definition}
    \textbf{Computationally secure} cryptography \textit{is} possible to crack if
    given infinite amounts of computing power, but using current computing 
    (including super-computing) is intractable.

    \begin{note}
        Quantum computers are known to be vastly superior to classical computers
        when factoring numbers, so traditional computationally secure methods will
        not be in an era of quantum computing.
    \end{note}
\end{definition}

\end{document}

\documentclass[titlepage, 12pt, leqno]{article}

% -------------------------------------------------- %
% -------------------- PACKAGES -------------------- %
% -------------------------------------------------- %
\usepackage{import}
\usepackage{pdfpages}
\usepackage{mathtools}
\usepackage{transparent}
\usepackage{enumitem}
\usepackage{xcolor}
\usepackage{tcolorbox}
\usepackage{amsmath}
\usepackage{amssymb}
\usepackage{parskip}
\usepackage{bbm}
\usepackage[margin = 1in]{geometry}
\tcbuselibrary{breakable}
\tcbset{breakable = true}


% -------------------------------------------------- %
% -------------- CUSTOM ENVIRONMENTS --------------- %
% -------------------------------------------------- %
\newtcolorbox{note}{colback=black!5!white,
                          colframe=black!55!white,
                          fonttitle=\bfseries,title=Note}

\newtcolorbox{ex}{colback=blue!5!white,
                          colframe=blue!55!white,
                          fonttitle=\bfseries,title=Example}

\newtcolorbox{definition}{colback=red!5!white,
                          colframe=red!55!white,
                          fonttitle=\bfseries,title=Definition}


% -------------------------------------------------- %
% ------------------- COMMANDS --------------------- %
% -------------------------------------------------- %
% Brackets, braces, etc. 
\newcommand{\abs}[1]{\lvert #1 \rvert}
\newcommand{\bigabs}[1]{\Bigl \lvert #1 \Bigr \rvert}
\newcommand{\bigbracket}[1]{\Bigl [ #1 \Bigr ]}
\newcommand{\bigparen}[1]{\Bigl ( #1 \Bigr )}
\newcommand{\ceil}[1]{\lceil #1 \rceil}
\newcommand{\floor}[1]{\lfloor #1 \rfloor}
\newcommand{\norm}[1]{\| #1 \|}
\newcommand{\bignorm}[1]{\Bigl \| #1 \Bigr \| #1}
\newcommand{\inner}[1]{\langle #1 \rangle}
\newcommand{\set}[1]{{ #1 }}


% -------------------------------------------------- %
% -------------------- SETUP ----------------------- %
% -------------------------------------------------- %
\title{\Huge{Lecture 8 - Cryptography (cont'd) and Online Advertising}}
\author{\large{Mitch Harrison}}
\date{\today}   
\begin{document}
\setlength{\parskip}{1\baselineskip}
\setlength{\parindent}{15pt}
\maketitle
\tableofcontents
\newpage


% -------------------------------------------------- %
% --------------------- BODY ----------------------- %
% -------------------------------------------------- %
\section{Prime Numbers}

\subsection{Generating large primes}
The most effecient known algorithm for creating a large prime number is to 
randomly generate a large number and check to see if it's prime. This is where
random number generation becomes key. As stated before, built-in random number
generation functions (e.g. \texttt{math.random()}) are insufficient for this
random number.

To check to see if a number is prime with 100\% certainty is computationally
intractable. Instead, several numbers are selected at random number and the 
pseudorandom number is checked for factorization with that number. If none are
factors (using \textit{Fermat's Little Theorem} normally), we say that the large
number is an \textbf{industry-grade prime}.

\begin{definition}
    \textbf{Lagrange's prime number theorem}. Let $\pi(N)$ be the number of primes
    less than or equal to $N$.
    \[
    \lim_{N \to \infty} \frac{\pi(N)}{N/ln(N)} = 1
    \]
    Effectively, this states that roughly 2,000 numbers will need to be
    generated before a prime (or at least an industry-grade prime) will be found.
\end{definition}

\subsection{Calculating a modular inverse}
To get "the inverse of $e \text{ mod } (p-1)(q-1)$, and call it $d$," (per the
last lecture notes) we use the \textbf{Extended Euclid Algorithm} (EEA), which 
works as follows:

\begin{enumerate}
    \item Input $a \ge b \ge 0$
    \item Output $x,y$ such that $ax + by = GCD(a,b)$
    \item Base case: if $b = 0$, return $(1,0)$
    \item $x,y = \mathbb{E}A(b,a \text{ mod }b)$ 
    \item Return $(y, x - |a/b|y)$.
\end{enumerate}

To show that this modular inverse calculation works, let $\phi = (p-1)(q-1)$. Get 
$x,y = EEA(\phi ,e)$. \textbf{Claim:} $y = \text{inverse of }e \text{ mod }\phi$.
Proof:
\begin{align*}
    \phi x + ye &= GCD(\phi,e) \\
                &= 1 - \phi x
                &\rightarrow ye \text{ mod }\phi 
\end{align*}

But given this modular inverse, how do we look up a user's public key $(N,e)$ and
send $y = M^{e} \text{ mod } N$. Say we want to compute $a^{d} \text{ mod }m$. 
The "slow way" would simply be to multiply $a$ by itself $b$ times. But with 
numbers this size, these multiplications are \textit{not} constant time on modern
arithmetic machines. Instead, we use
\[
\left(\left(\left(a \cdot a\right) \text{ mod }m \cdot a\right) \text{ mod }m
    \cdots a \right).
\]
We do this $b$ times. The slow way takes $O(2^{2(2^{n}})$, but the modulo way
takes $O(2^{n}n^{2})$, which is still exponential, but much better. But there is
a better algorithm that is cubic.

If you use an $n$-bit key for RSA, sending an $m>n$-bit message takes $O(mn^{2})$
time. Recall the message must be smaller than the key, so you break it into 
chunks. For small constant $n$ (i.e. a 64-bit key) this is reasonably fast, but
you need at least a 2000-bit key for good security in practice. This gives rise
to a $2000^{2}$-factor slowdown compared to just sending the plain text.

\subsection{HTTPS}
Symmetric private key encryption schemes are generally much faster, closer to
linear time in the length of the message. The problem was just how to communicate
the shared private key. That's where RSA comes in. We use the slower RSA to 
exchange these private keys, but can then use the faster symmetric encryption
for the remaining communications. This is how HTTPS works.

\pagebreak
\section{Online Advertising}

Google, Facebook, and effectively all other "free" services online are monetized
with advertizing in the 90\% or above of total revenue range. This raises
concerns about online advertisement and privacy.

A user's information, browsing, and shopping habits are sold to companies that
use that data to target advertisements. As recently as 2018, Verizon, AT\&T, and T
Mobile pledged to stop selling phone location data to data brokers. Cookies have
also been in the popular zeitgeist recently because of the requirements that
websites ask permission to use cookies.

\begin{definition}
    \textbf{Cookies }are short little files that are (typically) stored on a
    user's device to maintain status information (e.g. a "shopping cart"). 
    Cookies are usually devided into \textit{functional }cookies and
    \textit{third-party} cookies, the latter of which collects your use history,
    typically for selling to advertisers/data brokers.
\end{definition}

\subsection{The theory of auctions}

There are a few places a user could see an ad, some of which include banner ads or
search result ads. Those ads are sold (typically) using auction in real-time as
a page is loaded or search results are generated.

Many advertisers could be interested in showing you an ad (although not 
\textit{all}). Advertisers use your user data to decide if they are interested in
showing you an add. For example, a Google search ad would look at your search 
history.

\subsubsection{First price sealed bid auctions}
There are $n$ bidders interested in some item, say an ad slot. Each bidder $i$
has a valuation $v_{i}$ for the item. Each bidder reports a bid $b_{i}$. Give the
item to the bidder who reports the largest bid $b_{i}$ and charge $b_{i}$ dollars
for it. This raises a problem, however.

Suppose you know that you are likely to win. You have an incentive to 
\textit{shave} your bid (report $b_{i} < v_{i}$) so that you pay less. This is to
say that \textit{first price sealed bid auctions are not strategy-proof}.

\subsubsection{Second price sealed bid auctions}
The only change here from first price auctions is that instead of charging the
winner $b_{i}$ dollars, you charge the amount that the \textit{second-highest}
bidder bid. This prevents income loss from every bidder shaving their bids using
first price auctions.

\subsection{How the Google ads auction works}
Interestingly, Google does \textit{not} charge per display of an ad. They charge
per \textit{click} of the ad. You tell Google how much you are willing to pay for
a \textit{click} of the ad. This is a \textbf{conversion}. Advertisers are not
charged at all for the display of ads. This increases the amount that 
advertisers are willing to spend for a Google search ad slot.

On Google's end, they have to begin to try to figure out how likely a user is to
click on an ad before they can provide a valuation for that ad slot. Machine
learning is used for this, but we will discuss that further into the course.

There are $n$ bidders intrested in purchasing various ads. Each has a total 
budget $B_{i}$ that we cannot exceed. In each time step $t$ from 1 to $T$, we 
have an ad to auction and each bidder $i$ reports a bid $b_{i}$ for the ad. 
Suppose that for each ad, we just give it to the highest bidder and charge that
bidder their bid. Because of budget constraints set by advertisers, this may not
actually be an optimal solution.

\begin{note}
    Google advertisers can set a monthly/weekly/etc. budget that Google 
    \textit{cannot} exceed per its own terms of service.
\end{note}

\end{document}

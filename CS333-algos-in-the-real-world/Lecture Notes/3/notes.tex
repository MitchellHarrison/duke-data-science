\documentclass[titlepage, 12pt, leqno]{article}

% -------------------------------------------------- %
% -------------------- PACKAGES -------------------- %
% -------------------------------------------------- %
\usepackage{import}
\usepackage{pdfpages}
\usepackage{mathtools}
\usepackage{transparent}
\usepackage{xcolor}
\usepackage{tcolorbox}
\usepackage{amsmath}
\usepackage{amssymb}
\usepackage{parskip}
\usepackage{bbm}
\usepackage[margin = 1in]{geometry}
\tcbuselibrary{breakable}
\tcbset{breakable = true}


% -------------------------------------------------- %
% -------------- CUSTOM ENVIRONMENTS --------------- %
% -------------------------------------------------- %
\newtcolorbox{note}{colback=black!5!white,
                          colframe=black!55!white,
                          fonttitle=\bfseries,title=Note}

\newtcolorbox{ex}{colback=blue!5!white,
                          colframe=blue!55!white,
                          fonttitle=\bfseries,title=Example}

\newtcolorbox{definition}{colback=red!5!white,
                          colframe=red!55!white,
                          fonttitle=\bfseries,title=Definition}


% -------------------------------------------------- %
% ------------------- COMMANDS --------------------- %
% -------------------------------------------------- %
% Brackets, braces, etc. 
\newcommand{\abs}[1]{\lvert #1 \rvert}
\newcommand{\bigabs}[1]{\Bigl \lvert #1 \Bigr \rvert}
\newcommand{\bigbracket}[1]{\Bigl [ #1 \Bigr ]}
\newcommand{\bigparen}[1]{\Bigl ( #1 \Bigr )}
\newcommand{\ceil}[1]{\lceil #1 \rceil}
\newcommand{\floor}[1]{\lfloor #1 \rfloor}
\newcommand{\norm}[1]{\| #1 \|}
\newcommand{\bignorm}[1]{\Bigl \| #1 \Bigr \| #1}
\newcommand{\inner}[1]{\langle #1 \rangle}
\newcommand{\set}[1]{{ #1 }}


% -------------------------------------------------- %
% -------------------- SETUP ----------------------- %
% -------------------------------------------------- %
\title{\Huge{Lecture 3 - Research Projects and How to Research}}
\author{\large{Mitch Harrison}}
\date{\today}   
\begin{document}
\setlength{\parskip}{1\baselineskip}
\setlength{\parindent}{15pt}
\maketitle
\tableofcontents
\newpage


% -------------------------------------------------- %
% --------------------- BODY ----------------------- %
% -------------------------------------------------- %
\section{Project Overview}

We will be choosing a real-world algorithmic system, researching how it work,
developing a working example, and take a position on the societal impacts of that
algorithmic system. The requirement to choose \textit{any} algorithmic system,
but it has to have a societal impact component. 
\begin{note}
    Examples of algorithmic systems are YouTube recommendation systems, 
    self-driving cars, social media, and algorithmic gerrymandering/district
    drawing.
\end{note}

\subsection{Requirements}

The first component of the project is an individual proposal. We will use those 
proposals to find relevant group members, with whom we will submit a group 
proposal some time later. Proposals will identify which algorithmic systems will
be studied, choose a research question, and an annotated bibliography.
\begin{note}
    Groups will consist of four members.
\end{note}

\subsection{Technical demo}

This will be a working software artifact related to the chosen algorithmic system.
Any external libraries, APIs, packages, frameworks, etc. are allowed. Ideally,
this will be a piece of software that is public-facing and that you are proud to
show to peers, family, or potential employers. Substantial algorithmic components
must be implemented, however. Masking API calls behind an interface alone does 
not suffice. This project must be hosted using Git (Gitlab \textit{or} GitHub).

\begin{note}
    There will be a 10-12 minute presentation to the class at the end of the
    semester.
\end{note}

\subsection{Project site}

The final project website must include all of the following:
\begin{enumerate}
    \item An overview/landing page introducing the system to the user.
    \item A \textbf{how it works} page that explains the system underlying the
        website.
    \item A \textbf{technical demo} page that highlights the technical demo
        listed above.
    \item A \textbf{societal implications} page. Group members need not agree on
        the societal implications of the system, so each member should provide a
        separate reflection on the topic.
\end{enumerate}

\pagebreak
\section{Researching}
\subsection{What do researchers do?}
First, researchers must \textit{learn}. Diving into literature and finding the
current state of the art is a major component of research. Once a researcher is
comfortable in their space, they problematize.
\begin{definition}
    \textbf{Problematizing} is the act of turning forming a concrete problem which
    could be addressed with research.
\end{definition}
After problematizing, researchers theorize possible ways to attack the problem,
test those theories, and communicate their results.

\subsection{Posing a research question}
Research questions should be \textit{specific} (i.e. not "How does this 
algorithmic system work?"), \textit{testable}, \textit{novel}, 
\textit{substantial}, and \textit{relevant} to algorithms and algorithmic
systems in the real world.

\subsection{Searching literature and finding good sources}
In brief, observe that the citation network of a body of research is a 
\textit{directed graph}. Once a few "seed set" of papers are found, citations can
help track the movement of the body of research forward and backward in time. If a
paper is well-cited, that can be a good sign, but that can be difficult to 
evaluate.

A source is \textbf{credible} on a topic if the authors are experienced and
recognized experts and the source gives a nontrivial depth of exposition. Credible
sources are (usually) peer-reviewed publications in top journals or conferences,
tech reports, blogs, books, podcasts, etc. from recognized industry experts.

\begin{note}
    Duke libraries has access to major research databases. IEEE Explore digital
    libraries and ACM digital libraries are major professional societies for
    computer scientists. Others include AAAI (for AI) and arXiv (these are
    pre-print [i.e. not peer reviewed])

    Some publications that have a stronger societal focus include SIGCHI, 
    ACM FAccT, AI Ethics and Society Conference, and KDD.

    Some technical publications in AI and ML include AAAI, IJCAI, ICML, NeurIPS,
    CVPR, and ACL.
\end{note}

\subsection{How to read a paper}
Especially early in the research process, it is better to shallowly read several
pepers than to deeply read one, especially with how difficult and dense research
can be. The \textit{abstract}, the \textit{intro}, and the \textit{main results}
areas can be helpful here. If you run into a term you don't know, highlight it in
some way and move on.

After skimming, it may be wise to take 1-2 hours to read deeper and uncover
whether or not that work will be relevant to your own work. If it will be, then
deep reading and understanding of technical details becomes necessary.

\end{document}

\documentclass[titlepage, 12pt, leqno]{article}

% -------------------------------------------------- %
% -------------------- PACKAGES -------------------- %
% -------------------------------------------------- %
\usepackage{import}
\usepackage{pdfpages}
\usepackage{mathtools}
\usepackage{transparent}
\usepackage{xcolor}
\usepackage{tcolorbox}
\usepackage{amsmath}
\usepackage{amssymb}
\usepackage{parskip}
\usepackage{bbm}
\usepackage[margin = 1in]{geometry}
\tcbuselibrary{breakable}
\tcbset{breakable = true}


% -------------------------------------------------- %
% -------------- CUSTOM ENVIRONMENTS --------------- %
% -------------------------------------------------- %
\newtcolorbox{note}{colback=black!5!white,
                          colframe=black!55!white,
                          fonttitle=\bfseries,title=Note}

\newtcolorbox{ex}{colback=blue!5!white,
                          colframe=blue!55!white,
                          fonttitle=\bfseries,title=Example}

\newtcolorbox{definition}{colback=red!5!white,
                          colframe=red!55!white,
                          fonttitle=\bfseries,title=Definition}


% -------------------------------------------------- %
% ------------------- COMMANDS --------------------- %
% -------------------------------------------------- %
% Brackets, braces, etc. 
\newcommand{\abs}[1]{\lvert #1 \rvert}
\newcommand{\bigabs}[1]{\Bigl \lvert #1 \Bigr \rvert}
\newcommand{\bigbracket}[1]{\Bigl [ #1 \Bigr ]}
\newcommand{\bigparen}[1]{\Bigl ( #1 \Bigr )}
\newcommand{\ceil}[1]{\lceil #1 \rceil}
\newcommand{\floor}[1]{\lfloor #1 \rfloor}
\newcommand{\norm}[1]{\| #1 \|}
\newcommand{\bignorm}[1]{\Bigl \| #1 \Bigr \| #1}
\newcommand{\inner}[1]{\langle #1 \rangle}
\newcommand{\set}[1]{{ #1 }}


% -------------------------------------------------- %
% -------------------- SETUP ----------------------- %
% -------------------------------------------------- %
\title{\Huge{OLS Estimator (1)}}
\author{\large{Mitch Harrison}}
\date{\today}   
\begin{document}
\setlength{\parskip}{1\baselineskip}
\setlength{\parindent}{15pt}
\maketitle
\tableofcontents
\newpage


% -------------------------------------------------- %
% --------------------- BODY ----------------------- %
% -------------------------------------------------- %
\section{Linear Regression Models}

Linear regression models are linear \textit{in the parameters}. That is, for a 
given observation $Y_i$:
\[
    Y_i = \beta_0 + \beta_1f_1(X_{i1}) + \cdots + \beta_pf_p(X_{ip}) + \epsilon_i
\]
The functions $f_1, \cdots , f_p$ may themselves be non-linear, but as long as
$\beta$ is linear in $y$, we have a linear regression model.

\subsection{Loss Functions}
In STA210, we minimized the squared \textbf{residuals} to fit our models.
\begin{definition}
    \textbf{Residuals} are calculated as follows:
    \[
        \hat \epsilon = y_i - \hat{y_i} = y_i - (\beta_0 + \beta_1x_1)
    \]
\end{definition}
Minimizing squared residuals is only one of a potentially infinite number of 
possible loss functions.

\subsection{Interaction models}
Recall from STA210 that \textbf{interaction models} are models where some terms
contain multiples of more than one variable. For example:
\[
    y_i = \beta_0 + \beta_1x_1 + \beta_2x_2 + \beta_3x_1x_2
\]
Note that the $\beta_3$ term contains the interaction.

\pagebreak
\section{R tools}
The following are available in R:
\begin{itemize}
    \item \texttt{as.matrix()} function sets an object as a matrix object in R.
    \item \texttt{\%*\%} is the matrix multiplication operator in R. For example,
        \texttt{C <- A \%*\% B} where \texttt{A} and \texttt{B} are R matrix
        objects.
    \item \texttt{t()}  function takes the transpose of a matrix.
    \item \texttt{solve()} function inverts a matrix.
\end{itemize}


\end{document}

\documentclass[titlepage, 12pt, leqno]{article}

% -------------------------------------------------- %
% -------------------- PACKAGES -------------------- %
% -------------------------------------------------- %
\usepackage{import}
\usepackage{pdfpages}
\usepackage{mathtools}
\usepackage{transparent}
\usepackage{xcolor}
\usepackage{tcolorbox}
\usepackage{amsmath}
\usepackage{amssymb}
\usepackage{parskip}
\usepackage{bbm}
\usepackage[margin = 1in]{geometry}
\tcbuselibrary{breakable}
\tcbset{breakable = true}


% -------------------------------------------------- %
% -------------- CUSTOM ENVIRONMENTS --------------- %
% -------------------------------------------------- %
\newtcolorbox{note}{colback=black!5!white,
                          colframe=black!55!white,
                          fonttitle=\bfseries,title=Note}

\newtcolorbox{ex}{colback=blue!5!white,
                          colframe=blue!55!white,
                          fonttitle=\bfseries,title=Example}

\newtcolorbox{definition}{colback=red!5!white,
                          colframe=red!55!white,
                          fonttitle=\bfseries,title=Definition}


% -------------------------------------------------- %
% ------------------- COMMANDS --------------------- %
% -------------------------------------------------- %
% Brackets, braces, etc. 
\newcommand{\abs}[1]{\lvert #1 \rvert}
\newcommand{\bigabs}[1]{\Bigl \lvert #1 \Bigr \rvert}
\newcommand{\bigbracket}[1]{\Bigl [ #1 \Bigr ]}
\newcommand{\bigparen}[1]{\Bigl ( #1 \Bigr )}
\newcommand{\ceil}[1]{\lceil #1 \rceil}
\newcommand{\floor}[1]{\lfloor #1 \rfloor}
\newcommand{\norm}[1]{\| #1 \|}
\newcommand{\bignorm}[1]{\Bigl \| #1 \Bigr \| #1}
\newcommand{\inner}[1]{\langle #1 \rangle}
\newcommand{\set}[1]{{ #1 }}


% -------------------------------------------------- %
% -------------------- SETUP ----------------------- %
% -------------------------------------------------- %
\title{\Huge{OLS Estimator (2)}}
\author{\large{Mitch Harrison}}
\date{\today}   
\begin{document}
\setlength{\parskip}{1\baselineskip}
\setlength{\parindent}{15pt}
\maketitle
\tableofcontents
\newpage


% -------------------------------------------------- %
% --------------------- BODY ----------------------- %
% -------------------------------------------------- %
\section{Linear Albegra Review}

A set of vectors is \textbf{linearly independent} if no linear combination 
(besides all zeroes) of the vectors equals the zero vector. That is, if none of
the vectors can be written as a \textbf{linear combination} of the others (and
none are the zero vector).

The \textbf{span} of a set of vectors is the set of all possible linear 
combinations of them. You may also recall that a linearly independent set of
vectors that spans a subspace forms a \textbf{basis} for that subspace, but that
is less relevant for today.

The \textbf{column space} of $X$ is the span of the columns of $X$.

If a vector $Y$ does not sit within the column space of a matrix $X$, we find
the nearest solution that \textit{does} lie within the column space by
\textbf{projecting} onto that column space. 

\begin{definition}
    \textbf{Projecting} onto a column space involves finding the vector that is
    within the column space by finding a line that runs through the tip of a
    vector outside of the column space and the column space itself 
    perpendicularly. A vector pointing to the point where that line touches the
    column space is the \textbf{projection} of a vectore $Y$ onto $Col(X)$.
\end{definition}

\subsection{Finding projections}
A vector $e$ is orthogonal to the plane $C(X)$ (that is, the plane spanned by the
variables in $X$). This means that for any vector in $C(X)$, the inner product
between this vector and $e$ is 0.

This means that $X^Te =0$. Also notice that $e = y-z$, and $z = Xw$ for some 
vector $w$:
\[
    X^T(y-Xw) = 0
\]

\end{document}

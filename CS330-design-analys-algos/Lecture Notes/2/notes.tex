\documentclass[titlepage, 12pt, leqno]{article}

% -------------------------------------------------- %
% -------------------- PACKAGES -------------------- %
% -------------------------------------------------- %
\usepackage{import}
\usepackage{pdfpages}
\usepackage{mathtools}
\usepackage{transparent}
\usepackage{enumitem}
\usepackage{xcolor}
\usepackage{tcolorbox}
\usepackage{amsmath}
\usepackage{amssymb}
\usepackage{parskip}
\usepackage{bbm}
\usepackage{algorithm}
\usepackage{algpseudocode}
\usepackage{algpseudocodex}
\usepackage[margin = 1in]{geometry}
\tcbuselibrary{breakable}
\tcbset{breakable = true}


% -------------------------------------------------- %
% -------------- CUSTOM ENVIRONMENTS --------------- %
% -------------------------------------------------- %
\newtcolorbox{note}{colback=black!5!white,
                          colframe=black!55!white,
                          fonttitle=\bfseries,title=Note}

\newtcolorbox{ex}{colback=blue!5!white,
                          colframe=blue!55!white,
                          fonttitle=\bfseries,title=Example}

\newtcolorbox{definition}{colback=red!5!white,
                          colframe=red!55!white,
                          fonttitle=\bfseries,title=Definition}


% -------------------------------------------------- %
% ------------------- COMMANDS --------------------- %
% -------------------------------------------------- %
% Brackets, braces, etc. 
\newcommand{\abs}[1]{\lvert #1 \rvert}
\newcommand{\bigabs}[1]{\Bigl \lvert #1 \Bigr \rvert}
\newcommand{\bigbracket}[1]{\Bigl [ #1 \Bigr ]}
\newcommand{\bigparen}[1]{\Bigl ( #1 \Bigr )}
\newcommand{\ceil}[1]{\lceil #1 \rceil}
\newcommand{\floor}[1]{\lfloor #1 \rfloor}
\newcommand{\norm}[1]{\| #1 \|}
\newcommand{\bignorm}[1]{\Bigl \| #1 \Bigr \| #1}
\newcommand{\inner}[1]{\langle #1 \rangle}
\newcommand{\set}[1]{{ #1 }}


% -------------------------------------------------- %
% -------------------- SETUP ----------------------- %
% -------------------------------------------------- %
\title{\Huge{Lecture 2 - Proof by Induction and Iterative Algorithms}}
\author{\large{Mitch Harrison}}
\date{\today}   
\begin{document}
\setlength{\parskip}{1\baselineskip}
\setlength{\parindent}{15pt}
\maketitle
\tableofcontents
\newpage


% -------------------------------------------------- %
% --------------------- BODY ----------------------- %
% -------------------------------------------------- %
\section{Proof by Induction}

\begin{definition}
    \textbf{Induction} is useful when we want to prove something for every
    kind of statement in a sequence. For example, if we want to show that we
    can knock over all dominos in a line, we need only prove that one domino
    makes the next fall and that we can knock over the first domino.
\end{definition}

In induction we first want to prove the base case $P(0)$, and then show that
\[
P(k-1) \rightarrow P(k).
\]
The general workflow for proving by induction is
\begin{enumerate}
    \item Prove the \textit{base case}. This is $P(0)$.
    \item Show the inductive hypothesis for the problem. This usually takes the
        form of \textit{suppose this is true}.
    \item Take the \textit{inductive step}, showing that the inductive
        hypothesis holds for any $k$ (or whichever variable we are proving for).
\end{enumerate}

There are only two ways to test correctness of an algorithm. We can either test
experimentally or prove mathematically. The former option can never truly show
that an algorithm works for all cases, but the latter may introduce bugs in its
implementation that wouldn't be revealed until testing experimentally.

Inductive proofs are common in algorithms because you can do induction over
\textit{iterations} (for things like for loops) or \textit{sub-problems} for
problems involving recurrsion.

\pagebreak

\subsubsection{Example}
Take the following algorithm for finding the maximum in an array $A$:
\begin{algorithm}
\caption{Find the maximum}
\begin{algorithmic}
\Procedure{max}{$A,n$}
\State $m = A[0]$
\Comment{This is the base case.}
\For{$i=1$ to $n-1$}
\If{$A[i] > m$}
\State $m = A[i]$
    \EndIf
\EndFor
\State \Return $m$
\EndProcedure 
\end{algorithmic}
\end{algorithm}

We claim that $m_{k} = max(A[0], \cdots A[k])$ for all $0 \le k < n$. We 
establish a \textit{base case} where we set $m = A[0] = max(A[0])$. The
\textit{inductive hypothesis} tells us that 
$m_{k-1} = max(A[0], \cdots ,A[k-1])$. To take the inductive step, we need to 
consider two cases. Either
\begin{enumerate}
    \item $A[k] > m_{k-1}$ or
    \item $A[k] \le m_{k-1}$.
\end{enumerate}

The math (shown in course slides) shows that $m_{k} = max(A[0], \cdots ,A[k])$
for both cases.
\begin{note}
    No finite number of test cases can prove that an algorithm is \textit{always}
    correct.
\end{note}

\end{document}

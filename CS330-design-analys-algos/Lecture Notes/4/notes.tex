\documentclass[titlepage, 12pt, leqno]{article}

% -------------------------------------------------- %
% -------------------- PACKAGES -------------------- %
% -------------------------------------------------- %
\usepackage{import}
\usepackage{pdfpages}
\usepackage{mathtools}
\usepackage{transparent}
\usepackage{enumitem}
\usepackage{xcolor}
\usepackage{tcolorbox}
\usepackage{amsmath}
\usepackage{amssymb}
\usepackage{parskip}
\usepackage{bbm}
\usepackage{algorithm}
\usepackage{algpseudocode}
\usepackage{algpseudocodex}
\usepackage[margin = 1in]{geometry}
\tcbuselibrary{breakable}
\tcbset{breakable = true}


% -------------------------------------------------- %
% -------------- CUSTOM ENVIRONMENTS --------------- %
% -------------------------------------------------- %
\newtcolorbox{note}{colback=black!5!white,
                          colframe=black!55!white,
                          fonttitle=\bfseries,title=Note}

\newtcolorbox{ex}{colback=blue!5!white,
                          colframe=blue!55!white,
                          fonttitle=\bfseries,title=Example}

\newtcolorbox{definition}{colback=red!5!white,
                          colframe=red!55!white,
                          fonttitle=\bfseries,title=Definition}


% -------------------------------------------------- %
% ------------------- COMMANDS --------------------- %
% -------------------------------------------------- %
% Brackets, braces, etc. 
\newcommand{\abs}[1]{\lvert #1 \rvert}
\newcommand{\bigabs}[1]{\Bigl \lvert #1 \Bigr \rvert}
\newcommand{\bigbracket}[1]{\Bigl [ #1 \Bigr ]}
\newcommand{\bigparen}[1]{\Bigl ( #1 \Bigr )}
\newcommand{\ceil}[1]{\lceil #1 \rceil}
\newcommand{\floor}[1]{\lfloor #1 \rfloor}
\newcommand{\norm}[1]{\| #1 \|}
\newcommand{\bignorm}[1]{\Bigl \| #1 \Bigr \| #1}
\newcommand{\inner}[1]{\langle #1 \rangle}
\newcommand{\set}[1]{{ #1 }}


% -------------------------------------------------- %
% -------------------- SETUP ----------------------- %
% -------------------------------------------------- %
\title{\Huge{Lecture 4 - Divide and Conquer pt. 1}}
\author{\large{Mitch Harrison}}
\date{\today}   
\begin{document}
\setlength{\parskip}{1\baselineskip}
\setlength{\parindent}{15pt}
\maketitle
\tableofcontents
\newpage


% -------------------------------------------------- %
% --------------------- BODY ----------------------- %
% -------------------------------------------------- %
\section{Divide and Conquer}

\begin{definition}
    A \textbf{divide and conquer} algorithm is a recursive algorithm that makes
    \textit{non-overlapping} recursive calls. That is, all sub-problems are
    disjoint subsets.
\end{definition}

Divide and conquer algorithms are great for parallel algorithms (covered later
in the course), and sometimes more efficient non-parallel algorithms (covered
later this week).

Take the following algorithm to find the maximum value in an array:
\begin{algorithm}
\caption{Divide and conquer maximum-finding}
\begin{algorithmic}
\Procedure{DCMax}{$A,l,r$}
    \If{$l = r$}
    \State \Return $A[r]$
    \Comment{base case}
    \Else
        \State $m = \lfloor \frac{l+r}{2} \rfloor$
        \State $a = $ \Call{DCMax}{$A, l, m$}
        \State $b = $ \Call{DCMax}{$A, m+1, r$}
        \If{$a>b$}
            \State \Return $a$
        \Else
            \State \Return $b$
        \EndIf
    \EndIf
\EndProcedure 
\end{algorithmic}
\end{algorithm}

Our \textbf{base case} is when the left value is equal to the right value (i.e.
$l=r$). Observe that the recurssion tree is non-linear-- it branches left and
right into disjoint subtrees. That it, \textit{it is a divide-and-conquer
algorithm}.

In our base case (as is often the case with divide and conquer algorithms), the
asymptotic runtime is $O(1)$ because we are just returning a constant. Each
recursive call operates on a sub-problem that is half the size of the previous
one. Because there is a maximum of two calls on an input of size $n/2$, we
say that 
\[
T(n) - 2T\left(\frac{n}{2}\right) + c
\]
where $c$ is some unknown constant. To solve the recurrence relations, we can
use substitution:
\begin{align*}
    T(n) &\le 2T(n/2)+ c \\
         &\le 2[2T(n/2) + c] + c \\
         &= 4T(n/4) + 3c \\
         &\vdots\\
         &\le nT(1) + (n-1)c.
\end{align*}
This has $O(n)$ runtime complexity. This is not a formal proof but a good way
to quickly get an educated guess informally (possible on exam).

\section{Mergesort}
Let's say that we've designed an algorithm that runs in $O(n^{2})$ time. One
way to improve is to make a linear algorithm split subproblems geometrically
instead, which will allow much faster exploration of possible subproblems.
Enter \textbf{mergesort}.

Mergesort splits an array in half, and recursively sorts the left and right
halves. It then splits again until it is sorting sub-arrays of size $n=1$.
Once each half is sorted, they are merged again in sorted order. Interestingly,
the \textit{merge} step of mergesort requires a separate algorithm. The beauty
of mergesort is that you can reduce the problem of sorting array into a smaller
problem of merging two sorted arrays.

\subsection{Merge procedure}
Given two sorted arrays, we want to merge them into one with all values from
both, in sorted order. First, we look at the first element of each sub-array. We
take the smallest element of the two, copy it into our merged array, and
incriment the index of the sub-array we pulled it from. We continue comparing
in this way, incrimenting the index of whichever array we pull the smallest 
value from, to arrive at a final sorted list.

\subsection{Mergesort pseudocode}
\begin{algorithm}
\caption{Mergesort}
\begin{algorithmic}
\Procedure{Mergesort}{$A, l, r$}
    \If{$l < r$}
        \State $m = \lfloor (l+r) / 2 \rfloor$
        \State \Call{Mergesort}{$A, l, m$} 
        \State \Call{Mergesort}{$A, m+1, r$} 
        \State \Call{Merge}{$A,l,m,r$} 
    \EndIf
\EndProcedure 
\end{algorithmic}
\end{algorithm}
\begin{algorithm}
    \caption{Merge two sorted arrays into a new sorted array}
\begin{algorithmic}
\Procedure{Merge}{$A,l,m,r$}
    \State $n = r - l + 1$
    \State $B$ is new array of size $n$
    \State $i=l$
    \State $j = m+1$
    \For{$k=0$ to $n-1$}
        \If{$i>m$}
        \State $B[k] = A[j]$
        \State $j = j+1$
        \ElsIf{$j>r$}
        \State $B[k]= A[i]$
        \State $i = i+1$
        \ElsIf{$A[i] \le A[j]$}
        \State $B[k] = A[i]$
        \State $i = i+1$
        \Else
        \State $B[k] = A[j]$
        \State $j = j+1$
        \EndIf
        \State $k = k+1$
    \EndFor
    \For{$k=0$ to $n-1$}
    \State $A[l+k] = B[k]$
    \EndFor
\EndProcedure 
\end{algorithmic}
\end{algorithm}

\pagebreak

\subsection{Correctness of Mergesort}
Because \Call{Merge}{} doesn't depend on \Call{Mergesort}{}, but 
\Call{Mergesort}{} \textit{does} depend on \Call{Merge}{}. Thus, it is 
common to start with showing that \Call{Merge}{} is correct. Once we do, the
argument for the correctness of \Call{Mergesort}{} is relatively trivial. This is
a good general workflow for recursive procedure:
\begin{enumerate}
    \item Show that all sub-procedures are correct until you arrive at a
        procedure with no dependencies.
    \item Show that, given the sub-procedures are correct, a higher-level
        procedure is correct.
    \item Repeat until the main procedure is proved correct.
\end{enumerate}

\end{document}

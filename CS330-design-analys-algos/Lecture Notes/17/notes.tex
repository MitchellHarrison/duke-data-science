\documentclass[titlepage, 12pt, leqno]{article}

% -------------------------------------------------- %
% -------------------- PACKAGES -------------------- %
% -------------------------------------------------- %
\usepackage{import}
\usepackage{pdfpages}
\usepackage{mathtools}
\usepackage{transparent}
\usepackage{enumitem}
\usepackage{xcolor}
\usepackage{tcolorbox}
\usepackage{amsmath}
\usepackage{amssymb}
\usepackage{parskip}
\usepackage{bbm}
\usepackage{algorithm}
\usepackage{algpseudocode}
\usepackage{algpseudocodex}
\usepackage[margin = 1in]{geometry}
\tcbuselibrary{breakable}
\tcbset{breakable = true}


% -------------------------------------------------- %
% -------------- CUSTOM ENVIRONMENTS --------------- %
% -------------------------------------------------- %
\newtcolorbox{note}{colback=black!5!white,
                          colframe=black!55!white,
                          fonttitle=\bfseries,title=Note}

\newtcolorbox{ex}{colback=blue!5!white,
                          colframe=blue!55!white,
                          fonttitle=\bfseries,title=Example}

\newtcolorbox{definition}{colback=red!5!white,
                          colframe=red!55!white,
                          fonttitle=\bfseries,title=Definition}


% -------------------------------------------------- %
% ------------------- COMMANDS --------------------- %
% -------------------------------------------------- %
% Brackets, braces, etc. 
\newcommand{\abs}[1]{\lvert #1 \rvert}
\newcommand{\bigabs}[1]{\Bigl \lvert #1 \Bigr \rvert}
\newcommand{\bigbracket}[1]{\Bigl [ #1 \Bigr ]}
\newcommand{\bigparen}[1]{\Bigl ( #1 \Bigr )}
\newcommand{\ceil}[1]{\lceil #1 \rceil}
\newcommand{\floor}[1]{\lfloor #1 \rfloor}
\newcommand{\norm}[1]{\| #1 \|}
\newcommand{\bignorm}[1]{\Bigl \| #1 \Bigr \| #1}
\newcommand{\inner}[1]{\langle #1 \rangle}
\newcommand{\set}[1]{{ #1 }}


% -------------------------------------------------- %
% -------------------- SETUP ----------------------- %
% -------------------------------------------------- %
\title{\Huge{Lecture 17 - Flows and Cuts}}
\author{\large{Mitch Harrison}}
\date{\today}   
\begin{document}
\setlength{\parskip}{1\baselineskip}
\setlength{\parindent}{15pt}
\maketitle
\tableofcontents
\newpage


% -------------------------------------------------- %
% --------------------- BODY ----------------------- %
% -------------------------------------------------- %
\section{Definitions}

\subsection{Flows}

\begin{definition}
    A \textbf{capacitated directed graph} $G = (V,E,c)$ has a positive integer
    capacity $c:E \rightarrow \mathbb{N}$ for every edge.
\end{definition}

\begin{definition}
    An \textbf{$(s,t)$-flow} in a capacitated directed graph $G$ is a function
    $f:E \rightarrow \mathbb{R}_{\ge 0}$ that assigns a nonnegative value to
    every edge and satisfies
    \begin{enumerate}
        \item \textit{Capacity constraints} $f(u\rightarrow v) \le
            c(u \rightarrow v)$ for every edge $u \rightarrow v \in E$. That is,
            flow cannot exceed capacity.
        \item \textit{Flow conservation constraint}: inflow and outflow are the
            same for every edge. That is,
            \[
                \sum_{u\rightarrow v \in E}f(u\rightarrow v) =
                \sum_{v \rightarrow w \in E}f(v\rightarrow w),
            \]
            for every vertex \textit{except} $s$ and $t$.
    \end{enumerate}
\end{definition}

\begin{definition}
    The \textbf{value} of an $(s,t)$-flow $f$ is the net flow out of $s$. That
    is,
    \[
    |f| := \left(\sum_{s \rightarrow w \in E}f(s \rightarrow w)\right) -
    \left(\sum_{u \rightarrow s \in E}f(u \rightarrow s)\right).
    \]
    This necessarily equals the net flow into our target node $t$.
\end{definition}

\begin{definition}
    An edge is \textbf{saturated} if its flow is equivalent to its capacity and
    \textbf{unsaturated} if it is not.
\end{definition}

\pagebreak

\subsection{Cuts}

\begin{definition}
    An \textbf{$(s,t)$-cut} of a graph $G$ is a partition of the vertices into
    $S \subset V$, $T \subset V$ so that,
    \[
        S \cup T = V \text{, }S \cap T = \varnothing \text{, } 
        s \in S, \text{ and } t \in T
    \]
\end{definition}

\begin{definition}
    The \textbf{capacity} of an $(s,t)$-cut is the sum of the capacities of the
    \textit{cut edges} that cross from $S$ to $T$. That is,
    \[
        |(S,T)| = \sum_{u\rightarrow v, u \in S, v \in T}c(u\rightarrow v).
    \]
\end{definition}

\begin{definition}
    The \textbf{minimum cut problem} occurs when we, given $G$, we hope to
    compute a minimum capacity $(s,t)$-cut in $G$.
\end{definition}

\begin{definition}
    \textbf{Weak duality} claims that for a directed graph $G$, vertices
    $s,t \in V$, $(s,t)$-flow $f$ and $(s,t)$-cut $(S,T)$, the value of the
    flow will be at most the capacity of the cut. That is,
    \[
        |f| \le |(S,T)|.
    \]
    This necessitates the face that if $|f| = |(S,T)|$, we have found a
    maximal flow and a minimal cut, and every cut edge must be fully saturated.
\end{definition}

\begin{definition}
    \textbf{Strong duality} is a notion that claims (correctly) that a 
    flow and cut \textit{necessarily} exists such that
    \[
        |f^{*}| = |(S^{*}, T^{*})|.
    \]
    That is, an optimal solution exists.
\end{definition}

\pagebreak
\section{Residual Graphs}
A \textbf{residual graph} $G_{f}$ is constructed from the original graph $G$ by
\begin{enumerate}
    \item Drawing a forward edge from $u$ to $v$ with weight of whatever 
        capacity is leftover after sending flow over that edge in $G$.
    \item Drawing a backward edge from $v$ to $u$ with weight of whatever flow
        was being sent from $u$ to $v$ in the original graph $G$. Note that this
        is in the reverse direction of the flow of $G$.
\end{enumerate}

A path $P_{s\rightarrow t}$ from $s$ to $t$ in $G_{f}$ is an \textbf{augmenting
path} with \textbf{bottleneck capacity}
\[
    b(P_{s\rightarrow t}) = min_{u\rightarrow v \in P_{s\rightarrow t}}
    c_{f}(u \rightarrow v).
\]
Given an augmenting path $P$ in $G_{f}$ with bottleneck capacity $b$, we define
the augmented flow (in $G$) as
\[
f'(u\rightarrow v) =
\begin{cases}
    f(u\rightarrow v) +b & u\rightarrow v \in P \\
    f(u\rightarrow v)-b & v\rightarrow u \in P\\
    f(u\rightarrow v) & \text{otherwise}
\end{cases}
\]
\subsection{Augmented flow properties}
Augmented flow has some helpful properties. First, augmented flow obeys capacity
constraints. Second, augmented flows obey flow conservation extraints because
they are only changed by bottleneck capacity $b$ for vertices on the augmenting
path $P$.

\end{document}

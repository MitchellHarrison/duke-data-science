\documentclass[titlepage, 12pt, leqno]{article}

% -------------------------------------------------- %
% -------------------- PACKAGES -------------------- %
% -------------------------------------------------- %
\usepackage{import}
\usepackage{pdfpages}
\usepackage{mathtools}
\usepackage{transparent}
\usepackage{enumitem}
\usepackage{xcolor}
\usepackage{tcolorbox}
\usepackage{amsmath}
\usepackage{amssymb}
\usepackage{parskip}
\usepackage{bbm}
\usepackage{algorithm}
\usepackage{algpseudocode}
\usepackage{algpseudocodex}
\usepackage[margin = 1in]{geometry}
\tcbuselibrary{breakable}
\tcbset{breakable = true}


% -------------------------------------------------- %
% -------------- CUSTOM ENVIRONMENTS --------------- %
% -------------------------------------------------- %
\newtcolorbox{note}{colback=black!5!white,
                          colframe=black!55!white,
                          fonttitle=\bfseries,title=Note}

\newtcolorbox{ex}{colback=blue!5!white,
                          colframe=blue!55!white,
                          fonttitle=\bfseries,title=Example}

\newtcolorbox{definition}{colback=red!5!white,
                          colframe=red!55!white,
                          fonttitle=\bfseries,title=Definition}


% -------------------------------------------------- %
% ------------------- COMMANDS --------------------- %
% -------------------------------------------------- %
% Brackets, braces, etc. 
\newcommand{\abs}[1]{\lvert #1 \rvert}
\newcommand{\bigabs}[1]{\Bigl \lvert #1 \Bigr \rvert}
\newcommand{\bigbracket}[1]{\Bigl [ #1 \Bigr ]}
\newcommand{\bigparen}[1]{\Bigl ( #1 \Bigr )}
\newcommand{\ceil}[1]{\lceil #1 \rceil}
\newcommand{\floor}[1]{\lfloor #1 \rfloor}
\newcommand{\norm}[1]{\| #1 \|}
\newcommand{\bignorm}[1]{\Bigl \| #1 \Bigr \| #1}
\newcommand{\inner}[1]{\langle #1 \rangle}
\newcommand{\set}[1]{{ #1 }}


% -------------------------------------------------- %
% -------------------- SETUP ----------------------- %
% -------------------------------------------------- %
\title{\Huge{Lecture 1 - Algorithms, Data Structures, and Asymptotic Analysis}}
\author{\large{Mitch Harrison}}
\date{\today}   
\begin{document}
\setlength{\parskip}{1\baselineskip}
\setlength{\parindent}{15pt}
\maketitle
\tableofcontents
\newpage


% -------------------------------------------------- %
% --------------------- BODY ----------------------- %
% -------------------------------------------------- %
\section{Introduction to Algorithms}

\subsection{What is an algorithm?}
There are many definitions for an algorithm. Some include,
\begin{enumerate}
    \item An explicit, precise, unambiguous, mechanically-executable sequence of
        elementary steps intended to accomplish a specific purpose
    \item An algorithm terminates after a finite number of steps, each of which
        must be precisely defined, and all operations must be sufficiently 
        basic
    \item A well-defined effective procedure for computing desired output given
        an input
\end{enumerate}

\subsection{What are design and analysis?}
\subsubsection{Design}
Design covers the "how" and "what" of an algorithm. Given a precise specification
of the problem, we \textit{design} an algorithm to relate input to output to 
help solve that problem. It also provides a precise description of the
algorithm and its procedure.

\subsubsection{Analysis}
Analysis covers the "why" and "how well" of an algorithm, quanitifying its
performance at scale, providing counter examples, and giving mathematical
proofs.

\subsection{Describing algorithms}
In this course, we will mostly be working in pseudocode and precise English. We
will also write code for some assignments, but English and pseudocode will be
how we describe those algorithms. Our pseudocode will approximate our precise
English.

Even when we describe our algorithm with pseudocode, often accompanying it with
and English-language description is helpful for most human readers.

\pagebreak
\section{Asymptotic runtime review}
We will operate in the following abstract model of computation,
\begin{enumerate}
    \item We assume Random Access Memory (RAM). That is, memory searching is done
        in constant time.
    \item We assume that a \textit{constant time operation} is anything that
        does not depend on the size of the input.
    \item \textit{Runtime analysis} counts the number of constant time
        operations as a function of the size of the input.
\end{enumerate}

We will be using asymptotic analysis by way of \textit{Big-O notation}. That is,
 $T(N)$ is $O(g(N))$ if, for each $c$, $n_{0}$, such that for all $N \ge n_{0}$,
 \[
 T(N) \le c \cdot g(N).
 \]
\begin{note}
    Recall the common complexity classes in order of slowest to fastest:
    exponential, cubic, quadratic, nearly-linear, linear, logarithmic, and
    constant.
\end{note}
 
Other asymptotic notations other than Big-$O$ are Big-$\Theta$ and Big-$\Omega$.
They are similar to Big-$O$. $\Omega$ looks for lower bounds, such that
\[
T(N) \ge c \cdot g(N),
\]
and $\Theta$ looks for asymptotic equality,
\[
T(N) = c \cdot g(N).
\]
Thus, if an algorithm's Big-$O$ and Big-$\Omega$ are the same, then its
Big-$\Theta$ is the same. Additionally, Big-$O$ must be \textit{at least} equal
to Big-$\Theta$ and Big-$\Omega$ must be \textit{at most} equal to
Big-$\Theta$.

\pagebreak
\section{Data structures review}
The simplest data structure that we will use is the \textit{array}. That is, a
fixed-length collection of numbers.We will also use array-based \textit{lists}
that can grow by duplicating existing arrays' size if we need to add an
additional element past the number of free slots available in that array.

We also implement lists using \textit{linked lists}, where each node stores
both a value and a pointer to the next element in the list. The upside is that
adding elements is simpler, especially in the middle of the list. But we lose
constant-time lookup.

\textit{Binary heaps} are binary trees satisfying the following invariants:
\begin{enumerate}
    \item the \textbf{heap property}: every node is less than or equal to its
        successors
    \item the \textbf{shape property}: the tree is full at each layer except for
        (optionally) the last one
\end{enumerate}

The \textit{binary search tree} is a binary tree with the additional property
that each element of the left subtree of a node are less than the current one, 
and the right subtree is strictly greater than the current node. This enables
efficient search.

A \textit{hash table} is similar to an array, but we use a hash function to
determine both where to store an element and to lookup that element's location.
Sometimes, two elements will hash to the same position in the table. The 
simplest way to deal with this is that we can store arrays instead of individual
elements. This solves the collision problem, but makes lookup proportional to
the length of the array at a given position in the hash table.

\begin{note}
    By default, we will assuming \textit{worst-case} analysis. Because, with a
    careful choice of a hash function, we can assume the expectated performance
    of a hash table to be $O(1)$.
\end{note}

\end{document}

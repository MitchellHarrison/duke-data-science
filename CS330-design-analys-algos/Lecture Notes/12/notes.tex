\documentclass[titlepage, 12pt, leqno]{article}

% -------------------------------------------------- %
% -------------------- PACKAGES -------------------- %
% -------------------------------------------------- %
\usepackage{import}
\usepackage{pdfpages}
\usepackage{mathtools}
\usepackage{transparent}
\usepackage{enumitem}
\usepackage{xcolor}
\usepackage{tcolorbox}
\usepackage{amsmath}
\usepackage{amssymb}
\usepackage{parskip}
\usepackage{bbm}
\usepackage{algorithm}
\usepackage{algpseudocode}
\usepackage{algpseudocodex}
\usepackage[margin = 1in]{geometry}
\tcbuselibrary{breakable}
\tcbset{breakable = true}


% -------------------------------------------------- %
% -------------- CUSTOM ENVIRONMENTS --------------- %
% -------------------------------------------------- %
\newtcolorbox{note}{colback=black!5!white,
                          colframe=black!55!white,
                          fonttitle=\bfseries,title=Note}

\newtcolorbox{ex}{colback=blue!5!white,
                          colframe=blue!55!white,
                          fonttitle=\bfseries,title=Example}

\newtcolorbox{definition}{colback=red!5!white,
                          colframe=red!55!white,
                          fonttitle=\bfseries,title=Definition}


% -------------------------------------------------- %
% ------------------- COMMANDS --------------------- %
% -------------------------------------------------- %
% Brackets, braces, etc. 
\newcommand{\abs}[1]{\lvert #1 \rvert}
\newcommand{\bigabs}[1]{\Bigl \lvert #1 \Bigr \rvert}
\newcommand{\bigbracket}[1]{\Bigl [ #1 \Bigr ]}
\newcommand{\bigparen}[1]{\Bigl ( #1 \Bigr )}
\newcommand{\ceil}[1]{\lceil #1 \rceil}
\newcommand{\floor}[1]{\lfloor #1 \rfloor}
\newcommand{\norm}[1]{\| #1 \|}
\newcommand{\bignorm}[1]{\Bigl \| #1 \Bigr \| #1}
\newcommand{\inner}[1]{\langle #1 \rangle}
\newcommand{\set}[1]{{ #1 }}


% -------------------------------------------------- %
% -------------------- SETUP ----------------------- %
% -------------------------------------------------- %
\title{\Huge{Lecture 11 - Shortest Paths}}
\author{\large{Mitch Harrison}}
\date{\today}   
\begin{document}
\setlength{\parskip}{1\baselineskip}
\setlength{\parindent}{15pt}
\maketitle
\tableofcontents
\newpage


% -------------------------------------------------- %
% --------------------- BODY ----------------------- %
% -------------------------------------------------- %
\section{Wrapping Up DFS}

We previously discussed the use of DFS to calculate the \textbf{topological
ordering} of a directed acyclic graph (DAG). The simplest way to find this
ordering is by running \Call{allDFS}{} and then sorting in ascending order of
\texttt{post} time.

\subsection{The longest path in a DAG}
Fix a target vertex $t$, and we want to find the length of the \textit{longest
path} to $t$ for a given node $u$.
\[
L(u) =
\begin{cases}
    0 & u=t \\
    -\infty & u\ne t \text{ (no edges from $u$)}\\
    1 + max(L[v]) & \text{otherwise}
\end{cases}
\]
What is a valid iteration order in which to solve these subproblems? Suppose we
are given a topological ordering of our graph $G$. In a topological ordering,
we need to know the answers of the nodes to the "right" of the node for which 
we are currently solving (i.e. all of the \textit{out neighbors}). We therefore
must solve in reverse-topological order.

\subsection{Strongly connected components}
\begin{definition}
    A subgraph $G' \subseteq G$ is a \textbf{strongly connected component} (SCC)
    if it is a maximal subgraph such that for every pair of vertices 
    $u,v \in G'$, there is a bath \textit{both} from $u$ to $v$ \textit{and} 
    from $v$ to $u$ in $G'$.
\end{definition}

\begin{definition}
    A \textbf{condensation} is a "meta-graph" in which each "node" is an SCC
    of a larger graph $G$. The condensation is always a DAG.
\end{definition}

\subsection{Computing SCCs}
Let $u$ be a vertex in a sink of the condensation of $G$. The vertices
reachable from $u$ are an SCC of $G$.

\begin{definition}
The \textbf{reverse graph} $G^{R}=(V,E^{R})$ is simply the original graph $G$
with the direction of all edges flipped. In the reverse graph, all sinks become
sources and all sources become sinks. The structure of all SCCs remain the same
in $G$ and $G^{R}$.
\end{definition}

To find a vertex in a sink SCC, we use the reverse graph. We \textbf{claim} that
if we rull \Call{allDFS}{} on $G^{R}$, the vertex with the greates post time is
in a sink component of the condensation of $G$. 

Suppose $v$ in the SCC $C$ has the greatest post time. Suppose there is an edge
$x \rightarrow y \in E$ that exits $C$. In that case, $y \rightarrow x \in 
E^{R}$, so $v$ is reachable from $y$ in $G^{R}$. Consider two cases:
\begin{enumerate}
    \item \Call{dfs}{} gets called on $y$ first.
    \item \Call{dfs}{} gets called on $v$ first.
\end{enumerate}
If we draw the results of these cases, we see that $y$ has the greatest post 
time in both cases. However, we assumed that $v$ had the greatest post time. 
This gives rise to a contradiction, which shows that our initial claim was
correct.

\subsubsection{SCC computation abstract}
An informal idea for computing SCCs is that we repeatedly
\begin{enumerate}
    \item Find a vertex $u$ in a sink in the condensation of $G$.
    \item Find vertices reachable from $u$.
    \item Label this SCC and remove it from the graph.
\end{enumerate}

\pagebreak
\section{Breadth-First Search (BFS)}
\begin{definition}
    A path $P_{s\rightarrow t}$ from $s$ to $t$ is a \textbf{shortest path} if
    \[
    |P_{s\rightarrow t}| \le |P'_{s\rightarrow t}|
    \]
    Where $|P|$ is the number of edges in $P$ for an unweighted graph and
    $|P| = \sum_{(u,v) \in P}w(u,v)$ for a weighted path.
\end{definition}

\subsection{Single Source Shortest Path (SSSP)}
Given a graph $G = (V,E)$ and a source vertex $s$, we compute
\begin{enumerate}
    \item the shortest path distance $d[u]$ from $s$ to $u$
    \item The \textit{shortest path tree} (represented by \texttt{prev} 
        pointers) with a unique path $P_{s\rightarrow u}$ that is a shortest
        path from $s$ to $u$
\end{enumerate}
for every vertex $u$ reachable from $s$.

\subsubsection{Shortest path optimal substructure}
\textbf{Claim}: Suppose $P_{s\rightarrow u} + P_{u\rightarrow t}$ is a shortest
path from $s$ to $t$ that passes through $u$. Then $P_{s \rightarrow u}$ is a
shortest path from $s$ to $u$.

\textbf{Suppose} that a path $P'_{s\rightarrow u}$ exists that is shorter than
$P_{s\rightarrow u}$. Then, $|P'_{s\rightarrow u}| + P'_{u\rightarrow t}| <
|P_{s\rightarrow u}| + P_{u\rightarrow t}|$. This is a contradiction. Therefore,
our initial claim is correct.

Given a shortest path $P_{s\rightarrow u} + P_{u\rightarrow t}$, and 
$P_{s\rightarrow u}$ is the shortest path from $s$ to $u$, we should be able to
solve the SSSP problem by computing
\[
d(v) =
\begin{cases}
    0 & v=s \\
    1 + min[d(u)] & \text{otherwise}
\end{cases}
\]
This raises the problem of cycles, which are the bane of many shortest-path 
algorithms. We solve this problem with breadth-first search.

\subsection{Nearer nodes first pattern}
From a starting node $s$, we explore
\begin{enumerate}
    \item \textit{all} nodes reachable by 1 edge, then
    \item \textit{all} nodes with shortest path distance 2, then
    \item \textit{all} nodes with shortest path distance 3
    \item continue
\end{enumerate}

This is often called a \textit{wavefront} pattern of exploration, because it
resembles a wave eminating out from the spot into which a rock was thrown into a
pond.

\begin{algorithm}
\caption{breadth first search}
\begin{algorithmic}[1]
\Procedure{bfs}{$s$}
    \State Let $d,p$ be new length-$n$ lookup tables
    \State Let $Q$ be a new FIFO queue
    \State Add $s$ to $Q$
    \State $d[s]=0$
    \While{$Q$ is not empty}
        \State $u = Q$.remove()
        \For{$(u,v)$ in $E$}
            \If{$v \notin d$}
                \State $d[v] = 1 + d[u]$
                \State $p[v] = u$
                \State $Q$.add($v$)
            \EndIf
        \EndFor 
    \EndWhile
    \State \Return $d,p$
\EndProcedure 
\end{algorithmic}
\end{algorithm}

The total runtime of this \Call{bfs}{} algorithm is $O(n+m)$.

\end{document}

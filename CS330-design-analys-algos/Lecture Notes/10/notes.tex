\documentclass[titlepage, 12pt, leqno]{article}

% -------------------------------------------------- %
% -------------------- PACKAGES -------------------- %
% -------------------------------------------------- %
\usepackage{import}
\usepackage{pdfpages}
\usepackage{mathtools}
\usepackage{transparent}
\usepackage{enumitem}
\usepackage{xcolor}
\usepackage{tcolorbox}
\usepackage{amsmath}
\usepackage{amssymb}
\usepackage{parskip}
\usepackage{bbm}
\usepackage{algorithm}
\usepackage{algpseudocode}
\usepackage{algpseudocodex}
\usepackage[margin = 1in]{geometry}
\tcbuselibrary{breakable}
\tcbset{breakable = true}


% -------------------------------------------------- %
% -------------- CUSTOM ENVIRONMENTS --------------- %
% -------------------------------------------------- %
\newtcolorbox{note}{colback=black!5!white,
                          colframe=black!55!white,
                          fonttitle=\bfseries,title=Note}

\newtcolorbox{ex}{colback=blue!5!white,
                          colframe=blue!55!white,
                          fonttitle=\bfseries,title=Example}

\newtcolorbox{definition}{colback=red!5!white,
                          colframe=red!55!white,
                          fonttitle=\bfseries,title=Definition}


% -------------------------------------------------- %
% ------------------- COMMANDS --------------------- %
% -------------------------------------------------- %
% Brackets, braces, etc. 
\newcommand{\abs}[1]{\lvert #1 \rvert}
\newcommand{\bigabs}[1]{\Bigl \lvert #1 \Bigr \rvert}
\newcommand{\bigbracket}[1]{\Bigl [ #1 \Bigr ]}
\newcommand{\bigparen}[1]{\Bigl ( #1 \Bigr )}
\newcommand{\ceil}[1]{\lceil #1 \rceil}
\newcommand{\floor}[1]{\lfloor #1 \rfloor}
\newcommand{\norm}[1]{\| #1 \|}
\newcommand{\bignorm}[1]{\Bigl \| #1 \Bigr \| #1}
\newcommand{\inner}[1]{\langle #1 \rangle}
\newcommand{\set}[1]{{ #1 }}


% -------------------------------------------------- %
% -------------------- SETUP ----------------------- %
% -------------------------------------------------- %
\title{\Huge{Lecture 10 - Dynamic Programming and Graphs}}
\author{\large{Mitch Harrison}}
\date{\today}   
\begin{document}
\setlength{\parskip}{1\baselineskip}
\setlength{\parindent}{15pt}
\maketitle
\tableofcontents
\newpage


% -------------------------------------------------- %
% --------------------- BODY ----------------------- %
% -------------------------------------------------- %
\section{Matrix Chain Multiplication}

Given matrices $A_{0}, \cdots , A_{n}$, we can change the order in which they
are multiplied to reduce the number of scalar multiplications necessary to
find the solution. The question then becomes \textit{what is the smallest
number of scalar multiplications} required to compute $A_{0}A_{1} \cdots A_{n}$?

We will solve this problem using dynamic programming. First, we must divide the
problem. Let $MCM(i,j)$ be the minimum number of scalar multiplications to
multiply $A_{0}A_{1} \cdots A_{n}$. We will store the dimensions of an 
individual $A$ in an array $m[0, \cdots ,n]$. That is, matrix $A_{j}$ has
dimensions $m[j] \times m[j+1]$.

Then, for any $i\le k \le j$, $MCM(i,j)$ is at most
\[
    MCM(i,k) + MCM(k+1,j) + \left(m[i] \cdot m[k+1] \cdot m[j]\right),
\]
where the three terms being added are the cost to multiply matrices from $i$ to
$k$, the cost to multiply matrices $k+1$ to $j$, and the cost to multiply the
resulting two matrices respectively.

If we think about storing all of our solutions to $MCM(i,j)$ in a table, then
the \textbf{base cases} are when $i=j$. That is, they live on the diagonal. 
Observe that they are equivalent to zero, and that the upper triangle stores 
the only valid solutions because we never need a range where $i>j$.

In this table, solution $i,j$ depends on the values in the table to the left of
it in row $i$ and below it in column $j$. This has memory complexity $O(n^{2})$.

\pagebreak
\section{Graphs and Depth-First Search}

\textbf{Graphs} can be used to represent any type of pairwise relations. This
could be transportation networks, the internet, social networks, neural networks,
citation networks, and many other possible systems.

\subsection{Graph terminology}
We will say that a \textbf{graph} $G = (V,E)$ consists of a set of vertices 
(or nodes) $V$. A set of edges $E$ is between pairs of vertices 
($E \subseteq V\times V$). Note that $0\le m \le n^{2}$.

A graph $G' = (V', E')$ is a \textbf{subgraph} of $G = (V,E)$ if
$V' \subseteq V$ and $E' \subseteq E$ between the same vertices.

In an undirected graph, two nodes $a$ and $b$ are \textbf{neighbors} if they
are connected. That are also considered \textbf{adjacent}. The \textbf{degree}
of node $v$ is the number of neighbors of $v$.

In a directed graph, node $a$ is a \textbf{predecessor} of node $b$ if there is
an edge from $a$ to $b$. In this case, $b$ is a \textbf{successor} of $a$.
The \textbf{in-degree} of node $v$ is its number of predecessors, and its
\textbf{out-degree} is the number of sucessors.

A \textbf{path} in a graph is a sequence of adjacent vertices. A path is
\textbf{simple} if no vertices repeat. By default, when we discuss paths, we will
assume that they are simple. Conversely, a \textbf{cycle} is a sequence of 
adjacent vertices that begin and end at the same vertex.

In graph theory, a \textbf{tree} is any connected acyclic graph. That is,
there exists a pair between any pair of vertices, and there are no cycles. A
\textbf{forest} is a set of trees, and a tree or forest is \textbf{spanning} if
they include all vertices.

We usually use one of two data structures to store a graph in memory. The less
common one is the \textbf{adjacency matrix}, which is an $m\times m$ matrix
(where there are $m$ nodes in a graph) that has a 1 if there is an edge between
nodes $i$ and $j$, and 0 if not. Unfortunately, this takes $O(m^{2})$ memory
complexity.

Instead, we usually use an \textbf{adjacency list}. This is an array in which
index $i$ points to a linked list that contains all of the edges connected to
vertex $i$. This uses much less memory than an adjacency matrix at the cost of
slower lookups of edges.

\end{document}

\documentclass[titlepage, 12pt, leqno]{article}

% -------------------------------------------------- %
% -------------------- PACKAGES -------------------- %
% -------------------------------------------------- %
\usepackage{import}
\usepackage{pdfpages}
\usepackage{mathtools}
\usepackage{transparent}
\usepackage{enumitem}
\usepackage{xcolor}
\usepackage{tcolorbox}
\usepackage{amsmath}
\usepackage{amssymb}
\usepackage{parskip}
\usepackage{bbm}
\usepackage{algorithm}
\usepackage{algpseudocode}
\usepackage{algpseudocodex}
\usepackage[margin = 1in]{geometry}
\tcbuselibrary{breakable}
\tcbset{breakable = true}


% -------------------------------------------------- %
% -------------- CUSTOM ENVIRONMENTS --------------- %
% -------------------------------------------------- %
\newtcolorbox{note}{colback=black!5!white,
                          colframe=black!55!white,
                          fonttitle=\bfseries,title=Note}

\newtcolorbox{ex}{colback=blue!5!white,
                          colframe=blue!55!white,
                          fonttitle=\bfseries,title=Example}

\newtcolorbox{definition}{colback=red!5!white,
                          colframe=red!55!white,
                          fonttitle=\bfseries,title=Definition}


% -------------------------------------------------- %
% ------------------- COMMANDS --------------------- %
% -------------------------------------------------- %
% Brackets, braces, etc. 
\newcommand{\abs}[1]{\lvert #1 \rvert}
\newcommand{\bigabs}[1]{\Bigl \lvert #1 \Bigr \rvert}
\newcommand{\bigbracket}[1]{\Bigl [ #1 \Bigr ]}
\newcommand{\bigparen}[1]{\Bigl ( #1 \Bigr )}
\newcommand{\ceil}[1]{\lceil #1 \rceil}
\newcommand{\floor}[1]{\lfloor #1 \rfloor}
\newcommand{\norm}[1]{\| #1 \|}
\newcommand{\bignorm}[1]{\Bigl \| #1 \Bigr \| #1}
\newcommand{\inner}[1]{\langle #1 \rangle}
\newcommand{\set}[1]{{ #1 }}


% -------------------------------------------------- %
% -------------------- SETUP ----------------------- %
% -------------------------------------------------- %
\title{\Huge{Lecture 11 - Depth-First Search}}
\author{\large{Mitch Harrison}}
\date{\today}   
\begin{document}
\setlength{\parskip}{1\baselineskip}
\setlength{\parindent}{15pt}
\maketitle
\tableofcontents
\newpage


% -------------------------------------------------- %
% --------------------- BODY ----------------------- %
% -------------------------------------------------- %
\section{Depth-First Search}

\subsection{For undirected graphs}
\begin{algorithm}
\caption{depth-first search (undirected)}
\begin{algorithmic}[1]
\Procedure{dfs}{$u$}
\State visited[$u$] = true
\For{$(u,w)$ in $E$}
    \If{not visited[$w$]}
        \State \Call{DFS}{w}
    \EndIf
\EndFor
\EndProcedure 
\end{algorithmic}
\end{algorithm}

The runtime for this DFS algorithm is $O(n+m)$ (linear). We recurse on each of
$n$ nodes at most once each. We consider each of $m$ edges at most twice, once
from each endpoint.

DFS is great for some common questions about a graph $G = (V,E)$:
\begin{enumerate}
    \item Given $G$ and vertices $s \in V$ and $t \in V$, does there exist a
        path from $s$ to $t$?
    \item Given a vertex $s$, which vertices are reachable from $s$?
    \item Is the graph \textit{connected}? In other words, is it true that
        for all $u,v \in V$ there exists a path from $u$ to $v$?
\end{enumerate}

\subsubsection{Reductions}
Reducing $P$ to $P'$: if someone gave you an algorithm $A'$ for problem $P'$, 
can we use it to design an algorithm $A$ to solve problem $P$? This is possible
if we can convert the input of $P$ into a valid input for $P'$, solving $P'$
with $A'(P')$, and then converting $A'(P')$ into the answer for $P$. The total 
runtime of this procedute is the sum of those three steps.

\subsubsection{Pathfinding with DFS}
\begin{definition}
    \textbf{Back edges} are edges that connect a node to a node that we have 
    already traversed (in this case, using DFS). Note that if there are any
    back edges, the graph has a \textit{cycle}.
\end{definition}

\begin{definition}
    \textbf{Tree edges} are all edges in a graph that are not back edges.
\end{definition}

If we want the path to a node (and not just searching for a node), we can keep
track of previous nodes and recover paths by \textbf{backtracking}. We can
keep a \textit{previous lookup} table, in which one column is a node $n$, and the
other column contains the note from which we arrived at $n$. In that case, we
can traverse backwards using that previous lookup table, one node at a time,
until we arrive at node at which we started.

\subsubsection{Connected component}
\begin{definition}
    A \textbf{connected component} is a maximal subgraph with a path between
    every pair of vertices. A subgraph is \textbf{maximal} if it contains all
    possible vertices reachable from all vertices in the subgraph.
\end{definition}

Because \Call{dfs}{} only traverses vertices that can be reached from the
starting node. If a graph has multiple components, we can keep iterating over
vertices after our normal \Call{dfs}{} has completed. Once we have traversed
every vertex of every component, we will call this \Call{AllDFS}{}.

\begin{definition}
    A multiple DFS trees is called a \textbf{DFS forest}.
\end{definition}

\pagebreak
\section{DFS on Directed Graphs}
In a directed graph, connectivity is not symmetric. We will need some extra
vocabulary to distinguish some of the necessary parameters for traversing
directed graphs with DFS.

\subsection{Pre, Post, Clock}
We will represent \textbf{pre-} and \textbf{post-time} value in a global table.
We will get these pre and post times using a \texttt{clock} variable, which is
initially set to 0. In the DFS itself, we use the clock before and after
traversing (hence the names pre and post).
\subsection{For undirected graphs}
\begin{algorithm}
\caption{depth-first search with clock}
\begin{algorithmic}[1]
\Procedure{dfs}{$u$}
\State clock += 1
\State pre[u] = clock
\State visited[$u$] = true
\For{$(u,w)$ in $E$}
    \If{not visited[$w$]}
        \State \Call{DFS}{w}
    \EndIf
\EndFor
\State clock += 1
\State post[u] = clock
\EndProcedure 
\end{algorithmic}
\end{algorithm}

Note that in the above algorithm, \Call{dfs}{} only explores a single connected
component of a graph. We will later define \Call{allDFS}{}, which will ensure
that we traverse the entire graph, regardless of the number of connected
components.

Observe that \texttt{pre} grows on the way "down" a DFS tree, and \texttt{post}
incriments on the way back "up." 

\begin{definition}
    An edge is a \textbf{tree edge} if one node traverses to the next. A
    \textbf{forward edge} is one that jumps more than one node "down" the
    tree, skipping at least one node in the middle. A \textbf{back edge} was
    defined earlier, and is when an edge connects a node from lower in the
    tree to higher. A \textbf{cross edge} connects two connected components.
\end{definition}

\subsection{Directed Acyclic Graphs}
\begin{definition}
    A \textbf{directed acyclic graph} (DAG) has no back edges, at least one
    \textbf{source} (node with no incoming edges), and at least one
    \textbf{sink} (node with no outgoing edges).
\end{definition}

A DAG can be ordered so that all edges "go forward." That is, we number vertices
so that any edges go from lesser to greater. This is called \textbf{topological
ordering}. We get a topological ordering by running the \Call{allDFS}{}
procedure. This orders vertices from \textit{greatest to least post time}. This is
only a topological ordering if our graph is a DAG.

Consider an arbitrary edge $u \rightarrow v$. Since the graph is a DSG, it is
\textit{not} a back edge (DAGs have no back edges). There are some cases:
\begin{enumerate}
    \item This edge is a \textit{tree} or \textit{forward} edge. Then
        $post[v] < pre[u] < post[u]$.
    \item This edge is a \textit{cross} edge. In this case,
        $post[v] < pre[u] < post[u]$
\end{enumerate}
In either case, $u$ has the greater post time and will appear earlier in the
ordering. The runtime of this procedure is $O(nlog(n) + m)$ if we are explicitly
sorting, and $O(n+m)$ if we simply add vertices to a stack in post order as we
search.

\end{document}

\documentclass[titlepage, 12pt, leqno]{article}

% -------------------------------------------------- %
% -------------------- PACKAGES -------------------- %
% -------------------------------------------------- %
\usepackage{import}
\usepackage{pdfpages}
\usepackage{mathtools}
\usepackage{transparent}
\usepackage{enumitem}
\usepackage{xcolor}
\usepackage{tcolorbox}
\usepackage{amsmath}
\usepackage{amssymb}
\usepackage{parskip}
\usepackage{bbm}
\usepackage{algorithm}
\usepackage{algpseudocode}
\usepackage{algpseudocodex}
\usepackage[margin = 1in]{geometry}
\tcbuselibrary{breakable}
\tcbset{breakable = true}


% -------------------------------------------------- %
% -------------- CUSTOM ENVIRONMENTS --------------- %
% -------------------------------------------------- %
\newtcolorbox{note}{colback=black!5!white,
                          colframe=black!55!white,
                          fonttitle=\bfseries,title=Note}

\newtcolorbox{ex}{colback=blue!5!white,
                          colframe=blue!55!white,
                          fonttitle=\bfseries,title=Example}

\newtcolorbox{definition}{colback=red!5!white,
                          colframe=red!55!white,
                          fonttitle=\bfseries,title=Definition}


% -------------------------------------------------- %
% ------------------- COMMANDS --------------------- %
% -------------------------------------------------- %
% Brackets, braces, etc. 
\newcommand{\abs}[1]{\lvert #1 \rvert}
\newcommand{\bigabs}[1]{\Bigl \lvert #1 \Bigr \rvert}
\newcommand{\bigbracket}[1]{\Bigl [ #1 \Bigr ]}
\newcommand{\bigparen}[1]{\Bigl ( #1 \Bigr )}
\newcommand{\ceil}[1]{\lceil #1 \rceil}
\newcommand{\floor}[1]{\lfloor #1 \rfloor}
\newcommand{\norm}[1]{\| #1 \|}
\newcommand{\bignorm}[1]{\Bigl \| #1 \Bigr \| #1}
\newcommand{\inner}[1]{\langle #1 \rangle}
\newcommand{\set}[1]{{ #1 }}


% -------------------------------------------------- %
% -------------------- SETUP ----------------------- %
% -------------------------------------------------- %
\title{\Huge{Lecture 16 - Computational Hardness}}
\author{\large{Mitch Harrison}}
\date{\today}   
\begin{document}
\setlength{\parskip}{1\baselineskip}
\setlength{\parindent}{15pt}
\maketitle
\tableofcontents
\newpage


% -------------------------------------------------- %
% --------------------- BODY ----------------------- %
% -------------------------------------------------- %
\section{Hard Problems}

For problems like the longest path, max cut, knapsack, travelling salesman, and
vertex cover problems, no \textit{exact} polynomial time algorithms are known, 
only \textit{exponential} time algorithms. Such problems seem computationally
intractable.

\begin{definition}
    A \textbf{decision problem} is any computational problem that is answered by
    a single boolean value.
\end{definition}

Most optimization problem have a natural decision variant. The regular
(unweighted) shortest path, in which we are given a graph $G$ and vertices $s$
and $t$ and we are asked to compute the shortest path from $s$ to $t$. The
\textbf{decision variant} of this problem is, given $G$, $s$, and $t$, as well
as an integer $d$, does there exist a path from $s$ to $t$ with at most
$d$ degrees?

Forgetting solving the problem for a moment, can we at least \textit{verify}
a proposed solution? This proposal is called a \textbf{certificate}. Many
decision problems are hard to solve but easy to verify a proposed solution. For
example, with the longest path, \textit{solving} the problem involves finding
the path with at least $d$ edges. But to \textit{verify}, we just need to confirm
that, if there is such a path $P$, that $P$ is a simple path from $s$ to $t$ 
with at least $d$ edges.

Similarly, for the vertex cover problem where we are given a graph
$G = (V,E)$ and an integer $k$, solving requires us finding $C \subseteq V$ of
size $|C| le k$ such that for every $(u,v) \in E$, $u \in C$, or $v \in C$. But
to verify, we only need to confirm, given a $C$, that $|C|\le k$ and that
every edge is covered.

\subsection{Complexity classes}
\textbf{P} is the class of problems for which there exists a polynomial time
algorithm. This is most of the course so far. \textbf{NP}, is the class of
decision problems for which a certificate (or "proof") of a yes/true/1 answer
can be verified in polynomial time. Longest path falls into this category, for
example. \textbf{co-NP} is just like \textbf{NP}, but for no/false/0 answers.
For example, the question "is an integer $x$ prime?" A simple certificate for
showing that it is \textit{not} prime is finding a prime factorization.

Most theorists believe that there are \textit{not} polynomial time algorithms
for NP-hard problems. While potentially depressing, this gives us a way to know
what \textit{isn't} possible, allowing us to avoid wasting time trying to find
an exact polynomial time algorithm. For these problems, finding an 
\textit{approximation} is sufficient, so long as that solution is efficient.

\begin{note}
    NP stands for "non-deterministic polynomial."
\end{note}

\subsection{Relating P, NP, and co-NP}
Two observations are easy:
\begin{enumerate}
    \item $P \subseteq NP$. If we have a polynomial time algorithm, we verify
        by re-solving.
    \item $P \subseteq co-NP$, for the same reasoning as observation 1.
\end{enumerate}
This leaves one big question. Namely, are there problems in NP that 
\textit{cannot} be solved in polynomial time? That is, does $P = NP$ or
$P\ne NP$? This is the greatest outstanding question in theoretical computer
science.

\pagebreak
\section{NP-Hard and NP-Complete}
If we cannot prove $P = NP$, how do we characterize the problems we think are
hard to solve? We try to identify the \textit{hardest} problem(s) in NP.

\begin{definition}
    A problem $\Pi$ is \textbf{NP-hard} if a polynomial-time algorithm for
    $\Pi$ would imply a polynomial-time algorithm for \textit{every problem in
    NP}. In other words, $\Pi$ is NP-hard if an only if $\Pi$ can be solved in
    polynomial time, then $P = NP$.
\end{definition}

\begin{definition}
    A problem $\Pi$ is \textbf{NP-complete} if
    \begin{enumerate}
        \item it is in NP, and
        \item it is NP-hard.
    \end{enumerate}
\end{definition}

Recall reductions in algorithms. Reducing from $P$ to $P'$, if someone gave you
an algorithm $A'$ for problem $P'$, can we use it to design an algorithm $A$ to
solve problem $P$?

To prove that $P'$ is at least at hard as $P$ (up to polynomial factors), we 
give a polynomial time reduction from $P$ to $P'$. Given an arbitrary input $X$
to $P$, we construct input $Y$ to $P'$. Then, we argue the following: 
\begin{enumerate}
    \item The reduction is polynomial time,
    \item $P(X) = 1$ if and only if $P'(Y) = 1$, meaning prove
        \begin{enumerate}
            \item if $P(X) = 1$ then $P'(Y) = 1$, \textit{and},
            \item if $P'(Y) = 1$, then $P(X) = 1$.
        \end{enumerate}
\end{enumerate}
In other words, showing that a solution to $P(X)$ implies a solution to $P'(Y)
= 1$ and vice versa.

\subsection{Reduction example}
The \textbf{independent set (IS)}: Given a graph $G = (V,E)$ and an integer
$k$, is there a set of vertices $S \subseteq V$ of size $|S| \ge k$ such that
for every $(u,v) \in E$, $u \notin S$, or $v \notin S$. That is, there are no
edges between vertices in $S$. The \textbf{vertex cover (CV)} asks if there is a
set of vertices $C \subseteq V$ of size $|C| \le k$ such that for every $(u,v)
\in E$, $u \in C$, or $v \in C$. That is, at least one endpoint of every edge is
in $C$. 

We can reduce from IS to VC, which shows that VC is NP-hard if IS is NP-hard,
which it is. Given a graph $G=(V,E)$ and an integer $k$, we use the same graph
$G$, but consider an integer $k' = |V|-k$. We claim that $G$ has a VC of size
$k'$ if and only if $G$ has an IS of size $k$.

Suppose that $G$ has a VC of size $k'$, and call it $C$. Consider $S = V-C$.
Clearly $|S| = |V|-k' = k$. For $(u,v) \in E$, $u \in C$, or $v \in C$, so at
least one of $u$ or $v$ is not in $S = V-C$. So, $S$ is an IS of size $k$.

To show correctness in the other direction (which is required), we use the same
graph $G$, but consider an integer $k' = |V| -k$. We claim that $G$ has a VC of
size $k'$ if and only if $G$ has an IS of size $k$. Suppose $G$ has an IS of
size $k$, and call it $S$. Consider $C = V-S$. Clearly $|C| = |V|-k=k'$. For
$(u,v) \in  E$, $u \notin S$, or $v \notin S$, so at least one of $u$ or $v$ is
in $C=V-S$. So $C$ is a VC of size $k'$.

We have shown that $G$ has a VC of size $k'$ if an only if $G$ has an IS of size
$k$. Also, this reduction is polynomial-time computable. Computing $k'$ is 
$O(1)$ and computing $V-C$ is $O(|V|)$.

\subsection{A second example}
The \textit{clique} problem asks if there is a set of vertices $C \subseteq V$ of
size $|C| \ge k$, such that $C$ is a complete subgraph. That is, every pair of
vertices in $C$ are connected by an edge. Reducing from IS to clique shows that
clique is NP-hard if IS is NP-hard.

Given a graph $G = (V,E)$ and an integer $k$, we use the same integer $k$, but
consider $G' = (V,E')$, where $(u,v) \in E'$ if and only if $(u,v) \notin E$.
We claim that $G'$ has a clique of size $k$ if and only if $G$ has an IS of
size $k$. Suppose $G'$ has a clique of size $k$, and call it $C$. For
$u \in C$, $v \in C$, we must have $(u,v) \in E'$. Then, $(u,v) \notin E$ by
construction. So $C$ is an IS of size $k$ in $G$.

Now, suppose $G$ has an IS of size $k$, and call it $S$. For $u \in S$, $v \in 
S$, we must have $(u,v) \notin E$. Then $(u,v) \in E'$ by construction. So
$S$ is a clique of size $k$ in $G'$.

We have shown that $G'$ has a clique of size $k$ if and only if $G$ has an IS of
size $k$. This reduction is polynomial time computable. Computing $E'$ is
$O(|V|^{2})$.

\subsection{Common errors in reductions}
Common errors when trying to prove a problem $P'$ is NP-complete by reducing
from a known hard problem $P$ include:
\begin{enumerate}
    \item Starting from an arbitrary input to $P'$ instead of $P$. Remember:
        the problem of interest $P'$ is hard if it would allow us to solve an
        arbitrary input to the known hard problem $P$.
    \item Only proving one direction of $P(X) = 1$ if and only if $P'(Y)=1$.
        Remember: we want a one-to-one correspondence between solutions of
        both problems.
\end{enumerate}

\pagebreak
\section{Proving NP-Completeness}
To prove a problem $P'$ is NP-complete, we:
\begin{enumerate}
    \item argue that $P'$ is in NP (can verify a solution in polynomial time), 
        and
    \item show that $P'$ is NP-hard by reducing from a known NP-hard problem
        $P$ to $P'$.
\end{enumerate}
\textbf{In short, to prove that a problem $A$ is NP-hard, reduce a known NP-hard
problem to $A$.}

\end{document}

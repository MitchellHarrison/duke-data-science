\documentclass[titlepage, 12pt, leqno]{article}

% -------------------------------------------------- %
% -------------------- PACKAGES -------------------- %
% -------------------------------------------------- %
\usepackage{import}
\usepackage{pdfpages}
\usepackage{mathtools}
\usepackage{transparent}
\usepackage{enumitem}
\usepackage{xcolor}
\usepackage{tcolorbox}
\usepackage{amsmath}
\usepackage{amssymb}
\usepackage{parskip}
\usepackage{bbm}
\usepackage{algorithm}
\usepackage{algpseudocode}
\usepackage{algpseudocodex}
\usepackage[margin = 1in]{geometry}
\tcbuselibrary{breakable}
\tcbset{breakable = true}


% -------------------------------------------------- %
% -------------- CUSTOM ENVIRONMENTS --------------- %
% -------------------------------------------------- %
\newtcolorbox{note}{colback=black!5!white,
                          colframe=black!55!white,
                          fonttitle=\bfseries,title=Note}

\newtcolorbox{ex}{colback=blue!5!white,
                          colframe=blue!55!white,
                          fonttitle=\bfseries,title=Example}

\newtcolorbox{definition}{colback=red!5!white,
                          colframe=red!55!white,
                          fonttitle=\bfseries,title=Definition}


% -------------------------------------------------- %
% ------------------- COMMANDS --------------------- %
% -------------------------------------------------- %
% Brackets, braces, etc. 
\newcommand{\abs}[1]{\lvert #1 \rvert}
\newcommand{\bigabs}[1]{\Bigl \lvert #1 \Bigr \rvert}
\newcommand{\bigbracket}[1]{\Bigl [ #1 \Bigr ]}
\newcommand{\bigparen}[1]{\Bigl ( #1 \Bigr )}
\newcommand{\ceil}[1]{\lceil #1 \rceil}
\newcommand{\floor}[1]{\lfloor #1 \rfloor}
\newcommand{\norm}[1]{\| #1 \|}
\newcommand{\bignorm}[1]{\Bigl \| #1 \Bigr \| #1}
\newcommand{\inner}[1]{\langle #1 \rangle}
\newcommand{\set}[1]{{ #1 }}


% -------------------------------------------------- %
% -------------------- SETUP ----------------------- %
% -------------------------------------------------- %
\title{\Huge{Lecture 7 - Parallel Algorithms pt. 2}}
\author{\large{Mitch Harrison}}
\date{\today}   
\begin{document}
\setlength{\parskip}{1\baselineskip}
\setlength{\parindent}{15pt}
\maketitle
\tableofcontents
\newpage


% -------------------------------------------------- %
% --------------------- BODY ----------------------- %
% -------------------------------------------------- %
\section{Determinacy Races}

This is a new error that is unique to parallel algorithms. Say we have a
function that returns the sum of all elements in an array $A$. It is given
below

\begin{algorithm}
\caption{sum all elements in an array incorrectly}
\begin{algorithmic}[1]
\Procedure{PSum}{$A,l,r$}
    \If{$l=r$}
    \State \Return $A[r]$
    \Else
    \State $m = \lfloor \frac{l+r}{2}\rfloor$
    \State $s=0$
    \State \textbf{spawn} $s = s + $ \Call{PSum}{$A,l,m$}
    \State $s = s +$ \Call{PSum}{$A, m+1, r$}
    \State \textbf{sync}
    \State \Return $s$
    \EndIf
\EndProcedure 
\end{algorithmic}
\end{algorithm}

Lines 7 and 8 introduce a \textbf{determinacy race}. These occur when two
different threads running in parallel make changes to shared memory. These bugs
are called \textit{determinacy races} because the final output of the function
is not deterministic. That is, they can vary with different runs of the alorithm
as threads access memory in different orders. A determinacy race occurs when
\begin{enumerate}
    \item Two or more logically parallel instructions read from the same
        memory, \textit{and}
    \item one or more of those instructions write to that memory.
\end{enumerate}

To fix the determinacy race, we make the following change to our algorithm,

\begin{algorithm}
\caption{sum all elements in an array correctly}
\begin{algorithmic}[1]
\Procedure{PSum}{$A,l,r$}
    \If{$l=r$}
    \State \Return $A[r]$
    \Else
    \State $m = \lfloor \frac{l+r}{2}\rfloor$
    \State \textbf{spawn} $a = $ \Call{PSum}{$A,l,m$}
    \State $b =$ \Call{PSum}{$A, m+1, r$}
    \State \textbf{sync}
    \State \Return $a + b$
    \EndIf
\EndProcedure 
\end{algorithmic}
\end{algorithm}

This fix is common to parallel algorithms. First, we spawn two \textit{separate}
variables in memory, do some work on them, \textbf{sync}, and then combine them.
This removes the possibility of overwriting shared memory.

\pagebreak
\section{Parallel Matrix Multiplication}

The goal of this problem is to find the product of two $n\times n$ matrices
$A$ and $B$. The product matrix $C$ is also $n\times n$, where $c_{ij}$ is the
inner product of row $i$ in $A$ and column $j$ in $B$. That is,
\[
    C = AB \text{ where } c_{ij} = \sum_{k=1}^{n} a_{ik} \times b_{kj}.
\]
We \textit{could} solve this problem with a relatively simple iterative
iterative algorithm.
\begin{algorithm}
\caption{multiply matrices iteratively}
\begin{algorithmic}[1]
\Procedure{MMultiply}{$A,B,n$}
    \State $C = $ new $n$-by-$n$ matrix
    \For{$i = 1$ to $n$}
        \For{ $j = 1$ to $n$}
            \State $c_{ij} = 0$
            \For{$k=1$ to $n$}
                \State $c_{ij } = c_{ij + (a_{ik}b_{kj})}$
            \EndFor
        \EndFor
    \EndFor
    \State \Return $C$
\EndProcedure 
\end{algorithmic}
\end{algorithm}

Note that this algorithm is $O(n^{3})$. But in practice, it will be sub-cubic.
To parallelize, we must observe that we are only ever \textit{reading} from
$A$ or $B$, but never writing to them. We are only ever \textit{writing} to
$C$ (i.e. finding $c_{ij}$).

\subsection{Parallel "for" loops}
We will use \textbf{parallel for} to indicate a loop where each iteration of
that loop can be computed in parallel, syncing when all iterations are complete.
\textbf{Warning:} it is very easy to introduce a determinacy race in a 
parallel for loop by accident.

In \Call{MMultiply}{}, we can parallelize the \textit{for} loops on lines 3 and
4, but \textit{not} the one on line 6. Because line 7 (inside the inner-most
\textit{for} loop) makes changes to a shared variable in memory, parallelizing
the inner-most loop would cause a determinacy race. This would not occur in the
outer two loops.

\end{document}

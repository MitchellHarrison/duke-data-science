\documentclass[titlepage, 12pt, leqno]{article}

% -------------------------------------------------- %
% -------------------- PACKAGES -------------------- %
% -------------------------------------------------- %
\usepackage{import}
\usepackage{pdfpages}
\usepackage{mathtools}
\usepackage{transparent}
\usepackage{enumitem}
\usepackage{xcolor}
\usepackage{tcolorbox}
\usepackage{amsmath}
\usepackage{amssymb}
\usepackage{parskip}
\usepackage{bbm}
\usepackage{algorithm}
\usepackage{algpseudocode}
\usepackage{algpseudocodex}
\usepackage[margin = 1in]{geometry}
\tcbuselibrary{breakable}
\tcbset{breakable = true}


% -------------------------------------------------- %
% -------------- CUSTOM ENVIRONMENTS --------------- %
% -------------------------------------------------- %
\newtcolorbox{note}{colback=black!5!white,
                          colframe=black!55!white,
                          fonttitle=\bfseries,title=Note}

\newtcolorbox{ex}{colback=blue!5!white,
                          colframe=blue!55!white,
                          fonttitle=\bfseries,title=Example}

\newtcolorbox{definition}{colback=red!5!white,
                          colframe=red!55!white,
                          fonttitle=\bfseries,title=Definition}


% -------------------------------------------------- %
% ------------------- COMMANDS --------------------- %
% -------------------------------------------------- %
% Brackets, braces, etc. 
\newcommand{\abs}[1]{\lvert #1 \rvert}
\newcommand{\bigabs}[1]{\Bigl \lvert #1 \Bigr \rvert}
\newcommand{\bigbracket}[1]{\Bigl [ #1 \Bigr ]}
\newcommand{\bigparen}[1]{\Bigl ( #1 \Bigr )}
\newcommand{\ceil}[1]{\lceil #1 \rceil}
\newcommand{\floor}[1]{\lfloor #1 \rfloor}
\newcommand{\norm}[1]{\| #1 \|}
\newcommand{\bignorm}[1]{\Bigl \| #1 \Bigr \| #1}
\newcommand{\inner}[1]{\langle #1 \rangle}
\newcommand{\set}[1]{{ #1 }}


% -------------------------------------------------- %
% -------------------- SETUP ----------------------- %
% -------------------------------------------------- %
\title{\Huge{Lecture 8 - Dynamic Programming pt. 1}}
\author{\large{Mitch Harrison}}
\date{\today}   
\begin{document}
\setlength{\parskip}{1\baselineskip}
\setlength{\parindent}{15pt}
\maketitle
\tableofcontents
\newpage


% -------------------------------------------------- %
% --------------------- BODY ----------------------- %
% -------------------------------------------------- %
\section{Intro to Dynamic Programming}

Recall that \textit{Fibonacci numbers} take the form
\[
f(n) = 
\begin{cases}
    1 & 1 \le n \le 2 \\
    f(n-1) + f(n-2) & \text{else}
\end{cases}
\]
The problem with computing these numbers is that in practice, computing
these numbers recursively very quickly (factors of dozens) approach computation
times nearing the lifetime of the universe. 

Recall that we have so far been dealing with \textit{independent} recursive 
calls. That is, the two sub-problems are disjoint. However, Fibonacci numbers
are not disjoint; each depends on the results of the previous two calls. Thus,
we cannot divide and conquer or parallelize. In fact, recursively calculating
Fibonacci numbers is $\Omega(2^{n/2})$, so \textit{at least} exponential.

The reason calculating these numbers is slow is that there is an exponential
number of recursive calls. The problem is, we don't "remember" the previous 
sub-problems. That is, we re-compute $f(n)$ every time $f(m>n)$ is called. To
avoid this, we can have an auxiliary data structure for remembering past
sub-problem solutions (hash table, array, etc.). Then, we can first check to
see if we already have the answer to a given sub-problem in our data structure.
We will call this data structure the \textbf{memo}, and the process in which
we save results into a memo to speed up a solution \textbf{memoization}. Below
is \Call{MFib}{}, which uses an array $\mathcal{M}$ as the memo.

\begin{algorithm}
\caption{Memoized Fibonacci number calculator}
\begin{algorithmic}[1]
\Procedure{MFib}{$n,\mathcal{M}$}
    \If{$n\le 2$}
        \State \Return 1
    \Else
        \If{$n \notin \mathcal{M}$}
            \State $\mathcal{M}(n) = $ \Call{MFib}{$n-1,\mathcal{M}$} + 
            \Call{MFib}{$n-2,\mathcal{M}$}
        \EndIf
        \State \Return $\mathcal{M}[n]$
    \EndIf
\EndProcedure 
\end{algorithmic}
\end{algorithm}

Alternatively, we can find these numbers \textbf{bottom-up}, iteratively. This
method starts at a base case and works up from the bottom instead of starting at
the top and working down to the base case. The key property in bottom-up 
iteration is thet we \textit{iterate in an order so that sub-problems on which
the current iteration depends on already-solved sub-problems}. Below is 
\Call{DPFib}{}, which is a bottom-up iteration approach.

\begin{algorithm}
\caption{bottom-up iteration Fibonacci calculator}
\begin{algorithmic}[1]
\Procedure{DPFib}{$n$}
    \State $F = $ new array of length $n+1$
    \State $F[1] = F[2] = 1$
    \For{$i = 2$ to $n$}
        \State $F[i] = F[i-1] + F[i-2]$
    \EndFor
\EndProcedure 
\end{algorithmic}
\end{algorithm}

And in the next following \Call{DPFib2}{} algorithm, much less memory is used,
as we discard all but the previous two solutions. This is a substantial memory
savings as $n$ gets large.

\begin{algorithm}
\caption{Memory-saving Fibonacci calculator}
\begin{algorithmic}[1]
\Procedure{DPFib2}{$n$}
    \State $F = $ new array of length 2
    \State $F[0] = F[1] = 1$
    \For{$i = 3$ to $n$}
        \State $x = F[0] + F[1]$
        \State $F[0] = F[1]$
        \State $F[1] = x$
    \EndFor
    \State \Return $F[1]$
\EndProcedure 
\end{algorithmic}
\end{algorithm}

\subsection{Longest Common Subsequence problems}
The \textbf{Longest Common Subsequence} (LCS) is a type of problem in which we
find the length of the longest string (for example) that is a subsequence of
$a$ and $b$. For example, the longest common subsequence of "GATAC" and "ACGTC"
is "ATC" of length 3. A brute-force solution to find an LCS is given below.

\begin{algorithm}
\caption{find longest common subsequence (brute force)}
\begin{algorithmic}[1]
\Procedure{bruteLCS}{$a,b$}
    \State $c=0$
    \For{subsequence $s$ of $a$}
        \If{$s$ is a subsequence of $b$}
            \State $c =$ \Call{max}{$c$, \Call{length}{$s$}}
        \EndIf
    \EndFor
    \State \Return $c$
\EndProcedure 
\end{algorithmic}
\end{algorithm}

The problem with the brute force method is that it runs in \textit{exponential
time}, and thus we must use a better solution. We will do this using dynamic
programming. 

Suppose that (using the recursion fairy), we could magically get
the LCS of a \textit{prefix} of the strings. Then we have some possible cases:
\begin{enumerate}
    \item The last characters of the two substrings are the same. In this case,
        the LCS would certainly end with this. So the overall LCS is the LCS
        of the prefixes + 1.
    \item The last characters are not the same. In this case, then the LCS 
        cannot possibly end with both characters. Thus, the overall LCS is
        the LCS of one substring and the prefix of the other. Thus, there
        are only two possibilities in this case.
\end{enumerate}
Unfortunately, this process will involve a lot of recurrence, which can be
extremely slow. However, we've already discussed some methods for combatting
that slowness.

Let \Call{lcs}{$i,j$} be the length of the longest common subsequence between
$a_{0}, a_{1}, \cdots , a_{i}$ and $b_{0}, b_{1}, \cdots ,b_{j}$. Then, there
are $mn$ subproblems, one for each $0\le i < m$ and $0 \le j < n$. Thus,
\[
LCS(i,j) =
\begin{cases}
    0 & i < 0 \text{ or } j < 0 \\
    1 + LCS(i-1,j-1) & i \ge 0, j \ge 0, a[i] = b[j] \\
    max(LCS(i-1,j), LCS(i,j-1) & i \ge 0, j \ge 0, a[i] \ne b[j]
\end{cases}
\]
Suppose we store subproblems in a table where row $i$ and column $j$ contains
\Call{LCS}{$i,j$}. Then, our subproblem dependencies say we need to have the 
answers to subproblems in smaller row and column indices. We could then keep a
static $i$ and iterate $j$, and then incriment $i$ and repreat. This is the "easy 
case," row-by-row, filling left to right. Here is that algorithm.

\begin{algorithm}
\caption{fast LCS finding}
\begin{algorithmic}[1]
\Procedure{dplcs}{$a,b$}
    \State Let $m,n$ be the lengths of $a,b$.
    \State Let $T$ be a new $m$-by-$n$ table.
    \For{$i=0$ to $m-1$}
        \For{$j=0$ to $m-1$}
            \If{$a[i] = b[j]$}
                \State $T[i][j] = 1 + T[i-1][j-1]$
                \Comment{"diagonal" case}
            \Else
                \State $T[i][j] = max(T[i-1][j], T[i][j-1])$
                \Comment{"left/up" case}
            \EndIf
        \EndFor
    \EndFor
    \State \Return $T[m-1][n-1]$
\EndProcedure 
\end{algorithmic}
\end{algorithm}

This \Call{dplcs}{}, we fill a table and draw back arrows to the 
\textit{maximizing previous subproblem}, breaking ties arbitrarily. The runtime
for \Call{dplcs}{} is $O(mn)$, far better than the exponential-time brute force
solution.

\end{document}

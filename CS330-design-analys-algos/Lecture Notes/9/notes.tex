\documentclass[titlepage, 12pt, leqno]{article}

% -------------------------------------------------- %
% -------------------- PACKAGES -------------------- %
% -------------------------------------------------- %
\usepackage{import}
\usepackage{pdfpages}
\usepackage{mathtools}
\usepackage{transparent}
\usepackage{enumitem}
\usepackage{xcolor}
\usepackage{tcolorbox}
\usepackage{amsmath}
\usepackage{amssymb}
\usepackage{parskip}
\usepackage{bbm}
\usepackage{algorithm}
\usepackage{algpseudocode}
\usepackage{algpseudocodex}
\usepackage[margin = 1in]{geometry}
\tcbuselibrary{breakable}
\tcbset{breakable = true}


% -------------------------------------------------- %
% -------------- CUSTOM ENVIRONMENTS --------------- %
% -------------------------------------------------- %
\newtcolorbox{note}{colback=black!5!white,
                          colframe=black!55!white,
                          fonttitle=\bfseries,title=Note}

\newtcolorbox{ex}{colback=blue!5!white,
                          colframe=blue!55!white,
                          fonttitle=\bfseries,title=Example}

\newtcolorbox{definition}{colback=red!5!white,
                          colframe=red!55!white,
                          fonttitle=\bfseries,title=Definition}


% -------------------------------------------------- %
% ------------------- COMMANDS --------------------- %
% -------------------------------------------------- %
% Brackets, braces, etc. 
\newcommand{\abs}[1]{\lvert #1 \rvert}
\newcommand{\bigabs}[1]{\Bigl \lvert #1 \Bigr \rvert}
\newcommand{\bigbracket}[1]{\Bigl [ #1 \Bigr ]}
\newcommand{\bigparen}[1]{\Bigl ( #1 \Bigr )}
\newcommand{\ceil}[1]{\lceil #1 \rceil}
\newcommand{\floor}[1]{\lfloor #1 \rfloor}
\newcommand{\norm}[1]{\| #1 \|}
\newcommand{\bignorm}[1]{\Bigl \| #1 \Bigr \| #1}
\newcommand{\inner}[1]{\langle #1 \rangle}
\newcommand{\set}[1]{{ #1 }}


% -------------------------------------------------- %
% -------------------- SETUP ----------------------- %
% -------------------------------------------------- %
\title{\Huge{Lecture 9 - Dynamic Programming cont'd}}
\author{\large{Mitch Harrison}}
\date{\today}   
\begin{document}
\setlength{\parskip}{1\baselineskip}
\setlength{\parindent}{15pt}
\maketitle
\tableofcontents
\newpage


% -------------------------------------------------- %
% --------------------- BODY ----------------------- %
% -------------------------------------------------- %
\section{Dynamic Programming Problem-Solving Guide}
Solving dynamic programming problems generally takes on the following workflow;
\begin{enumerate}
    \item Consider some examples of the problem, debelop recursive intuition. 
        For example, "if someone told me the answer to part of this problem..."
        and then consider cases.
    \item Define subproblems.
    \item Develop a recurrence using 1) and 2).
    \item Determine a valid iteration order using 3).
    \item Write the iterative algorithm using 3) and 4).
    \item (Maybe) optimize space and time complexity using 5).
\end{enumerate}

\subsection{LCS example cont'd}
Recall that a \textit{longest common subsequence} is the collection of letters
whose order is shared between two strings. Characters can be removed, but not
re-ordered. Our dynamic algorithm involved filling a table of $m \times n$
dimension, where $n$ and $m$ are the length of the two strings. Tracing this
algorithm iteratively, we can see that some cells of that table depend on 
entries to the left of the current element, and some are to the upper-left. We
can fill a matrix $P$ that contains "back-pointers" that point to the element
in our data table $T$ depend on. This allows us to \textit{backtrack} from the
last subproblem, accumulating chatacters that "match" on the \textit{diagonal}
case.

\pagebreak
\section{The Knapsack Problem}
We will look at the 0/1 knapsack problem. Given $n$ items
$0, 1, \cdots n$ with costs $c[0], \cdots , c[n-1]$. Our goal is to choose a
set of items with total cost of \textit{at most} $B$ of maximum total value. In
this 0/1 version of this problem, there is only one of each item on the shelf,
so we have either zero or one of each.

Take the issue of \textbf{participatory budgeting}, where public leaders are
trying to choose how to allocate funding between multiple public improvement
projects. The value of each item may be complex, but in this case we can think
of them simply as projects. So given a budget constraint, choosing the optimal
public works project is an applied knapsack problem.

Let us define our subproblems and think recursively. We will call our procedure
\Call{ks}{}. We want \Call{ks}{$i,b$} to return the maximum value using items
$0, \cdots ,i$ and budget $b$. The obvious base cases are
\begin{enumerate}
    \item There is no budget left. That is, $b \le 0$.
    \item There are no items left. That is, $i=0$.
\end{enumerate}
Once we have our base case, there are two possible other non-base cases.
\begin{enumerate}
    \item If the current cost is higher than our budget (i.e. $c[i]>b$), then
        we cannot buy it (i.e. returning \Call{ks}{i-1,b}).
    \item We can afford the current item. We then buy the item if 
        \[
            v[i] + \text{\Call{ks}{$i-1, b-c[i]$}}
        \]
        is greater than \Call{ks}{$i-1,b$}. If it is not, we check the next
        item as before.
\end{enumerate}
Combining these steps make the following algorithm trivial to write.
\Call{ks}{} has runtime complexity of $O(nB)$ and space complexity $O(B)$.

\begin{algorithm}
\caption{knapsack problem}
\begin{algorithmic}[1]
\Procedure{ks}{$i,b$}
    \If{$b\le0$ \textbf{or} $i<0$}
        \State \Return 0
    \ElsIf{$c[i]>b$}
        \State \Return \Call{ks}{$i-1,b$}
    \Else
    \State include = $v[i] +$ \Call{ks}{$i-1,b-c[i]$}
    \State exclude = \Call{ks}{$i-1,b$}
    \EndIf
    \State \Return \Call{max}{include, exclude}
\EndProcedure
\end{algorithmic}
\end{algorithm}

\end{document}

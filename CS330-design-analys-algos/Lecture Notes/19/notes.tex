\documentclass[titlepage, 12pt, leqno]{article}

% -------------------------------------------------- %
% -------------------- PACKAGES -------------------- %
% -------------------------------------------------- %
\usepackage{import}
\usepackage{pdfpages}
\usepackage{mathtools}
\usepackage{transparent}
\usepackage{enumitem}
\usepackage{xcolor}
\usepackage{tcolorbox}
\usepackage{amsmath}
\usepackage{amssymb}
\usepackage{parskip}
\usepackage{bbm}
\usepackage{algorithm}
\usepackage{algpseudocode}
\usepackage{algpseudocodex}
\usepackage[margin = 1in]{geometry}
\tcbuselibrary{breakable}
\tcbset{breakable = true}


% -------------------------------------------------- %
% -------------- CUSTOM ENVIRONMENTS --------------- %
% -------------------------------------------------- %
\newtcolorbox{note}{colback=black!5!white,
                          colframe=black!55!white,
                          fonttitle=\bfseries,title=Note}

\newtcolorbox{ex}{colback=blue!5!white,
                          colframe=blue!55!white,
                          fonttitle=\bfseries,title=Example}

\newtcolorbox{definition}{colback=red!5!white,
                          colframe=red!55!white,
                          fonttitle=\bfseries,title=Definition}


% -------------------------------------------------- %
% ------------------- COMMANDS --------------------- %
% -------------------------------------------------- %
% Brackets, braces, etc. 
\newcommand{\abs}[1]{\lvert #1 \rvert}
\newcommand{\bigabs}[1]{\Bigl \lvert #1 \Bigr \rvert}
\newcommand{\bigbracket}[1]{\Bigl [ #1 \Bigr ]}
\newcommand{\bigparen}[1]{\Bigl ( #1 \Bigr )}
\newcommand{\ceil}[1]{\lceil #1 \rceil}
\newcommand{\floor}[1]{\lfloor #1 \rfloor}
\newcommand{\norm}[1]{\| #1 \|}
\newcommand{\bignorm}[1]{\Bigl \| #1 \Bigr \| #1}
\newcommand{\inner}[1]{\langle #1 \rangle}
\newcommand{\set}[1]{{ #1 }}


% -------------------------------------------------- %
% -------------------- SETUP ----------------------- %
% -------------------------------------------------- %
\title{\Huge{Lecture 19 - Approximation Algorithms}}
\author{\large{Mitch Harrison}}
\date{\today}   
\begin{document}
\setlength{\parskip}{1\baselineskip}
\setlength{\parindent}{15pt}
\maketitle
\tableofcontents
\newpage


% -------------------------------------------------- %
% --------------------- BODY ----------------------- %
% -------------------------------------------------- %
\section{Intro to Approximation}

Given $n$ items $0,1, \cdots , n-1$ with costs $c[0, \cdots, n-1]$ and values
$v[0, \cdots ,n-1]$, and a budget $B$. Our goal is to choose a set of items with
total cost at most $B$ of maximum total value. In optimizing an approximate
solution, we are concerned with \textbf{approximation ratios}. Say our optimal
solution is a value of 108, and using some algorithm, we find a solution that
has total value 100. Our approximation ratio is then $\frac{108}{100} = 1.08$.

For a given approximation problem, let $A(X)$ be the value of the algorithm's
solution $X$, and let $OPT(X)$ be the optimal solution. The algorithm is an
\textbf{$\alpha(n)$-approximation algorithm} if an only if
\[
max\left(\frac{OPT(X)}{A(X)}, \frac{A(X)}{OPT(X)}\right) \le \alpha(n).
\]
Here, $\alpha(n)$ is the \textbf{approximation factor}.  For a given
\textit{minimization} problem and algorithm, the algorithm is a
\textit{constant} $\alpha$-approximation algorithm if and only if
\[
\frac{A(X)}{OPT(X)} \le \alpha,
\]
where $\alpha$ is the approximation factor.

\pagebreak
\section{Makespan and Job Scheduling}
We are given $m$ identical machines and $n$ jobs with runtimes $T[1], 
\cdots T[n]$. We want to compute an assignment $A[1], \cdots ,A[n]$ where
$A[j] = i$ means "schedule job $j$ on machine $i$." We want to minimize the
\[
    makespan(A) = max\left(\sum _{j:A[j]=i}T[j]\right).
\]
In other words, the time to complete all the jobs. Say our three machines have
total loads of 5, 10, and 15. The max of these is 15, which is the makespan.
Despite the seeming simplicity of the problem, this problem is NP-hard.
There is a simple, intuitive greedy algorithm to approximate a solution. It is
defined below.

\begin{algorithm}
\caption{greedy makespan approximator}
\begin{algorithmic}[1]
\Procedure{greedySolver}{$T,m$}
    \For{$i\leftarrow 1$ to $m$}
        \State $Total[i]\leftarrow 0$
    \EndFor
    \For{$j\leftarrow 1$ to $n$}
        \State $mini \leftarrow argmin_{i} Total[i]$
        \State $A[j] \leftarrow mini$
        \State $Total[mini] \leftarrow Total[mini] = T[j]$
    \EndFor
    \State \Return $A$
\EndProcedure 
\end{algorithmic}
\end{algorithm}

In effect, this algorithm assigns jobs to the least loaded machine at each
step. To find a lower bound on the approximation factor, we can plug in
examples by hand and calculating those factors. But to find an upper bound,
which is more important, requires bounding $OPT(X)$. 

\subsection{Proof}
The \textit{largest job lemma} states that $OPT \ge max_{j}T[j]$. That
is, the makespan must be at least the size of the longest job. The
\textit{total jobs lemma} states that $OPT\le \frac{1}{m}\sum_{j=1}^{m}T[j]$.
That is, the optimal solution cannot be better than the average runtime
perfectly split among all machines.

Using these two facts, we can prove that our greedy approximation algorithm
is a 2-approximation algorithm. Let $i$ be the machine with the greatest load.
This machine determines the makespan. Let $j$ be the last job greedily assigned
to machine $i$. Let $L(i,j)$ be the load on $i$ before assigning $j$. Then,
$makespan = T[j] = L(i,j)$.

By the largest job lemma, $T[j] \le OPT$. Since $i$ was the least loaded machine,
$L(i,j) \le \frac{1}{m}\sum T[j]$. Combining with the total jobs lemma, we see 
that that average is at most $OPT$. Thus, $L(i,j) \le OPT$. We conclude that
$makespan \le 2 \cdot OPT$. Thus, our greedy solution is a 2-approximation
algorithm.

\pagebreak
\section{The Travelling Salesman}
We have $n$ nodes $1, \cdots ,n$ and positive weights $l(u,v)$ between
every pair of nodes $u$ and $v$. We want to find a Hamiltonian Cycle (also
called a \textbf{tour}) that visits every node exactly once. Among such tours,
we want the one with minimum total length. Real examples of this problems are
bus routes, trucking/supply logistics, local deliveries, and others.

Crucially, exhaustive search is computationally intractable for TSP. The
brute force solution involves checking $n!$ possible orderings. Suppose a
machine can check $2^{30}$ permutations per second. If there are only 10
stops, this machine takes less than a second. But for 20 locations, it would
take this machine over 70 years. With 30 possible locations, we would need
several lifetimes of the universe.

We will argue that this problem is NP-hard, and discuss approximate solutions,
in the next lecture.

\end{document}

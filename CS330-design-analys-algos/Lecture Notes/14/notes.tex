\documentclass[titlepage, 12pt, leqno]{article}

% -------------------------------------------------- %
% -------------------- PACKAGES -------------------- %
% -------------------------------------------------- %
\usepackage{import}
\usepackage{pdfpages}
\usepackage{mathtools}
\usepackage{transparent}
\usepackage{enumitem}
\usepackage{xcolor}
\usepackage{tcolorbox}
\usepackage{amsmath}
\usepackage{amssymb}
\usepackage{parskip}
\usepackage{bbm}
\usepackage{algorithm}
\usepackage{algpseudocode}
\usepackage{algpseudocodex}
\usepackage[margin = 1in]{geometry}
\tcbuselibrary{breakable}
\tcbset{breakable = true}


% -------------------------------------------------- %
% -------------- CUSTOM ENVIRONMENTS --------------- %
% -------------------------------------------------- %
\newtcolorbox{note}{colback=black!5!white,
                          colframe=black!55!white,
                          fonttitle=\bfseries,title=Note}

\newtcolorbox{ex}{colback=blue!5!white,
                          colframe=blue!55!white,
                          fonttitle=\bfseries,title=Example}

\newtcolorbox{definition}{colback=red!5!white,
                          colframe=red!55!white,
                          fonttitle=\bfseries,title=Definition}


% -------------------------------------------------- %
% ------------------- COMMANDS --------------------- %
% -------------------------------------------------- %
% Brackets, braces, etc. 
\newcommand{\abs}[1]{\lvert #1 \rvert}
\newcommand{\bigabs}[1]{\Bigl \lvert #1 \Bigr \rvert}
\newcommand{\bigbracket}[1]{\Bigl [ #1 \Bigr ]}
\newcommand{\bigparen}[1]{\Bigl ( #1 \Bigr )}
\newcommand{\ceil}[1]{\lceil #1 \rceil}
\newcommand{\floor}[1]{\lfloor #1 \rfloor}
\newcommand{\norm}[1]{\| #1 \|}
\newcommand{\bignorm}[1]{\Bigl \| #1 \Bigr \| #1}
\newcommand{\inner}[1]{\langle #1 \rangle}
\newcommand{\set}[1]{{ #1 }}


% -------------------------------------------------- %
% -------------------- SETUP ----------------------- %
% -------------------------------------------------- %
\title{\Huge{Lecture 14 - Shortest Path cont'd}}
\author{\large{Mitch Harrison}}
\date{\today}   
\begin{document}
\setlength{\parskip}{1\baselineskip}
\setlength{\parindent}{15pt}
\maketitle
\tableofcontents
\newpage


% -------------------------------------------------- %
% --------------------- BODY ----------------------- %
% -------------------------------------------------- %
\section{Shortest Path}

\subsection{Bellman-Ford}
Recall that Bellman-Ford is helpful for finding shortest paths when dealing with
negative edge weights. Typically it takes the form
\begin{algorithm}
\caption{Bellman-Ford typically}
\begin{algorithmic}[1]
\Procedure{BellmanFord}{$s$}
    \State \Call{InitSSSP}{$s$}
    \For{$V-1$ times}
    \Comment{$V$ is the number of vertices}
        \For{every edge $u \rightarrow v$}
            \If{$u \rightarrow v$ is tense}
                \State \Call{Relax}{$u \rightarrow v$}
            \EndIf
        \EndFor
    \EndFor
    \For{every edge $u \rightarrow v$}
        \If{$u \rightarrow v$ is tense}
            \State \Return "Negative cycle!"
        \EndIf
    \EndFor
\EndProcedure 
\end{algorithmic}
\end{algorithm}

\begin{definition}
    An edge is \textbf{tense} if $d[v]>d[u]+w(u,v)$, where $d[i]$ is the
    distance from the current vertex to vertex $i$ and $w(u,v)$ is the weight
    of the edge from $u$ to $v$.
\end{definition}

This implementation of \Call{BellmanFord}{} has $O(|V||E|)$, or $O(nm)$,
runtime complexity and requires $O(n)$ space complexity.

\pagebreak
\section{All Pairs Shortest Path Cases}
There are three possible cases for finding all shortest paths. Either a graph
is unweighted, weighted with non-negative weights, and weighted with negative
weights. 

For an unweighted graph, we could run a BFS from every vertex and write the
solutions to a table. This runs in $O(n(n+m))$, which is $O(n^{2})$ for a 
sparse graph and $O(n^{3})$ on a dense one.

For non-negative weights, we could run Dijkstra from every vertex. In that 
case, our asymptotic runtime would be $O(n(n+m)log(n))$. This is $O(n^{2}log(n))$
for a sparse graph and $O(n^{3}log(n))$ on a dense one.

Including negative weights requires us to run Bellman-Ford on every vertex. This
is $O(n(nm))$ time. This results in $O(n^{3})$ time on a sparse graph and
$O(n^{4})$ time on a dense one. Let's work thorugh a better way.

\subsection{Dynamic programming for all-pairs shortest paths}
Let us number our vertices arbitrarily from 1 to $n = |V|$. Then, for every pair
of vertices $u$ and $v$ and every integer $r \in [0, \cdots ,n]$, we define
the subproblem:

$\pi(u,v,r)$ is the shortest path (if any) from $u$ to $v$ that passes through
only verticees numbered at most $r$. We define \textit{pass through} to mean
that there are intermediate vertices between $u$ and $v$. Similarly, let
$d(u,v,r)$ be the corresponding distance.

In this formulation, there are three cases:
\begin{enumerate}
    \item $d(u,v,0) = w(u,v)$ if there is an edge $u \rightarrow v$, else
        $\infty$.
    \item $d(u,v,r) = d(u,v,r-1)$ if the shortest path \textit{does not} use
        vertex $r$.
    \item $d(u,v,r) = d(u,r,r-1) + d(r,v,r-1)$ if the shortest path \textit{does}
        use vertex $r$.
\end{enumerate}
Thus we arrive at a recurrence
\[
dist(u,v,r) -
\begin{cases}
    w(u \rightarrow v) & r=0 \\
    min
       \begin{cases}
           dist(u,v,r-t) \\
           dist(u,r,r-1) + dist(r,v,r-1)
       \end{cases}
       & \text{otherwise}
\end{cases}
\]
This is the \textbf{Floyd-Warshall algorithm}, and it is quantified below.

\pagebreak

\begin{algorithm}
\caption{Floyd-Warshall algorithm for all-pair shortest paths}
\begin{algorithmic}[1]
\Procedure{FloydWarshall}{$V,E,w$}
    \For{all vertices $u$}
        \For{all vertices $v$}
            \State $dist(u,v) = w(u,v)$
        \EndFor
    \EndFor
    \For{all vertices $r$}
        \For{all vertices $u$}
            \For{all vertices $v$}
                \If{$dist[u,v]>dist[u,r] + dist[r,v]$}
                    \State $dist[u,v] = dist[u,r] + dist[r,v]$
                \EndIf
            \EndFor
        \EndFor
    \EndFor
\EndProcedure
\end{algorithmic}
\end{algorithm}

This results in an all-pairs shortest path calculation for graphs containing
negative weights in at most $O(|V|^{3})$, or $O(n^{3})$ time instead of 
$O(n^{4})$ as we found with \Call{BellmanFord}{}. Also note that, in a dense
graph, it is slightly faster that Dijkstra as well. But in a sparse graph,
Dijkstra is much faster at $O(n^{2}log(n))$.

\end{document}

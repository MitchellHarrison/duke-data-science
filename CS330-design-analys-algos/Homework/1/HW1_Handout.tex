\documentclass[11pt]{article}
\usepackage[margin=1in]{geometry}
\usepackage{amsmath, amsfonts}
\usepackage[noend]{algpseudocode}
\usepackage{algorithm}
\usepackage[parfill]{parskip}
\usepackage{enumerate}
\usepackage[shortlabels]{enumitem}
\usepackage{hyperref}
\usepackage[english]{babel}
\usepackage[autostyle]{csquotes}
\usepackage{enumitem}
\usepackage{tikz}
\usetikzlibrary{decorations.pathreplacing}
\definecolor{color1}{rgb}{0.7, 0.2, 0.2}
\definecolor{color2}{rgb}{0.0, 0.4, 0.0}
\definecolor{color3}{rgb}{0.2, 0.4, 0.7}

%--------------------------------------------------------%


\title{\vspace{-0.5in}Compsci 330 Design and Analysis of Algorithms \\
Assignment 1, Spring 2024 Duke University}
\author{Dav King, Mitchell Harrison}
\date{Due Date: Thursday, January 25, 2024}

%--------------------------------------------------------%

\begin{document}

\maketitle


%--------------------------------------------------------%

\paragraph{How to Do Homework.} We recommend the following three step process 
for homework to help you learn and prepare for exams.
\begin{enumerate}
	\item Give yourself ~15-20 minutes per problem to try to solve on your 
            own, without help or external materials, as if you were taking an 
            exam. Try to brainstorm and sketch the algorithm for applied 
            problems. Don't try to type anything yet.
	\item After a break, review your answers. Lookup resources or get help 
            (from peers, office hours, Ed discussion, etc.) about problems you 
            weren't sure about.
	\item Rework the problems, fill in the details, and typeset your final 
            solutions.
\end{enumerate}

\paragraph{Typesetting and Submission.} Your solutions should be typed and 
submitted as a single pdf on Gradescope. Handwritten solutions or pdf files 
that cannot be opened will not be graded. \LaTeX \footnote{If you are new to
\LaTeX, you can download it for free at 
\href{https://www.latex-project.org}{latex-project.org} or you can use the 
popular and free (for a personal account) cloud-editor 
\href{https://www.overleaf.com}{overleaf.com}. We also recommend 
\href{https://www.overleaf.com/learn}{overleaf.com/learn} for tutorials and 
reference.} is preferred but not required. %If you use another editor for your solutions (such as Microsoft Word), you should convert the final document to a pdf, confirm that the symbolic math from the equation editor is properly formatted, and submit the pdf. 
You must mark the locations of your solutions to individual problems on 
Gradescope as explained in 
\href{https://help.gradescope.com/article/ccbpppziu9-student-submit-work#submitting_a_pdf}{the documentation}.
Any applied problems will request that you submit code separately on Gradescope
to be autograded. 

\paragraph{Writing Expectations.} If you are asked to provide an algorithm, you
should clearly and unambiguously define every step of the procedure as a
combination of precise sentences in plain English or pseudocode. If you are 
asked to explain your algorithm, its runtime complexity, or argue for its
correctness, your written answers should be clear, concise, and should show your
work. Do not skip details but do not write paragraphs where a sentence suffices.

\paragraph{Collaboration and Internet.} If you wish, you can work with a single
partner (that is, in groups of 2), in which case you should submit a single
solution 
\href{https://help.gradescope.com/article/m5qz2xsnjy-student-add-group-members}{as a group on gradescope}. 
You can use the internet, but looking up solutions or using large language 
models is unlikely to help you prepare for exams. 
See the 
\href{https://sites.duke.edu/spring24compsci330/policies/}{course policies webpage} for more details.

\paragraph{Grading.} Theory problems will be graded by TAs on an S/U scale (for
each sub-problem). Applied problems typically have a separate autograder where
you can see your score. The lowest scoring problem is dropped. See the
\href{https://sites.duke.edu/spring24compsci330/assignments/}{course assignments webpage} for more details.


%--------------------------------------------------------%

\newpage

\paragraph{Problem 1 (Asymptotics).} 

For each of the following statements, decide whether the statement is true or
false, and briefly explain your reasoning.

\begin{enumerate}[(a)]
    \item $3n$ is $O(n)$.
    
    \item $n^2 + 2n$  is $\Omega(n)$.

    \item $n^2 + 2n$  is $\Theta(n)$.
    
	\item $n^2 \log(n)$ is $O(n^3)$.
    
    \item $n\log(n^3)$ is $\Theta(n \log(n))$.

    \item $2^n$ is $\Theta(n^2)$.

    \item If Algorithm A has runtime complexity $\Theta(n)$ and Algorithm B 
        has runtime complexity $\Theta(n)$, then the empirical runtime of the
        algorithms will be approximately equal for a given input size $n$.

    \item If Algorithm A has runtime complexity $\Theta(n^2)$, then doubling 
        the input size should roughly double the empirical runtime of the
        algorithm for large input sizes $n$.
\end{enumerate}

\paragraph{Solution 1.}

\begin{enumerate}[(a)]
    	\item True. In asymptotic reasoning, we do not care about coefficients.
            $3n$ is still on the order of linear growth, and so we can say that
            it is $O(n)$.
	\item True. $\forall n$ there will be some linear function that is 
            always less than $n^2 + 2n$ - in this case, we don't even need to
            define a point $n_0$ where this takes effect, since it will always 
            be true. Since this function is bounded below by a linear-order
            function, it is $\Omega(n)$.
	\item False. In order to be $\Theta(n)$, the function would need to be
            $O(n)$ and $\Omega(n)$. However, this function is $O(n^2)$, since 
            it increases in quadratic complexity. We cannot remove this term, 
            and so we are bounded by $O(n^2)$ and the function cannot be
            $\Theta(n)$.
	\item True. While it could be bounded by a lower complexity
            ($O(n^2\text{log}(n))$), it is certainly upper-bounded by $n^3$ 
            which is greater than $n^2\text{log}(n)$.
    \item True. Another way to write $n\text{log}(n^3)$ would be
        $3n\text{log}(n)$. From this, we can clearly see that this function is
        $O(n\text{log}(n))$, and it is $\Omega(n\text{log}(n))$ as well (we can
        find constants $c$ such that it is bounded above and below by functions
        on that order). Thus, the function is $\Theta(n\text{log}(n))$.
	\item False. $2^n$ is $O(2^n)$, which is greater than $O(n^2)$. Thus,
            this function cannot be upper-bounded by a function on the quadratic
            order, and we therefore cannot say it is $\Theta(n^2)$.
	\item Not necessarily. A bound of $\Theta(n)$ merely says that they
            should grow at the same rate (e.g., doubling the size of the input
            should roughly double the empirical runtime). However, $\Theta$
            notation excludes coefficients and other implementation differences
            that could cause the two algorithms to have wildly different
            empirical runtimes.
	\item False. If Algorithm A has complexity $\Theta(n^2)$, then doubling
            the input size should roughly increase the empirical runtime of the
            algorithm by a factor of 4.
\end{enumerate}

%--------------------------------------------------------%

\newpage
\paragraph{Problem 2 (Iteration and Induction).}
Consider an array $A$ storing $n$ integers. Consider the following algorithm 
for sorting from least to greatest. Note that it sorts the same array passed 
as input in place; there is no return value.

\begin{algorithmic}[1]
\Procedure{sort}{$A$}
    \For{$i=0$ to $n-2$}
        \State $m = i$
        \For{$j=i+1$ to $n-1$}
            \If{$A[j] < A[m]$}
                \State $m = j$
            \EndIf
        \EndFor
        \State temp $=A[i]$
        \State $A[i] = A[m]$
        \State $A[m] =$ temp
    \EndFor
\EndProcedure
\end{algorithmic}

\begin{enumerate}[(a)]
    \item Derive the asymptotic runtime complexity of the \Call{Sort}{} 
        algorithm as a function of $n$. Briefly explain your answer.
    
    \item Prove that the \Call{Sort}{} procedure is correct using mathematical 
        induction. You may assume that for a given iteration $i$ of the outer 
        loop, lines $3-9$ swap the minimum value from $A[i], \dots, A[n-1]$ 
        with $A[i]$.
\end{enumerate}


\paragraph{Solution 2.}

\begin{enumerate}[(a)]
    \item The \Call{Sort}{} algorithm has $O(n^{2})$ runtime complexity. The 
        outer loop runs $n-1$ times (i.e. linear with $n$), and the inner loop
        similarly grows linearly with $n$. Multiplying $O(n) \cdot O(n)$, we
        arrive at the final runtime complexity, $\boxed{O(n^{2})}$.
    \item Let the \Call{Sort}{} function be correct if, after it has terminated,
        array $A$ is ordered in ascending order. That is, 
        $A[0]\le A[1] \le \cdots \le A[n-1]$.         

        Our \textbf{base case} will let $n=2$. That is,
        array $A_{0} = [a_{0}, a_{1}]$. If $a_{0} > a_{1}$. Line 5 will
        evaluate to \texttt{True}, and the two elements will switch positions.
        If $a_{0} \le a_{1}$, no changes will be made. Then the algorithm will
        terminate. Thus, \Call{Sort}{} is correct in the base case where $n=2$.

        \textbf{Assume} that at iteration $k$, the first $k-1$ elements of $A$
        are both correct and sorted. That is, they are the smallest $k-1$
        integers in $A$ and are in ascending order. The inner loop of 
        \Call{Sort}{} will operate on the sub-array 
        $A_{k} = [A_{k-1}, \cdots ,A_{n-1}]$. As in the base case, it will
        find the minimum value of that sub-array and shift it to the first
        position of $A_{k}$.

        \textbf{Therefore}, after $n-2$ iterations of \Call{Sort}{}, any array 
        $A$ of arbitrary length $n$ will be sorted. Note; \Call{Sort}{} does 
        not take $n-1$ iterations because the final sub-array is length 1 and 
        will always be the largest value in $A$.
\end{enumerate}

%--------------------------------------------------------%

\newpage
\paragraph{Problem 3 (Tree Recursion and Induction).}
Consider a binary search tree $T$ storing $n$ integers. Each node $r$ has
attributes $r$.value storing the integer element at that node, as well as 
references $r$.left and $r$.right to the left and right child nodes in the tree
respectively. If there is no child node, the reference will be NULL.
\begin{enumerate}[(a)]
    \item Given integers $a$ and $b$ where $a < b$, describe an efficient
        recursive algorithm that returns the number of elements of $T$ between
        $a$ and $b$ (inclusive).
    
    \item Use induction to prove that your algorithm is correct. Consider using
        the size of a tree for your induction variable, and note that a subtree
        of a binary search tree is itself a binary search tree.
    
    \item Derive the worst-case asymptotic runtime complexity of your algorithm
        as a function of $n$, the size of $T$. Make no assumptions about the
        shape of the binary search tree or the number of elements of $T$ between
        $a$ and $b$.
    
    \item Let $m$ be the number of elements in $T$ between $a$ and $b$
        (inclusive). Suppose that $T$ is perfect, that is, every internal node
        has 2 children and the leaf nodes are all at the same depth 
        (intuitively, a perfectly balanced tree). Derive the runtime complexity
        of your algorithm as a function of $n$ and $m$. For an efficient
        algorithm, this runtime should be smaller than in the previous step
        when $m$ is much smaller than $n$.
\end{enumerate}

\paragraph{Solution 3.}

\begin{enumerate}[(a)]
    \item \begin{algorithmic}[1]
	    \Procedure{Search}{node curr, int a, int b}
            \If{curr == NULL}
            \Comment base case
                \State \Return 0
            \EndIf
            \If{curr.value $\geq a$ \&\& curr.value $\leq b$}
                \State \Return 1 + \Call{Search}{curr.left, a, b} + 
                \Call{Search}{curr.right, a, b}
            \ElsIf{curr.value $ < a$}
            \Comment continue traversal if out of range
                \State \Return \Call{Search}{curr.right, a, b}
            \Else
                \State \Return \Call{Search}{curr.left, a, b}
            \EndIf
        \EndProcedure
	\end{algorithmic}

    \item Our \textbf{base case} is where $n=0$. That is, when \Call{Search}{} 
        is called on an empty tree. In that case, \Call{Search}{} returns 0, 
        which is correct, since \texttt{NULL} cannot be within range of 
        $a$ and $b$, regarless of their values.

        \textbf{Suppose} that \Call{Search}{} has traversed correctly through
        step $k-1$. At iteration $k$, there are three \textbf{cases} to 
        consider:
        \begin{itemize}
            \item The current value of the node is less than $a$. That is, it
                it below the lower bound of our search range. In that case,
                \Call{Search}{} traverses to the right and does not add to the
                total returned by the \Call{Search}{} procedure. Because every
                node to the right of the current node is larger than its
                value, this is correct.
            \item The current value is greater than $b$. That is, it is above the
                upper bound of our search range. In that case, \Call{Search}{}
                traverses to the left and does not add to the total returned
                by \Call{Search}{}. Becuase each node to the left of the current
                one has a lower value than the current one, this is correct.
            \item The current value is within the search range. In this case,
                the node is counted among the total that \Call{Search}{}
                returns, and traversal continues. This is correct.
        \end{itemize}
        Since all iterations up to iteration $k-1$ were correct, all
        possible cases for iteration $k$ are correct, and our base case is
        correct, \Call{Search}{} is correct.

    \item Regardless of the size or shape of our tree $T$, the worst-case
        asymptotic runtime complexity is $O(n)$, in which \Call{Search}{} is
        called for every node in the tree (i.e. when every node is between
        $a$ and $b$).

    \item Let our tree $T$ have $n$ nodes, $m$ of which are between $a$ and 
        $b$. $T$ will be split in half $n-m$ times; once whenever the algorithm
        lands on a node outside of $[a,b]$. For each $m$, the algorithm will
        recurse twice, checking both the subtree to the left and right of $m$.
        This does not reduce the size of the space. Since there are 2 recursive
        calls for each $m$, and one for each $n$, which splits the tree, we
        are left with a runtime of
        \[
        O\left(n\left(\frac{m}{n}\right) + log(n) \cdot \left(1 -
                \frac{m}{n}\right)\right)
        \]
        which simplifies to
        \[
            \boxed{O\left(m + \left(1 - \frac{m}{n}log(n)\right)\right)}
        \]
        
\end{enumerate}

%--------------------------------------------------------%

%--------------------------------------------------------%

\newpage
\paragraph{Problem 4 (Hash Table and Probability).}
Consider a hash table $T$ with $m$ positions indexed from $0$ to $m-1$ and a 
hash function $h(x) = |x| \mod m$. The hash table stores an element $x$ in
position $h(x)$ and resolves collisions by linear chaining (that is, by storing 
a linked list at each position for all elements mapped to that index). You also
have an array $A$ of $n$ integers, not necessarily unique. Consider the 
following algorithm for counting the number of unique integers in $A$ using $T$.

\begin{algorithmic}[1]
\Procedure{CountUnique}{$A, T$}
    \State $c = 0$;
    \For{$i=0$ to $n-1$}
        \State $x = A[i]$
        \If{$x \notin T[h(x)]$}
            \State $c = c + 1$
            \State Add $x$ to $T[h(x)]$
        \EndIf
    \EndFor
    \State \Return $c$
\EndProcedure
\end{algorithmic}

\begin{enumerate}[(a)]
    \item What is the \textit{worst-case} asymptotic runtime complexity of the
        \Call{CountUnique}{} algorithm? Do not make any assumptions about the
        values in $A$. Briefly explain your answer.
    \item Suppose that $A$ consists of $n$ integers drawn independently and
        uniformly at random from $\{0,1,\dots,km-1\}$ for some integer $k>0$. 
        For a given index $0 \leq i \leq m-1$,  what is the probability that
        \textit{none} of the $n$ elements of $A$ are hashed to $T[i]$? What is
        the probability that \textit{all} of the the $n$ elements of $A$ are
        hashed to $T[i]$? Briefly explain your answers.
    \item Again suppose that $A$ consists of $n$ integers drawn independently 
        and uniformly at random from $\{0,1,\dots,km-1\}$ for some integer 
        $k>0$. For a given index $0 \leq i \leq m-1$, what is the
        \textit{expected} number of elements of $A$ (counting possible
        duplicates) that hash to $T[i]$? Based on your answer, derive the
        expected asymptotic runtime complexity of the \Call{CountUnique}
        procedure as a function of $n$ and/or $m$, under this random input
        assumption.
\end{enumerate}

\paragraph{Solution 4.}

\begin{enumerate}[(a)]
    	\item The worst case asymptotic runtime complexity is $O(n^2)$. This is
            because, in specific cases, all elements of $A$ will hash to the 
            same position in $T$, and be linearly chained together. Thus, we
            already must traverse the array (reaching each element is in total
            $O(n)$), and for each one we may have to look through a list of 
            $i - 1$ integers, which in the worst case is $O(n)$. Thus, the 
            worst-case runtime is $O(n^2)$. 
	\item The number of elements bucket $i$ can be described by a Binomial
            distribution, where $n = n$ and $p = \frac{1}{m}$. Then the
            probability that \textit{none} of the $n$ elements are hashed to
            $T[i]$ is the product of each one not being hashed to $T[i]$, i.e.
            $\prod_{i = 0}^{km} 1 - \frac{1}{m} = (1 - \frac{1}{m})^{km}$. The
            probability that \textit{all} of the $n$ elements are hashed to
            $T[i]$ is the product of each one being hashed to $T[i]$, i.e.
            $\prod_{i = 0}^{km} \frac{1}{m} = (\frac{1}{m})^{km}$.
	\item For $n$ independent integers across $m$ buckets (each following 
            the same Binom($n, \frac{1}{m}$) distribution), we expect the number
            of elements from $A$ that hash to each $T[i]$ to be $\frac{n}{m}$.
            Earlier, it was noted that the worst-case runtime was $O(n^2)$,
            because for each of the $n$ indices in $A$, we might have to
            traverse $n$ possibilities to see if the element were already
            contained there. However, we only expect to see $\frac{n}{m}$
            elements in each bucket in this procedure. Thus, in the expected
            case, this function is $O(\frac{n^2}{m})$.
\end{enumerate}

%--------------------------------------------------------%

%--------------------------------------------------------%

\newpage
\paragraph{Problem 5 (Applied Problem).}
Given a sorted array $A$ of size $N$ ($1 \leq N \leq 5\cdot10^{5}$), and two
numbers $x,y$ $(0 \leq x \leq y \leq N)$, determine the number of elements of 
$A$ in the range $[x,y]$ (inclusive). Language-specific details follow. You can
use whichever of Python or Java you prefer.

Your solution's efficiency will be checked by comparing if its empirical 
runtime is within constant factors of an $O(\log(N))$ reference solution. Your
submission will be automatically graded against test cases for correctness and
efficiency. You will be able to see your score and you can resubmit as many 
times as you like up to the deadline. \textbf{NOTE:} Unlike the theory problems,
the applied problem grade \textbf{is the raw score shown on Gradescope}. See the
\href{https://sites.duke.edu/spring24compsci330/assignments/}{course assignments webpage} for more details. 

\begin{itemize}
	\item \textbf{Python.} You should submit a file called \texttt{range.py}
            to the Gradescope item ``Homework 1 (Python)." The file should 
            define (at least) a top level function \texttt{range\_count} that
            looks like: 
    \begin{itemize}
        \item \verb|def range_count(A:[int], x:int, y:int)|
    \end{itemize}

	and returns the number of elements of $\texttt{A}$ in the range
        $\texttt{[x,y]}$
	
    \item \textbf{Java.} You should submit a file called \texttt{Range.java} to
        the Gradescope item ``Homework 1 (Java)." The file should define (at
        least) a top level function \texttt{rangeCount} that looks like: 
    \begin{itemize}
        \item \verb|public int rangeCount(int[] A, int x, int y)|
    \end{itemize}
    where $\texttt{A}$ is a sorted \verb|int[]| and \verb|[x,y]| is the range of
    our query. Return the number of elements of \verb|A| in the range 
    \verb|[x,y]| 
\end{itemize}

\end{document}

\documentclass[titlepage, 12pt, leqno]{article}

% -------------------------------------------------- %
% -------------------- PACKAGES -------------------- %
% -------------------------------------------------- %
\usepackage{import}
\usepackage{pdfpages}
\usepackage{mathtools}
\usepackage{transparent}
\usepackage{xcolor}
\usepackage{tcolorbox}
\usepackage{amsmath}
\usepackage{amssymb}
\usepackage{parskip}
\usepackage{bbm}
\usepackage{multicol}
\usepackage{blindtext}
\usepackage[margin = .5in]{geometry}
\tcbuselibrary{breakable}
\tcbset{breakable = true}


% -------------------------------------------------- %
% -------------- CUSTOM ENVIRONMENTS --------------- %
% -------------------------------------------------- %
\newtcolorbox{note}{colback=black!5!white,
                          colframe=black!55!white,
                          fonttitle=\bfseries,title=Note}

\newtcolorbox{ex}{colback=blue!5!white,
                          colframe=blue!55!white,
                          fonttitle=\bfseries,title=Example}

\newtcolorbox{definition}{colback=red!5!white,
                          colframe=red!55!white,
                          fonttitle=\bfseries,title=Definition}


% -------------------------------------------------- %
% ------------------- COMMANDS --------------------- %
% -------------------------------------------------- %
% Brackets, braces, etc. 
\newcommand{\abs}[1]{\lvert #1 \rvert}
\newcommand{\bigabs}[1]{\Bigl \lvert #1 \Bigr \rvert}
\newcommand{\bigbracket}[1]{\Bigl [ #1 \Bigr ]}
\newcommand{\bigparen}[1]{\Bigl ( #1 \Bigr )}
\newcommand{\ceil}[1]{\lceil #1 \rceil}
\newcommand{\floor}[1]{\lfloor #1 \rfloor}
\newcommand{\norm}[1]{\| #1 \|}
\newcommand{\bignorm}[1]{\Bigl \| #1 \Bigr \| #1}
\newcommand{\inner}[1]{\langle #1 \rangle}
\newcommand{\set}[1]{{ #1 }}


% -------------------------------------------------- %
% -------------------- SETUP ----------------------- %
% -------------------------------------------------- %
\begin{document}
\setlength{\parskip}{0.5\baselineskip}
\setlength{\parindent}{15pt}
\setlength{\belowdisplayskip}{0pt}
\setlength{\abovedisplayskip}{0pt}
\setlength{\belowdisplayshortskip}{0pt}
\setlength{\abovedisplayshortskip}{0pt}


% -------------------------------------------------- %
% --------------------- BODY ----------------------- %
% -------------------------------------------------- %

\begin{multicols*}{2}

\textbf{Control:}
\begin{itemize}
    \item The \texttt{cmp src1, src2} instruction explicitly sets condition 
        codes
    \item \texttt{jmp} is an unconditional jump. All other \texttt{j*} opcodes
        are conditional jumps.
\end{itemize}

\textbf{Procedures}:
\begin{itemize}
    \item The \textbf{callee} is called by the \textbf{caller}. To do so, the
        \texttt{callq} and \texttt{retq} move the instruction pointer in and out 
        of the callee.
    \item \textit{Caller-saved} register values need to be pushed onto the stack
        before making a procedure call only if the caller needs that value later.
    \item \textit{Callee-saved} register values need to be pushed onto the stack
        only if the callee intends to use those registers.
\end{itemize}

\textbf{Arrays/Overflows}:
\begin{itemize}
    \item Casting a pointer from one type to another will \textit{not} change its
        address.
\end{itemize}

\textbf{Pointers, Compilation, Linking:}
\begin{itemize}
    \item The program's \textit{entry point} can be found in an executable object
        file but not a relocatable object file.
\end{itemize}

\textbf{Exceptions, Processes}
\begin{itemize}
    \item A \textbf{preempt} is when the OS temporarily suspends an executing
        process to execute a different process.
    \item When a system call is performed in x86, the \texttt{\%rax} register
        contains the system call index.
    \item A divide-by-zero error \textit{always} results in the termination of a
        process on Unix.
    \item \textbf{Interrupts} are caused by hardware connected to the CPU.
    \item A \textbf{fault} happens unintentionally, but may potentially be 
        recovered from.
\end{itemize}

\columnbreak
\textbf{Files}
\begin{itemize}
    \item \textbf{syscall stub procedures} contain assembly instructions, are
        responsible for enforcing register conventions, and need to understand
        how syscalls map to specific numbers
    \item The CPU uses the exception table to identify handlers for both syscalls
        and interrupts.
    \item Both syscalls and interrupts jump to the next instruction of the
        user-mode code when they finish, instead of re-executing the current
        instruction.
    \item regular files, directories, virtual divices, and process information
        are exposed as files.
\end{itemize}

\textbf{Recurrsion}:
\begin{itemize}
    \item A \textbf{register spill} is when the register allocator runs out of
        registers and must save and restore values from memory.
\end{itemize}

\textbf{File I/O and processes}
\begin{itemize}
    \item \textbf{Stubs} place the syscall number as an argument into registers
        as needed, effectively giving the illusion that they are normal procedure
        calls.
    \item The stubs handle the return value from syscalls, which are placed in
        \texttt{\%rax} by the kernel.
    \item Hardware periodically raises interrupts that give control to other
        processes to run.
\end{itemize}

\textbf{Storage}
\begin{itemize}
    \item Main memory is DRAM, off-chip L2 caches are SRAM.
    \item A sequence of accesses has \textbf{stride} $k$ if it is $k$ elements 
        from the previous access. (Stride of 1 is sequential)
\end{itemize}

\textbf{Cache Mapping Question:}
Given a capacity $C$ of $2^n$ bytes, and a block size of $2^m$, the number of 
blocks is $2^{n-m}$. The width of the block offset field is $m$. The total number of bits in an address is $B$, and the number of \textit{tag bits} is $B-n-m$.
Basically, $m+n$ bits are used for the block offset and index, and the rest are
the tag.

\pagebreak
\textbf{Caching:}
\begin{itemize}
    \item A \textbf{hit} is when data is in the selected block. A \textbf{miss} is
        when it is not in that block and is fetched from memory (slower).
    \item Three types of misses, \textit{Cold Miss} on the first request,
        \textit{Capacity Miss} when the set of blocks in use doesn't fit, or
        \textit{Conflict Miss} that arrises from a previous eviction.
    \item \textit{Tag bits} differentiate blocks in the same index of a software
        hash table.
    \item Hash tables have a certain number of elements per slot in the table.
        Old entries are evicted FIFO-style.
\end{itemize}


\end{multicols*}

\end{document}

\documentclass[titlepage, 12pt, leqno]{article}

% -------------------------------------------------- %
% -------------------- PACKAGES -------------------- %
% -------------------------------------------------- %
\usepackage{import}
\usepackage{pdfpages}
\usepackage{mathtools}
\usepackage{transparent}
\usepackage{xcolor}
\usepackage{tcolorbox}
\usepackage{amsmath}
\usepackage{amssymb}
\usepackage{parskip}
\usepackage{bbm}
\usepackage[margin = 1in]{geometry}


% -------------------------------------------------- %
% -------------- CUSTOM ENVIRONMENTS --------------- %
% -------------------------------------------------- %
\newtcolorbox{note}{colback=black!5!white,
                          colframe=black!55!white,
                          fonttitle=\bfseries,title=Note}

\newtcolorbox{ex}{colback=blue!5!white,
                          colframe=blue!55!white,
                          fonttitle=\bfseries,title=Example}

\newtcolorbox{definition}{colback=red!5!white,
                          colframe=red!55!white,
                          fonttitle=\bfseries,title=Definition}


% -------------------------------------------------- %
% ------------------- COMMANDS --------------------- %
% -------------------------------------------------- %
% Brackets, braces, etc. 
\newcommand{\abs}[1]{\lvert #1 \rvert}
\newcommand{\bigabs}[1]{\Bigl \lvert #1 \Bigr \rvert}
\newcommand{\bigbracket}[1]{\Bigl [ #1 \Bigr ]}
\newcommand{\bigparen}[1]{\Bigl ( #1 \Bigr )}
\newcommand{\ceil}[1]{\lceil #1 \rceil}
\newcommand{\floor}[1]{\lfloor #1 \rfloor}
\newcommand{\norm}[1]{\| #1 \|}
\newcommand{\bignorm}[1]{\Bigl \| #1 \Bigr \| #1}
\newcommand{\inner}[1]{\langle #1 \rangle}
\newcommand{\set}[1]{{ #1 }}


% -------------------------------------------------- %
% -------------------- SETUP ----------------------- %
% -------------------------------------------------- %
\title{\Huge{Lecture 14 - Peer Review}}
\author{\large{Mitch Harrison}}
\date{\today}   
\begin{document}
\setlength{\parskip}{1\baselineskip}
\setlength{\parindent}{15pt}
\maketitle
\tableofcontents
\newpage


% -------------------------------------------------- %
% --------------------- BODY ----------------------- %
% -------------------------------------------------- %
\section{What is peer review?}

\begin{definition}
    \textbf{Peer review} is the "expert assessment of submitted materials" and is
    used to help ensure that published materials are of high quality and free
    from mistakes.
\end{definition}

Peer review is used to maintain quality standards and credibility in shared
results. Many times peer review is blinded, which allows for more honest and
constructive comments.

\begin{note}
    Peer review is \textit{not} limited to academic settings and journal
    submissions!
\end{note}

Peer reviewers are tasked with providing a dispassionate evaluation of the work
in question - this review is often used for decision-making purposes on behalf of
a larger institution. 

Useful peer reviews provide helpful feedback (both positive and negative) about
the relative strengths and weaknesses of the piece being reviewed. Just because
there are no mistakes doesn't necessarily mean that an article is worthy of
publication!

\pagebreak
\section{Evaluating the introduction}
\begin{itemize}
    \item Was the main goal of the analysis easy to identify and appropriate for
        addressing the research problem?
    \item Was the rationale for the data analysis explained well?
    \item Did the manuscript describe the context/background of the work and
        its relation to existing literature?
    \item Were the variables (response and predictors) clearly identified and
        discussed?
    \item Did the manuscript explain how the data were collected and/or how they
        were derived?
    \item If provided, was any EDA helpful and informative in addressing the main
        goal(s)?i
\end{itemize}

\pagebreak
\section{Evaluating the methodology}
\begin{itemize}
    \item Is the proposed analysis appropriate given the main goal(s) and 
        dataset?
    \item Why was this particular methodology chosed over competing choices?
    \item Are the specific methods described in enough detail that the work could
        be replicated by other researchers without access to the original
        analysis code?
    \item Is it clear which approaches/models were used to evaluate specific 
        goals?
    \item What assumptions are needed for the model(s), and how do you plan to
        assess whether they hold?
    \item What sensitivity analyses, if any, are planned, and how do they relate 
        to your analysis approach?
\end{itemize}

\pagebreak
\section{Evaluating the results}
\begin{itemize}
    \item Were tables formatted cleanly and precisely?
    \item Did visualizations follow good practices (e.g., clean axis labels,
        clear titles, appropriate figures given data types, etc.)?
    \item Did tables/figures refer to raw variables names, or were all references
        clearly made in context of the data?
    \item Were appropriate conventions re: formatting (e.g., an acceptable
        number of decimal places, table/figure captions, etc.) followed when
        displaying results?
    \item Was there an appropriate quantification of uncertainty of estimates?
    \item Were all results interpreted correctly?
\end{itemize}

\pagebreak
\section{Evaluating the discussion}
\begin{itemize}
    \item How did results address or fail to address the goal(s) of the 
        manuscript?
    \item Did the manuscript provide clear, correct, and effective interpretation
        of the analysis results?
    \item Were all conclusions made directly supported by the data?
    \item Were there any issues with reliability or validity of the data?
\end{itemize}

\end{document}

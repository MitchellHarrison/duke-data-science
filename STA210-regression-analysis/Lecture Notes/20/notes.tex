\documentclass[titlepage, 12pt, leqno]{article}

% -------------------------------------------------- %
% -------------------- PACKAGES -------------------- %
% -------------------------------------------------- %
\usepackage{import}
\usepackage{pdfpages}
\usepackage{mathtools}
\usepackage{transparent}
\usepackage{xcolor}
\usepackage{tcolorbox}
\usepackage{amsmath}
\usepackage{amssymb}
\usepackage{parskip}
\usepackage{bbm}
\usepackage[margin = 1in]{geometry}
\tcbuselibrary{breakable}
\tcbset{breakable = true}


% -------------------------------------------------- %
% -------------- CUSTOM ENVIRONMENTS --------------- %
% -------------------------------------------------- %
\newtcolorbox{note}{colback=black!5!white,
                          colframe=black!55!white,
                          fonttitle=\bfseries,title=Note}

\newtcolorbox{ex}{colback=blue!5!white,
                          colframe=blue!55!white,
                          fonttitle=\bfseries,title=Example}

\newtcolorbox{definition}{colback=red!5!white,
                          colframe=red!55!white,
                          fonttitle=\bfseries,title=Definition}


% -------------------------------------------------- %
% ------------------- COMMANDS --------------------- %
% -------------------------------------------------- %
% Brackets, braces, etc. 
\newcommand{\abs}[1]{\lvert #1 \rvert}
\newcommand{\bigabs}[1]{\Bigl \lvert #1 \Bigr \rvert}
\newcommand{\bigbracket}[1]{\Bigl [ #1 \Bigr ]}
\newcommand{\bigparen}[1]{\Bigl ( #1 \Bigr )}
\newcommand{\ceil}[1]{\lceil #1 \rceil}
\newcommand{\floor}[1]{\lfloor #1 \rfloor}
\newcommand{\norm}[1]{\| #1 \|}
\newcommand{\bignorm}[1]{\Bigl \| #1 \Bigr \| #1}
\newcommand{\inner}[1]{\langle #1 \rangle}
\newcommand{\set}[1]{{ #1 }}


% -------------------------------------------------- %
% -------------------- SETUP ----------------------- %
% -------------------------------------------------- %
\title{\Huge{Lecture 19 - Linear Mixed Models pt. 1}}
\author{\large{Mitch Harrison}}
\date{\today}   
\begin{document}
\setlength{\parskip}{1\baselineskip}
\setlength{\parindent}{15pt}
\maketitle
\tableofcontents
\newpage


% -------------------------------------------------- %
% --------------------- BODY ----------------------- %
% -------------------------------------------------- %
\section{Mixed Models}

Say we have a dataset of various coffee species and their properties. There may
be an independence violation because coffee grown in the same region may have
similar properties. This is a \textbf{independence violation due to nesting.} 
Thus, we are left with two "levels" of variance: one inter-
region variance and one intra-region variance. Using a single model would not be
able to get at both of these variances.

\subsection{Random intercept models}
\begin{definition}
    \textbf{Random effects} are a second course of variability (not $\epsilon$).
    For each region in the coffee example, all slope coefficients may be the same,
    but the intercept term would vary such that $\beta_{0j} = \gamma_{00} + 
    u_{0j}$ for the $j$th region in the dataset.
\end{definition}

Taking these random effects into a linear model with identical slopes but varying
intercepts, we arrive at the following model for the $i$th observation of region
$j$.
\[
    y_{ij} = (\gamma_{00} + u_{0j}) + \gamma_1x_1 + \cdots + \epsilon_{ij}
\]
With a \textbf{random intercept model} like the one above, we can answer questions
like "What is the relationship between region and $y$ while taking into account
region-level clustering?" or "how much variability in $y$ occurs at the region-
specific level (while knowing that higher $x$ are associated with higher $y$?"

These random effect can also be used with slopes, not just intercepts. These are
\textbf{random slopes} and look like the following:
\[
    y_{ij} = (\gamma_{00} + u_{0j}) + (\gamma_{10} + u_{1j})x_{1j} + 
    \cdots + \epsilon_{ij}
\]

\end{document}

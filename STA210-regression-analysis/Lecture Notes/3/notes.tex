\documentclass[titlepage, 12pt, leqno]{article}

% -------------------------------------------------- %
% -------------------- PACKAGES -------------------- %
% -------------------------------------------------- %
\usepackage{import}
\usepackage{pdfpages}
\usepackage{mathtools}
\usepackage{transparent}
\usepackage{xcolor}
\usepackage{tcolorbox}
\usepackage{amsmath}
\usepackage{amssymb}
\usepackage{parskip}
\usepackage[margin = 1in]{geometry}


% -------------------------------------------------- %
% -------------- CUSTOM ENVIRONMENTS --------------- %
% -------------------------------------------------- %
\newtcolorbox{note}{colback=black!5!white,
                          colframe=black!55!white,
                          fonttitle=\bfseries,title=Note}

\newtcolorbox{ex}{colback=blue!5!white,
                          colframe=blue!55!white,
                          fonttitle=\bfseries,title=Example}

\newtcolorbox{definition}{colback=red!5!white,
                          colframe=red!55!white,
                          fonttitle=\bfseries,title=Definition}


% -------------------------------------------------- %
% ------------------- COMMANDS --------------------- %
% -------------------------------------------------- %
% Brackets, braces, etc. 
\newcommand{\abs}[1]{\lvert #1 \rvert}
\newcommand{\bigabs}[1]{\Bigl \lvert #1 \Bigr \rvert}
\newcommand{\bigbracket}[1]{\Bigl [ #1 \Bigr ]}
\newcommand{\bigparen}[1]{\Bigl ( #1 \Bigr )}
\newcommand{\ceil}[1]{\lceil #1 \rceil}
\newcommand{\floor}[1]{\lfloor #1 \rfloor}
\newcommand{\norm}[1]{\| #1 \|}
\newcommand{\bignorm}[1]{\Bigl \| #1 \Bigr \| #1}
\newcommand{\inner}[1]{\langle #1 \rangle}
\newcommand{\set}[1]{{ #1 }}


% -------------------------------------------------- %
% -------------------- SETUP ----------------------- %
% -------------------------------------------------- %
\title{\Huge{Lecture 3 - Visualizing Data}}
\author{\large{Mitch Harrison}}
\date{\today}   
\begin{document}
\setlength{\parskip}{1\baselineskip}
\setlength{\parindent}{15pt}
\maketitle
\tableofcontents
\newpage


% -------------------------------------------------- %
% --------------------- BODY ----------------------- %
% -------------------------------------------------- %
\section{Exploratory Data Analysis (EDA)}

\begin{definition}
    \textbf{Exploratory data analysis (EDA)} is an approach to analyzing data to summarise their main characteristics.
\end{definition}

EDA is often visual. Today, we will focus on visualizing data in R using \textit{ggplot2}. Familiarizing oneself with a data set comes before this step. Knowing which variables (and types of variables) are in the data set is the first step in appropriately visualizing it.
\pagebreak

\section{Visualizing Data}
\subsection{Grammar of Graphics}
The \textbf{Grammar of Graphics} concisely describes the elements of data visualization. The following list is the order that these elements appear. The \textit{ggplot2} package (part of the \textit{tidyverse}) implements the grammar of graphics in R.
\begin{enumerate}
    \item Data
    \item Aesthetics
    \item Geometries
    \item Facets
    \item Statistics
    \item Coordinates
    \item Theme
\end{enumerate}
\pagebreak
\section{Manipulating Data}
\subsection{Tidy Data}
\begin{definition}
    \textbf{Tidy data} is a dataset in which each row in an observation and each column is a variable.
\end{definition}
Data may not be tidy for any number of reasons, but the vast majority of the time, we will tidy our data prior to visualizing it.

\subsection{Why do we visualize data?}
Visualizing data makes it much easier to intuitively see patterns in data. These patterns are sometimes not even apparent by calculating summary statistics. The \textit{datasauRus\_dozen} datasets shows several plots with a similar \textit{correlation coefficient} that are visibly completely different.

\end{document}

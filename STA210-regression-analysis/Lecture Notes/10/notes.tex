\documentclass[titlepage, 12pt, leqno]{article}

% -------------------------------------------------- %
% -------------------- PACKAGES -------------------- %
% -------------------------------------------------- %
\usepackage{import}
\usepackage{pdfpages}
\usepackage{mathtools}
\usepackage{transparent}
\usepackage{xcolor}
\usepackage{tcolorbox}
\usepackage{amsmath}
\usepackage{amssymb}
\usepackage{parskip}
\usepackage[margin = 1in]{geometry}


% -------------------------------------------------- %
% -------------- CUSTOM ENVIRONMENTS --------------- %
% -------------------------------------------------- %
\newtcolorbox{note}{colback=black!5!white,
                          colframe=black!55!white,
                          fonttitle=\bfseries,title=Note}

\newtcolorbox{ex}{colback=blue!5!white,
                          colframe=blue!55!white,
                          fonttitle=\bfseries,title=Example}

\newtcolorbox{definition}{colback=red!5!white,
                          colframe=red!55!white,
                          fonttitle=\bfseries,title=Definition}


% -------------------------------------------------- %
% ------------------- COMMANDS --------------------- %
% -------------------------------------------------- %
% Brackets, braces, etc. 
\newcommand{\abs}[1]{\lvert #1 \rvert}
\newcommand{\bigabs}[1]{\Bigl \lvert #1 \Bigr \rvert}
\newcommand{\bigbracket}[1]{\Bigl [ #1 \Bigr ]}
\newcommand{\bigparen}[1]{\Bigl ( #1 \Bigr )}
\newcommand{\ceil}[1]{\lceil #1 \rceil}
\newcommand{\floor}[1]{\lfloor #1 \rfloor}
\newcommand{\norm}[1]{\| #1 \|}
\newcommand{\bignorm}[1]{\Bigl \| #1 \Bigr \| #1}
\newcommand{\inner}[1]{\langle #1 \rangle}
\newcommand{\set}[1]{{ #1 }}


% -------------------------------------------------- %
% -------------------- SETUP ----------------------- %
% -------------------------------------------------- %
\title{\Huge{Lecture 9 - Transformations}}
\author{\large{Mitch Harrison}}
\date{\today}   
\begin{document}
\setlength{\parskip}{1\baselineskip}
\setlength{\parindent}{15pt}
\maketitle
\tableofcontents
\newpage


% -------------------------------------------------- %
% --------------------- BODY ----------------------- %
% -------------------------------------------------- %
\section{Logarithmic Transformation}
Below is an example model predicting diamond price given carats:
\[
log( \text{price}_i | \text{carat}_i) = \beta_0 + \beta_1( \text{carat}_i)
+\epsilon_i
\]
\begin{note}
    $E(log(Y)) \ne log(E(Y))$
\end{note}

\subsection{Interpreting logarithmic models}
The following model uses the natural log, so 
\begin{align*}
    \widehat{log( \text{price}_i)} &= 6.15 + 2.07( \text{carat}_i) \\
    \widehat{ \text{price}} &= exp(6.15 + 2.07( \text{carat}_i)) \\
                                   &= exp(6.15)exp(2.07 \times \text{carat}_i)
\end{align*}

With this model, we predict an $e^{2.07}$-times (or 7.92-times) increase in price 
for each additional carat. When logarithms are involved, we expect a 
multiplicative change per unit of explanatory variables, not an additive one.

\begin{note}
    We \textit{do not} use the word \textbf{expect} in this context. We use the
    term \textbf{predict}.
\end{note}

\subsection{Logarithmic Predictors}
Assume the following model:
\[
y_i = \beta_0 + \beta_1log_10(x_i) = \epsilon_i
\]
To make the logarithmic term more easy to interpret, we will let $log_{10}(x_i)
= z_i$. Now, for a one-unit increase in $z_i$, there is a ten-fold increase in
$x_1$.

\pagebreak
\section{Exponential Transformations}
Take the following model:
\[
    y_i = \beta_0 + \beta_1x_{i} + \beta_2x_{i}^2 + \epsilon_i
\]
We include a linear term and a quadratic term for the same variable $x$. This does
introduce \textit{some} tradeoffs, specifically with the fact that computational
resources are spent to calculate $\beta_1$ and also the fact that introducing 
another coefficient reduces our degrees of freedom by 1.

\pagebreak
\section{Transformations on both sides}
Take the following model:
\[
    log(y_i) = \beta_o + \beta_1(log(x_i)) + \epsilon_1
\]
In this model, a one-unit increase in $x_i$ produces a multiplicative difference
in $y_i$. Since $x_i$ is logarithmic, this is also a multiplicative change.

\end{document}

\documentclass[titlepage, 12pt, leqno]{article}

% -------------------------------------------------- %
% -------------------- PACKAGES -------------------- %
% -------------------------------------------------- %
\usepackage{tcolorbox}
\usepackage{amsmath}
\usepackage[margin = 1in]{geometry}


% -------------------------------------------------- %
% -------------- CUSTOM ENVIRONMENTS --------------- %
% -------------------------------------------------- %
\newtcolorbox{note}{colback=black!5!white,
                          colframe=black!55!white,
                          fonttitle=\bfseries,title=Note}

\newtcolorbox{ex}{colback=blue!5!white,
                          colframe=blue!55!white,
                          fonttitle=\bfseries,title=Example}

\newtcolorbox{definition}{colback=red!5!white,
                          colframe=red!55!white,
                          fonttitle=\bfseries,title=Definition}

% -------------------------------------------------- %
% ------------------- SETTINGS --------------------- %
% -------------------------------------------------- %
\setlength{\parindent}{0pt}
\pagenumbering{arabic}

% -------------------------------------------------- %
% -------------------- SETUP ----------------------- %
% -------------------------------------------------- %
\title{\Huge{Lecture 21 - Linear Mixed Models pt.2}}
\author{\large{Mitch Harrison}}
\date{\today}
\begin{document}
\maketitle
\tableofcontents
\newpage

% -------------------------------------------------- %
% --------------------- BODY ----------------------- %
% -------------------------------------------------- %
\section{Code for Fitting Mixed Effects Models}

In the below code, the \texttt{c1} variable is categorical, while all \texttt{x} variables
are numeric. Also note the \texttt{1} term that is constant. It is telling R that we want
an intercept. The \texttt{|} operator tells R that that term is a mixed effect, while
all "normal" terms are fixed.

\begin{verbatim}
library(lme4)
model <- lmer(y ~ 1 + x1 + (1 + x2 | c1), data = data)
\end{verbatim}

\begin{note}
    The \texttt{summary()} function will then show different sections for mixed and fixed
    effects. Additionally, \texttt{summary(model)\$coef} will not show the random effects.
\end{note}

\pagebreak
\section{Longitudinal Data}
\begin{definition}
    \textbf{Longitudinal data} are collected on the same subjects over time. For example,
    surveying the same participants every year for 10 years.
\end{definition}

Longitudinal data is \textit{not} independent across time by definition, since the same
participants are involved at each time period. This is true whether or not each participant
is independent from others.

\pagebreak
\section{Conditional vs. Marginal Models}
Interpreting the fixed effects are the \textit{conditional slopes} for a participant. Even
though we have only a single fixed effect here, we are still performing estimation
conditionally on something.

If there is large between-group variability, then maybe the relative associations due to 
the fixed effects would be small. We may want to examine the slopes corresponding to
fixed effects at different levels of the random effects, or even what the average fixed
effects are while \textbf{marginalizing} out the random effects.

\subsection{Generalized estimating equations}
GEE's are often used for clustered or longitudinal data when we are interested in marginal
effects of parameter estimates. That is, a \textbf{population average effect}. This is in
contrast to the cluster-specific model being estimated previously.

\begin{definition}
    \textbf{Generalized estimated equations (GEEs)} estimate the population
    \textit{average} without trying to model the covariance struction "within each subject"
    like when we estimate subject-specific random effects in mixed models.
\end{definition}

The type of question we use GEEs with is "For a one-unit increase in $X$ in the
\textit{population}, how much would the average $Y$ be expected to change?

\subsection{Correlation structure}
We can specify a correlation structure based on prior knowledge of the way observations
are correlated.
\begin{itemize}
    \item Independent (time points are independent)
    \item Exchangeable (pairs of points have identical correlations)
    \item Autoregressive (time points have autoregressive structure, so correlation
        between two observations one unit apart is $p$, two units apart is $p^2$, etc.)
    \item Unstructured (unique correlation estimated for each pair of points
    \item ... and others.
\end{itemize}

\pagebreak
\subsection{Code for fitting a GEE}
This will not be asked on any in-class assessment, but is good for possible final
project/future work. \textit{Note: the \texttt{subject} variable is whichever variable
is tracking the identity of each participant.}

\begin{verbatim}
library(gee)
model <- gee(
    y ~ x1,
    data = data,
    id = subject,
    family = "gaussian",
    corstr = "exchangeable"
)
\end{verbatim}

\begin{note}
    The \texttt{corstr} parameter is \textit{correlation structure}. There are many
    possible values, and each correspond to a different type of correlation described
    above.
\end{note}

\end{document}

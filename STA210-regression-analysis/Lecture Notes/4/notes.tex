\documentclass[titlepage, 12pt, leqno]{article}

% -------------------------------------------------- %
% -------------------- PACKAGES -------------------- %
% -------------------------------------------------- %
\usepackage{import}
\usepackage{pdfpages}
\usepackage{mathtools}
\usepackage{transparent}
\usepackage{xcolor}
\usepackage{tcolorbox}
\usepackage{amsmath}
\usepackage{amssymb}
\usepackage{parskip}
\usepackage[margin = 1in]{geometry}


% -------------------------------------------------- %
% -------------- CUSTOM ENVIRONMENTS --------------- %
% -------------------------------------------------- %
\newtcolorbox{note}{colback=black!5!white,
                          colframe=black!55!white,
                          fonttitle=\bfseries,title=Note}

\newtcolorbox{ex}{colback=blue!5!white,
                          colframe=blue!55!white,
                          fonttitle=\bfseries,title=Example}

\newtcolorbox{definition}{colback=red!5!white,
                          colframe=red!55!white,
                          fonttitle=\bfseries,title=Definition}


% -------------------------------------------------- %
% ------------------- COMMANDS --------------------- %
% -------------------------------------------------- %
% Brackets, braces, etc. 
\newcommand{\abs}[1]{\lvert #1 \rvert}
\newcommand{\bigabs}[1]{\Bigl \lvert #1 \Bigr \rvert}
\newcommand{\bigbracket}[1]{\Bigl [ #1 \Bigr ]}
\newcommand{\bigparen}[1]{\Bigl ( #1 \Bigr )}
\newcommand{\ceil}[1]{\lceil #1 \rceil}
\newcommand{\floor}[1]{\lfloor #1 \rfloor}
\newcommand{\norm}[1]{\| #1 \|}
\newcommand{\bignorm}[1]{\Bigl \| #1 \Bigr \| #1}
\newcommand{\inner}[1]{\langle #1 \rangle}
\newcommand{\set}[1]{{ #1 }}


% -------------------------------------------------- %
% -------------------- SETUP ----------------------- %
% -------------------------------------------------- %
\title{\Huge{Lecture 4 - Ordinary Least Squares}}
\author{\large{Mitch Harrison}}
\date{\today}   
\begin{document}
\setlength{\parskip}{1\baselineskip}
\setlength{\parindent}{15pt}
\maketitle
\tableofcontents
\newpage


% -------------------------------------------------- %
% --------------------- BODY ----------------------- %
% -------------------------------------------------- %
\section{What is a linear model?}
Let's consider a situation with a single continuous predictor:
\[
    y_i = \beta_0 + \beta_1x_i + \epsilon_i 
\]
In this case:
\begin{align*}
    y_i &= \text{the outcome (dependent variable) for observation } i \\
    \beta_0 &= \text{the intercept parameter} \\
    \beta_1 &= \text{the slope parameter} \\
    x_i &= \text{the predictor variable} \\
    \epsilon_i &= \text{the error}
\end{align*}

A function for an estimated $y_i$ given an $x$ value $x_i$ is:
\[
    E\left(y_i \;\middle|\; x_i\right) = \beta_0 + \beta_1x_i
\]

We want to find the best estimates $\hat{\beta}_0$ and $\hat{\beta}_1$.

\begin{definition}
    The \textbf{residual} $\epsilon_i$ is the difference between the \textit{observed} $y_i$ and the \textit{predicted} value $\hat{y}_i$:
   \begin{align*}
       \hat{\epsilon}_i &= y_i - \left(\hat{\beta}_0 + \hat{\beta}_1x_i\right) \\
       \Aboxed{\hat{\epsilon}_i &= y_i - \hat{y}_i}
   \end{align*}
\end{definition}
\pagebreak

\section{Loss functions}
\begin{definition}
    A \textbf{loss function} is a way of quantifying the accuracy of a model that sums the total residuals of all points in some way. Least squares is one, but any function could be used in theory.
\end{definition}

The \textbf{formula for ordinary least squares} is as follows:
\[
    f(y_1, \cdots ,y_n, \hat{y}_1, \cdots , \hat{y}_n) = \frac{1}{n} \sum_{i=1}^{n}\left(y_i - \hat{y}_i\right)^2 = \boxed{\frac{1}{n} \sum_{i=1}^{n}\left(y_i - \left(\hat{\beta}_0 + \hat{\beta}_1x_i\right)\right)^2}
\]

The R code for fitting a least-squares linear model is as follows. The \texttt{lm()} function returns a model:  
\begin{verbatim}
    m1 <- lm(y ~ x, data = myData)
\end{verbatim}

And using \texttt{tidymodels} package, we have the following syntax:
\begin{verbatim}
    model <- linear_reg() |>
        set_engine("lm") |>
        fit(y ~ x, data = myData)
\end{verbatim}

\end{document}

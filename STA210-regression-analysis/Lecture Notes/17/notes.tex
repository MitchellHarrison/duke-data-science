\documentclass[titlepage, 12pt, leqno]{article}

% -------------------------------------------------- %
% -------------------- PACKAGES -------------------- %
% -------------------------------------------------- %
\usepackage{import}
\usepackage{pdfpages}
\usepackage{mathtools}
\usepackage{transparent}
\usepackage{xcolor}
\usepackage{tcolorbox}
\usepackage{amsmath}
\usepackage{amssymb}
\usepackage{parskip}
\usepackage{bbm}
\usepackage[margin = 1in]{geometry}
\tcbuselibrary{breakable}
\tcbset{breakable = true}


% -------------------------------------------------- %
% -------------- CUSTOM ENVIRONMENTS --------------- %
% -------------------------------------------------- %
\newtcolorbox{note}{colback=black!5!white,
                          colframe=black!55!white,
                          fonttitle=\bfseries,title=Note}

\newtcolorbox{ex}{colback=blue!5!white,
                          colframe=blue!55!white,
                          fonttitle=\bfseries,title=Example}

\newtcolorbox{definition}{colback=red!5!white,
                          colframe=red!55!white,
                          fonttitle=\bfseries,title=Definition}


% -------------------------------------------------- %
% ------------------- COMMANDS --------------------- %
% -------------------------------------------------- %
% Brackets, braces, etc. 
\newcommand{\abs}[1]{\lvert #1 \rvert}
\newcommand{\bigabs}[1]{\Bigl \lvert #1 \Bigr \rvert}
\newcommand{\bigbracket}[1]{\Bigl [ #1 \Bigr ]}
\newcommand{\bigparen}[1]{\Bigl ( #1 \Bigr )}
\newcommand{\ceil}[1]{\lceil #1 \rceil}
\newcommand{\floor}[1]{\lfloor #1 \rfloor}
\newcommand{\norm}[1]{\| #1 \|}
\newcommand{\bignorm}[1]{\Bigl \| #1 \Bigr \| #1}
\newcommand{\inner}[1]{\langle #1 \rangle}
\newcommand{\set}[1]{{ #1 }}


% -------------------------------------------------- %
% -------------------- SETUP ----------------------- %
% -------------------------------------------------- %
\title{\Huge{Lecture 17 - Ordinal Regression}}
\author{\large{Mitch Harrison}}
\date{\today}   
\begin{document}
\setlength{\parskip}{1\baselineskip}
\setlength{\parindent}{15pt}
\maketitle
\tableofcontents
\newpage


% -------------------------------------------------- %
% --------------------- BODY ----------------------- %
% -------------------------------------------------- %
\section{Ordinal Data}

\begin{definition}
    \textbf{Ordinal data} is categorical bit has a logical order. For example,
    levels of education or pain scales.
\end{definition}

\subsection{Cumulative link model}
We might consider an outcome $Y$ that looks at all outcomes at once. For $j$
total ordered categories, we might model the \textbf{cumulative probability}
for observation $i$:
\begin{align*}
    \gamma_{ij} &= \mathbb{P}(Y_i \le j) \\
                &= \sum_{n=1}^{j}\mathbb{P}(Y_i = n)
\end{align*}
\begin{note}
    $\gamma$ is limited to values between 0 and 1 since it's a probability.
\end{note}

\begin{definition}
    A \textbf{cumulative link model} takes the following form:
    \[
        logit(\gamma_{ij}) = \beta_{0_j} + \beta_1x_{i1} + 
        \cdots + \beta_px_{ip}
    \]
\end{definition}

In a cumulative link model, the $\beta_{0;j}$ term is what we would predict the
logit of $\gamma$ would be if an observation had zeroes for all predictors.

\subsection{Interpreting ordered logistic regression models}
Exponentiating our model, we say that for a one-unit increase in $x_i$, we
predict an $e^{\beta_i}$-fold increase in the odds of being in the next 
highest level of the response variable.

\pagebreak
\subsection{The proportional odds assumption}
\begin{definition}
     The \textbf{proportional odds assumption} implies that changes in $X_k$ 
     have the same conditional relationship with odds of being in categories
     $a$ vs. $b$, $b$ vs $c$, $c$ vs. $d$, etc for any $j$ vs. $j-1$.
\end{definition}

\pagebreak
\section{Code}
The code for ordinal logistic regression models is as follows:
\begin{verbatim}
library(MASS)
model <- polr(factor(y) ~ x1 + x2, data = data)
\end{verbatim}

To interpret:
\begin{verbatim}
exp(coef(model))
\end{verbatim}

Say we have a predictor called \texttt{got\_treatment} and another called
\texttt{BMI}. The number ($n$) listed under \texttt{got\_treatment} means that 
patients who receive treatment have $n$\% of the odds of having the next
higher level of response variable compared to those who did not get treatment,
while controlling form BMI.

\end{document}

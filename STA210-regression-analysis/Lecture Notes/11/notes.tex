\documentclass[titlepage, 12pt, leqno]{article}

% -------------------------------------------------- %
% -------------------- PACKAGES -------------------- %
% -------------------------------------------------- %
\usepackage{import}
\usepackage{pdfpages}
\usepackage{mathtools}
\usepackage{transparent}
\usepackage{xcolor}
\usepackage{tcolorbox}
\usepackage{amsmath}
\usepackage{amssymb}
\usepackage{parskip}
\usepackage[margin = 1in]{geometry}


% -------------------------------------------------- %
% -------------- CUSTOM ENVIRONMENTS --------------- %
% -------------------------------------------------- %
\newtcolorbox{note}{colback=black!5!white,
                          colframe=black!55!white,
                          fonttitle=\bfseries,title=Note}

\newtcolorbox{ex}{colback=blue!5!white,
                          colframe=blue!55!white,
                          fonttitle=\bfseries,title=Example}

\newtcolorbox{definition}{colback=red!5!white,
                          colframe=red!55!white,
                          fonttitle=\bfseries,title=Definition}


% -------------------------------------------------- %
% ------------------- COMMANDS --------------------- %
% -------------------------------------------------- %
% Brackets, braces, etc. 
\newcommand{\abs}[1]{\lvert #1 \rvert}
\newcommand{\bigabs}[1]{\Bigl \lvert #1 \Bigr \rvert}
\newcommand{\bigbracket}[1]{\Bigl [ #1 \Bigr ]}
\newcommand{\bigparen}[1]{\Bigl ( #1 \Bigr )}
\newcommand{\ceil}[1]{\lceil #1 \rceil}
\newcommand{\floor}[1]{\lfloor #1 \rfloor}
\newcommand{\norm}[1]{\| #1 \|}
\newcommand{\bignorm}[1]{\Bigl \| #1 \Bigr \| #1}
\newcommand{\inner}[1]{\langle #1 \rangle}
\newcommand{\set}[1]{{ #1 }}


% -------------------------------------------------- %
% -------------------- SETUP ----------------------- %
% -------------------------------------------------- %
\title{\Huge{Lecture 11 - Comparing Models}}
\author{\large{Mitch Harrison}}
\date{\today}   
\begin{document}
\setlength{\parskip}{1\baselineskip}
\setlength{\parindent}{15pt}
\maketitle
\tableofcontents
\newpage


% -------------------------------------------------- %
% --------------------- BODY ----------------------- %
% -------------------------------------------------- %
\section{Comparing Models}

Suppose one has two models to work with. What sorts of ways might you compare the 
models? How might one model be "better" than another?

First, there is an argument to be made that given similar performance, a model
that is easier to interpret for humans (fewer parameters, non-transformed 
parameters, etc.) is the "superior" model. Second, a model having smaller 
residuals/more accurate predictions is essential. 

\pagebreak
\section{Metrics for model evaluation}
\subsection{Mean Squared Error}
\begin{definition}
    The formula for \textbf{mean squared error (MSE)} is given by:
    \[
        MSE = \frac{1}{n}\sum_{i=1}^{n}(y_i-\hat y_i)^2
    \]
\end{definition}

The \texttt{rmse()} function in R can help calculate the MSE or RMSE (root
mean squared error). Augmented models are passed into that function, not
models themselves. An example code chunk is as follows:
\begin{verbatim}
model_aug <- aug(model)
rmse(model_aug, truth = y, estimate = .fitted)
\end{verbatim}

\begin{note}
    In the above chunk, $y$ is the response variable.
\end{note}

\subsection{F-test for linear models}
The null and alternative hypotheses for an \textbf{F-test} is as follows:
\begin{align*}
    H_0 &= \beta_1 = \cdots = \beta_p = 0\\
    H_1 &= \text{at least one }\beta_k \ne 0
\end{align*}

We use \textit{F-tests} to check for any non-zero slopes in the model. It does not
tell us which slopes are non-zero, but does tell us whether or not there are any
predictors which are helping us make predictions.

The \textit{F-test} is named so because it uses an \textbf{F-distribution}.

\begin{note}
    The \texttt{summary()} function in R will produce the \textit{F-statistic}
    and the corresponding p-value for that F-statistic. That is how we will 
    evaluate the above hypotheses. The given p-value is the distribution of the
    test statistic under $H_0$.
\end{note}

\pagebreak
\subsection{Nested Models}
\begin{definition}
    Two linear models are \textbf{nested} if one of the models consists solely of
    terms found "within" another. For instance:
   \begin{align*}
       y_i &= \beta_0 + \beta_1x_{i1} + \beta_2x_{i2} + \epsilon_i \\
       y_i &= \beta_0 + \beta_1x_{i1} + \epsilon_i
   \end{align*}
\end{definition}

To compare multiple nested models, we run \textit{F-tests} on them. For three
nested models \texttt{m0, m1, m2}, we use the \texttt{anova()} function in R.

\begin{verbatim}
anova(m0, m1, m2)
\end{verbatim}

The output of the \texttt{anova()} function will displays several numbers, one of
which is the p-value for the F-test. If that p-value is smaller than our
significance level, we reject the null hypothesis that the additional predictors
have no impact on our model.

\begin{note}
    We can \textbf{only use F-tests with nested models}, never with models that
    are not nested.
\end{note}

\end{document}

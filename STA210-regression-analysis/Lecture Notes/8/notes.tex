\documentclass[titlepage, 12pt, leqno]{article}

% -------------------------------------------------- %
% -------------------- PACKAGES -------------------- %
% -------------------------------------------------- %
\usepackage{import}
\usepackage{pdfpages}
\usepackage{mathtools}
\usepackage{transparent}
\usepackage{xcolor}
\usepackage{tcolorbox}
\usepackage{amsmath}
\usepackage{amssymb}
\usepackage{parskip}
\usepackage[margin = 1in]{geometry}


% -------------------------------------------------- %
% -------------- CUSTOM ENVIRONMENTS --------------- %
% -------------------------------------------------- %
\newtcolorbox{note}{colback=black!5!white,
                          colframe=black!55!white,
                          fonttitle=\bfseries,title=Note}

\newtcolorbox{ex}{colback=blue!5!white,
                          colframe=blue!55!white,
                          fonttitle=\bfseries,title=Example}

\newtcolorbox{definition}{colback=red!5!white,
                          colframe=red!55!white,
                          fonttitle=\bfseries,title=Definition}


% -------------------------------------------------- %
% ------------------- COMMANDS --------------------- %
% -------------------------------------------------- %
% Brackets, braces, etc. 
\newcommand{\abs}[1]{\lvert #1 \rvert}
\newcommand{\bigabs}[1]{\Bigl \lvert #1 \Bigr \rvert}
\newcommand{\bigbracket}[1]{\Bigl [ #1 \Bigr ]}
\newcommand{\bigparen}[1]{\Bigl ( #1 \Bigr )}
\newcommand{\ceil}[1]{\lceil #1 \rceil}
\newcommand{\floor}[1]{\lfloor #1 \rfloor}
\newcommand{\norm}[1]{\| #1 \|}
\newcommand{\bignorm}[1]{\Bigl \| #1 \Bigr \| #1}
\newcommand{\inner}[1]{\langle #1 \rangle}
\newcommand{\set}[1]{{ #1 }}


% -------------------------------------------------- %
% -------------------- SETUP ----------------------- %
% -------------------------------------------------- %
\title{\Huge{Lecture 8 - Interaction Terms}}
\author{\large{Mitch Harrison}}
\date{\today}   
\begin{document}
\setlength{\parskip}{1\baselineskip}
\setlength{\parindent}{15pt}
\maketitle
\tableofcontents
\newpage


% -------------------------------------------------- %
% --------------------- BODY ----------------------- %
% -------------------------------------------------- %
\section{Review: Multiple Regression}

The \textbf{multiple linear regression} formula is as follows:
\[
    y_i =\beta_0 +\beta_1x_{i1} + \cdots +\beta_px_{ip} +\epsilon_i
\]
Where:
\begin{itemize}
    \item $p$ is the number of predictor/explanatory variables
    \item $y_i$ is the outcome (dependent variable) of interest
    \item $\beta_0$ is the intercept parameter
    \item $\beta_1, \cdots , \beta_p$ are the slope parameters
    \item $x_{i1}, \cdots , x_{ip}$ are predictor variables
    \item $\epsilon_i$ is the error (like the $\beta$'s, it is not observed.
\end{itemize}

\subsection{Vocabulary}
\begin{itemize}
    \item The \textbf{response variable} is the variable whose behavior or
        variation we are trying to understand
    \item \textbf{Explanatory variables} are other variables that we use to 
        explain the variation in the response
    \item The output of the \textbf{model function} is the \textbf{predicted
        value}.
    \item \textbf{Residuals} show how far each case is from its predicted value.
        \[
        \text{Residual} = \text{Observed value} - \text{Predicted value}
        \]
\end{itemize}

\pagebreak
\section{Interaction Effects}
\begin{definition}
    An \textbf{interaction effect} between a continuous predictor and a
    categorical predictor in a model allows for different slops for different
    levels of a categorical predictor.\\[.1in]
    To accomplish this, we add an \textbf{interaction variable} that is the 
    product of two explanatory variables.
\end{definition}

In words, we would say that "The relationship between the response variable and
a continuous variable depends on the value of a categorical variable."

To \textbf{build an interaction model}, we use the following code:
\begin{verbatim}
model <- linear_reg() |>
    set_engine("lm") |>
    fit(price ~ age + age * make, data = car_prices)
\end{verbatim}

\begin{ex}
    The fomula for a model comparing the price decrease per year between Porsche
    and Jaguar:
    \[
        \text{price}_i = \beta_0 + \beta_1( \text{age}_i) +
        \beta_2(I( \text{Porche})_i) + \beta_3( \text{age}_i \times I( \text{
        Porsche})_i) + \epsilon_i
    \]
    How do you \textbf{interpret the coefficients} in this model?\\[.1in]
    $\beta_0$ is the expected value of a new Jaguar. $\beta_1$ gives
    the expected rate of price decay per year of age for Jaguars.
    $\beta_2$ gives the change in expected value for a new car given that
    the car is a Porsche. $\beta_3$ gives the expected change in rate of
    decay in price per year of age in Porsche compared to Jaguar.
\end{ex}

\begin{note}
    If there are interaction terms in your model, it's often quite a bit trickier
    to interpret main effects. First, ask yourself whether the main effects are
    what you even care about!
\end{note}

An example \textbf{hypothesis test} using an interactive model might be something
like "The relationship between age and price does not depend on the make of a 
car." The effect of this on the model in the previous example would be that
$\beta_3 = 0$.

To check whether or not that hypothesis test is true, we check the p-value of the
interaction coefficient in our models summary. If it's less than our significance
level, we fail to reject the null hypothesis that the relationship between age
and make does not impact the predicted price.
\end{document}

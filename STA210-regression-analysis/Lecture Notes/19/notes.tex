\documentclass[titlepage, 12pt, leqno]{article}

% -------------------------------------------------- %
% -------------------- PACKAGES -------------------- %
% -------------------------------------------------- %
\usepackage{import}
\usepackage{pdfpages}
\usepackage{mathtools}
\usepackage{transparent}
\usepackage{xcolor}
\usepackage{tcolorbox}
\usepackage{amsmath}
\usepackage{amssymb}
\usepackage{parskip}
\usepackage{bbm}
\usepackage[margin = 1in]{geometry}
\tcbuselibrary{breakable}
\tcbset{breakable = true}


% -------------------------------------------------- %
% -------------- CUSTOM ENVIRONMENTS --------------- %
% -------------------------------------------------- %
\newtcolorbox{note}{colback=black!5!white,
                          colframe=black!55!white,
                          fonttitle=\bfseries,title=Note}

\newtcolorbox{ex}{colback=blue!5!white,
                          colframe=blue!55!white,
                          fonttitle=\bfseries,title=Example}

\newtcolorbox{definition}{colback=red!5!white,
                          colframe=red!55!white,
                          fonttitle=\bfseries,title=Definition}


% -------------------------------------------------- %
% ------------------- COMMANDS --------------------- %
% -------------------------------------------------- %
% Brackets, braces, etc. 
\newcommand{\abs}[1]{\lvert #1 \rvert}
\newcommand{\bigabs}[1]{\Bigl \lvert #1 \Bigr \rvert}
\newcommand{\bigbracket}[1]{\Bigl [ #1 \Bigr ]}
\newcommand{\bigparen}[1]{\Bigl ( #1 \Bigr )}
\newcommand{\ceil}[1]{\lceil #1 \rceil}
\newcommand{\floor}[1]{\lfloor #1 \rfloor}
\newcommand{\norm}[1]{\| #1 \|}
\newcommand{\bignorm}[1]{\Bigl \| #1 \Bigr \| #1}
\newcommand{\inner}[1]{\langle #1 \rangle}
\newcommand{\set}[1]{{ #1 }}


% -------------------------------------------------- %
% -------------------- SETUP ----------------------- %
% -------------------------------------------------- %
\title{\Huge{Lecture 19 - Missing Data}}
\author{\large{Mitch Harrison}}
\date{\today}   
\begin{document}
\setlength{\parskip}{1\baselineskip}
\setlength{\parindent}{15pt}
\maketitle
\tableofcontents
\newpage


% -------------------------------------------------- %
% --------------------- BODY ----------------------- %
% -------------------------------------------------- %
\section{Missing Data}
\subsection{Visualizing Missing Data}
The code for visualizing missing data in a table:
\begin{verbatim}
library(naniar)
vis_miss(data) # option 1
gg_miss_fct(x = data, fct = x1) # option 2, levels of single categorical var
\end{verbatim}

Other options from a different library:
\begin{verbatim}
library(UpSetR)
gg_miss_upset(nhanes2)
\end{verbatim}

\subsection{Terminology}
Suppose $Z = (Z_1, \cdots , Z_k)$ is the full data, which may be completely or
partially missing for some observations, and suppose $R = (R_1, \cdots , R_k)$ is
a vactor of indicators for whether each $Z_i$ is observed (1 if observed, 0 if
missing).

\begin{definition}
    \textbf{Missing Completely at Random (MCAR)}: 
    $ \mathbb{P}\left(R=r \;\middle|\; Z\right) $ does not depend on $Z$ (i.e.
    $R$ and $Z$ are independent).
\end{definition}

\begin{definition}
    \textbf{Missing at Random (MAR)}: 
    $ \mathbb{P}\left(R=r \;\middle|\; Z\right) $ only depends on elements of $Z$
    that are observed for $R=r$.
\end{definition}

\begin{definition}
    \textbf{Missing Not at Random (MNAR)}:
    $ \mathbb{P}\left(R=r \;\middle|\; Z\right) $ depends on elements of $Z$ that 
    are \textit{not} observed for $R=r$.
\end{definition}

\subsection{Examples of missingness mechanisms}
Suppose you are designing a written survey examining alcohol consumption among
students and your goal is to quantify alcohol consumption and examine potential
predictors.

Under MCAR, \textit{no systematic differences exist} between those with missing
data and those with complete data. Those with missing data are representative of
the entire population.

Suppose some survey responses are destroyed because the building that the 
responses were housed in burns down. In this case, whether a response is missing
or not is completely unrelated to the data at hand.

\begin{note}
    MCAR is a very strong assumption and is usually unrealistic unless the study 
    was deliberately designed to include missing data and account for it
    appropriately.
\end{note}

Under MAR, missing data are related to observed data, but not with the unobserved
data. Taking the same survey example, suppose statistics students recognize the
importance of having complete data and were more likely to complete the survey
compared to other departments. Also suppose whether someone completed the survey
was due solely to major, and that we fully observed everyone's major. In this
case, we would have MAR data.

Under MNAR, missing data are related to \textit{unobserved} data. For instance,
suppose instead that students who have higher alcohol consumption are less likely
to respond to the survey. In this case, whether a survey is missing depends 
directly on the value of that unobserved response itself: the data are missing
\textit{not} at random.

\begin{note}
    You cannot use your dataset to find whether your data or MAR vs MNAR. But if
    we make the MAR assumption, there \textit{are} methods to test MAR vs. MCAR.
    However, MAR is an \textit{unverifiable} assumption and must be justified in
    the context of each problem.
\end{note}

\subsection{Imputation}
\begin{definition}
    The idea behind \textbf{imputation} is to "fill in" missing values in the 
    data in a reasonable way.
\end{definition}

Often times for missing continuous variables, researchers may simply plug in the
mean or median of observed values for missing values. For categorical variables,
researchers may plug in the most common category.

\begin{note}
    Typically, imputation artificially deflates standard error, as many
    observations now contain the same value.
\end{note}

The previous approach is often termed \textit{unconditional} mean/median/mode
imputation, because it doesn't take into consideration the other covariates in
the dataset (which may be especially undesirable for MAR mechanisms, for example).

We might instead try to create a model for the missing values based on the
observed data and use predictions from these models as imputed values (e.g. a
linear model for missing continuous predictors). We may also try \textbf{hot
deck imputation}, which fills in missing data based on observed values from 
other observations which "match" in some sense.

To have an "ideal" imputation procedure, we may want to consider methods that:
\begin{itemize}
    \item provide consistent estimation of parameters of interest
    \item take into consideration extra variability due to imputation procedure
    \item allow for principles quantification of variability of estimated
        parameters
\end{itemize}
\end{document}

\documentclass[titlepage, 12pt, leqno]{article}

% -------------------------------------------------- %
% -------------------- PACKAGES -------------------- %
% -------------------------------------------------- %
\usepackage{import}
\usepackage{pdfpages}
\usepackage{transparent}
\usepackage{xcolor}
\usepackage{tcolorbox}
\usepackage{amsmath}
\usepackage{amssymb}
\usepackage{parskip}
\usepackage[margin = 1in]{geometry}


% -------------------------------------------------- %
% -------------- CUSTOM ENVIRONMENTS --------------- %
% -------------------------------------------------- %
\newtcolorbox{note}{colback=black!5!white,
                          colframe=black!55!white,
                          fonttitle=\bfseries,title=Note}

\newtcolorbox{ex}{colback=blue!4!white,
                          colframe=blue!55!white,
                          fonttitle=\bfseries,title=Example}

\newtcolorbox{definition}{colback=red!5!white,
                          colframe=red!55!white,
                          fonttitle=\bfseries,title=Definition}


% -------------------------------------------------- %
% ------------------- COMMANDS --------------------- %
% -------------------------------------------------- %
% Brackets, braces, etc. 
\newcommand{\abs}[1]{\lvert #1 \rvert}
\newcommand{\bigabs}[1]{\Bigl \lvert #1 \Bigr \rvert}
\newcommand{\bigbracket}[1]{\Bigl [ #1 \Bigr ]}
\newcommand{\bigparen}[1]{\Bigl ( #1 \Bigr )}
\newcommand{\ceil}[1]{\lceil #1 \rceil}
\newcommand{\floor}[1]{\lfloor #1 \rfloor}
\newcommand{\norm}[1]{\| #1 \|}
\newcommand{\bignorm}[1]{\Bigl \| #1 \Bigr \| #1}
\newcommand{\inner}[1]{\langle #1 \rangle}
\newcommand{\set}[1]{{ #1 }}


% -------------------------------------------------- %
% -------------------- SETUP ----------------------- %
% -------------------------------------------------- %
\title{\Huge{Lecture 1 - Intro and Syllabus Day}}
\author{\large{Mitch Harrison}}
\date{\today}   
\begin{document}
\setlength{\parskip}{1\baselineskip}
\setlength{\parindent}{15pt}
\maketitle
\tableofcontents
\newpage


% -------------------------------------------------- %
% --------------------- BODY ----------------------- %
% -------------------------------------------------- %
\section{Notes}

\subsection{Considerations for model construction}
\begin{itemize}
    \item Prediction vs. Explanation
    \item Association vs. Causation
    \item Data-driven vs. Theory-driven
    \item Prospective vs. Retrospective
\end{itemize}

\subsection{Why linear models?}
\begin{itemize}
    \item they are foundational for more "sophisticated" models
    \item they are transparent for the modeler and the audience
    \item they are translational (like Optical Character Recognition [OCR])
    \item they are useful
\end{itemize}

\begin{note}
The primary question of a linear model is "How does $X$ change $Y$?". Variability is natural for these relationships due to randomness. This is a \textbf{bivariate} relationship.
\end{note}

\subsection{Means and variances}
\textbf{Expectations} are averages:
\[
E(Y) = \sum_y \times p_Y(y)
\]
\textbf{Variances} are expected squared deviations around the mean.
\[
Var(Y) = E\big[(Y-E(Y))^2\big]
\]
\textbf{Conditional} means and variances ($Y$ given that $X = x$):
\[
E(Y|X = x) = \text{... some function}
\]
\[
Var(Y|X = x) = \text{... some other function}
\]
\begin{ex}
    One possible sample of conditional variance and expectations:
    \[
    E(Y|X = x) = \beta_0 + \beta_1x
    \]
    \[
    Var(Y|X = x) = \sigma^2
    \]
\end{ex}
\end{document}

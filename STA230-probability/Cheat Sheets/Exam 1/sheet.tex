\documentclass[titlepage, 12pt, leqno]{article}

% -------------------------------------------------- %
% -------------------- PACKAGES -------------------- %
% -------------------------------------------------- %
\usepackage{import}
\usepackage{multicol}
\usepackage{pdfpages}
\usepackage{mathtools}
\usepackage{transparent}
\usepackage{xcolor}
\usepackage{tcolorbox}
\usepackage{amsmath}
\usepackage{amssymb}
\usepackage{parskip}
\usepackage[margin = 0.5in]{geometry}


% -------------------------------------------------- %
% -------------- CUSTOM ENVIRONMENTS --------------- %
% -------------------------------------------------- %
\newtcolorbox{note}{colback=black!5!white,
                          colframe=black!55!white,
                          fonttitle=\bfseries,title=Note}

\newtcolorbox{ex}{colback=blue!5!white,
                          colframe=blue!55!white,
                          fonttitle=\bfseries,title=Example}

\newtcolorbox{definition}{colback=red!5!white,
                          colframe=red!55!white,
                          fonttitle=\bfseries,title=Definition}


% -------------------------------------------------- %
% ------------------- COMMANDS --------------------- %
% -------------------------------------------------- %
% Brackets, braces, etc. 
\newcommand{\abs}[1]{\lvert #1 \rvert}
\newcommand{\bigabs}[1]{\Bigl \lvert #1 \Bigr \rvert}
\newcommand{\bigbracket}[1]{\Bigl [ #1 \Bigr ]}
\newcommand{\bigparen}[1]{\Bigl ( #1 \Bigr )}
\newcommand{\ceil}[1]{\lceil #1 \rceil}
\newcommand{\floor}[1]{\lfloor #1 \rfloor}
\newcommand{\norm}[1]{\| #1 \|}
\newcommand{\bignorm}[1]{\Bigl \| #1 \Bigr \| #1}
\newcommand{\inner}[1]{\langle #1 \rangle}
\newcommand{\set}[1]{{ #1 }}


% -------------------------------------------------- %
% -------------------- SETUP ----------------------- %
% -------------------------------------------------- %
\begin{document}
\setlength{\parskip}{1\baselineskip}
\setlength{\parindent}{0pt}
\newpage


% -------------------------------------------------- %
% --------------------- BODY ----------------------- %
% -------------------------------------------------- %
\begin{multicols}{2}
\textbf{Bayes' Rule}
\begin{align*}
    \mathbb{P}\left(A \;\middle|\; B\right) &= \frac{ 
    \mathbb{P}\left(B \;\middle|\; A\right) \mathbb{P}(A)}{
\mathbb{P}\left(B \;\middle|\; A\right) \mathbb{P}(A) +
\mathbb{P}\left(B \;\middle|\; A^\complement \right) \mathbb{P}(A^\complement )}\\
&= \frac{ \mathbb{P}\left(B \;\middle|\; A\right) \mathbb{P}(A)}{\mathbb{P}(B)}
\end{align*}

\textbf{Binomial Distribution Formula }
\begin{align*}
    \mathbb{P}(k) &= \mathbb{P}( k\text{ success out of } n \text{ trials}) \\
                  &= \binom{n}{k}p^k(1-p)^{n-k}
\end{align*}

\textbf{Standard Deviation}
\begin{enumerate}
    \item For \textit{number of successes} $X$:
        \[
            \sqrt{np(1-p)}
        \]
    \item For \textit{fraction of successes} $\frac{X}{n}$:
        \[
        \frac{\sqrt{p(1-p)}}{\sqrt{n}}
        \]
\end{enumerate}

\textbf{Multinomial Distribution}

Let $n_i$ be the number of results of type $i$ out of the $n$ trials and $p_i$ be
the probability of that result:
\[
\mathbb{P}(N_1=n_1, \cdots , N_k=n_k) = \frac{n!}{n_1! \cdots n_k!}
p_1^{n_1} \cdots p_k^{n_k}
\]
For example, $n_1 =1$ if we are looking for an ace in a deck, and $n_2=2$ if
we are looking for two face cards in a hand of five cards:
\[
\mathbb{P}(N_1=1, N_2=2) = \frac{5!}{1!2!}\left(\frac{1}{13}\right)^1 
\left(\frac{4}{13}\right)^2 \left(\frac{8}{13}\right)^2
\]
And $\frac{8}{13}$ represents the "other" category, which is the odds of neither
ace nor face cards.

\textbf{Helpful Probability Calculations}
\begin{itemize}
    \item $\mathbb{P}(A^\complement \cap B^\complement) = 
        \mathbb{P}((A \cup B)^\complement) =
        1-\mathbb{P}(A \cup B)$
    \item $\mathbb{P}(A \cup B) = \mathbb{P}(A) + \mathbb{P}(B) -
        \mathbb{P}(A \cap B)$
    \item $ \mathbb{P}\left(A \;\middle|\; B\right) = 
        \frac{\mathbb{P}(A \cap B)}{\mathbb{P}(B)}$
\end{itemize}

\columnbreak

\textbf{Empirical Rule}
\begin{align*}
    \Phi(1) - \Phi(-1) &\approx 68\% \\
    \Phi(2) - \Phi(-2) &\approx 95\% \\
    \Phi(3) - \Phi(-3) &\approx 99.7\% \\
\end{align*}
\textbf{Phi Function Notes}
\begin{itemize}
    \item If you need to look up a negative number in the Phi table, use this
        trick: $\Phi(-z) = 1 - \Phi(z)$
    \item If $p$ is not given, we use the worst-case scenario $p=\frac{1}{2}$
    \item $\Phi(z) - \Phi(-z) = 2\Phi(z)-1$
\end{itemize}

\textbf{Multiplication Rule}
\begin{align*}
    \mathbb{P}(A \cap B) &= \mathbb{P}(B) \mathbb{P}\left(A \;\middle|\; B\right) 
    \\
    \mathbb{P}(A \cap B) &= \mathbb{P}(A) \mathbb{P}\left(B \;\middle|\; A\right) 
\end{align*}
\begin{note}
    If $A$ and $B$ are \textbf{independent}, then 
    \[
    \mathbb{P}\left(A \;\middle|\; B\right) = \mathbb{P}(A)\mathbb{P}(B)
    \]
\end{note}

\textbf{Rule of Average Conditional Probabilities}

This divides conditions into cases to find an overall probability.
\[
\mathbb{P}(A) = \mathbb{P}\left(A \;\middle|\; B\right) \mathbb{P}(B) +
\mathbb{P}\left(A \;\middle|\; C\right) \mathbb{P}(C) + \cdots 
\]
\textbf{Independence}
\begin{itemize}
    \item Two events are \textit{independent} when $ \mathbb{P}\left(A 
        \;\middle|\; B\right) = \mathbb{P}\left(A \;\middle|\; 
        B^\complement \right) = \mathbb{P}(A)$. This is the \textit{mathematical
        definition of independence.}
    \item Two random variables $X$ and $Y$ are \textit{independent} when
        $ \mathbb{P}\left(X=x \;\middle|\; Y=y\right) = \mathbb{P}(Y=y)$ or if 
        their joint distribution table has scalar-multiple rows and columns
    \item Multiple events are \textit{independent} if 

        $\mathbb{P}(A \cap B \cap C) = \mathbb{P}(A)\mathbb{P}(B)\mathbb{P}(C)$
\end{itemize}
\end{multicols}
\pagebreak
\begin{multicols}{2}
    \textbf{Binomial Confidence Intervals}
    \begin{itemize}
        \item We use \textit{binomial distribution confidence intervals} when 
            there are $n$ independent trials that have binary success/failure
            outcomes. Here, $k$ is the number of std. deviations from the mean:
            \[
                \mathbb{P}(p-c \le \frac{k}{n} \le p+c) = 2\Phi(k)-1
            \]
            And:
            \[
                c = k \cdot \left(\frac{\sqrt{p(1-p)}}{\sqrt{n}}\right)
            \]
        \item Given a fraction of successes $\hat p$, the \textit{approximate}
            probability that $a \le k \le b$ is:
            \[
                \Phi\left(\frac{b+\frac{1}{2}-np}{\sqrt{np(1-p)}}\right) -
                \Phi\left(\frac{a-\frac{1}{2}-np}{\sqrt{np(1-p)}}\right)
            \]
            \begin{note}
            The above can be used to find the probability of a single value. Set
            $a=b$ and use the same formula. This works because it includes
            the \textit{continuity correction}.
            \end{note}
            
        \item The \textit{approximate} probability that $a \le k$ is:
            \[
                \mathbb{P}(a \le k) = 1-\Phi\left(\frac{a-\frac{1}{2}-
                np}{\sqrt{np(1-p)}}\right)
            \]
        \item The probability that less than $x$\% of events will be successes to
            a $c$\% confidence is:
            \[
            \mathbb{P}(\hat p < x) = \Phi\left(\frac{x-p}{\sigma}\right) \ge c
            \]
            
    \end{itemize}

\textbf{Hypergeometric Distribution (for sampling \textit{without} replacement}
\[
    \mathbb{P}(k \text{ good items out of }n \text{ samples}) =
    \frac{\binom{G}{k}\binom{B}{n-k}}{\binom{N}{n}}
\]
Where $G$ is the number of "good" items in the sample, $N$ is the total sample
size, $B$ is the number of "bad" items, $n$ is the amount taken from the sample
(five cards out of an $N=52$ card deck for example), and $k$ is the number of
"good" samples out of $n$.

\columnbreak
\textbf{Relevant Integrals}
\begin{itemize}
    \item \textit{Integral of the standard normal bell curve}
        \[
            \Phi(z) = \int_{-\infty}^{z}\frac{1}{\sqrt{2\pi }}e^{-x^2/2}
        \]
    \item \textit{Phi functions for non-standard normal distributions}
       \begin{align*}
            &\int_{a}^{b}\frac{1}{\sigma\sqrt{2\pi }}e^{-(x-\mu)^2/2\sigma^2}\\
            &=\Phi\left(\frac{b-\mu}{\sigma}\right) -
            \Phi\left(\frac{a-\mu}{\sigma}\right) \\
            &= \int_{\frac{a-\mu}{\sigma}}^{\frac{b-\mu}{\sigma}}
            \frac{1}{\sqrt{2\pi}}e^{-\frac{x^2}{2}}
       \end{align*}
       Here, $\mu$ is the \textit{mean} of the distribution.
\end{itemize}

\textbf{Misc. Facts}
\begin{itemize}
    \item The \textbf{mode of the binomial distribution} is the greatest integer
        less than or equal to $np+p$. If this is an integer $m$, there are two
        modes, $m$ and $m-1$
    \item \textit{Mutually exclusive events} are not necessarily independent. If
        $A$ is rolling a 3, and $B$ is rolling a two, for example.
    \item For testing uniqueness over multiple occurences \textit{without
        replacement}, consider making a tree similar to the birthday problem.
        Recall in those cases:
        \[
            \mathbb{P}( \text{any non-unique}) = 1 - \mathbb{P}( \text{all 
            unique})
        \]
    \item The \textit{continuity correction} should only be used when dealing wit
        integer values.
\end{itemize}
\end{multicols}

\end{document}

\documentclass[titlepage, 12pt, leqno]{article}

% -------------------------------------------------- %
% -------------------- PACKAGES -------------------- %
% -------------------------------------------------- %
\usepackage{import}
\usepackage{pdfpages}
\usepackage{mathtools}
\usepackage{transparent}
\usepackage{xcolor}
\usepackage{tcolorbox}
\usepackage{amsmath}
\usepackage{amssymb}
\usepackage{parskip}
\usepackage{bbm}
\usepackage[margin = 1in]{geometry}
\tcbuselibrary{breakable}
\tcbset{breakable = true}


% -------------------------------------------------- %
% -------------- CUSTOM ENVIRONMENTS --------------- %
% -------------------------------------------------- %
\newtcolorbox{note}{colback=black!5!white,
                          colframe=black!55!white,
                          fonttitle=\bfseries,title=Note}

\newtcolorbox{ex}{colback=blue!5!white,
                          colframe=blue!55!white,
                          fonttitle=\bfseries,title=Example}

\newtcolorbox{definition}{colback=red!5!white,
                          colframe=red!55!white,
                          fonttitle=\bfseries,title=Definition}


% -------------------------------------------------- %
% ------------------- COMMANDS --------------------- %
% -------------------------------------------------- %
% Brackets, braces, etc. 
\newcommand{\abs}[1]{\lvert #1 \rvert}
\newcommand{\bigabs}[1]{\Bigl \lvert #1 \Bigr \rvert}
\newcommand{\bigbracket}[1]{\Bigl [ #1 \Bigr ]}
\newcommand{\bigparen}[1]{\Bigl ( #1 \Bigr )}
\newcommand{\ceil}[1]{\lceil #1 \rceil}
\newcommand{\floor}[1]{\lfloor #1 \rfloor}
\newcommand{\norm}[1]{\| #1 \|}
\newcommand{\bignorm}[1]{\Bigl \| #1 \Bigr \| #1}
\newcommand{\inner}[1]{\langle #1 \rangle}
\newcommand{\set}[1]{{ #1 }}


% -------------------------------------------------- %
% -------------------- SETUP ----------------------- %
% -------------------------------------------------- %
\title{\Huge{Lecture 17 - Joint Distributions Pt. 1}}
\author{\large{Mitch Harrison}}
\date{\today}   
\begin{document}
\setlength{\parskip}{1\baselineskip}
\setlength{\parindent}{15pt}
\maketitle
\tableofcontents
\newpage


% -------------------------------------------------- %
% --------------------- BODY ----------------------- %
% -------------------------------------------------- %
\section{Review}

\begin{ex}
    Suppose $X \sim Unif(2,6)$ and $Y = g(X) = 10X+5$. \textbf{What is the
    range of} $Y$?
    \vspace{10px}
    
    Plugging in the minimum and maximum values for $X$ (2 and 6 respsectively),
    we arrive at the range of possible values of $Y$, which is $g(2)$ and $g(6)$.
    Thus:
    \[
        \boxed{range(Y) = [2,6]} 
    \]
\end{ex}

\pagebreak
\section{Max/Min Type Problems with Continuous Random Variables}
\begin{ex}
    You have a string of 5 light bulds, hooked up in series. The lifespan of each 
    individual bulb is exponentially distributied with an average lifespan of 10
    months. Each bulb's lifespan is independent of the others. Let $X$ be the time
    it takes until the string goes dark. \textbf{Find the CDF, PDF, and Expected
    value of $X$}.
    \vspace{10px}
    
    We are given that $\lambda = \frac{1}{10}$ because we lose 1 bulb (which we 
    define as an "arrival") per 10 months. The string goes dark when the 
    \textit{first} bulb goes dark because they are connected in series. Thus, we
    are looking for $min(X_1, \cdots ,X_5)$. The CDF is $\mathbb{P}(X\le t)$:
    \begin{align*}
       \mathbb{P}(X\le t) &= \mathbb{P}(min(X_1, \cdots ,X_5) \le t) \\
                          &= 1 - \mathbb{P}(min(X_1, \cdots ,X_5) > t) \\
                          &= 1 - \left(e^{-\lambda t} \right) \\
       \Aboxed{\mathbb{P}(X \le t) &= 1 - \left(e^{-t/10}\right)^5} 
    \end{align*}
    \vspace{10px}
    
    Once we have the CDF, we can differentiate to find the PDF:
    \begin{align*}
       \frac{d}{dt}\left(F(t)\right) = \frac{1}{2}e^{-t/2}
    \end{align*}
    We see that the PDF is an exponential distribution where $X \sim Exp\left(
    \frac{1}{2}\right)$.

    Now that we have the PDF, we can find the expected value:
    \begin{align*}
        \mathbb{E}(X) &= \int_{0}^{\infty}x \cdot PDF \\
                      &= \frac{1}{\lambda} \\
                      &= \frac{1}{\frac{1}{2}} \\
        \Aboxed{ \mathbb{E}(X) &= 2} 
    \end{align*}
\end{ex}

\subsection{General expressions}
Let $X_1, \cdots , X_n$ be independent (but not necessarily independently 
distributed) random variables with CDF of $X_i$ equal to $F_i(t)$. Recall that
$F_i(t) = \mathbb{P}(X_i \le t)$. 
\begin{ex}
    Let $Y$ be $min(X_1, \cdots , X_n)$. \textbf{Find the CDF} of $Y$.
   \begin{align*}
       \mathbb{P}(Y\le t) &= 1 - \mathbb{P}( \text{all $X_i>t$}) \\
                          &= 1 - \mathbb{P}(X_1>t) \cdots \mathbb{P}(X_n > t) \\
       \Aboxed{\mathbb{P}(Y \le t)&= 1 - \prod_{i=1}^n\left(1-F_i(t)\right)}
   \end{align*}
\end{ex}

\begin{ex}
    Let $Z$ be $max(X_1, \cdots , X_n)$. \textbf{Find the CDF of $Z$}.
   \begin{align*}
       \mathbb{P}(Z \le t) &= \mathbb{P}( \text{all $X_i \le t$}) \\
       \Aboxed{\mathbb{P}(Z \le t) &= \prod_{i=1}^{n}F_i(t)}
   \end{align*}
\end{ex}

\pagebreak
\section{Joint Distributions for Continuous Random Variables}
What is we have two related random variables? Maybe we measure the heights of a
parent and child (on average, these are related random variables). We could ask a
question like "what is the probability that both a parent and child are three
standard deviations above the mean height?"
\begin{definition}
    A \textbf{joint density function} $f(x,y)$ for two random variables $X$ and
    $Y$ gives a probability per unit area near the point $(x,y)$. It's still a
    density function, but this time \textit{volumes} correspond to probabilities.
\end{definition}

For a range of $X$ values $(a,b)$ and range of $Y$ values $(c,d)$ on the $X$ and
$Y$ axes, the volume above a rectangle in the $XY$-plane with those bounds is:
\[
    \mathbb{P}(a<X<b \text{ and } c<Y<d)
\]
For brevity, we use the following notation:
\[
    \mathbb{P}(X \in \Delta x, Y \in \Delta y)
\]
For a bivariate joint distribution, 
\[
    \boxed{f(x,y) = \frac{\mathbb{P}(X \in \Delta x, Y \in \Delta y)}{\Delta X 
            \Delta Y}} 
\]
\subsection{Uniform densities in $\mathbb{R}^2$ and $\mathbb{R}^3$}
For uniform densities, areas (or volumes) corespond to probabilities. If we can 
find the area (or volume), we can get the probability.
\vspace{10px}

If a random point $(X,Y)$ has uniform distribution over a region $R$, then
\[
    \mathbb{P}((X,Y) \in C) = \frac{area(C)}{area(D)}
\]
If $X$ and $Y$ are independent random variables, each uniformly distributed on an
interval, then $(X,Y)$ is uniformly distributed on a rectangle.
\vspace{10px}

If $(X,Y)$ are uniform over some reigon $R$, then:
\[
    f(x,y) = 
    \begin{cases}
        \frac{1}{area(R)} & (x,y) \in R\\
        0 & else
    \end{cases}
\]
\begin{note}
    If two uniformly distributed random variables are independent, then they are
    jointly uniform. For example, if $X \sim unif(0,1)$ and $Y \sim unif(3,7)$, 
    then:
    \[
        f(x,y) = 
        \begin{cases}
            \frac{1}{4} & 0<x<1; 3<y<7\\
            0 & \text{else}
        \end{cases}
    \]
\end{note}
\subsection{Method of infinitesimals}
For a uniform distribution, we can use the idea that area proportions correspond
to probabilities.
\begin{ex}
    Consider a right triangle with vertices at (0,0), (0,1), and (1,0). Let
    $X$ be the $x$-coordinate of a point picked uniformly at random (i.e. all
    points are equally likely). \textbf{What is the PDF of $X$}? Additionally,
    \textbf{what is $ \mathbb{E}(X)$}?
    \vspace{10px}
    
    We know, since the $(X,Y)$ is jointly uniform, that the CDF of is:
    \[
    f(x,y) = 
    \begin{cases}
        2 & (x,y) \text{ in triangle} \\
        0 & else
    \end{cases}
    \]
    \textbf{Method 1 (using CDF)}: We know that the CDF is $\mathbb{P}(X \le t)$.
    Therefore, the ratio of the area of the trapezoid formed within the triangle
    by the vertical line $X=t$ and the whole triangle is the total probability.
    to simplify the arithmetic, we can use the complement rule:
    \begin{align*}
        \mathbb{P}(X \le t) &= 1 - \mathbb{P}(X > t) \\
                            &= 1 = \frac{ \text{area of triangle to the 
                        right of $t$}}{ \text{area of the whole triangle}} \\
                            &= 1 - \frac{\frac{(1-t)^2}{2}}{\frac{1}{2}} \\
        \Aboxed{&= 1 - (1-t)^2} 
    \end{align*}
    Once we have the CDF, we differentiate to arrive at the PDF $f(x)$:
    \begin{align*}
        f(x) &= \frac{d}{dt} = -2(1-t)(-1) = 2(1-t) \\
        f(x) &=
        \begin{cases}
            2(1-x) & 0<x<1 \\
            0 & \text{else}
        \end{cases}
    \end{align*}
    \textbf{Method 2 (method of infinitesimals)}: We will use the following
    formula:
    \[
        \boxed{ f(x) = \lim_{\Delta x \to \infty} \frac{\mathbb{P}(X \in 
        \Delta x)}{\Delta x}}
    \]
    First, we handle the numerator of that function. \textit{Note:} we approximate
    the area of a trapezoid over a small $x$ interval using a rectangle.
    \begin{align*}
        \mathbb{P}(X \in  \Delta x) &= \frac{ \text{area of 
        trapezoid formed on interval $x \to \Delta x$}}{ \text{area of whole
        triangle}} \\
                                    &\approx \frac{ \text{area of rectangle
                                    formed on that interval}}{ \text{area of the
                                    whole triangle}} \\
                                    &= \frac{(1-x)\Delta x}{\frac{1}{2}} \\
                \Aboxed{&= 2(1-x)\Delta x} 
    \end{align*}
    Now that we've handled the numerator, we complete the function:
    \begin{align*}
        f_X(x) &= \lim_{\Delta x \to \infty} \frac{2(1-x)\Delta x}{\Delta x} \\
               &= 2(1-x)
    \end{align*}
    Therefore, we arrive at the solution:
    \[
        \boxed{f_X(x) = 
        \begin{cases}
            2(1-x) & 0<x<1 \\
            0 & \text{else}
        \end{cases}
        } 
    \]
\end{ex}

\end{document}

\documentclass[titlepage, 12pt, leqno]{article}

% -------------------------------------------------- %
% -------------------- PACKAGES -------------------- %
% -------------------------------------------------- %
\usepackage{import}
\usepackage{mathtools}
\usepackage{transparent}
\usepackage{xcolor}
\usepackage{tcolorbox}
\usepackage{amsmath}
\usepackage{amssymb}
\usepackage{parskip}
\usepackage[margin = 1in]{geometry}


% -------------------------------------------------- %
% -------------- CUSTOM ENVIRONMENTS --------------- %
% -------------------------------------------------- %
\newtcolorbox{note}{colback=black!5!white,
                          colframe=black!55!white,
                          fonttitle=\bfseries,title=Note}

\newtcolorbox{ex}{colback=blue!5!white,
                          colframe=blue!55!white,
                          fonttitle=\bfseries,title=Example}

\newtcolorbox{definition}{colback=red!5!white,
                          colframe=red!55!white,
                          fonttitle=\bfseries,title=Definition}


% -------------------------------------------------- %
% ------------------- COMMANDS --------------------- %
% -------------------------------------------------- %
% Brackets, braces, etc. 
\newcommand{\abs}[1]{\lvert #1 \rvert}
\newcommand{\bigabs}[1]{\Bigl \lvert #1 \Bigr \rvert}
\newcommand{\bigbracket}[1]{\Bigl [ #1 \Bigr ]}
\newcommand{\bigparen}[1]{\Bigl ( #1 \Bigr )}
\newcommand{\ceil}[1]{\lceil #1 \rceil}
\newcommand{\floor}[1]{\lfloor #1 \rfloor}
\newcommand{\norm}[1]{\| #1 \|}
\newcommand{\bignorm}[1]{\Bigl \| #1 \Bigr \| #1}
\newcommand{\inner}[1]{\langle #1 \rangle}
\newcommand{\set}[1]{{ #1 }}


% -------------------------------------------------- %
% -------------------- SETUP ----------------------- %
% -------------------------------------------------- %
\title{\Huge{Lecture 10 - Expectation}}
\author{\large{Mitch Harrison}}
\date{\today}   
\begin{document}
\setlength{\parskip}{1\baselineskip}
\setlength{\parindent}{15pt}
\maketitle
\tableofcontents
\newpage


% -------------------------------------------------- %
% --------------------- BODY ----------------------- %
% -------------------------------------------------- %
\section{Review}
\textbf{Empirical Rule:}

\begin{enumerate}
    \item Given $p$: $\mathbb{P}(p-c \le \hat p \le p+c) = x$, $x$ is the 
        desired confidence interval (i.e. 0.95), and helps us get the number of
        standard deviations. $c$ is the number of standard deviations and:
        \[
            c = ( \text{\# std. dev's})\frac{\sqrt{p(1-p)}}{\sqrt{n}}
        \]
    \item If we don't know $p$, we use $\hat p$ to get an estimate of $p$. The
        confidence interval will end up being $(\hat p-c, \hat p+c)$, where:
        \[
        c = ( \text{\# std. dev's}) \times ( \text{size of std. dev})
        \]
        And the size of the standard deviations is $\frac{\sqrt{p(1-
        p)}}{\sqrt{n}}$. Sinc we don't know $p$, we operate under the worst-
        case scenario of $p=\frac{1}{2}$. Thus, the biggest possible standard
        deviation is $\frac{1}{2\sqrt{n}}$.
\end{enumerate}

\textbf{Question:} When do we define the standard deviation as $\frac{\sqrt{p(1-p)}}{\sqrt{n}}$ vs. $\sqrt{np(1-p)}$?

When we are looking at the total number of successes (i.e. $X =$ \# of heads
and we are looking for $\mathbb{P}(X\ge 7)$), we use $\sqrt{np(1-p)}$. But if we
are looking for the fraction of successes $X$ out of $n$ trials, we get the 
fraction (or proportion) of successes $\frac{X}{n}$, and we use the standard
deviation form $\frac{\sqrt{p(1-p)}}{\sqrt{n}}$. Typically, the former case
involves using the continuity correction.

\begin{ex}
    Three cards are drawn without replacement from a standard deck. What is the
    probability that the minimum of three cards is less than or equal to 10?
    \\[.1in]
    Here,
   \begin{align*}
       N &= 52 \\
       G &= 12 \text{(the number of face cards)}\\
       n &= 3 \\
       k &= 3
   \end{align*}
   Thus,
    \[
        \boxed{1 - \frac{\binom{40}{0}\binom{12}{3}}{\binom{52}{3}}} 
    \]
\end{ex}

\pagebreak
\section{Consequences of Independence}
\begin{enumerate}
    \item Any subgroup of independent variables are also independent.
    \item Functions of independent random variables are also independent. For
        example, if $X_1$ and $X_2$ are independent, then $2X_1$ and $X_2 + 3$ 
        are also independent.
    \item Functions of disjoint collections of independent random variables are
        also independent. For example, if $X_1, X_2, X_3, X_4$ are independent,
        then $min(X_1, X_2)$ and $max(X_3, X_4)$ are also independent.
\end{enumerate}

\pagebreak
\section{Introduction to expected value}
Suppose a class ahs the following twenty quiz grades:
\begin{center}
\begin{tabular}{c|c}
    Grade out of 6 & Number of quizzes \\
    \hline
    2&1 \\
    3&3 \\
    4&6 \\
    5&6 \\
    6&4
\end{tabular}
\end{center}

How can we compute the average?
\begin{align*}
    \text{total} &= \sum_{ \text{all grades }}( \text{grade})
    \frac{ \text{\# quizzes of that grade}}{ \text{total \# of quizzes}}
\end{align*}
This is effectively a simple \textbf{weighted average}.

\begin{definition}
    The \textbf{expected value} of a random variable $X$ is the "weighted 
    average." If $X$ has probability mass function $p(x)$, then:
    \[
    \mathbb{E}(X) = \sum_{ \text{all possible }x} \cdot p(x) = 
    \sum_{x}x \mathbb{P}(X=x)
    \]
\end{definition}

\begin{ex}
    $X$ has the following mass function. What is the expected value of $X$?
    \begin{center}
    \begin{tabular}{c|c c c}
        $x$&-2&3&5\\
        \hline
        $\mathbb{P}(X=x)$&1/2&1/3&1/6
    \end{tabular}
    \end{center}

    We use the expected value function to arrive at a solution:
   \begin{align*}
       \mathbb{E}(X) &= -2\left(\frac{1}{2}\right) + 3\left(\frac{1}{3}\right) +
       5\left(\frac{1}{6}\right) \\
       \Aboxed{ \mathbb{E}(X) &= \frac{5}{6}} 
   \end{align*}
\end{ex}

\pagebreak
\subsection{Properties of Expected Value}
Let $X$ be a (discrete) random variable.
\begin{enumerate}
    \item For any constant $k$:
        \[
        \mathbb{E}(k) = k
        \]
    \item If $ \mathbb{E}(X)$ is finite, then for any constant $k$:
        \[
        \mathbb{E}(kX) = k \mathbb{E}(X)
        \]
    \item The $k$th \textbf{moment} of $X$ is:
        \[
        \mathbb{E}(X^k) = \sum_{ \text{all }x}x^k \mathbb{P}(X=x)
        \]
    \item For a function $f$ defined on the range of $X$:
        \[
        \mathbb{E}(f(X)) = \sum_{ \text{all }x}f(x)\mathbb{P}(X=x)
        \]
    \item If $X$ and $Y$ are independent random variables:
        \[
        \mathbb{E}(XY) = \sum_{ \text{all }x,y}(xy)\mathbb{P}(X=x, Y=y)
        \]
    \item \textbf{Addition rule}
        \[
        \mathbb{E}(X+Y) = \mathbb{E}(X) + \mathbb{E}(Y)
        = \sum_{ \text{all }x,y}(x+y)\mathbb{P}(X=x, Y=y)
        \]
        \begin{note}
            The \textit{addition rule} holds whether $X$ and $Y$ are 
            \textit{dependent} or \textit{independent}.
        \end{note}
        
    \item \textbf{Tail sum formula}: For $X$ with possible range values
        $\{0,1,2,3, \cdots ,n\}$:
        \[
        \mathbb{E}(X) = \sum_{k=1}^{n}\mathbb{P}(X \ge k)
        \]
        \begin{note}
            The \textit{tail sum formula} holds for any set of non-negative
            integers. It's called a "tail sum" because it starts at $k$ and adds
            up the distribution from $k$ to the tip of the tail of the
            distribution graph $n$.
        \end{note}
\end{enumerate}

\pagebreak
\section{Indicator Functions}

\begin{ex}
    We deal five cards from a deck of 52 without replacement. Let $X$ denote the
    number of aces among the chosen cards. Find the expected value of $X$.
    \\[.1in]
    Here, we are looking for $ \mathbb{E}(X)$ where $X$ is the number of aces.
   \begin{align*}
       \mathbb{E}(X) &= \sum_{x=0}^{4}x \mathbb{P}(X=x) \\
                     &= 0\mathbb{P}(X=0)+1\left(\frac{\binom{4}{1}
                         \binom{48}{4}}{\binom{52}{4}}\right) +
                         2\mathbb{P}(X=2) + 3\mathbb{P}(X=3) + 
                         4\mathbb{P}(X=4)\\
       \Aboxed{ \mathbb{E}(X) &= \frac{5}{13}} 
   \end{align*}
\end{ex}

\begin{definition}
    The \textbf{indicator function} of an event $A$ is the random variable which
    has the range $\{0,1\}$ such that:
    \[
        1_A(s) =
    \begin{cases}
        1 & \text{if } s \in A \\
        0 & \text{if }s \notin A
    \end{cases}
    \]
    Basically:
    \[
        1_A =
    \begin{cases}
        1 & \text{if A happened} \\
        0 & \text{if not}
    \end{cases}
    \]
\end{definition}

\begin{ex}
    Let $A$ be an event inside of the outcome space $\Omega $. What is
    $ \mathbb{E}(1_A)$?
   \begin{align*}
       \mathbb{E}(1_A) &= 1\mathbb{P}(A) + 0\mathbb{P}(A^\complement)\\
        \Aboxed{ \mathbb{E}(1_A) = \mathbb{P}(A)} 
   \end{align*}
\end{ex}

\begin{ex}
    We're going to redo the problem using the method of indicators. \textbf{Which 
    indicator functions} will be useful in this case?
    \begin{enumerate}
        \item $1_n = $ the indicator on the event that the $n$th card is an ace
        \item $1_n =$ the indicator on the event that $n$ aces are among the
            five chosen cards
        \item $1_n =$ the indicator on the event that greater than or equal to
            $n$ aces are among the five chosen cards
    \end{enumerate}
    \vspace{10px}
    $\boxed{ \text{Answer 1}}$ is correct. If $X$ is the number of aces, then
    summing the indicator functions in Answer 1 will give $X$. By the
    \textit{addition rule}:
   \begin{align*}
       \mathbb{E}(X) = \mathbb{E}(1_1) + \mathbb{E}(1_2) + \cdots + 
       \mathbb{E}(1_5)
   \end{align*}
\end{ex}

\pagebreak
\subsection{Indicator Example}
There are 15 workers in the office. Management allows a birthday party on the 
last day of a month if at least one person had a birthday during that month. How
many parties are there on average during a year? Assume for simplicity that the 
birth months of the individuals are independent and each month is equally likely.

\begin{ex}
    For this question, we're going to use the method of indicators. Which 
    indicator functions seem useful for this problem?
    \begin{enumerate}
        \item $1_n =$ the indicator on the event that at least 1 birthday is in 
            the $n$th month.
        \item $1_n = $ the number of people with a birthday in the $n$th month.
        \item $1_n = $ the indicator on the event that $n$ months have parties
            out of the 12 possible months.
        \item $1_n = $ the indicator on the event that person $n$ does not share
            a birthday month with any other person in the office.
    \end{enumerate}
\end{ex}

\subsection{Expected value of binomial distribution}
\textbf{FILL HERE}

\subsection{Tail Sum Formula}
For $X$ with possible values $\{0,1,2, \cdots ,n\}$:
\[
    \mathbb{E}(X) = \sum_{j=1}^{n}\mathbb{P}(X \ge j)
\]

This is equivalent to $\sum_{x=0}^{n}x \mathbb{P}(X=x)$, just written in a 
different way.

\begin{ex}
    Suppose 3 fair six-sided dice are rolled. Let $Y$ be the minimum of three 
    rolls. \textbf{Compute} $ \mathbb{E}(Y)$.\\[.1in]
    Let's start with the sub-goal of finding an expression for $ \mathbb{P}(Y
    \ge k)$. \textbf{Compute} $ \mathbb{P}(Y \ge 3)$.
\end{ex}

\end{document}

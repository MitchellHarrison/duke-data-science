\documentclass[titlepage, 12pt, leqno]{article}

% -------------------------------------------------- %
% -------------------- PACKAGES -------------------- %
% -------------------------------------------------- %
\usepackage{import}
\usepackage{pdfpages}
\usepackage{mathtools}
\usepackage{transparent}
\usepackage{xcolor}
\usepackage{tcolorbox}
\usepackage{amsmath}
\usepackage{amssymb}
\usepackage{parskip}
\usepackage[margin = 1in]{geometry}


% -------------------------------------------------- %
% -------------- CUSTOM ENVIRONMENTS --------------- %
% -------------------------------------------------- %
\newtcolorbox{note}{colback=black!5!white,
                          colframe=black!55!white,
                          fonttitle=\bfseries,title=Note}

\newtcolorbox{ex}{colback=blue!5!white,
                          colframe=blue!55!white,
                          fonttitle=\bfseries,title=Example}

\newtcolorbox{definition}{colback=red!5!white,
                          colframe=red!55!white,
                          fonttitle=\bfseries,title=Definition}


% -------------------------------------------------- %
% ------------------- COMMANDS --------------------- %
% -------------------------------------------------- %
% Brackets, braces, etc. 
\newcommand{\abs}[1]{\lvert #1 \rvert}
\newcommand{\bigabs}[1]{\Bigl \lvert #1 \Bigr \rvert}
\newcommand{\bigbracket}[1]{\Bigl [ #1 \Bigr ]}
\newcommand{\bigparen}[1]{\Bigl ( #1 \Bigr )}
\newcommand{\ceil}[1]{\lceil #1 \rceil}
\newcommand{\floor}[1]{\lfloor #1 \rfloor}
\newcommand{\norm}[1]{\| #1 \|}
\newcommand{\bignorm}[1]{\Bigl \| #1 \Bigr \| #1}
\newcommand{\inner}[1]{\langle #1 \rangle}
\newcommand{\set}[1]{{ #1 }}


% -------------------------------------------------- %
% -------------------- SETUP ----------------------- %
% -------------------------------------------------- %
\title{\Huge{Lecture 2 - Foundations}}
\author{\large{Mitch Harrison}}
\date{\today}   
\begin{document}
\setlength{\parskip}{1\baselineskip}
\setlength{\parindent}{15pt}
\maketitle
\tableofcontents
\newpage


% -------------------------------------------------- %
% --------------------- BODY ----------------------- %
% -------------------------------------------------- %
\section{Review from Previous Lecture}

\begin{ex}
    Two fair dice are rolled. $A$ is the event that both are 3's and $B$ is the event that one roll is a 5 and the other is a 6. \textbf{Which is more likely}, $A$ or $B$?
   \begin{align*}
       A &= \{(3,3)\} \\
       B &= \{(5,6), (6,5)\}
   \end{align*}
    Since $B$ has more elements, it is more likely than $A$, therefore:
    \[
    \boxed{\mathbb{P}(B) > \mathbb{P}(A)} 
    \]
\end{ex}

\subsection{Useful Formulas/Rules for Calculations}
\begin{itemize}
    \item $P(\Omega) = 1$ for an outcome space $\Omega$ 
    \item $0 \le \mathbb{P}(B) \le 1$ for any event $B$ 
    \item If $B_1 \cap B_2 = \phi $, then $\mathbb{P}(B_1 \cup  B_2) = \mathbb{P}(B_1) + \mathbb{P}(B_2)$ 
    \item $\mathbb{P}(A^\complement) = 1 - \mathbb{P}(A)$ and vice versa.
    \item $\mathbb{P}(A \cup B) = \mathbb{P}(A) + \mathbb{P}(B) - \mathbb{P}(A \cap B)$ 
\end{itemize}

\begin{ex}
    You roll three fair dice. \textbf{What is the probability that at least one roll is even?} 
   \begin{align*}
       \mathbb{P}(\text{At least 1 even}) &= 1 - \mathbb{P}(\text{No even rolls)} \\
        &= 1 - \frac{1}{8} 
        \Aboxed{\mathbb{P}(\text{At least 1 even}) = \frac{7}{8}}
   \end{align*}

\end{ex}
\pagebreak

\section{Permutations and Combinations}
\subsection{Permutations}
\begin{definition}
    \textbf{Permutations} are the number of possible ways to order $n$ different things, as is $n!$ For example, there are $3! = 6$ possible ways to order 3 elements.
\end{definition}

\subsection{Combinations}
\begin{definition}
    \textbf{Combinations} are the number of ways to choose $k$ items out of a collections of $n$ possible items. The notation for this is as follows:
    \[
        \binom{n}{k}
    \]
\end{definition}
To calculate the \textbf{binomial coefficient} $\binom{n}{k}$, the following formula is used:
\[
    \frac{n!}{(n-k)! k!}
\]
\begin{ex}
    \textbf{How many ways can we arrange 5 red and 3 blue chairs?} 
    \[
      \binom{8}{3} \quad \text{or} \quad \binom{8}{5} = \frac{8!}{5!3!} = \boxed{56} 
    \]
\end{ex}
\pagebreak
\section{Distributions}
\begin{ex}
    Teams play a dice rolling game:
    \begin{itemize}
        \item Each team has 36 tokens they can distribute in bins labelled 2 through 12
        \item Once all tokens are distributed, the host rolls a pair of dice and announces the sum.
        \item Teams may remove one token from the bin labelled with that sum.
        \item The host continues reporting the sums until a team runs out of tokens, and that team is the winner.

        \textit{(a)} \textbf{What is the most likely sum?} 
        \[
        \boxed{7} 
        \]
        \textit{(b)} \textbf{What is the best way to distribute the tokens?} 
        \[
        \boxed{\text{A bell curve centered on 7}}
        \]
    \end{itemize}
\end{ex}

The distribution of possible sums on a 2-column table with one column being a possible sum of two dice and the second being the probability of that sum. This is a \textbf{Probability Mass Function} 

\textbf{Heuristic Idea}: 
The probability \textbf{distribution} describes the probability of all outcomes in the outcome space.

\begin{definition}
    A probability \textbf{distribution} is a function, whose input is an event and the output is the liklihood of that event.
    \[
    F(outcome)
    \]
\end{definition}

The previous 2-column \textbf{probability mass function} is best used for small numbers of possible outcomes.

A \textbf{histogram} (discrete) or \textbf{normal/bell curve} (continuous) of probabilities of possible sums is best for larger number of possible outcome spaces.

A \textbf{function} is best used  when the number of experiments is potentially infinite. For example, flipping a coin until it lands on heads.

\subsection{Mathematical Definition of Distribution}
\begin{definition}
    A \textbf{partition} is a way of dividing up outcomes. The partition of some set $B$, written as $(B_1,\dots,B_n)$, is a list of subsets of an outcome space $\Omega$ such that:
    \[
    B_i \cap B_j = \phi \quad \text{and}
    \]
    \[
    \bigcup_{i=1}^{n}B_i = B
    \]
   In other words, together the $B_i$'s make up all of $B$, but no two $B_i$'s overlap with each other.
\end{definition}

\begin{ex}
    For a partition of $B$, $(B_1, \dots, B_n$, what is $\mathbb{P}(B_1 \cup \dots \cup B_n$?
\[
    \boxed{\sum_{i=1}^{n}B_i} 
\]
\end{ex}

The \textbf{probability distribution} is a function. The input into the function is any event (subset) of the outcome space. The output of the function is the probability of that ecent. Any function, $P$, can be a distribution over $\Omega$ as long as it satisfies three properties:
\begin{itemize}
    \item $P(B) \ge 0$ for any event $B$ 
    \item For any event $B$ and any partition $(B_1, \dots, B_n)$ of $B$, $P(B) = \sum_{i=1}^{n}P(B_i)$ 
    \item $P(\Omega) = 1$ 
\end{itemize}
\end{document}

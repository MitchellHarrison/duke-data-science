\documentclass[titlepage, 12pt, leqno]{article}

% -------------------------------------------------- %
% -------------------- PACKAGES -------------------- %
% -------------------------------------------------- %
\usepackage{import}
\usepackage{mathtools}
\usepackage{transparent}
\usepackage{xcolor}
\usepackage{tcolorbox}
\usepackage{amsmath}
\usepackage{amssymb}
\usepackage{parskip}
\usepackage[margin = 1in]{geometry}


% -------------------------------------------------- %
% -------------- CUSTOM ENVIRONMENTS --------------- %
% -------------------------------------------------- %
\newtcolorbox{note}{colback=black!5!white,
                          colframe=black!55!white,
                          fonttitle=\bfseries,title=Note}

\newtcolorbox{ex}{colback=blue!5!white,
                          colframe=blue!55!white,
                          fonttitle=\bfseries,title=Example}

\newtcolorbox{definition}{colback=red!5!white,
                          colframe=red!55!white,
                          fonttitle=\bfseries,title=Definition}


% -------------------------------------------------- %
% ------------------- COMMANDS --------------------- %
% -------------------------------------------------- %
% Brackets, braces, etc. 
\newcommand{\abs}[1]{\lvert #1 \rvert}
\newcommand{\bigabs}[1]{\Bigl \lvert #1 \Bigr \rvert}
\newcommand{\bigbracket}[1]{\Bigl [ #1 \Bigr ]}
\newcommand{\bigparen}[1]{\Bigl ( #1 \Bigr )}
\newcommand{\ceil}[1]{\lceil #1 \rceil}
\newcommand{\floor}[1]{\lfloor #1 \rfloor}
\newcommand{\norm}[1]{\| #1 \|}
\newcommand{\bignorm}[1]{\Bigl \| #1 \Bigr \| #1}
\newcommand{\inner}[1]{\langle #1 \rangle}
\newcommand{\set}[1]{{ #1 }}


% -------------------------------------------------- %
% -------------------- SETUP ----------------------- %
% -------------------------------------------------- %
\title{\Huge{Lecture 4 - Bayes' Rule, Binomial Distribution, Approximation}}
\author{\large{Mitch Harrison}}
\date{\today}   
\begin{document}
\setlength{\parskip}{1\baselineskip}
\setlength{\parindent}{15pt}
\maketitle
\tableofcontents
\newpage


% -------------------------------------------------- %
% --------------------- BODY ----------------------- %
% -------------------------------------------------- %
\section{Review}
\begin{ex}
    Let $\Omega$ be the outcome space for rolling a pair of 6-sided dice. Let $A$ be the event that the first roll is 6. Let $B$ be the event that the sum of the rolls is 7. \textbf{Are $A$ and $B$ independent?} 
   \begin{align*}
       \mathbb{P}(A) = \frac{1}{6} \\
       \mathbb{P}\left(A \;\middle|\; B\right) = \frac{1}{6} 
   \end{align*}
Therefore, $\boxed{A \text{ and } B \text{ are independent.}}$    
\end{ex}
\pagebreak

\section{Independence}
\subsection{Independence of two events}
 $A$ and $B$ are \textbf{independent} if any of the following equations are satisfied:
\begin{align*}
    \mathbb{P}\left(A \;\middle|\; B\right) &= \mathbb{P}(A) \\
    \mathbb{P}\left(A \;\middle|\; B^\complement \right) &= \mathbb{P}(A) \\
    \mathbb{P}\left(A \;\middle|\; B\right) &= \mathbb{P}\left(A \;\middle|\; B^\complement \right) \\
    \mathbb{P}(A \cap B) &= \mathbb{P}(A)\mathbb{P}(B) 
\end{align*}

\subsection{Independence of more than two events}
For $A$, $B$, and $C$ to be independent, check the following:

\textit{(1)} $A$ and $B$ are independent. Use any of the above equations for this.

\textit{(2)} $C$ does not depend on the occurence of $A$ or $B$ or both.
\[
\mathbb{P}\left(C \;\middle|\; A \cap B\right) = \mathbb{P}\left(C \;\middle|\; A^\complement \cap B\right) = \mathbb{P}\left(C \;\middle|\; A \cap B^\complement \right) = \mathbb{P}\left(C \;\middle|\; A^\complement \cap B^\complement \right) = \mathbb{P}(C)
\]

Also, three events are independent if \textbf{both} of the following hold:

\textit{(1)} All possible pairs of events are independent.

\textit{(2)} $\mathbb{P}(A \cap B \cap C) = \mathbb{P}(A)\mathbb{P}(B)\mathbb{P}(C)$ 

\begin{definition}
    The \textbf{Multiplication Rule for Three Independent Events} shows that if events $A$, $B$, and $C$ are mutually independent, then:
    \[
    \mathbb{P}(A \cap B \cap C) = \mathbb{P}(A)\mathbb{P}(B)\mathbb{P}(C)
    \]
\end{definition}

\begin{ex}
    Let $\Omega $ be an outcome space associated to rolling one die twice. Let $A$ be the event that the first roll is even. Let $B$ be the event that the second roll is even. Let $C$ be the event that the sum of the rolls is 7.

    \textbf{Are these three events independent? Are they \textit{pairwise} independent?} 
   \begin{align*}
       \mathbb{P}(A) &= \frac{1}{2} \\
       \mathbb{P}(B) &= \frac{1}{2} \\
       \mathbb{P}(C) &= \frac{6}{36} = \frac{1}{6} 
   \end{align*}
   Events $A$ and $B$ are independent because:
    \begin{align*}
      \mathbb{P}\left(A \;\middle|\; B\right) = \mathbb{P}(A) \\
      \mathbb{P}\left(B \;\middle|\; A\right) = \mathbb{P}(B)
    \end{align*}
    We also find that $C$ is independent of both $A$ and $B$:
    \[
    \mathbb{P}\left(C \;\middle|\; A\right) = \mathbb{P}\left(C \;\middle|\; B\right) = \mathbb{P}(C)
    \]
    
    However, if both $A$ and $B$ occur, then $C$ cannot, thus:
    \[
        \mathbb{P}(A \cap B \cap C) = 0 \ne \mathbb{P}(A)\mathbb{P}(B)\mathbb{P}(C)
    \]
    Thus, $\boxed{\text{They are pairwise independent but not independent.}}$. 
\end{ex}
\pagebreak
\section{Bayes' Rule}
\textbf{Main Idea}: We can use the multiplication rule to translate a difficult-to-calculate conditional probability into an easy-to-calculate conditional probability.

\begin{ex}
    Suppose there are three similar boxes. Box $i$ contains $i$ white balls and one black ball for $i = 1,2,3$. I pick a box at random and don't tell you which one. From that box, I remove a ball and show it to you. I offer you a prize if you can guess correctly which box the ball came from. 

    \textbf{Which box would you guess if the ball is white? What is your chance of guessing correctly?} \\[.1in]
    Intuitively, we would guess Box 3, since it had the highest number of white balls.
    \[
    \mathbb{P}\left(\text{Box}_i \;\middle|\; \text{White}\right) = \frac{\mathbb{P}(\text{Box}_i \cap \text{White})}{\mathbb{P}(\text{White})}
    \]
    Fortunately, we know that $\mathbb{P}(B_3) = \frac{1}{3}$ and $\mathbb{P}(\text{White}| B_3) = \frac{3}{4}$. Thus:
    \[
    \mathbb{P}(B_3 \cap \text{white}) = \frac{1}{3}\left(\frac{3}{4}\right) 
    \]
    To find $\mathbb{P}(\text{white}$, we use the following:
   \begin{align*}
       \mathbb{P}(\text{white}) &= \mathbb{P}(B_1)( \mathbb{P}\left(\text{white} \;\middle|\; B_1\right) + \mathbb{P}(B_2)( \mathbb{P}\left(\text{white} \;\middle|\; B_2\right) + \mathbb{P}(B_3)( \mathbb{P}\left(\text{white} \;\middle|\; B_3\right) \\
                                &= \frac{1}{3} \left(\frac{1}{2}\right) + \frac{1}{3} \left(\frac{2}{3} \right) + \frac{1}{3} \left(\frac{3}{4} \right) \\
       \mathbb{P}(\text{white}) &= \frac{23}{36}
   \end{align*}
    We know that $\mathbb{P}(B_3 \cap \text{white}) = \frac{1}{3} \cdot \frac{3}{4} = \frac{1}{4}$, so our final probability of correctness given that we selected Box 3 is as follows:
   \begin{align*}
       \mathbb{P}(B_3) &= \frac{\frac{1}{4} }{\frac{23}{36}} \\
       \Aboxed{\mathbb{P}(B_3) = \frac{9}{23}} 
   \end{align*}
\end{ex}

\pagebreak
\textbf{Big Idea:} We can translate between the following using Bayes' Rule:
\[
    \mathbb{P}\left(A \;\middle|\; B\right) \quad \mathbb{P}\left(B \;\middle|\; A\right) 
\]
To do so, we acknolwedge the following equalities:
\[
\boxed{\mathbb{P}\left(B \;\middle|\; A\right) = \frac{\mathbb{P}(A \cap B)}{\mathbb{P}(A)} = \frac{ \mathbb{P}\left(A \;\middle|\; B\right) \mathbb{P}(B)}{\mathbb{P}(A)}}
\]

\begin{definition}
    More generally than above, \textbf{Bayes' Rule} states that for a partition $B_1, \cdots , B_n$ or an outcome space $\Omega $:
    \[
    \mathbb{P}\left(B_i \;\middle|\; A\right) = \frac{ \mathbb{P}\left(A \;\middle|\; B_i\right) \mathbb{P}(B_i)}{\mathbb{P}(A)} = \frac{ \mathbb{P}\left(A \;\middle|\; B_i\right) \mathbb{P}(B_i)}{ \mathbb{P}\left(A \;\middle|\; B_1\right) \mathbb{P}(B_1) + \cdots + \mathbb{P}\left(A \;\middle|\; B_n\right) \mathbb{P}(B_n)} 
    \]
    We used this method to find the denominator using probabilities of various boxes in the previous example.
\end{definition}

\subsection{False Positives}
\begin{ex}
    A blood test for a certain disease returns a value of either positive or negative. 95\% of people with the disease test positive. 2\% of people without the disease test positive (false positive). Only 1\% of the population has the disease.

    If a person is chosen at random from the population, tested, and the result comes back positive, \textbf{what is the likelihood that the person has the disease?} \\[.1in]
    Let:
   \begin{align*}
       T &= \text{test positive}\\
       A &= \text{has disease}\\
   \end{align*}

   We are asked to find $ \mathbb{P}\left(A \;\middle|\; T\right) $. Using Baye's rule, we arrive at the following:
  \begin{align*}
      \mathbb{P}\left(A \;\middle|\; T\right) &= \frac{\mathbb{P}(A \cap B)}{\mathbb{P}(A)} \\
                                              &= \frac{ \mathbb{P}\left(T \;\middle|\; A\right) \mathbb{P}(A)}{\mathbb{P}(A)} \\
                                              &= \frac{.01 \cdot .95}{(.01 \cdot .95)+(.99 \cdot .02)} \\
  \Aboxed{\mathbb{P}\left(A \;\middle|\; T\right) &\approx 32\%}
  \end{align*}
  
\end{ex}
\pagebreak

\section{The Birthday Problem}
Given that there are 48 people in a room, \textbf{what are the odds that two of them share a birthday}? 

Let $A$ be the event that two people share a birthday and $A^\complement $ be the event that all 48 birthdays are unique. Note:
\[
    \mathbb{P}(A) = 1 - \mathbb{P}(A^\complement )
\]
Iterating over all people, the first is guaranteed to be unique, the second has a $\frac{364}{365}$ chance of being unique, and so on. For 48 people we are left with the following chances of being entirely unique:
\[
    \mathbb{P}(A^\complement ) = \frac{365!}{(365-48)!} \cdot \frac{1}{365^{48}} 
\]

And using the complement rule, we arrive at the following solution:
\begin{align*}
    \mathbb{P}(A) &= 1 - \mathbb{P}(A^\complement ) \\
    \Aboxed{\mathbb{P}(A) &\approx 95\%} 
\end{align*}
\pagebreak
\end{document}

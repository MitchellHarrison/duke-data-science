\documentclass[titlepage, 12pt, leqno]{article}

% -------------------------------------------------- %
% -------------------- PACKAGES -------------------- %
% -------------------------------------------------- %
\usepackage{import}
\usepackage{pdfpages}
\usepackage{mathtools}
\usepackage{transparent}
\usepackage{xcolor}
\usepackage{tcolorbox}
\usepackage{amsmath}
\usepackage{amssymb}
\usepackage{parskip}
\usepackage{bbm}
\usepackage[margin = 1in]{geometry}
\tcbuselibrary{breakable}
\tcbset{breakable = true}


% -------------------------------------------------- %
% -------------- CUSTOM ENVIRONMENTS --------------- %
% -------------------------------------------------- %
\newtcolorbox{note}{colback=black!5!white,
                          colframe=black!55!white,
                          fonttitle=\bfseries,title=Note}

\newtcolorbox{ex}{colback=blue!5!white,
                          colframe=blue!55!white,
                          fonttitle=\bfseries,title=Example}

\newtcolorbox{definition}{colback=red!5!white,
                          colframe=red!55!white,
                          fonttitle=\bfseries,title=Definition}


% -------------------------------------------------- %
% ------------------- COMMANDS --------------------- %
% -------------------------------------------------- %
% Brackets, braces, etc. 
\newcommand{\abs}[1]{\lvert #1 \rvert}
\newcommand{\bigabs}[1]{\Bigl \lvert #1 \Bigr \rvert}
\newcommand{\bigbracket}[1]{\Bigl [ #1 \Bigr ]}
\newcommand{\bigparen}[1]{\Bigl ( #1 \Bigr )}
\newcommand{\ceil}[1]{\lceil #1 \rceil}
\newcommand{\floor}[1]{\lfloor #1 \rfloor}
\newcommand{\norm}[1]{\| #1 \|}
\newcommand{\bignorm}[1]{\Bigl \| #1 \Bigr \| #1}
\newcommand{\inner}[1]{\langle #1 \rangle}
\newcommand{\set}[1]{{ #1 }}


% -------------------------------------------------- %
% -------------------- SETUP ----------------------- %
% -------------------------------------------------- %
\title{\Huge{Lecture 16 - Change of Variable}}
\author{\large{Mitch Harrison}}
\date{\today}   
\begin{document}
\setlength{\parskip}{1\baselineskip}
\setlength{\parindent}{15pt}
\maketitle
\tableofcontents
\newpage


% -------------------------------------------------- %
% --------------------- BODY ----------------------- %
% -------------------------------------------------- %
\section{Review}

\begin{ex}
    Calls arrive at a call center according to a Poisson distribution. The
    average rate at which calls arrive is 1 call per 3 minutes. \textbf{What is
    the probability at least 40 calls arrive in 2 hours?}
    \vspace{10px}
    
    We are given that $\lambda = \frac{1}{3}$ and $t$ is two hours, or $t=120$.
    We then calculate $\mu = \lambda t = \frac{120}{3} = 40$. Plugging into the 
    formula for Poisson arrival, we arrive at the following:
    \[
        \boxed{\mathbb{P}( \text{at least 40 calls in 2 hours}) =
        \sum_{k=40}^{\infty}\frac{e^{-40}\cdot40^k}{k!}} 
    \]
\end{ex}

\pagebreak
\section{Change of Variable}
Suppose we know the density function for $X$. If we have a related random variable
$Y = g(X)$, how can we find the density function of $Y$? (Example: $X$ is height
and $Y = |x-\mu|$.)

\subsection{Linear change of variable}
\begin{ex}
    Suppose $X \sim $ Unif(0,1). Let $a > 0$. If $Y = aX + b$, what is the 
    density function for $Y$?
    \vspace{10px}
    
    Plugging in the limits of our range of $X$ (i.e. 0 and 1), we arrive at the
    possible range of $Y$, which is $[b, a+b]$. Thus, the width of our uniform
    distribution for $Y$ is $a+b-a$, or simply $a$. Since the area under the
    PDF must be 1, we know the height of the rectangle formed by the uniform
    distribution is $\frac{1}{a}$. Thus, we arrive at the PDF of $Y$:
    \[
        \boxed{\text{PDF}(Y) = 
        \begin{cases}
            \frac{1}{a} & b<y<a+b \\
            0 & \text{else}
        \end{cases}}
    \]
\end{ex}

More generally, let $X$ be a continuous random variable with density function
$f_X(x)$. Let $Y = aX+b$, then:
\[
    \boxed{f_Y(y) = \frac{1}{|a|}\cdot f_X\left(\frac{y-b}{a}\right)} 
\]
\subsection{One-to-one differentiable functions}
Let $X$ be a continuous random variable with density $f_X$ on the interval 
$(a,b)$. Let $Y=g(X)$ for some one-to-one differentiable function $g$ on the
interval $(a,b)$. Then:
\[
    \boxed{f_Y(y) = \frac{1}{\left|\frac{dy}{dx}\right|}\cdot f_X(g^{-1}(y)} 
\]
The endpoints of the range of $Y$ must necessarily be $g(a)$ and $g(b)$.
\begin{ex}
    Let $X$ have the following density function:
    \[
        f_X(x) = 
        \begin{cases}
            e^{-x} & x>0 \\
            0 & \text{else}
        \end{cases}
    \]
    Let $Y = \sqrt{X}$. What is the densty function for $Y$?
    \vspace{10px}
    
    First, we find the range of $Y$, which is all possible non-negative numbers
    for the function $Y = \sqrt{X}$. Then, we confirm that $\sqrt{X}$ is one-to-
    one differentiable, which it is. Next we solve for $X$ in terms of $Y$. In
    this case, since $Y = \sqrt{X}$, then $X = Y^2$. Then we find $\frac{dy}{dx}$.
    For this problem, $\frac{dy}{dx}(\sqrt{X}) = \frac{1}{2\sqrt{X}}$. Finally,
    we plug in our findings:
   \begin{align*}
       f_Y(y) &= \frac{d_X(y^2)}{\left|\frac{dy}{dx}\right|} \\
              &= \frac{e^{-x}}{\frac{1}{2\sqrt{X}}} \\
              &= 2\sqrt{X}e^{-x}
              &= 2ye^{-y^2}
   \end{align*}
    
   Therefore,
   \[
   \boxed{
        f_Y(y) = 
        \begin{cases}
            2ye^{-y^2} & y>0 \\
            0 & \text{else}
        \end{cases}
   } 
   \]
\end{ex}

\subsection{Many-to-one functions}
Recall the discrete version:

\vspace{10px}
\begin{center}
\begin{tabular}{c|c c c c c}
    $x$ & -2 & -2 & 0 & 1 & 2 \\
    \hline
    $\mathbb{P}(X=x)$ & 1/6 & 1/6 & 1/3 & 1/6 & 1/6
\end{tabular}
\end{center}
\vspace{10px}
The, for $y = x^2$:
\begin{center}
\begin{tabular}{c|c c c}
    $y$ & 0 & 1 & 4 \\
    \hline
    $\mathbb{P}(Y=y)$ & 1/3 & 1/3 & 1/3
\end{tabular}
\end{center}
\vspace{10px}

Let $X$ be a continuous random variable with a known density function $f_X$. Let
$g$ be a differentiable function and define $Y=g(X)$. Then, the density function
of $Y$ is:
\[
f_Y(y) = \sum_{x:y=g(x)} \frac{f_X}{\left|\frac{dy}{dx}\right|}
\]
where $x:y=g(x)$ denotes "all $x$ with $g(x) = y$".

\end{document}

\documentclass[titlepage, 12pt, leqno]{article}

% -------------------------------------------------- %
% -------------------- PACKAGES -------------------- %
% -------------------------------------------------- %
\usepackage{import}
\usepackage{pdfpages}
\usepackage{mathtools}
\usepackage{transparent}
\usepackage{xcolor}
\usepackage{tcolorbox}
\usepackage{amsmath}
\usepackage{amssymb}
\usepackage{parskip}
\usepackage{bbm}
\usepackage[margin = 1in]{geometry}


% -------------------------------------------------- %
% -------------- CUSTOM ENVIRONMENTS --------------- %
% -------------------------------------------------- %
\newtcolorbox{note}{colback=black!5!white,
                          colframe=black!55!white,
                          fonttitle=\bfseries,title=Note}

\newtcolorbox{ex}{colback=blue!5!white,
                          colframe=blue!55!white,
                          fonttitle=\bfseries,title=Example}

\newtcolorbox{definition}{colback=red!5!white,
                          colframe=red!55!white,
                          fonttitle=\bfseries,title=Definition}


% -------------------------------------------------- %
% ------------------- COMMANDS --------------------- %
% -------------------------------------------------- %
% Brackets, braces, etc. 
\newcommand{\abs}[1]{\lvert #1 \rvert}
\newcommand{\bigabs}[1]{\Bigl \lvert #1 \Bigr \rvert}
\newcommand{\bigbracket}[1]{\Bigl [ #1 \Bigr ]}
\newcommand{\bigparen}[1]{\Bigl ( #1 \Bigr )}
\newcommand{\ceil}[1]{\lceil #1 \rceil}
\newcommand{\floor}[1]{\lfloor #1 \rfloor}
\newcommand{\norm}[1]{\| #1 \|}
\newcommand{\bignorm}[1]{\Bigl \| #1 \Bigr \| #1}
\newcommand{\inner}[1]{\langle #1 \rangle}
\newcommand{\set}[1]{{ #1 }}


% -------------------------------------------------- %
% -------------------- SETUP ----------------------- %
% -------------------------------------------------- %
\title{\Huge{Lecture 12 - Central Limit Theorem}}
\author{\large{Mitch Harrison}}
\date{\today}   
\begin{document}
\setlength{\parskip}{1\baselineskip}
\setlength{\parindent}{15pt}
\maketitle
\tableofcontents
\newpage


% -------------------------------------------------- %
% --------------------- BODY ----------------------- %
% -------------------------------------------------- %
\section{Review(ish) Scaling, Shifting, Standardization}
Finding variance:
\begin{align*}
    Var(X) &= \mathbb{E}((X-\mu_x)^2) \\
    Var(X) &= \mathbb{E}(X^2) - ( \mathbb{E}(X))^2
\end{align*}

\begin{ex}
    \textbf{How does} $Var(Y)$ compare to $Var(aY+b)$, given that $a$ and $b$ are 
    constants?
    \vspace{10px}
    Since the variance of a constant is zero,
    \[
    Var(aY + b) = Var(aY) + Var(b) = Var(aY)
    \]
    Extrapolating, 
   \begin{align*}
       Var(aY) &= \mathbb{E}((aY-a\mu_y)^2) \\
               &= \mathbb{E}((a(Y=\mu_y))^2) \\
               &= a^2 \mathbb{E}((Y-\mu_y)^2) \\
       \Aboxed{Var(aY) &= a^2 Var(Y)} 
   \end{align*}
\end{ex}

\begin{ex}
    \textbf{How does} $SD(Y)$ compare to $SD(aY+b)$?
    \vspace{10px}
   \begin{align*}
       SD = \sqrt{Var} : SD(aY+b) &= \sqrt{Var(aY+b)} \\
                                  &= \sqrt{a^2 Var(Y)} \\
                                  &= |a| \sqrt{Var(Y)} \\
   \end{align*}
   Therefore,
   \[
   \boxed{SD(aY+b) = |a| SD(Y)} 
   \]
\end{ex}

\begin{definition}
    Using \textbf{standardization}, we can rewrite probabilities in terms of a
    random variable with mean 0 and std. deviation 1.
    \vspace{10px}
    If a random variable $X$ has $ \mathbb{E}(X) = \mu$ and $SD(X) = \sigma$, 
    then:
    \[
        X^* = \frac{X-\mu}{0}
    \]
    $X^*$ has mean 0 and standard deviation 1. Effectively, this shifts the 
    distribution to 0 and divides by the standard deviation to make it 1.
    \vspace{10px}
    \textbf{Always}, $ \mathbb{E}(X^*) = 0$ and $SD(X^*) = 1$
\end{definition}


\pagebreak
\section{Normal, Binomial, Standardization}
The standard normal density function is
\[
    \frac{1}{\sqrt{2\pi }}e^{-x^2/2}
\]
It looks like a bell curve. It corresponds to a perfect bell curve centered at 0.

\subsection{Review of normal approximation of binomial}
Let $X \sim binomial(n,p)$. Use a normal approximation to estimate $\mathbb{P}(
a \le X \le b)$.
\vspace{10px}
For binomial distributions, we use the following formula, including the 
\textit{continuity correction}, which will not be used today:
\[
\Phi\left(\frac{b+\frac{1}{2}-\mu}{\sigma}\right) - 
\Phi\left(\frac{a-\frac{1}{2}-\mu}{\sigma}\right)
\]

\subsection{Goal: Extend to non-binomial}
Let's say Researcher 1 grabs 100 students and get their average number of dogs.
Researcher 2, 3, etc. does the same thing. They are at the same university, so
there may be overlap. If we record the average number of dogs from 100 
researchers, we will end up with 100 data points. Plotting a histogram of these
data will approximate a bell curve. 

Doing anything similar (i.e. taking many sample's averages and plotting them) will
always result in a bell curve, regardless of the distribution of the number of
dogs in the student body.

\pagebreak
\section{Law of Averages and Central Limit Theorem}
Distributions that are not normal, start to look normal if you look at the
\textit{average} or \textit{sum} of repeated trials. We saw this for normal
distributions, but it's true for \textit{any} distribution.

\begin{definition}
    The \textbf{Law of Averages}, or \textbf{Law of Large Numbers}. Let $X_1,X_2,
    \cdots $ be a sequence of independent random variables with the same 
    distribution as $X$. Let $ \mathbb{E}(X) = \mu$ and define the averages of
    $n$ random variables:
    \[
    \bar X_n = \frac{X_1+X_2+ \cdots + X_n}{n}    
    \]
    \begin{note}
        Two random variables have the \textit{same distribution} if they have the
        same means and standard deviations.
    \end{note}
    Then, for every $\epsilon$, no matter how small:
    \[
    \mathbb{P}(| \bar X_n - \mu | < \epsilon) \rightarrow 1 
    \text{ as } n \rightarrow \infty
    \]
    "The average of the average is the average."
    \begin{note}
        For this to hold true, $X_1, \cdots , X_n$ must be \textit{independent}.
    \end{note}
    Here:
    \[
    \mathbb{E}(X_1) = \cdots = \mathbb{E}(X_n) = \mu
    \]
    Thus,
    \begin{align*}
        \mathbb{E}( \bar X_n) &= \mu \\
        SD( \bar X_n) &\approx c \cdot \frac{1}{\sqrt{n}}
    \end{align*}
    Where $c$ is some constant.
\end{definition}

\begin{definition}
    The \textbf{Central Limit Theorem}: Let $X_1, \cdots , X_n$ be $n$ 
    independent random variables with the same distribution over a finite range.
    \begin{itemize}
        \item For large $n$, the \textit{sum} of the $X_i$'s is roughly 
            normally distributed.
                Let:
                \begin{align*}
                    S &= \text{sum} = X_1 + \cdots + X_n \\
                    \mathbb{E}(S) &= \mu_s \\
                    SD(S) &= \sigma_s
                \end{align*}
                Then,
                \begin{align*}
                    \mathbb{P}(a\le S \le b) &= \mathbb{P}\left(\frac{1-
                        \mu_s}{\sigma_s}\le
                        S^* \le \frac{b-\mu_s}{\sigma_s}\right) \\
                                             &\approx \Phi\left(\frac{b-
                                             \mu_s}{\sigma_s}\right) -
                                                \Phi\left(\frac{a-
                                                \mu_s}{\sigma_s}\right)
                \end{align*}
        \item For large $n$, the \textit{average} of the $X_i$'s is roughly
            normally distributed.
            Let:
           \begin{align*}
               \bar X_n &= \text{sample average} = \frac{X_1 + \cdots 
               + X_n}{n} \\
                   \mathbb{E}( \bar X_n) &= \mu_x \\
                   SD( \bar X_n) = \sigma_{ \bar X}
           \end{align*}
            Then,
           \begin{align*}
               \mathbb{P}(a \le \bar X_n \le b) &= \mathbb{P}\left(
                   \frac{a-\mu_{ \bar X_n}}{\sigma_{ \bar X}}\right) \\
                                                &\approx \Phi\left(
                                                    \frac{b-\mu_{ \bar
                                                    X}}{\sigma_{ \bar X}}\right) -
                                                    \Phi\left(
                                                    \frac{a-\mu_{ \bar
                                                    X}}{\sigma_{ \bar X}}\right)
           \end{align*}
    \end{itemize}
    \begin{note}
        Here, we could say that all $X_i$ are $i.i.d.$ or \textit{independent,
        identically distributed}
    \end{note}
\end{definition}

\pagebreak
\subsection{Examples.}
\begin{enumerate}
    \item Let $X$ be the number on one roll of a die. $X$ is uniformly 
        distributed. But $S_n = \sum X_n$ will be roughly normally distributed as
        $n$ gets large.
    \item Let $X$ be the number on one roll of a die. $X$ is uniformly 
        distributed. But $ \bar X_n = \frac{X_1+ \cdots +X_n}{n}$ will be
        roughly normally distributed as $n$ gets large.
    \item A pollster samples 1000 people and asks if they prefer candidate $X$ or
        candidate $Y$. Let $Z$ be the number of poeple who prefer candidate $X$
        out of the 1000 people sampled. The random variable $Z$ has a 
        hypergeometric distribution. If 30 pollsters all take random samples of
        1000 people, then $Z_1 + \cdots  + Z_{30}$ will be roughly normally
        distributed.
\end{enumerate}

\begin{ex}
    Give an example of a sum of random variables $X_1 + \cdots + X_n$ that will
    NOT approach a normal distribution no matter how large $n$ is.
\end{ex}

\pagebreak
\section{Mean and standard deviation}
Suppose $X_1, \cdots , X_n$ are independent, all with the same distribution.
$ \mathbb{E}(X_i) = \mu$ and $SD(X_i) = \sigma$.
\begin{enumerate}
    \item What is $ \mathbb{E}(X_1 + \cdots + X_n)$?
        \[
        \boxed{n\mu} 
        \]
    \item What is $SD(X_1 + \cdots + X_n)$?
       \begin{align*}
           SD^2 &= Var(X_1 + \cdots + X_n) \\
                &= Var(X_1) + \cdots + Var(X_n) \\
           SD^2 &= n\sigma^2
       \end{align*}
       Thus,
       \[
       \boxed{SD = \sqrt{n\sigma^2} = \sigma\sqrt{n}} 
       \]
       \begin{note}
           We can split up variances in a sum if and only if $X_1, \cdots , X_n$
           are \textit{independent}.
       \end{note}
       
    \item What is $ \mathbb{E}\left(\frac{X_1+ \cdots +X_n}{n}\right)$?
        \[
        \boxed{\mu} 
        \]
    \item What is $ SD\left(\frac{X_1+ \cdots +X_n}{n}\right)$?
       \begin{align*}
           SD &= \frac{1}{n} SD(X_1 + \cdots + X_n) \\
           &= \frac{1}{n}\sqrt{n}\sigma \\
           \Aboxed{SD &= \frac{\sigma}{\sqrt{n}}} 
       \end{align*}
\end{enumerate}

\begin{ex}
    Let $D_i$ be the \textbf{first digit} of a randomly chosen Fibonacci number
    (from amongst the first 100,000 Fibonacci numbers). $ \mathbb{E}(D_i) = 3.44$
    and $SD(D_i) = 2.4611$.
    \vspace{10px}

    Define $ \bar D_n$ as the \textit{average} of $n$ randomly selected Fibonacci
    numbers:
    \[
    \bar D_n = \frac{D_1 + \cdots + D_n}{n}
    \]
    \begin{enumerate}
        \item Predict the value of $ \bar D_n$ when $n$ is large.
            \[
                \mathbb{E}( \bar D_n) = \mathbb{E}(D_i) = \boxed{3.44}
            \]
        \item Find the number $\epsilon$ such that for $n=10,000$ the chance
            that your prediction is off by more than $\epsilon$ is about 10\%
           \begin{align*}
               \mathbb{P}\left(\frac{3.44-\epsilon}{SD( \bar D_n)} \le \bar D_n
                   \le \frac{3.44+\epsilon}{SD( \bar D_n)}\right) = 0.9 \\
                   \mathbb{P}(| \bar D_n - 3.44| > \epsilon) \approx 0.1
           \end{align*}

        \item Find approximately the least value of $n$ such that your prediction
            of $ \bar D_n$ is correct to within 0.01 with probability at least
            0.95

            \textbf{Not completed in class}
            \vspace{10px}

        \item If you had to predict the first digit of $ \bar D_{100}$, what 
            digit should you choose to maximize your chance of being correct, and
            what is that chance?

            \textbf{Not completed in class}
    \end{enumerate}
\end{ex}
\end{document}

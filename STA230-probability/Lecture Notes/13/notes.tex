\documentclass[titlepage, 12pt, leqno]{article}

% -------------------------------------------------- %
% -------------------- PACKAGES -------------------- %
% -------------------------------------------------- %
\usepackage{import}
\usepackage{pdfpages}
\usepackage{mathtools}
\usepackage{transparent}
\usepackage{xcolor}
\usepackage{tcolorbox}
\usepackage{amsmath}
\usepackage{amssymb}
\usepackage{parskip}
\usepackage{bbm}
\usepackage[margin = 1in]{geometry}


% -------------------------------------------------- %
% -------------- CUSTOM ENVIRONMENTS --------------- %
% -------------------------------------------------- %
\newtcolorbox{note}{colback=black!5!white,
                          colframe=black!55!white,
                          fonttitle=\bfseries,title=Note}

\newtcolorbox{ex}{colback=blue!5!white,
                          colframe=blue!55!white,
                          fonttitle=\bfseries,title=Example}

\newtcolorbox{definition}{colback=red!5!white,
                          colframe=red!55!white,
                          fonttitle=\bfseries,title=Definition}


% -------------------------------------------------- %
% ------------------- COMMANDS --------------------- %
% -------------------------------------------------- %
% Brackets, braces, etc. 
\newcommand{\abs}[1]{\lvert #1 \rvert}
\newcommand{\bigabs}[1]{\Bigl \lvert #1 \Bigr \rvert}
\newcommand{\bigbracket}[1]{\Bigl [ #1 \Bigr ]}
\newcommand{\bigparen}[1]{\Bigl ( #1 \Bigr )}
\newcommand{\ceil}[1]{\lceil #1 \rceil}
\newcommand{\floor}[1]{\lfloor #1 \rfloor}
\newcommand{\norm}[1]{\| #1 \|}
\newcommand{\bignorm}[1]{\Bigl \| #1 \Bigr \| #1}
\newcommand{\inner}[1]{\langle #1 \rangle}
\newcommand{\set}[1]{{ #1 }}


% -------------------------------------------------- %
% -------------------- SETUP ----------------------- %
% -------------------------------------------------- %
\title{\Huge{Lecture 12 - Central Limit Theorem}}
\author{\large{Mitch Harrison}}
\date{\today}   
\begin{document}
\setlength{\parskip}{1\baselineskip}
\setlength{\parindent}{15pt}
\maketitle
\tableofcontents
\newpage


% -------------------------------------------------- %
% --------------------- BODY ----------------------- %
% -------------------------------------------------- %
\section{Review}

\begin{ex}
    Let $X_1, X_2, \cdots $ be i.i.d. random variables, each with mean $\mu$ and
    (finite) standard deviation $\sigma$
    \vspace{10px}

    \textbf{Sums:}

    Let $S_n = X_1 + \cdots + X_n$
    \[
    \mathbb{P}(a \le S_n \le b) = \mathbb{P}\left(
        \frac{a - n\mu}{\sqrt{n}\cdot\sigma} \le S_n^* \le
        \frac{b - n\mu}{\sqrt{n}\cdot\sigma}\right)
        \approx \Phi\left(\frac{b-n\mu}{\sqrt{n}\cdot\sigma}\right) -  
        \Phi\left(\frac{a-n\mu}{\sqrt{n}\cdot\sigma}\right)
    \]
    \begin{note}
        The $\Phi$ function here are basically finding \textit{z-scores}.
    \end{note}
    
    \textbf{Averages}

    Let $\bar X_n = \frac{X_1 + \cdots + X_n}{n}$:
    \[
    \mathbb{P}(a \le \bar X_n \le b) = \mathbb{P}\left(
        \frac{a - \mu}{\frac{\sigma}{\sqrt{n}}}\le \bar X_n^* \le
        \frac{b - \mu}{\frac{\sigma}{\sqrt{n}}}\right)
        \approx \Phi\left(\frac{b-\mu}{\frac{\sigma}{\sqrt{n}}}\right) -  
        \Phi\left(\frac{a-n\mu}{\frac{\sigma}{\sqrt{n}}}\right)
    \]
    These are true no matter what the distribution of $X$ is, and it also holds
    for $X_i$'s with infinite range as long as the standard deviation is finite.
\end{ex}

\begin{note}
    Anything with star (i.e. $\bar X_n^*$, $S_n^*$) is \textit{normallized}, thus
    it has standard deviation 1 and mean 0. It is calculated as follows:
    \[
    S^*_n = \frac{S_n - \mathbb{E}(S_n)}{SD(S_n)}
    \]
\end{note}

\pagebreak
\section{Discrete Distributions}
\subsection{Geometric distribution}
\begin{ex}
    You keep shooting free throws until you make one. You have about
    a 1/6 chance of making each shot. Let $X$ be the number of shots you take. 
    What is the distribution of $X$?
    \vspace{10px}
    
    We are looking for $\mathbb{P}(X=k)$ in this problems:
   \begin{align*}
       X& \sim Geom\left(\frac{1}{6}\right) \\
       X &= \text{\# of trials to and including success with 1/6 probability
       of success} \\
           \Aboxed{X &= \left(\frac{5}{6}\right)^{k-1}\left(\frac{1}{6}\right)}
   \end{align*}
\end{ex}

You keep shooting free throws until you make one. You have about
a 1/6 chance of making each shot. Let $X$ be the number of shots you take. What is
the distribution of $X$.

\begin{definition}
    A random variable $Y$ with range \{1,2,3,...\} has a 
    \textbf{geomtric-$p$ distribution} if $\mathbb{P}(Y=k) = (1-p)^{k-1}p$
    \begin{itemize}
        \item We say $Y ~ Geom(p)$ or $Y ~ Geometric(p)$
        \item $ \mathbb{E}(Y) = \frac{1}{p}$
        \item $SD(Y) = \frac{\sqrt{1-p}}{p}$
    \end{itemize}
    
    A random variable $Y$ with range \{0,1,2,3,...\} has a 
    \textbf{geometric-$p$ distribution} if $\mathbb{P}(Y=k) = (1-p)^kp$
    \begin{itemize}
        \item We say $Y ~ Geom(p)$ or $Y ~ Geometric(p)$
        \item $ \mathbb{E}(Y) = \frac{1}{p}-1$
        \item $SD(Y) = \frac{\sqrt{1-p}}{p}$
    \end{itemize}
\end{definition}

\textbf{Examples of random variables with geometric distribution}
\begin{itemize}
    \item $X = $ number of flips until you get heads
    \item $Y = $ number of unsuccessful trials before a success
    \item $Z = $ number of packages of baseball cards purchased until finding
        Honus Wagner
\end{itemize}

\pagebreak

\end{document}

\documentclass[titlepage, 12pt, leqno]{article}

% -------------------------------------------------- %
% -------------------- PACKAGES -------------------- %
% -------------------------------------------------- %
\usepackage{import}
\usepackage{pdfpages}
\usepackage{transparent}
\usepackage{xcolor}
\usepackage{tcolorbox}
\usepackage{amsmath}
\usepackage{amssymb}
\usepackage{parskip}
\usepackage[margin = 1in]{geometry}


% -------------------------------------------------- %
% -------------- CUSTOM ENVIRONMENTS --------------- %
% -------------------------------------------------- %
\newtcolorbox{note}{colback=black!5!white,
                          colframe=black!55!white,
                          fonttitle=\bfseries,title=Note}

\newtcolorbox{ex}{colback=blue!5!white,
                          colframe=blue!55!white,
                          fonttitle=\bfseries,title=Example}

\newtcolorbox{definition}{colback=red!5!white,
                          colframe=red!55!white,
                          fonttitle=\bfseries,title=Definition}


% -------------------------------------------------- %
% ------------------- COMMANDS --------------------- %
% -------------------------------------------------- %
% Brackets, braces, etc. 
\newcommand{\abs}[1]{\lvert #1 \rvert}
\newcommand{\bigabs}[1]{\Bigl \lvert #1 \Bigr \rvert}
\newcommand{\bigbracket}[1]{\Bigl [ #1 \Bigr ]}
\newcommand{\bigparen}[1]{\Bigl ( #1 \Bigr )}
\newcommand{\ceil}[1]{\lceil #1 \rceil}
\newcommand{\floor}[1]{\lfloor #1 \rfloor}
\newcommand{\norm}[1]{\| #1 \|}
\newcommand{\bignorm}[1]{\Bigl \| #1 \Bigr \| #1}
\newcommand{\inner}[1]{\langle #1 \rangle}
\newcommand{\set}[1]{{ #1 }}


% -------------------------------------------------- %
% -------------------- SETUP ----------------------- %
% -------------------------------------------------- %
\title{\Huge{Lecture 1 - Equally Likely Outcomes/Syllabus Day}}
\author{\large{Mitch Harrison}}
\date{\today}   
\begin{document}
\setlength{\parskip}{1\baselineskip}
\setlength{\parindent}{15pt}
\maketitle
\tableofcontents
\newpage


% -------------------------------------------------- %
% --------------------- BODY ----------------------- %
% -------------------------------------------------- %
\section{Syllabus Notes}

\subsection{Exams}
There will be two tests and a final exam. Exams are worth 20\% each, and the final exam is worth 30\%. Final will replace the lowest grade.

\subsection{Homework}
Homeworks are due Thursdays at 11:59 pm. Homework is worth 30\%. Professor says yes to most homework extensions. If all homeworks are done on time, 0.5\% will be added to the homework average. Problem sets are on Sakai, and there will be no homework on test weeks.

\subsection{Help Rooms}
Help rooms are Tues/Wed/Thurs 7pm - 10pm in Physice 259.

\section{Interpretations of Probability}
There are two ways of discussing probability: \textbf{frequency} and \textbf{degree of belief}.

\begin{definition}
    The frequency interpretation for coin tosses says that as the number of flips foes to $\infty$, $\frac{1}{2} $ will be heads. We can also call this the \textbf{long-run average}.
\end{definition}

\begin{definition}
    The \textbf{degree of belief} interpretation discusses how confident a person is. This interpretation is best for one-time events, or events are not repeatable.
\end{definition}

\begin{ex}
    Which interpretation of probability is best for elections, weather predictions, and medical studies? \\[.1in]
    The \textbf{degree of belief interpretation}, since the election is an experiment that can only be run once.
\end{ex}

\pagebreak
\section{Course Overview}
We will directly compute probability. Then, we will describe the \textit{distribution} of values. Finally, we update our probabilities based on new information. The final step is \textit{conditional probability}.

\subsection{Discrete Probability}
Dice, cards, and coins are \textbf{discrete} since they have a fixed set of possible values. Calculating probabilities for discrete values will typically involve sums.

\subsection{Continuous Probability}
Time, height, and length are \textbf{continuous} since their possible values are infinite. Probability calculations will involve

\section{Definitions}
\begin{definition}
    An \textbf{experiment} is a well-defined procedure that returns a result.
    \begin{itemize}
        \item Flip a coin
        \item Measure someone's height
    \end{itemize}
\end{definition}

\begin{definition}
    The \textbf{outcome space} or \textbf{sample space} is a set of all possible outcomes. The notation for this set is $\Omega$. We also call this the \textbf{universal set}.
\end{definition}

\begin{ex}
    Define $\Omega$ for dice rolls.
    \[
    \Omega = \{1,2,3,4,5,6\}
    \]
\end{ex}

\begin{ex}
    Define $\Omega$ for height.
    \[
    \Omega = (0\text{ft.}, 8\text{ft.})
    \]
\end{ex}

\begin{definition}
    An \textbf{event} is a \textbf{subset} of possible outcomes. We notate this with $A$.
\end{definition}

For even-number dice rolls, $A = \{2,4,6\}$ 

\begin{definition}
    The \textbf{probability} of an event $A$ is denoted $\mathbb{P}(A)$
\end{definition}
 
\begin{note}
    If all outcomes in $\Omega$ are equally likely:
    \[
    \mathbb{P} = \frac{\#A}{\#\Omega} 
    \]
\end{note}


\begin{ex}
    What is the likelihood if you flip a fair coin twice, that the coin will land on heads at least once?
    \[
    \mathbb{P}(B) = 1 - \mathbb{P}(\text{no heads}) = 1 - \frac{1}{4}
    \]
    \[
    \mathbb{P}(B) = \frac{3}{4} 
    \]
\end{ex}

\begin{definition}
    An impossible event is an \textbf{empty set} and is denoted with $\phi $.
\end{definition}

\begin{definition}
    The \textbf{complement} of a set is the opposite set of solutions. If $A$ is rolling an even number, then $A^C$ is rolling odd.
\end{definition}

\subsection{Set Language}
For an inclusive OR (meaning both sets and their overlap), we use a \textbf{union}, which is annotated as $A \cup B$.

To get only where $A$ and $B$ overlap, we use an \textbf{intersection}, annotated as $A \cap B$.

\begin{definition}
    Two sets are \textbf{disjoint} if there is no overlap, or:
    \[
    A \cap B = \phi 
    \]
\end{definition}

\pagebreak
\section{Formulas and Rules for Calculations}
\begin{equation}
    \mathbb{P}(\Omega) = 1
\end{equation}

\begin{equation}
    0 \le \mathbb{P}(B) \le 1
\end{equation}

If two sets $A$ and $B$ are \textit{disjoint}, then:
\begin{equation}
    \mathbb{P}(A \text{ or } B) = \mathbb{P}(A) + \mathbb{P}(B)
\end{equation}

\begin{equation}
    \mathbb{P}(A \cup B) = \mathbb{P}(A) + \mathbb{P}(B) - \mathbb{P}(A \cap B)
\end{equation}

\end{document}

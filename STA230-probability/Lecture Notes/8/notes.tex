\documentclass[titlepage, 12pt, leqno]{article}

% -------------------------------------------------- %
% -------------------- PACKAGES -------------------- %
% -------------------------------------------------- %
\usepackage{import}
\usepackage{pdfpages}
\usepackage{mathtools}
\usepackage{transparent}
\usepackage{xcolor}
\usepackage{tcolorbox}
\usepackage{amsmath}
\usepackage{amssymb}
\usepackage{parskip}
\usepackage[margin = 1in]{geometry}


% -------------------------------------------------- %
% -------------- CUSTOM ENVIRONMENTS --------------- %
% -------------------------------------------------- %
\newtcolorbox{note}{colback=black!5!white,
                          colframe=black!55!white,
                          fonttitle=\bfseries,title=Note}

\newtcolorbox{ex}{colback=blue!5!white,
                          colframe=blue!55!white,
                          fonttitle=\bfseries,title=Example}

\newtcolorbox{definition}{colback=red!5!white,
                          colframe=red!55!white,
                          fonttitle=\bfseries,title=Definition}


% -------------------------------------------------- %
% ------------------- COMMANDS --------------------- %
% -------------------------------------------------- %
% Brackets, braces, etc. 
\newcommand{\abs}[1]{\lvert #1 \rvert}
\newcommand{\bigabs}[1]{\Bigl \lvert #1 \Bigr \rvert}
\newcommand{\bigbracket}[1]{\Bigl [ #1 \Bigr ]}
\newcommand{\bigparen}[1]{\Bigl ( #1 \Bigr )}
\newcommand{\ceil}[1]{\lceil #1 \rceil}
\newcommand{\floor}[1]{\lfloor #1 \rfloor}
\newcommand{\norm}[1]{\| #1 \|}
\newcommand{\bignorm}[1]{\Bigl \| #1 \Bigr \| #1}
\newcommand{\inner}[1]{\langle #1 \rangle}
\newcommand{\set}[1]{{ #1 }}


% -------------------------------------------------- %
% -------------------- SETUP ----------------------- %
% -------------------------------------------------- %
\title{\Huge{Lecture 8 - Random Variables}}
\author{\large{Mitch Harrison}}
\date{\today}   
\begin{document}
\setlength{\parskip}{1\baselineskip}
\setlength{\parindent}{15pt}
\maketitle
\tableofcontents
\newpage


% -------------------------------------------------- %
% --------------------- BODY ----------------------- %
% -------------------------------------------------- %
\section{Review}
\begin{ex}
    20 cards are selected from a standard deck without replacement. 
    \textbf{What is the probability} that at least 10 of the cards are diamonds
    or at least 10 of the cards are spades?\\[.1in]
    We will use the following formula:
    \[
    \mathbb{P}(k \text{ successes without replacements}) = \frac{\binom{G}{k}\binom{N-G}{n-k}}{\binom{N}{n}} 
    \]
    Where $N$ is the population size, $n$ is the number of samples, $k$ is 
    the number of successes, and $G$ is the number of "good" cards (diamonds or
    spades). We will find these probabilities independently and add them in the 
    following way:
    \[
    \mathbb{P}(A \cup B) = \mathbb{P}(A) + \mathbb{P}(B) - \mathbb{P}(A \cap B)
    \]
    Where here, $\mathbb{P}(A) = \mathbb{P}(B)$ \\[.1in]
    Also, since we need \textit{at least} 10 of either suit with 13 total of each
    suit in the deck,
    \[
    \mathbb{P}( \text{at least 10 diamonds}) = \sum_{k=10}^{13}\frac{\binom{13}{k}\binom{39}{20-k}}{\binom{52}{20}} 
    \]
    We also calculate the following:
    \[
    \mathbb{P}(A \cap B) = \frac{\binom{13}{10}\binom{13}{10}}{\binom{52}{20}} 
    \]
    Thus for the final total:
    \[
       \mathbb{P}( \text{at least 10 diamonds or spades}) = \boxed{2\sum_{k=10}^{13}\frac{\binom{13}{k}\binom{39}{20-k}}{\binom{52}{20}} + \frac{\binom{13}{10}\binom{13}{10}}{\binom{52}{20}}}
    \]
\end{ex}

\textbf{Previous topics} 
\begin{enumerate}
    \item The \textit{exact} calculation of a binomial is 
        \[
            \binom{n}{k}p^k(1-p)^{n-k}
        \]
    \item but, that's computationally expensive and time-consuming by hand. So we
        \textit{estimate} with the following:
        \[
            \Phi\left(\frac{1 \pm \frac{1}{2} - \mu}{\sigma} \right)
        \]
\end{enumerate}

\pagebreak
\section{Random Variables}
Previously, we defined $X$ as the number of success out of $n$ trials. Today, we
will formalize that \textit{random variable} and properties about it.
\textbf{Examples} 
\begin{enumerate}
    \item Experiment: Flip a coin 10 times.\\[.1in]
        $X$ = the number of heads out of the 10 flips\\
        $Y$ = the number of heads minus the number of tails

        If our outcomes were $\{TTTTTTTTTH\}$, then $X=1$.

    \item Experiment: Rolling a pair of dice \\[.1in]
        $X$ = the sum of the two dice
        $Y$ = the minimum of the two dice

        If we rolled (1,3), then $X=4$ and $Y=1$.
\end{enumerate}

We think of $X$ and $Y$ as \textbf{functions.} The \textit{domain} of this 
function is the elements of the outcome space. The \textit{range} (or
\textit{output}) is real numbers. These random variables represent some way to
assign numbers to all of the outcomes.
\begin{note}
    Typically, we use capital letters to represent random variables. ($X$, $Y$, 
    etc.)
\end{note}

\begin{definition}
    A \textbf{random variable} is a function defined on an outcome space 
    $\Omega $ that assigns each outcome a real number. That is, if $X$ is the
    name of the random variable and $s$ is a possible outcome, then $X(s)$ is a
    real number.
\end{definition}

\textbf{Experiment:} Flip a fair coin three times.
\begin{enumerate}
    \item Give an example of a \textbf{random variable}.

        $\boxed{X = \text{number of heads} - \text{number of tails}}$
    \item Give an example of an \textbf{event}.

        $\boxed{(X = 2) = \{HHH\}}$
\end{enumerate}

\begin{ex}
    Which of the following is \textit{not} a random variable?
    \begin{enumerate}
        \item $X$ = the number rolled on a fair 6-sided die
        \item $Y$ = the number of spades in a set of 5 cards
        \item $Z$ = 1
        \item $\boxed{W = \text{rolling 2 fair 6-sided dice}}$
    \end{enumerate} 
    Here, number 4 is an \textit{experiment}, not a random variable.
\end{ex}

\subsection{When are two random variables equal?}
Random variables are \textbf{equal} if they give the same output for all input.
For example, when we roll 2 fair dice:
\begin{align*}
    X &= \text{number on first roll} \\
    Y &= \text{number of second roll}
\end{align*}

\begin{enumerate}
    \item Do $X$ and $Y$ have the same distribution?

        $\boxed{ \text{Yes!}}$ Since both dice are fair and identical, the 
        probability of each face is the same.

    \item Are $X$ and $Y$ equal?

        $\boxed{ \text{No.}}$ If the event was (1,2), then using that same input,
        $X=1$ and $Y=2$. Thus, they are not equal.
\end{enumerate}

\begin{definition}
    Let $X$ be a random variable with finitely namy range values. The 
    \textbf{probability mass function} of $X$ is the function that gives the 
    probability of each possible value of $X$. The mass function is:
    \[
    p(x) = \mathbb{P}(X=x)
    \]
   To represent the mass function, you can plot it, list it as a table, or write
   it as a function. For random variables with finitely many range values, the
   mass function determines the distribution of $X$. 
\end{definition}

\begin{ex}
    You flip a coin 3 times. Let $X$ be the random variable that gives the number
    of heads. \textbf{Write down the mass function} of $X$.

    \begin{center}
    \begin{tabular}{c|c c c c}
        $x$ & 0 & 1&2&3 \\
        \hline
        $\mathbb{P}(X=x)$ & $\left(\frac{1}{2}\right)^3$
                          & $3\left(\frac{1}{2} \right)^3$
                          & $3\left(\frac{1}{2} \right)^3$
                          & $\left(\frac{1}{3} \right)^3$
    \end{tabular}
    \end{center}
    
    or, as a function:
    \[
        \boxed{\mathbb{P}(X = x) = \binom{3}{x}\left(\frac{1}{2}\right)^x\left(\frac{1}{2} \right)^{3-x}}
    \]
\end{ex}

\subsection{Functions of random variables}
Let $X$ be a random variable. We can create a new random variable $Y$ as a 
function of $X$. For example, $X$ could be the number of successes out of 100
independent trials with probability $p$ of success. Let $Y$ be distance between
the number of successes out of 100 trials and the average (or expected) number
of successes. So $Y = |X - 100p|$.

If $Y = f(X)$, to get the distribution of $Y$, we can use our knowledge of the
distribution of $X$.

\[
\mathbb{P}(Y = y) = \mathbb{P}(f(X)=y) = \sum_{ \text{all }x \text{ with }f(x)=y}\mathbb{P}(X=x) = \sum_{x \text{ with } f(x) = y}\mathbb{P}(X=x)
\]

\begin{ex}
    Let $X$ be a random variable \textbf{uniformly distributed} over the set
    $\{-2,-1,0,1,2\}$.

    \begin{definition}
        \textbf{Uniformly distributed} means that all values in the set have
        equal probability.
    \end{definition}

    \begin{center}
    \begin{tabular}{c|c c c c c}
        $x$ & -2 & -1 & 0 & 1 & 2 \\
        \hline
        $\mathbb{P}(X=x)$ & $\frac{1}{5}$ &$\frac{1}{5}$&$\frac{1}{5}$
                          &$\frac{1}{5}$&$\frac{1}{5}$
    \end{tabular}
    \end{center}
    \vspace{10px}

    Let $Y = X^2$. \textbf{What is the distribution of $Y$}?

    \vspace{10px}
    \begin{center}
    \begin{tabular}{c|c c c}
        $y$ &0&1&4\\
        \hline
        $\mathbb{P}(Y=y)$&1/5&2/5&2/5
    \end{tabular}
    \end{center}
\end{ex}

\subsection{Joint distributions}
\begin{definition}
    The \textbf{joint distribution} of two random variables gives the probability
    of all possible pairs of outcomes. So, the joint distribution of $X$ and $Y$
    is a list of (or description of) the probabilities 
    $\mathbb{P}(X = x \text{ and } Y = y)$ for all possible pairs ($x$, $y$).
\end{definition}

Ultimately, a \textit{joint distribution} is a way to get $\mathbb{P}(X=x 
\text{ and } Y=y)$. Can use chart, function, etc.

\begin{ex}
    Two dice are rolled. $X$ is the value of the first die, $Y$ is the value of
    the second. \textbf{What is the joint distribution} of $X$ and $Y$?\\[.1in]
    Since we have a $\frac{1}{6} $ chance of rolling a given number, there is a
    $\frac{1}{6} $ chance of rolling that number on the second roll. Thus:
    \[
    \mathbb{P}(X=x \text{ and } Y=y) = \frac{1}{6} \cdot \frac{1}{6} = 
    \boxed{\frac{1}{36}} 
    \]
\end{ex}

\begin{ex}
    A fair coin is tossed three times. Let $X$ be the number of heads on the first
    two tosses, $Y$ the number of heads on the last two tosses. Provide the joint
    distribution of $X$ and $Y$.

    \begin{center}
    \begin{tabular}{c|c |c |c}
        &$x=0$& 1&2 \\
        \hline
        $y=0$ & 1/8&1/8&0\\
        \hline
        1&1/8&2/8&1/8\\
        \hline
        2&0&1/8&1/8
    \end{tabular}
    \end{center}
\end{ex}

\end{document}

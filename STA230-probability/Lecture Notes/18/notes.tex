\documentclass[titlepage, 12pt, leqno]{article}

% -------------------------------------------------- %
% -------------------- PACKAGES -------------------- %
% -------------------------------------------------- %
\usepackage{import}
\usepackage{pdfpages}
\usepackage{mathtools}
\usepackage{transparent}
\usepackage{xcolor}
\usepackage{tcolorbox}
\usepackage{amsmath}
\usepackage{amssymb}
\usepackage{parskip}
\usepackage{bbm}
\usepackage{multicol}
\usepackage[margin = 1in]{geometry}
\tcbuselibrary{breakable}
\tcbset{breakable = true}


% -------------------------------------------------- %
% -------------- CUSTOM ENVIRONMENTS --------------- %
% -------------------------------------------------- %
\newtcolorbox{note}{colback=black!5!white,
                          colframe=black!55!white,
                          fonttitle=\bfseries,title=Note}

\newtcolorbox{ex}{colback=blue!5!white,
                          colframe=blue!55!white,
                          fonttitle=\bfseries,title=Example}

\newtcolorbox{definition}{colback=red!5!white,
                          colframe=red!55!white,
                          fonttitle=\bfseries,title=Definition}


% -------------------------------------------------- %
% ------------------- COMMANDS --------------------- %
% -------------------------------------------------- %
% Brackets, braces, etc. 
\newcommand{\abs}[1]{\lvert #1 \rvert}
\newcommand{\bigabs}[1]{\Bigl \lvert #1 \Bigr \rvert}
\newcommand{\bigbracket}[1]{\Bigl [ #1 \Bigr ]}
\newcommand{\bigparen}[1]{\Bigl ( #1 \Bigr )}
\newcommand{\ceil}[1]{\lceil #1 \rceil}
\newcommand{\floor}[1]{\lfloor #1 \rfloor}
\newcommand{\norm}[1]{\| #1 \|}
\newcommand{\bignorm}[1]{\Bigl \| #1 \Bigr \| #1}
\newcommand{\inner}[1]{\langle #1 \rangle}
\newcommand{\set}[1]{{ #1 }}


% -------------------------------------------------- %
% -------------------- SETUP ----------------------- %
% -------------------------------------------------- %
\title{\Huge{Lecture 18 - Joint Distributions}}
\author{\large{Mitch Harrison}}
\date{\today}   
\begin{document}
\setlength{\parskip}{1\baselineskip}
\setlength{\parindent}{15pt}
\setlength{\columnsep}{0pt}
\maketitle
\tableofcontents
\newpage


% -------------------------------------------------- %
% --------------------- BODY ----------------------- %
% -------------------------------------------------- %
\section{Review}

Probabilities for $(x,y)$ jointly uniform can be calculated as a ratio of areas.

\begin{ex}
    $X$ and $Y$ are jointly uniformly distributed over a triangle with vertices
    (0,0), (1,0), and (0,1). \textbf{What is the probability} that 
    $X+Y > \frac{1}{2}$?
    \vspace{10px}
    
    The region of the triangle that makes $X+Y>\frac{1}{2}$ be true forms a
    trapezoid with a base that is the hypotenuse of the triangle. The smaller base
    of that trapezoid is the line $Y = \frac{1}{2}-X$. The probability of landing 
    in that trapezoid is:
   \begin{align*}
       \frac{ \text{whole triangle} - \text{not in trapezoid}}{ \text{
       whole triangle}} &= \frac{\frac{1}{2}-\frac{1}{8}}{\frac{1}{2}} \\
       \Aboxed{\mathbb{P}\left(X+Y>\frac{1}{2}\right) &= \frac{3}{4}} 
   \end{align*}
\end{ex}

\pagebreak
\section{Non-uniform Joint Probability Densities}
\subsection{Big example}
$X$ and $Y$ have a joint density $6d^{-2x-3y}$ for $x,y>0$. \textbf{What is the
probability} that $X<Y$?
\vspace{10px}

 We are given the following density function:
 \[
 f(x,y) = 
 \begin{cases}
     6e^{-2x-3y} & x,y>0 \\
     0 & \text{else}
 \end{cases}
 \]
First, we find our bounding region in the $XY$-plane where $X<Y$. Then we will
use double integration to find the volume of the shape above those $XY$ bounds. 
To find $X<Y$, we simple shade the region above the line $Y=X$ on the $XY$-plane.

Once we find our $XY$ bounds, we double integrate over the CDF (since we are 
looking for the volume under the CDF). For this example, we will integrate over
$y$ first, then $x$. 

To find the bounds of integration, we want all possible $Y$ values for a fixed $X$
(for which the bounds may depend on $X$, i.e. the inner bounds depend on the
outer variable). For $Y$ in this case, the lower bound is $X$ (because $(X,Y)$ is
the point where we enter the region we found in the $XY$-plane. Because our
region extends to infinity, then the upper bound of the inner integration is
$\infty$.

The outer bounds of integration will be \textit{constant}. In this case, the
range of possible $X$ values is 0 to $\infty$. Thus, those will be our outer 
bounds. We then arrive at our final integral.
\[
    \int_{0}^{\infty}\int_{y=x}^{y=\infty}6e^{-2x-3y}dydx
\]
Plugging into an integral solver, we arrive at a solution:

\[
    \boxed{\int_{0}^{\infty}\int_{y=x}^{y=\infty}6e^{-2x-3y}dydx =
    \frac{2}{5}}
\]
\pagebreak
\newgeometry{left=.5in, right = .5in}
\subsection{Comparing discrete and continuous joint distributions}
\begin{multicols}{2}
    \textbf{Discrete Joint Distributions:}
    \begin{enumerate}
        \item Probability at a point:
            \[
            \mathbb{P}(X=x, Y=y)
            \]
        \item Probability of a set:
            \[
            \mathbb{P}((X,Y) \in B) = 
            \sum_{(x,y) \in B}\mathbb{P}(X=x,Y=y)
            \]
        \item Marginals:
           \begin{align*}
               \mathbb{P}(X=x) &= \sum_{ \text{all $y$}} \mathbb{P}(X=x, Y=y) \\
               \mathbb{P}(Y=y) &= \sum_{ \text{all $x$}}\mathbb{P}(X=x,Y=y)
           \end{align*}

        \item Expected value of a function of $X$ and $Y$:
            \[
            \mathbb{E}(g(X,Y)) = \sum_{x,y}g(x,y)\mathbb{P}(X=x,Y=y)
            \]
        \item Independence:
            \[
            \mathbb{P}(X=x,Y=y) = \mathbb{P}(X=x)\mathbb{P}(Y=y)
            \]
    \end{enumerate}

\columnbreak

    \textbf{Continuous Joint Distributions:}
    \begin{enumerate}
        \item Probability density at a point:
            \[
            f(x,y) = \frac{\mathbb{P}(X \in \Delta x, Y \in \Delta y)}{
            \Delta x \Delta y}
            \]
        \item Probability of a set:
            \[
            \mathbb{P}((X,Y) \in B) = \int \int_{B}f(x,y)dydx
            \]
        \item Marginals:
           \begin{align*}
               f_X(x) &= \int_{-\infty}^{\infty}f(x,y)dy \\
               f_Y(y) &= \int_{-\infty}^{\infty}f(x,y)dx
           \end{align*}

        \item Expected value of a function of $X$ and $Y$:
            \[
            \mathbb{E}(g(X,Y)) = \int \int g(x,y)f(x,y)dydx
            \]
        \item Independence
            \[
            f(x,y) = f_X(x)f_Y(y)
            \]
    \end{enumerate}
\end{multicols}
\restoregeometry
\pagebreak
\subsection{Sample problem}
Let $T$ be the trapezoidal region defined by the vertices (0,0), (1,0), (1,1), 
and (0,2). Suppose $(X,Y)$ are jointly uniformly distributed over the region $T$.

\begin{enumerate}
    \item Find the joint density of $(X,Y)$.
        \vspace{10px}
        
        We can calculate the area of this trapezoid is $\frac{3}{2}$. Knowing 
        this, we can find the density function:
       \begin{align*}
           f(x,y) &= 
           \begin{cases}
               \frac{1}{ \text{area}} & (x,y) \in T \\
               0 & \text{else}
           \end{cases}
                  &=
                  \begin{cases}
                      \frac{2}{3} & (x,y) \in T \\
                      0 & \text{else}
                  \end{cases}
       \end{align*}
    \item Find the marginal densities $f_X(x)$ and $f_Y(y)$. In this case:
       \begin{align*}
           f_X(x) &= \int_{0}^{2-x}\frac{2}{3}dy =
           \begin{cases}
               \frac{2}{3}(2-x) & 0<x<1\\
               0 & \text{else}
           \end{cases} \\
           f_Y(y) &=
           \begin{cases}
               \int_{0}^{1}\frac{2}{3}dx & 0<y<1\\
               \int_{0}^{2-y}\frac{2}{3}dx & 1<y<2 \\
               0 & \text{else}
           \end{cases}
       \end{align*}
\end{enumerate}

\end{document}

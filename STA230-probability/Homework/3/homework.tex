\documentclass[titlepage, 12pt, leqno]{article}

% -------------------------------------------------- %
% -------------------- PACKAGES -------------------- %
% -------------------------------------------------- %
\usepackage{import}
\usepackage{pdfpages}
\usepackage{mathtools}
\usepackage{transparent}
\usepackage{xcolor}
\usepackage{tcolorbox}
\usepackage{amsmath}
\usepackage{amssymb}
\usepackage{parskip}
\usepackage[margin = 1in]{geometry}


% -------------------------------------------------- %
% -------------- CUSTOM ENVIRONMENTS --------------- %
% -------------------------------------------------- %
\newtcolorbox{note}{colback=black!5!white,
                          colframe=black!55!white,
                          fonttitle=\bfseries,title=Note}

\newtcolorbox{ex}{colback=blue!5!white,
                          colframe=blue!55!white,
                          fonttitle=\bfseries,title=Example}

\newtcolorbox{definition}{colback=red!5!white,
                          colframe=red!55!white,
                          fonttitle=\bfseries,title=Definition}


% -------------------------------------------------- %
% ------------------- COMMANDS --------------------- %
% -------------------------------------------------- %
% Brackets, braces, etc. 
\newcommand{\abs}[1]{\lvert #1 \rvert}
\newcommand{\bigabs}[1]{\Bigl \lvert #1 \Bigr \rvert}
\newcommand{\bigbracket}[1]{\Bigl [ #1 \Bigr ]}
\newcommand{\bigparen}[1]{\Bigl ( #1 \Bigr )}
\newcommand{\ceil}[1]{\lceil #1 \rceil}
\newcommand{\floor}[1]{\lfloor #1 \rfloor}
\newcommand{\norm}[1]{\| #1 \|}
\newcommand{\bignorm}[1]{\Bigl \| #1 \Bigr \| #1}
\newcommand{\inner}[1]{\langle #1 \rangle}
\newcommand{\set}[1]{{ #1 }}


% -------------------------------------------------- %
% -------------------- SETUP ----------------------- %
% -------------------------------------------------- %
\title{\Huge{Homework 3}}
\author{\large{Mitch Harrison}}
\date{\today}   
\begin{document}
\setlength{\parskip}{1\baselineskip}
\setlength{\parindent}{15pt}
\maketitle
\tableofcontents
\newpage


% -------------------------------------------------- %
% --------------------- BODY ----------------------- %
% -------------------------------------------------- %
\section{Question 1}
\textit{(a)} 

There are only four possible ways of arranging the boxes $A/B$ with the red $R$ balls and green $G$ balls:
\begin{align*}
    \{R/GGGG\} \\
    \{RG/GGG\} \\
    \{RGG/GG\} \\
    \{RGGG/G\}
\end{align*}


\textit{(b)} 

To give my friend the worst possible chance of success, I would use a $\{RGG/GG\}$ configuration, which gives the following probability of success:
\begin{align*}
    \mathbb{P}\left(B_1 \;\middle|\; \text{green}\right) &= \mathbb{P}(B_1) \mathbb{P}\left(\text{green} \;\middle|\; B_1\right) + \mathbb{P}(B_2) \mathbb{P}\left(\text{green} \;\middle|\; B_2\right) \\
                                                         &= \frac{1}{2} \left(\frac{1}{3} \right) + \frac{1}{2} \left(\frac{1}{2} \right) \\
                                                         &= \frac{1}{6} + \frac{1}{4} \\
\Aboxed{\mathbb{P}\left(B_1 \;\middle|\; \text{green}\right) &= \frac{5}{12}} 
\end{align*}

\textit{(c)}

Yes. If we place the red ball alone in a box $B_r$, and all green balls in the second box $B_g$, then as long as my friend is playing optimally, he will always be correct in his guesses, regardless of color, and:
\[
\boxed{ \mathbb{P}\left(\text{Correct} \;\middle|\; \text{red}\right) = \mathbb{P}\left(\text{Correct} \;\middle|\; \text{green}\right) = 1} 
\]
\pagebreak
\section{Question 2}
Let:
\begin{align*}
    T &= \text{test positive} \\
    A &= \text{has Lyme disease}
\end{align*}
\textit{(a)} 

The probability of a positive test result regardless of having Lyme is as follows:
\begin{align*}
    \mathbb{P}(T) &= \mathbb{P}\left(T \;\middle|\; A\right)\mathbb{P}(A) + \mathbb{P}\left(T \;\middle|\; A^\complement \right)\mathbb{P}(A^\complement) \\
                  &= \frac{9}{10} \left(\frac{6}{100,000} \right) + \frac{3}{100} \left(\frac{99,994}{100,000}\right) \\
    \Aboxed{\mathbb{P}(T) &\approx .030052} 
\end{align*}

\textit{(b)}
The probability of a randomly selected person receiving a false negative test is as follows:
\begin{align*}
    \mathbb{P}(A \cap T^\complement ) &= \mathbb{P}(A) \mathbb{P}\left(T^\complement  \;\middle|\; A\right) \\
                                      &= \frac{6}{100,000} \left(\frac{1}{10} \right) \\
    \Aboxed{\mathbb{P}(A \cap T^\complement) &= .000006} 
\end{align*}

\textit{(c)} 

Given a positive test, the probability that a person has Lyme is:
\begin{align*}
    \mathbb{P}\left(A \;\middle|\; T\right) &= \frac{ \mathbb{P}\left(A \;\middle|\; T\right) }{ \mathbb{P}\left(A \;\middle|\; T\right) + \mathbb{P}\left(A^\complement  \;\middle|\; T\right) } \\
                                            &= \frac{.00006 \cdot .9}{(.00006 \cdot .9)+(.99994 \cdot .03)} \\
    \Aboxed{\mathbb{P}\left(A \;\middle|\; T\right) &\approx .001796}
\end{align*}

\textit{(d)} 
The doctor estimates $\mathbb{P}(A) = 0.9$. Correcting for the negative test, we arrive at the following probability that the test was a false negative:
\begin{align*}
    \mathbb{P}\left(A \;\middle|\; T^\complement \right) &= \frac{\mathbb{P}(A) \mathbb{P}\left(T^\complement  \;\middle|\; A\right) }{\mathbb{P}(A) \mathbb{P}\left(T^\complement  \;\middle|\; A\right) + \mathbb{P}(\neg A) \mathbb{P}\left(T^\complement  \;\middle|\; \neg A\right) } \\
                                                         &= \frac{.9 \cdot .1}{(.9 \cdot .1)+(.97 \cdot .1)} \\
    \Aboxed{ \mathbb{P}\left(A \;\middle|\; T^\complement \right) &\approx 0.481283} 
\end{align*}

\pagebreak
\section{Question 3}
\[
    \boxed{C = \text{ the second roll is greater than or equal to 5}} 
\]
\pagebreak
\section{Question 4}
The first card drawn has a 100\% chance of being unique. In a deck of $n$ cards, each subsequent draw has a $\frac{n-3}{n-1}$ chance of being unique, since each card has two duplicates and one card is removed (without replacement) per draw. Thus, we arrive at the following probability of uniqueness:
\begin{align*}
    \mathbb{P}(\text{All unique}) &= \frac{60}{60} \cdot \frac{57}{59} \cdot \frac{54}{58} \cdot \frac{51}{57} \cdot \frac{48}{56} \\
                                  &= \frac{54 \cdot 51 \cdot 48}{59 \cdot 58 \cdot 56} \\
    \mathbb{P}(\text{All unique}) &\approx 0.6898
\end{align*}

Since we know that $\mathbb{P}(\text{duplicates}) = 1 - \mathbb{P}(\text{all unique})$, we arrive at the probability of having duplicates:
\[
    \boxed{\mathbb{P}(\text{duplicates}) \approx 0.3102}
\]
\pagebreak
\section{Question 5}

Let:
\begin{align*}
    A &= \text{made 1st shot} \\
    M_{i,j} &= \text{made } i \text{ out of } j \text{ shots}
\end{align*}

\textit{(a)} 
We are looking for $ \mathbb{P}\left(A \;\middle|\; M_{30,40}\right) $, which we transform into the following fraction:
\[
    \frac{\mathbb{P}(A \cap M_{30,40})}{\mathbb{P}(M_{30,40})} 
\]
To calculate the numerator, we use the multiplication rule:
\begin{align*}
    \mathbb{P}(A \cap M_{30,40}) &= \mathbb{P}(A) \mathbb{P}\left(M_{30,40} \;\middle|\; A\right) \\
                                 &= \mathbb{P}(A)\mathbb{P}(M_{29,30})
\end{align*}

Using the binomial coefficient formula, we calculate:
\begin{align*}
    \mathbb{P}\left(A \;\middle|\; M_{30,40}\right) &= \frac{.53(.0034)}{.0024} \\
    \Aboxed{\mathbb{P}\left(A \;\middle|\; M_{30,40}\right) &\approx .7508} 
\end{align*}

\textit{(b)} 

Here, we will define a new event $F_{i,j}$ where Shaq makes the first $i$ shots out of $j$. So:
\[
    \mathbb{P}(\text{made at least 4}) = \mathbb{P}(F_{4,5}) + \mathbb{P}(F_{5,5})
\]

Calculating these individually, we find:
\begin{align*}
    \mathbb{P}(F_{4,5}) &= \mathbb{P}(A)\mathbb{P}(F_{4,5} | A)\mathbb{P}(M_{30,40} | F_{4,5} \cap A) \\
                        &= .53\mathbb{P}(F_{3,4})\mathbb{P}(M_{26,35})
\end{align*}

Using the binomial coefficient formula, we calculate:
\begin{align*}
    \mathbb{P}\left(F_{4,5} \;\middle|\; A \cap M_{30,40}\right) &= .53 \cdot .2799 \cdot .0054 \\
                                                                 &= .0008
\end{align*}

With similar reasoning, we calculate:
\begin{align*}
    \mathbb{P}\left(F_{5,5} \;\middle|\; A \cap M_{30,40}\right) &= .53 \cdot .0789 \cdot .0124 \\
    \mathbb{P}\left(F_{5,5} \;\middle|\; A \cap M_{30,40}\right) &= .0005
\end{align*}

Adding these, we arrive at a total solution:
\[
    \boxed{\mathbb{P}(\text{made at least 4 of first 5} | A \cap M_{30,40}) = .0013} 
\]

\pagebreak 
\section{Question 6}
My work partners were Aseda Asomani and Blake Morris.

\pagebreak
\section{Question 7}
To generalize, we will call the doors $D_i, D_j, D_k$. We will select $D_i$ as our initial guess.
\begin{align*}
    &\mathbb{P}(D_i) = \frac{1}{3} \\
    &\mathbb{P}(D_j \cup D_k) = \mathbb{P}(D_j) + \mathbb{P}(D_k) = \frac{2}{3} 
\end{align*}

Once we reveal a door ($D_k$ for example), we are left with the following:
\begin{align*}
    &\mathbb{P}\left(D_j \cap D_k \;\middle|\; \text{not }D_k\right) = \frac{2}{3} \\ 
    &\mathbb{P}(D_j) = \frac{2}{3} 
\end{align*}

Thus, the odds of success from switching becomes $\mathbb{P}(D_j \cap D_k)$, increasing from $\frac{1}{3}$ to $\frac{2}{3}$.

\textit{(b)}

No matter the reasoning for which door was opened, if there are $n$ doors, we choose one, and the host reveals $n-2$ goats, the odds of success from switching are $\frac{n-1}{n}$. 

\end{document}

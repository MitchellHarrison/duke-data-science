\documentclass[titlepage, 12pt, leqno]{article}

% -------------------------------------------------- %
% -------------------- PACKAGES -------------------- %
% -------------------------------------------------- %
\usepackage{import}
\usepackage{pdfpages}
\usepackage{mathtools}
\usepackage{transparent}
\usepackage{xcolor}
\usepackage{tcolorbox}
\usepackage{amsmath}
\usepackage{amssymb}
\usepackage{parskip}
\usepackage{bbm}
\usepackage[margin = 1in]{geometry}


% -------------------------------------------------- %
% -------------- CUSTOM ENVIRONMENTS --------------- %
% -------------------------------------------------- %
\newtcolorbox{note}{colback=black!5!white,
                          colframe=black!55!white,
                          fonttitle=\bfseries,title=Note}

\newtcolorbox{ex}{colback=blue!5!white,
                          colframe=blue!55!white,
                          fonttitle=\bfseries,title=Example}

\newtcolorbox{definition}{colback=red!5!white,
                          colframe=red!55!white,
                          fonttitle=\bfseries,title=Definition}


% -------------------------------------------------- %
% ------------------- COMMANDS --------------------- %
% -------------------------------------------------- %
% Brackets, braces, etc. 
\newcommand{\abs}[1]{\lvert #1 \rvert}
\newcommand{\bigabs}[1]{\Bigl \lvert #1 \Bigr \rvert}
\newcommand{\bigbracket}[1]{\Bigl [ #1 \Bigr ]}
\newcommand{\bigparen}[1]{\Bigl ( #1 \Bigr )}
\newcommand{\ceil}[1]{\lceil #1 \rceil}
\newcommand{\floor}[1]{\lfloor #1 \rfloor}
\newcommand{\norm}[1]{\| #1 \|}
\newcommand{\bignorm}[1]{\Bigl \| #1 \Bigr \| #1}
\newcommand{\inner}[1]{\langle #1 \rangle}
\newcommand{\set}[1]{{ #1 }}


% -------------------------------------------------- %
% -------------------- SETUP ----------------------- %
% -------------------------------------------------- %
\begin{document}
\setlength{\parskip}{1\baselineskip}
\setlength{\parindent}{15pt}
\newpage


% -------------------------------------------------- %
% --------------------- BODY ----------------------- %
% -------------------------------------------------- %
\section{Question 1}

\textit{(A)}
\[
\boxed{ \mathbb{E}( \text{first}) = \sum_{k=1}^{13}\frac{1}{13}k = 7} 
\]
\textit{(B)}
\begin{align*}
    \mathbb{E}( \text{sum}) &= \sum_{k=1}^{13} \mathbb{E}(X_k) \\
                            &= 13 \mathbb{E}(X_1) \\
                            &= 13 \times 7 \\
                        \Aboxed{\mathbb{E}( \text{sum}) = 91}
\end{align*}

\textit{(C)}

Let $X = \# \text{ of kings}$:
\begin{align*}
    \mathbb{E}(X) &= \sum_{x=0}^{4}x \mathbb{P}(X=x) \\
                  &= 0(\mathbb{P}(X=0)) + 1\left(\frac{\binom{4}{1}
                      \binom{48}{12}}{\binom{52}{13}}\right) +
                        2\left(\frac{\binom{4}{2}
                      \binom{48}{11}}{\binom{52}{13}}\right) +
                        3\left(\frac{\binom{4}{3}
                      \binom{48}{10}}{\binom{52}{13}}\right) +
                        4\left(\frac{\binom{4}{4}
                      \binom{48}{9}}{\binom{52}{13}}\right) \\
    \Aboxed{ \mathbb{E}(X) &= 1} 
\end{align*}

\pagebreak
\section{Question 2}
Let $X = \# $ of players remaining:
\begin{align*}
    \mathbb{E}(X) &= 12 \mathbb{E}( \mathbbm{1}_{i}) \\
    \mathbb{E}( \mathbbm{1}_{1}) &=\mathbb{P}(\text{person }i 
    \text{ has unique sum}) \\
                                 &= \frac{1}{36}\left(1-
                                     \frac{1}{36}\right)^{11}+\cdots+
                      \frac{6}{36}\left(1-\frac{6}{36}\right)^{11} + 
                      \frac{5}{36}\left(1-\frac{5}{36}\right)^{11} + \cdots +
                  \frac{1}{36}\left(1-\frac{1}{36}\right)^{11}\\
\end{align*}
Thus, 
\begin{align*}
    \mathbb{E}(X) &\approx 12\times 0.3008 \\
    \Aboxed{ \mathbb{E}(X) &\approx 3.61} 
\end{align*}

\pagebreak
\section{Question 3}
\textit{(A)}

Let $X = \# $ of unique rolls:
\begin{align*}
    \mathbb{E}(X) &= \mathbb{E}( \mathbbm{1}_{1} + \cdots + \mathbbm{1}_{k}) \\
                  &= k \mathbb{E}( \mathbbm{1}_{1})
\end{align*}

Solving for $ \mathbb{E}( \mathbbm{1}_{1})$, we find:
\begin{align*}
    \mathbb{E}( \mathbbm{1}_{1}) &= 1- \mathbb{P}( \text{color doesn't appear}) \\
                                 &= 1 - \binom{n}{0} \left(\frac{1}{k}\right)^0
                                 \left(1-\frac{1}{k}\right)^n \\
                                 &= 1 - \left(1-\frac{1}{k}\right)^n
\end{align*}

Therefore,
\[
    \boxed{ \mathbb{E}(X) = k\left[1-\left(1-\frac{1}{k}\right)^n\right]} 
\]
\textit{(B)}

\[
    \lim_{n \to \infty} \mathbb{E}(X) = \lim_{n \to \infty} k\left[
    1-\left(1-\frac{1}{k}\right)^n\right] = \boxed{k}
\]
\textit{(C)}

Here, $ \mathbb{E}(X) = k \mathbb{E}( \mathbbm{1}_{i})$. Solving for $ 
\mathbb{E}( \mathbbm{1}_{i})$:
\begin{align*}
    \mathbb{E}( \mathbbm{1}_{i}) &= 1 - \mathbb{P}( \text{color doesn't appear})\\
                                 &= 1 - \left[\frac{
                                     \binom{N}{0}\binom{(k-
                             1)N}{n}}{\binom{NK}{n}}\right] \\
\end{align*}

Therefore:
\[
\boxed{ \mathbb{E}(X) = k\left(1 - \left[
    \frac{\binom{N}{0}\binom{(k-1)N}{n}}{\binom{NK}{n}}\right]\right)} 
\]
\pagebreak
\section{Question 4}
My work partners were Aseda Asomani and Victor Crespo.
\end{document}

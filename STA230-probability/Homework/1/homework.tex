\documentclass[titlepage, 12pt, leqno]{article}

% -------------------------------------------------- %
% -------------------- PACKAGES -------------------- %
% -------------------------------------------------- %
\usepackage{import}
\usepackage{mathtools}
\usepackage{pdfpages}
\usepackage{transparent}
\usepackage{xcolor}
\usepackage{tcolorbox}
\usepackage{amsmath}
\usepackage{amssymb}
\usepackage{parskip}
\usepackage[margin = 1in]{geometry}


% -------------------------------------------------- %
% -------------- CUSTOM ENVIRONMENTS --------------- %
% -------------------------------------------------- %
\newtcolorbox{note}{colback=black!5!white,
                          colframe=black!55!white,
                          fonttitle=\bfseries,title=Note}

\newtcolorbox{ex}{colback=blue!5!white,
                          colframe=blue!55!white,
                          fonttitle=\bfseries,title=Example}

\newtcolorbox{definition}{colback=red!5!white,
                          colframe=red!55!white,
                          fonttitle=\bfseries,title=Definition}


% -------------------------------------------------- %
% ------------------- COMMANDS --------------------- %
% -------------------------------------------------- %
% Brackets, braces, etc. 
\newcommand{\abs}[1]{\lvert #1 \rvert}
\newcommand{\bigabs}[1]{\Bigl \lvert #1 \Bigr \rvert}
\newcommand{\bigbracket}[1]{\Bigl [ #1 \Bigr ]}
\newcommand{\bigparen}[1]{\Bigl ( #1 \Bigr )}
\newcommand{\ceil}[1]{\lceil #1 \rceil}
\newcommand{\floor}[1]{\lfloor #1 \rfloor}
\newcommand{\norm}[1]{\| #1 \|}
\newcommand{\bignorm}[1]{\Bigl \| #1 \Bigr \| #1}
\newcommand{\inner}[1]{\langle #1 \rangle}
\newcommand{\set}[1]{{ #1 }}


% -------------------------------------------------- %
% -------------------- SETUP ----------------------- %
% -------------------------------------------------- %
\title{\Huge{Homework 1}}
\author{\large{Mitch Harrison}}
\date{\today}   
\begin{document}
\setlength{\parskip}{1\baselineskip}
\setlength{\parindent}{15pt}
\setlength{\jot}{10pt}
\maketitle
\newpage


% -------------------------------------------------- %
% --------------------- BODY ----------------------- %
% -------------------------------------------------- %
\section{Question 1}
\textit{(a)} 
\[
    \int_{1}^{\infty} \frac{c}{x^n}dx = c \int_{1}^{\infty} x^{-n}dx = 1 \quad \text{and} 
    \quad n > 1
\]
\begin{align*}
    \int_{1}^{\infty}x^{-n}dx &= \lim_{t \to \infty} \int_{1}^{t} x^{-n}dx \\
    &= \lim_{t \to \infty} \bigbracket{\frac{x^{1-n}}{1-n}}_1^t \quad n > 1 \\
    &= \lim_{t \to \infty} \bigbracket{\frac{\frac{1}{x^{n-1}}}{1-n}}_1^t \\
    &= \lim_{t \to \infty} \bigbracket{\frac{1}{(1-n)t^{n-1}} - \frac{1}{1-n}} \\
    &= \lim_{t \to \infty} \bigbracket{\frac{1}{1-n}\bigparen{\frac{1}{t^{n-1}} - 1}} \\
    &= -\frac{1}{1-n} \\
    \int_{1}^{\infty}x^{-n}dx &= \frac{1}{n-1} \\
    \therefore \frac{c}{n-1} &= 1 \\
    \Aboxed{c &= n-1}
\end{align*}

\pagebreak
\textit{(b)} 
\[
    \sum_{k=2}^{\infty}\frac{2^k}{5\cdot k!} =\frac{1}{5}\sum_{k=2}^{\infty}\frac{2^k}{k!}
\]
Here, $\sum_{k=2}^{\infty}\frac{2^k}{k!}$ is similar to the Taylor Series representation of $e^x$, but $x = 2$ and the series does not start at $k=0$. Thus:
\begin{align*}
    \sum_{k=2}^{\infty}\frac{2^k}{k!} &= e^{2} - \sum_{k=0}^{1}\frac{2^k}{k!} \\
    &= e^{2} - 3 \\
    \Aboxed{\sum_{k=2}^{\infty}\frac{2^k}{5 \cdot k!} &= \frac{e^{2} - 3}{5}}
\end{align*}

\textit{(c)} 
\[
    \sum_{k=1}^{\infty}3p^{2k+1} = \sum_{k=0}^{\infty}3p^{2k+1}-3p
\]

\[
    3p^{2k+1} = (3p)(p^2)^k
\]

\[
    \sum_{k=0}^{\infty}3p(p^2)^k = \frac{3p}{1-p^2} 
\]

\[
    \boxed{\sum_{k=1}^{\infty}3p^{2k+1} = \frac{3p}{1-p^2} - 3p} 
\]
\pagebreak

\textit{(d)} 
\begin{align*}
    e^{-y} &= \frac{1}{\frac{dy}{dx}} \\
    \frac{1}{e^y} &= \frac{dx}{dy} \\
    e^y &= \frac{dy}{dx} \\
    dx &= e^{-y}dy
\end{align*}

\[
    \int e^{-y}dy = -e^{-y} + c_1 \quad \int dx = x + c_2
\]

\begin{align*}
    -e^{-y} &= x + c \\
    e^{-y} &= -x - c \\
    -y &= \ln(-x-c) \\
    y &= -\ln(-x-c)
\end{align*}

We know $y(0) = 0$, therefore:
\begin{align*}
    0 &= \ln(0-c) \\
    &= \ln(-c) \\
    e^0 &= e^{\ln(-c)} \\
    1 &= -c \\
    -1 &= c
\end{align*}

Now that $c = -1$ is known, we find a general solution:
\[
    \boxed{f(x) = -\ln(-x+1)} 
\]
\pagebreak

\section{Question 2}
The set of all possible sums of two dice rolls is as follows:
\[
    \boxed{\Omega = \{2,3,4,5,6,7,8,9,10,11,12\}}
\]
These are not all equally likely. For example, there is only one possible combination of dice rolls that sum to $2$ or $12$, but more than one for $7$ ($6+1$, $4+3$, etc.).
\pagebreak

\section{Question 3}
\setlength{\jot}{5pt}
\begin{align*}
    \omega = \{&(1,1), (1,2), (1,3), (1,4), (1,5), (1,6), \\
           &(2,1), (2,2), (2,3), (2,4), (2,5), (2,6), \\
           &(3,1), (3,2), (3,3), (3,4), (3,5), (3,6), \\
           &(4,1), (4,2), (4,3), (4,4), (4,5), (4,6), \\
           &(5,1), (5,2), (5,3), (5,4), (5,5), (5,6), \\
           &(6,1), (6,2), (6,3), (6,4), (6,5), (6,6) \}
\end{align*}

\textit{(a)} 
\setlength{\jot}{5pt}
\begin{align*}
    A = \{&(1,1), (1,2), (1,3), (1,4), (1,5), (1,6), \\
          &(2,1), (2,2), (2,3), (2,4), (2,5), (2,6), \\
          &(3,1), (3,2), (3,3), (3,4), (3,5), (3,6)\}
\end{align*}
\begin{align*}
    \mathbb{P}(A) &= \frac{\#A}{\#\Omega} = \frac{18}{36} \\
    \Aboxed{\mathbb{P}(A) &= \frac{1}{2}}
\end{align*}

\textit{(b)} 
\[
    B = \{(2,6), (3,5), (4,4), (5,3), (6,2)\}
\]
\begin{align*}
    \mathbb{P}(B) &= \frac{\#B}{\#\Omega} = \frac{5}{36} \\
    \Aboxed{\mathbb{P}(B) &= \frac{5}{36}}
\end{align*}

\textit{(c)} 
\begin{align*}
     A \cup B = \{&(1,1), (1,2), (1,3), (1,4), (1,5), (1,6), \\
             &(2,1), (2,2), (2,3), (2,4), (2,5), (2,6), \\
             &(3,1), (3,2), (3,3), (3,4), (3,5), (3,6), \\
             &(4,4), (5,3), (6,2)\}
\end{align*}

\begin{align*}
    \mathbb{P}(A \cup B) &= \frac{\#(A \cup B)}{\#\Omega} = \frac{21}{36} \\
    \Aboxed{\mathbb{P}(A \cup B) &= \frac{7}{12}}
\end{align*}

\textit{(d)} 
\[
    A \cap B = \{(2,6), (3,5)\}
\]
\begin{align*}
    \mathbb{P}(A \cap B) &= \frac{\#(A \cap B)}{\#\Omega} = \frac{2}{36} \\
    \Aboxed{\mathbb{P}(A \cap B) &= \frac{1}{18}}
\end{align*}


\textit{(e)} 
\begin{align*}
    B^\complement = \{&(1,1), (1,2), (1,3), (1,4), (1,5), (1,6), \\
          &(2,1), (2,2), (2,3), (2,4), (2,5), \\
          &(3,1), (3,2), (3,3), (3,4), (3,6), \\
          &(4,1), (4,2), (4,3), (4,5), (4,6), \\
          &(5,1), (5,2), (5,4), (5,5), (5,6), \\
          &(6,1), (6,3), (6,4), (6,5), (6,6) \} \\[.1in]
    A \cap B^\complement = \{&(1,1), (1,2), (1,3), (1,4), (1,5), (1,6), \\
             &(2,1), (2,2), (2,3), (2,4), (2,5), \\
             &(3,1), (3,2), (3,3), (3,4), (3,6)\}
\end{align*}

\begin{align*}
    \mathbb{P}(A \cap B^\complement) &= \frac{\#(A \cap B^\complement)}{\#\Omega} = \frac{16}{36}  \\
    \Aboxed{\mathbb{P}(A \cap B^\complement) &= \frac{4}{9}}
\end{align*}

\textit{(f)} 
\begin{align*}
    C = \{&(1,1), (1,3), (1,5), \\
          &(3,1), (3,3), (3,5), \\
          (5,1), (5,3), (5,5)\}
\end{align*}
\begin{align*}
    A \cap B \cap C &= \{(3,5)\} \\
    \mathbb{P}(A \cap B \cap C) &= \frac{\#(A \cap B \cap C)}{\#\Omega} = \frac{1}{36} \\
    \Aboxed{\mathbb{P}(A \cap B \cap C) &= \frac{1}{36}}
\end{align*}
\pagebreak

\textit{(g)} 
\begin{align*}
    A \cup A^\complement &= \Omega \\
    \mathbb{P}(\Omega) &= 1 \\
    \Aboxed{\mathbb{P}(A \cup A^\complement) &= 1}
\end{align*}
\pagebreak

\section{Question 4}
My work partner was Aseda Asomani.

\end{document}

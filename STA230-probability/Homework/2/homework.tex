\documentclass[titlepage, 12pt, leqno]{article}

% -------------------------------------------------- %
% -------------------- PACKAGES -------------------- %
% -------------------------------------------------- %
\usepackage{import}
\usepackage{pdfpages}
\usepackage{mathtools}
\usepackage{transparent}
\usepackage{xcolor}
\usepackage{tcolorbox}
\usepackage{amsmath}
\usepackage{amssymb}
\usepackage{parskip}
\usepackage[margin = 1in]{geometry}
\usepackage{graphicx}

\graphicspath{{./images/}}


% -------------------------------------------------- %
% -------------- CUSTOM ENVIRONMENTS --------------- %
% -------------------------------------------------- %
\newtcolorbox{note}{colback=black!5!white,
                          colframe=black!55!white,
                          fonttitle=\bfseries,title=Note}

\newtcolorbox{ex}{colback=blue!5!white,
                          colframe=blue!55!white,
                          fonttitle=\bfseries,title=Example}

\newtcolorbox{definition}{colback=red!5!white,
                          colframe=red!55!white,
                          fonttitle=\bfseries,title=Definition}


% -------------------------------------------------- %
% ------------------- COMMANDS --------------------- %
% -------------------------------------------------- %
% Brackets, braces, etc. 
\newcommand{\abs}[1]{\lvert #1 \rvert}
\newcommand{\bigabs}[1]{\Bigl \lvert #1 \Bigr \rvert}
\newcommand{\bigbracket}[1]{\Bigl [ #1 \Bigr ]}
\newcommand{\bigparen}[1]{\Bigl ( #1 \Bigr )}
\newcommand{\ceil}[1]{\lceil #1 \rceil}
\newcommand{\floor}[1]{\lfloor #1 \rfloor}
\newcommand{\norm}[1]{\| #1 \|}
\newcommand{\bignorm}[1]{\Bigl \| #1 \Bigr \| #1}
\newcommand{\inner}[1]{\langle #1 \rangle}
\newcommand{\set}[1]{{ #1 }}


% -------------------------------------------------- %
% -------------------- SETUP ----------------------- %
% -------------------------------------------------- %
\title{\Huge{Homework 2}}
\author{\large{Mitch Harrison}}
\date{\today}   
\begin{document}
\setlength{\parskip}{1\baselineskip}
\setlength{\parindent}{15pt}
\maketitle
\tableofcontents
\newpage


% -------------------------------------------------- %
% --------------------- BODY ----------------------- %
% -------------------------------------------------- %
\section{Question 1}

\[
\Omega = \{a,b,c,d,e,f,g,h\}
\]
A \textit{partition} of $\Omega$ is any collection of disjoint subsets that combine to form $\Omega$. For example:
\begin{align*}
    \Omega_1 &= \{a,d,c,d\} \\
    \Omega_2 &= \{e,f,g,h\}
\end{align*}
\[
    \boxed{\Omega_1 \cup \Omega_2 = \Omega} 
\]
\pagebreak
\section{Question 2}
A \textit{distribution} is a function that takes an event as input and returns the probability of that event occurring. \textbf{Option \textit{(a)} is is a distribution} because the output of the function relies on the input value of $A$. However, \textbf{option \textit{(b)} is not a distribution} because the output is always $\frac{1}{5}$, regardless of the event given as input.

\pagebreak
\section{Question 3}
\textit{(a)} $\boxed{10!}$ \\[.1in]
\textit{(b)} 

We begin by taking a line  of 10 people and forming it into a circle. For each of all $10!$ permutations of that line, there are 10 possible identical circle permutations, one for each position that circle can be rotated to. Therefore, the possible circles are as follows:
\[
    \frac{10!}{10} = \boxed{9! \text{ possible circles}}
\]
\textit{(c)} 

We know that Ana must be among the selectees (meaning $\binom{1}{1}$), leaving 9 possible choices for the remaining 3 selectees. Thus, we are left with the following number of possible combinations:
\begin{align*}
    \binom{1}{1} \binom{9}{3} &= \frac{9 \cdot 8 \cdot 7}{3 \cdot 2 \cdot 1} \\
                              &= \boxed{84 \text{ possible groups}}
\end{align*}

\pagebreak
\section{Question 4}
There are two slots in the 6-digit passcode for duplicates. There are $\binom{6}{2}$ possible slots to choose. For the remaining 4 slots, each much be unique. Thus, we find the total number of combinations as follows:
\begin{align*}
    \binom{6}{2} \frac{9!}{5!} &= \left(\frac{6 \cdot 5}{2}\right) \left(9 \cdot 8 \cdot 7 \cdot 6\right) \\
                               &= 15 \cdot 9 \cdot 8 \cdot 7 \cdot 6 \\
                               &= \boxed{45360 \text{ possible passcodes}} 
\end{align*}

\pagebreak
\section{Question 5}
The outcome space for two dice rolls is as follows:
\begin{align*}
    \Omega = \{&(1,1), (1,2), (1,3), (1,4), (1,5), (1,6), \\
           &(2,1), (2,2), (2,3), (2,4), (2,5), (2,6), \\
           &(3,1), (3,2), (3,3), (3,4), (3,5), (3,6), \\
           &(4,1), (4,2), (4,3), (4,4), (4,5), (4,6), \\
           &(5,1), (5,2), (5,3), (5,4), (5,5), (5,6), \\
           &(6,1), (6,2), (6,3), (6,4), (6,5), (6,6) \}
\end{align*}
To arrive at a probability mass function, we count the occurrences for each digit where that digit is the minimum value of the two rolls. By reducing fractions, we arrive at the following function:
\renewcommand{\arraystretch}{1.5}
\begin{center}
    \begin{tabular}{ c|c|c|c|c|c|c| }
        $A$ & 1 & 2 & 3 & 4 & 5 & 6 \\
        \hline
        $\mathbb{P}(A)$ & $\frac{11}{36}$ & $\frac{1}{4}$ & $\frac{7}{36}$ & $\frac{5}{36}$ & $\frac{1}{12}$ & $\frac{1}{36}$ 
    \end{tabular}
\end{center}

\pagebreak
\section{Question 6}
When flipping a fair coin thrice, we are given the following events:
\[
    A = \text{Flip at least 2 heads} \quad B = \text{Exactly 2 flips are heads}
\]
\textit{(a)}

Given that at least two flips are heads (event $A$), the only way for event $B$ to be true is if the outcome is contained in the following event:
\[
    \{HHT,\, HTH,\, THH\}
\]
Since there is only one possible event that makes $B$ false given that $A$ occurred ($HHH$), we arrive at the conditional probability:
\[
    \boxed{\mathbb{P}\left(B \;\middle|\; A\right) = \frac{3}{4}}
\]
\textit{(b)}

Since $A$ must be true if $B$ is true, we know that $B \subseteq A$, and therefore:
\[
    \boxed{\mathbb{P}\left(A\;\middle|\;B\right) = 1}
\]
\pagebreak
\section{Question 7}
The only possible coordinates from which the ant can reach (4,2) are (4,1) and (3,2). Since the probability for each possible step is $\frac{1}{2}$, we arrive at the following:
\begin{align*}
    \mathbb{P}(\text{Arrives at }(4,2)) &= \frac{1}{2}\mathbb{P}((4,1)) + \frac{1}{2} \mathbb{P}((3,2)) \\
                                        &= \frac{1}{2} \left(\frac{5}{32}\right) + \frac{1}{2} \left(\frac{10}{32} \right) \\
                                        &= \frac{5}{64} + \frac{10}{64} \\
    \Aboxed{\mathbb{P}(\text{Arrives at } (4,2)) &= \frac{15}{64}}
\end{align*}


\pagebreak
\section{Question 8}
Let:
\begin{align*}
    A &= {C_1 \text{ works}} \\
    B &= {C_2 \text{ works}}
\end{align*}
and,
\begin{align*}
    \mathbb{P}(A) &= .9 \\
    \mathbb{P}(B) &= .8 \\
    \mathbb{P}\left(B \;\middle|\; A^\complement \right) &= .4
\end{align*}

\textit{(a)} 

Since We know that the probability of $C_2$ functioning only changes in the event that $C_1$ fails, thus if $C_1$ is functional, the probability that $C_2$ is:
\begin{align*}
    \mathbb{P}\left(B \;\middle|\; A\right) &= \mathbb{P}(B) \\
                                            &= \boxed{80\%}
\end{align*}

\textit{(b)} 

Given that only one component needs to fail, the only possible chance of the system functioning is that both componenents remain working. The chances of both components functioning on a given day is given by the following:
\begin{align*}
    \mathbb{P}(\text{system functions}) &= \mathbb{P}(A)\mathbb{P}(B) \\
                                       &= .9(.8) \\
    &= \boxed{72\%} 
\end{align*}

\textit{(c)} 

Given that the overall system failed, there are three possible causes. First, $C_1$ failed and $C_2$ did not. Second, $C_2$ failed and $C_1$ did not. Third, both $C_1$ and $C_2$ failed. The probability of each is as follows:
\begin{align*}
    \mathbb{P}(C_1\text{ failed}) &= 1 - .9 = 10\% \\
    \mathbb{P}(C_2\text{ failed}) &= .1 \cdot .4 = 4\% \\
    \mathbb{P}(\text{both failed}) &= .1 \cdot .6 = 6\%
\end{align*}

This leaves a 20\% total chance of failure under normal conditions. However, we know that the system failed, and thus we multiply each probability by 5 to arrive at 100\% total. Thus:
\begin{align*}
    \mathbb{P}(C_1\text{ failed}) &= 50\% \\
    \mathbb{P}(C_2\text{ failed}) &= 20\% \\
    \mathbb{P}(\text{both failed}) &= 30\%
\end{align*}

And we see our final probability that $C_1$ is still operational:
\[
    \boxed{\mathbb{P}(\text{only }C_2 \text{ failed}) = 20\%}
\]
\pagebreak
\section{Question 9}
Since $A$ and $B$ are independent,
\[
    \mathbb{P}\left(A \;\middle|\; B\right) = \mathbb{P}(A) = \frac{1}{7} 
\]
and,
\[
    \mathbb{P}(A \cap B) = \mathbb{P}(A)\mathbb{P}(B)
\]
Since there are 14 possible outcomes in $\Omega$, each with a $\frac{1}{14}$ chance of occurring:
\begin{align*}
    \mathbb{P}(A \cap B) &= \frac{1}{7} \cdot \mathbb{P}(B) = \frac{1}{14} \\
    \Aboxed{\mathbb{P}(B) &= \frac{1}{2}} 
\end{align*}
\pagebreak
\section{Question 10}
Let,
\begin{align*}
    A &= \text{Heads after odd number of flips} \\
    B &= \text{3 or more flips} \\
    C &= \text{4 or more flips}
\end{align*}
\textit{(a)} 

To lose, there must be an odd number of flips before arriving at heads. For the first flip, there is a $\frac{1}{2}$ chance of that occurring. If not, we must get tails on the second throw (another $\frac{1}{2}$ chance), and heads on the third throw for a third $\frac{1}{2}$ chance. This leaves us with the following geometric series:
\[
\frac{1}{2} + \frac{1}{8} + \cdots = \sum_{n=0}^{\infty}\frac{1}{2} \left(\frac{1}{4}\right)^n
\]
Solving this geometric series, we arrive at the probability of our loss $A$:
\begin{align*}
    \sum_{n=0}^{\infty}\frac{1}{2} \left(\frac{1}{4}\right)^n &= \frac{\frac{1}{2} }{1 - \frac{1}{4}} \\
                                                              &= \frac{1}{2} \cdot \frac{4}{3} \\
                                                              &= \frac{4}{6} \\
    \Aboxed{\mathbb{P}(A) &= \frac{2}{3}} 
\end{align*}



\textit{(b)} 

As long as the starting point is an odd number of turns (1 in part \textit{(a)}, 3 here), the probability of each player's victory remains the same. Thus:
\[
    \mathbb{P}\left(A \;\middle|\; B\right) = \mathbb{P}(A) = \frac{2}{3} 
\]
Therefore, $\boxed{A \text{ and } B \text{ are independent.}}$ 

\textit{(c)} 

If we begin counting at an even number of turns, the probability of each player's victory is reversed, since the first coin toss being heads gives the opposite player a victory and so on. Thus: 
\[
    \mathbb{P}\left(A \;\middle|\; C\right) \ne \mathbb{P}(A)
\]
Therefore, $\boxed{A \text{ and } C\text{ are dependent}}$

\pagebreak
\section{Question 11}
My work partners were Aseda Asomani and Blake Morris.

\pagebreak
\section{Question 12}
Translating the problem into a geometric series gives us:
\[
    \sum_{n=2}^{\infty}\frac{1}{c^n} 
\]
which is equivalent to:
\[
    \sum_{n=0}^{\infty}\frac{1}{c^n} - \sum_{n=0}^{1}\frac{1}{c^n} = \sum_{n=0}^{\infty}\frac{1}{c^n} -1-\frac{1}{c}
\]
Since we are solving a probability mass function, the sum of this series must be 1. Therefore:
\[
    \sum_{n=0}^{\infty}\frac{1}{c^n} - 1 - \frac{1}{c} = 1 = 2 + \frac{1}{c} 
\]
Using the formula to solve an infinite geometric series starting at 0, where $a = 1$ and $r = \frac{1}{c}$, we arrive at:
\begin{align*}
    \sum_{n=0}^{\infty}\frac{1}{c^n} = \frac{1}{1-\frac{1}{c}} &= 2+\frac{1}{c} \\
    1 &= \left(2+\frac{1}{c}\right)\left(1-\frac{1}{c}\right) \\
    1 &= 2 + \frac{1}{c} - \frac{2}{c} - \frac{1}{c^2} \\
    1 &= 2 - \frac{1}{c} - \frac{1}{c^2} \\
    1 &= \frac{1}{c^2} + \frac{1}{c} \\
    0 &= \frac{1}{c^2} + \frac{1}{c} - 1
\end{align*}

Plugging into the quadratic formula, we arrive at two possible solutions:
\begin{align*}
    c_1 &= \frac{1+\sqrt{5}}{2} &c_2 = \frac{1-\sqrt5}{2} 
\end{align*}

But, since $|c_2| < 1$, the only possible solution is $c_1$, as the demoniator $c$ must be greater than 1 or the series diverges, thus:
\[
\boxed{c = \frac{1+\sqrt5}{2}} 
\]

\end{document}

\documentclass[titlepage, 12pt, leqno]{article}

% -------------------------------------------------- %
% -------------------- PACKAGES -------------------- %
% -------------------------------------------------- %
\usepackage{import}
\usepackage{pdfpages}
\usepackage{mathtools}
\usepackage{transparent}
\usepackage{xcolor}
\usepackage{tcolorbox}
\usepackage{amsmath}
\usepackage{amssymb}
\usepackage{parskip}
\usepackage[margin = 1in]{geometry}


% -------------------------------------------------- %
% -------------- CUSTOM ENVIRONMENTS --------------- %
% -------------------------------------------------- %
\newtcolorbox{note}{colback=black!5!white,
                          colframe=black!55!white,
                          fonttitle=\bfseries,title=Note}

\newtcolorbox{ex}{colback=blue!5!white,
                          colframe=blue!55!white,
                          fonttitle=\bfseries,title=Example}

\newtcolorbox{definition}{colback=red!5!white,
                          colframe=red!55!white,
                          fonttitle=\bfseries,title=Definition}


% -------------------------------------------------- %
% ------------------- COMMANDS --------------------- %
% -------------------------------------------------- %
% Brackets, braces, etc. 
\newcommand{\abs}[1]{\lvert #1 \rvert}
\newcommand{\bigabs}[1]{\Bigl \lvert #1 \Bigr \rvert}
\newcommand{\bigbracket}[1]{\Bigl [ #1 \Bigr ]}
\newcommand{\bigparen}[1]{\Bigl ( #1 \Bigr )}
\newcommand{\ceil}[1]{\lceil #1 \rceil}
\newcommand{\floor}[1]{\lfloor #1 \rfloor}
\newcommand{\norm}[1]{\| #1 \|}
\newcommand{\bignorm}[1]{\Bigl \| #1 \Bigr \| #1}
\newcommand{\inner}[1]{\langle #1 \rangle}
\newcommand{\set}[1]{{ #1 }}


% -------------------------------------------------- %
% -------------------- SETUP ----------------------- %
% -------------------------------------------------- %
\title{\Huge{Homework 4}}
\author{\large{Mitch Harrison}}
\date{\today}   
\begin{document}
\setlength{\parskip}{1\baselineskip}
\setlength{\parindent}{15pt}
\maketitle
\tableofcontents
\newpage


% -------------------------------------------------- %
% --------------------- BODY ----------------------- %
% -------------------------------------------------- %
\section{Question 1}

Let $B$ be the event that the dice are biased and $R_7$ be the event that a
seven is rolled. Bayes' Rule gives us:
\[
\mathbb{P}\left(B \;\middle|\; R_7\right) = \frac{ 
\mathbb{P}\left(R_7 \;\middle|\; B\right) \mathbb{P}(B)}
{ \mathbb{P}\left(R_7 \;\middle|\; B\right) \mathbb{P}(B) +
\mathbb{P}\left(R_7 \;\middle|\; \neg B \right) \mathbb{P}(\neg B)}
\]

We know that $\mathbb{P}(B) = \mathbb{P}(\neg B) = \frac{1}{2}$.
To calculate $ \mathbb{P}\left(R_7 \;\middle|\; B\right) $, we use a binomial 
distribution:
\begin{align*}
    \mathbb{P}\left(R_7 \;\middle|\; B\right) &= binom
    \left(3,\frac{1}{3}\right) \\
                                              &= \binom{3}{1}\left(\frac{1}{2}
                                                  \right)\left(\frac{2}{3} 
                                                      \right)^2 \\
                                              &= \frac{4}{9} 
\end{align*}
Similar calculations gives $ \mathbb{P}\left(R_7 \;\middle|\; \neg B\right) 
= \frac{25}{72} $. Plugging into the formula given by Bayes' Rule, we arrive at
a solution:

\[
    \boxed{ \mathbb{P}\left(B \;\middle|\; R_7\right) \approx 0.5614}
\]
\pagebreak
\section{Question 2}
We are given the following:
\begin{align*}
    n &= 1000 \\
    p &= 0.1 \\
    a &= 90 \\
    b &= 115 \\
    \sigma &= \sqrt{1000 \cdot .1 \cdot .9} = \sqrt{90}
\end{align*}

Plugging these in to the $\Phi$ functions for non-standard normal distributions,
we arrive at a solution:
\begin{align*}
    \mathbb{P}(90 \le \hat p \le 115) &\approx \Phi\left(\frac{115+\frac{1}{2} -
    100}{\sqrt{90}}\right) \\
    &\approx \Phi(1.63) - \Phi(-1.11) \\
    &\approx \Phi(1.63)-(1-\Phi(1.11)) \\
    &\approx 0.9484 - (1-0.8665) \\
            \Aboxed{\mathbb{P}(90 \le \hat p \le 115) &\approx 0.8149}
\end{align*}

\pagebreak
\section{Question 3}
\textit{(a)} 

We are given that $n = 500$ and $p = .002$. We find the exact solution with a
summation:

\[
    \sum_{k=0}^{2}\binom{500}{k}(.002)^k(.998)^{500-k}
\]
Plugging that summation into a computer algebra system, we arrive at a solution:

\[
    \boxed{\mathbb{P}(k<3) \approx 0.9198} 
\]

To estimate using a normal approximation, we first calculate the mean and 
standard deviation.
\begin{align*}
    \sigma &= \sqrt{500(.002)(.998)} \approx 1 \\
    \mu &= 500(.002) = 1
\end{align*}
Plugging these into a $\Phi$ function with a continuity correction, we estimate
the probability of 3 or fewer faults:
\[
    \Phi\left(\frac{3-\frac{1}{2} -1}{1}\right) = \Phi\left(\frac{3}{2} \right)
    \approx \boxed{0.9332}
\]

\textit{(b)} 

We are given $p$ and population size, but need to solve for $n$. To do so, we use
a $\Phi$ function:

\begin{align*}
    1 - \Phi\left(\frac{10,000-\frac{1}{2} -.998n}{\sqrt{.998n-.002}}\right)
    &= .997 \\
        \Phi\left(\frac{10,000-\frac{1}{2} -.998n}{\sqrt{.998n-.002}}\right)
                                                                      &= .003 \\
            \Phi\left(\frac{.998n - 9999.5}{\sqrt{.998n-.002}}
        \right) &=.997
\end{align*}

Plugging into the $\Phi^{-1}$ table and solving for $n$, we arrive at a solution:
\begin{align*}
    \frac{.998n - 9999.5}{\sqrt{.998n-.002}} &= 2.75 \\
    \Aboxed{n &\approx 10,299 \text{ batteries}} 
\end{align*}

\pagebreak
\section{Question 4}
\textit{(a)} 

Our desired confidence is 99\%, but our number of trials $n$ is unknown. We will
use $p = \frac{1}{2} $ to get a worst-case solution. Using $\Phi$ functions and
$p = \frac{1}{2} $, we arrive at a solution:
\begin{align*}
    0.99 &\approx \Phi\left(\frac{p+.01-p}{\frac{\sqrt{p(1-p)}}{\sqrt{n}} }\right)
    - \Phi\left(\frac{p-.01-p}{\frac{\sqrt{p(1-p)}}{\sqrt{n}} }\right) \\
         &\approx 2\Phi\left(\frac{0.01}{\frac{\sqrt{p(1-p)}}{\sqrt{n}} }\right)
         -1 \\
    .995 &\approx \Phi\left(\frac{0.01\sqrt{n}}{\sqrt{p(1-p)}}\right) \\
        2.675 &\approx \frac{.01\sqrt{n}}{\sqrt{p(1-p)}} = \frac{
        .01\sqrt{n}}{\frac{1}{2} } \\
        2.675 &\approx 0.02\sqrt{n} \\
        133.75 &\approx \sqrt{n} \\
        \Aboxed{17,889 &\approx n} 
\end{align*}

\textit{(b)} 

We are given $\hat p = \frac{80}{100} $. To calculate the worst-case standard
deviation, we let $p = \frac{1}{2} $, thus:
\[
 \sigma = \frac{\sqrt{p(1-p)}}{10} = \frac{\sqrt{\frac{1}{4} }}{10} =
 \frac{1}{20} 
\]

Therefore, our final confidence interval is:
\[
    \left(p-\frac{1}{20}, p+\frac{1}{20} \right) = 
    \boxed{\left(\frac{3}{4}, \frac{17}{20} \right) }
\]

\pagebreak
\section{Question 5}
Let $H_4$ be the event that 4 people have sickle-cell anemia, $A$ represent
selecting population A, and $B$ represent selection population B. Bayes' Rule
states that:
\[
\mathbb{P}\left(A \;\middle|\; H_4\right) = \frac{
\mathbb{P}\left(H_4 \;\middle|\; A\right) \mathbb{P}(A)}{
\mathbb{P}\left(H_4 \;\middle|\; A\right) \mathbb{P}(A) + 
\mathbb{P}\left(H_4 \;\middle|\; B\right) \mathbb{P}(B)} 
\]
We use a hypergeometric distribution to calculate $ \mathbb{P}\left(H_4 
\;\middle|\; A\right) $:
\[
\mathbb{P}\left(H_4 \;\middle|\; A\right) = \frac{
 \binom{500}{4}\binom{500}{6}}{\binom{1000}{10}} \approx 0.2057
\]

Using the same method, we find that $ \mathbb{P}\left(H_4 \;\middle|\; B\right) 
\approx 0.0104$. Plugging into the formula given by Bayes' Rule, we arrive at
a solution:
\begin{align*}
    \mathbb{P}\left(A \;\middle|\; H_4\right) &= \frac{
\mathbb{P}\left(H_4 \;\middle|\; A\right) \mathbb{P}(A)}{
\mathbb{P}\left(H_4 \;\middle|\; A\right) \mathbb{P}(A) + 
\mathbb{P}\left(H_4 \;\middle|\; B\right) \mathbb{P}(B)} \\
&= \frac{.2057(.5)}{.2057(.5)+.0104(.5)} \\
        \Aboxed{ \mathbb{P}\left(A \;\middle|\; H_4\right) &\approx .9519} 
\end{align*}

\pagebreak
\section{Question 6}
My work partners were Aseda Asomani, Blake Morris, and Nathaniel Wullar.

\end{document}

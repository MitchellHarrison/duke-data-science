\documentclass[titlepage, 12pt, leqno]{article}

% -------------------------------------------------- %
% -------------------- PACKAGES -------------------- %
% -------------------------------------------------- %
\usepackage{import}
\usepackage{pdfpages}
\usepackage{mathtools}
\usepackage{transparent}
\usepackage{enumitem}
\usepackage{xcolor}
\usepackage{tcolorbox}
\usepackage{amsmath}
\usepackage{amssymb}
\usepackage{parskip}
\usepackage{bbm}
\usepackage{algorithm}
\usepackage{algpseudocode}
\usepackage{algpseudocodex}
\usepackage[margin = 1in]{geometry}
\tcbuselibrary{breakable}
\tcbset{breakable = true}


% -------------------------------------------------- %
% -------------- CUSTOM ENVIRONMENTS --------------- %
% -------------------------------------------------- %
\newtcolorbox{note}{colback=black!5!white,
                          colframe=black!55!white,
                          fonttitle=\bfseries,title=Note}

\newtcolorbox{ex}{colback=blue!5!white,
                          colframe=blue!55!white,
                          fonttitle=\bfseries,title=Example}

\newtcolorbox{definition}{colback=red!5!white,
                          colframe=red!55!white,
                          fonttitle=\bfseries,title=Definition}


% -------------------------------------------------- %
% ------------------- COMMANDS --------------------- %
% -------------------------------------------------- %
% Brackets, braces, etc. 
\newcommand{\abs}[1]{\lvert #1 \rvert}
\newcommand{\bigabs}[1]{\Bigl \lvert #1 \Bigr \rvert}
\newcommand{\bigbracket}[1]{\Bigl [ #1 \Bigr ]}
\newcommand{\bigparen}[1]{\Bigl ( #1 \Bigr )}
\newcommand{\ceil}[1]{\lceil #1 \rceil}
\newcommand{\floor}[1]{\lfloor #1 \rfloor}
\newcommand{\norm}[1]{\| #1 \|}
\newcommand{\bignorm}[1]{\Bigl \| #1 \Bigr \| #1}
\newcommand{\inner}[1]{\langle #1 \rangle}
\newcommand{\set}[1]{{ #1 }}


% -------------------------------------------------- %
% -------------------- SETUP ----------------------- %
% -------------------------------------------------- %
\title{\Huge{Lecture 1 - Introductions}}
\author{\large{Mitch Harrison}}
\date{\today}   
\begin{document}
\setlength{\parskip}{1\baselineskip}
\setlength{\parindent}{15pt}
\maketitle
\tableofcontents
\newpage


% -------------------------------------------------- %
% --------------------- BODY ----------------------- %
% -------------------------------------------------- %
\section{What, Why, and How?}

\subsubsection{What?}
This is \textit{the plot}. This is basically the choice of plot type for a 
certain type of data. Lines for chronological, maps for geospacial, etc. are
examples.

\subsubsection{How?}
This is the biggest portion of the course. It will cover design, pre-processing
data, mapping data to aesthetics, making visual encoding decisions (e.g.
making plots accessible), and post-processing for visual appeal and annotation.

\subsubsection{Why?}
This will tie together the "how" and "what" through the grammar of graphics
theoretical framework.

\pagebreak
\section{Grammar of Graphics}
\begin{definition}
    \textbf{Data visualization} si the creation and study of the visual
    representation of data. In this course, we will use R will \texttt{ggplot} 
    mostly.
\end{definition}

The book that inspires the phrase "grammar of graphics" is the book
\textit{Grammar of Graphics} by Leland Wilkinson. It is a theoretical framework
that enables us to concisely desscribe the components of graphics using
layers.

\subsection{Accessability}
The cheapest and easiest way to increase accessability is to use \textit{double
encoding}, where differences in groups are shown with two different changes.
For example, a scatter plot's groups may be differentiated by both 
\textit{color} and \textit{shape}. That way, those with differences in visual
ability can use either. Choosing colorblind-friendly colors (e.g. using
\\ \texttt{ggthemes::scale\_color\_colorblind()}) is also more inclusive.

\end{document}

\documentclass[titlepage, 12pt, leqno]{article}

% -------------------------------------------------- %
% -------------------- PACKAGES -------------------- %
% -------------------------------------------------- %
\usepackage{import}
\usepackage{pdfpages}
\usepackage{mathtools}
\usepackage{transparent}
\usepackage{enumitem}
\usepackage{xcolor}
\usepackage{tcolorbox}
\usepackage{amsmath}
\usepackage{amssymb}
\usepackage{parskip}
\usepackage{bbm}
\usepackage{algorithm}
\usepackage{algpseudocode}
\usepackage{algpseudocodex}
\usepackage[margin = 1in]{geometry}
\tcbuselibrary{breakable}
\tcbset{breakable = true}


% -------------------------------------------------- %
% -------------- CUSTOM ENVIRONMENTS --------------- %
% -------------------------------------------------- %
\newtcolorbox{note}{colback=black!5!white,
                          colframe=black!55!white,
                          fonttitle=\bfseries,title=Note}

\newtcolorbox{ex}{colback=blue!5!white,
                          colframe=blue!55!white,
                          fonttitle=\bfseries,title=Example}

\newtcolorbox{definition}{colback=red!5!white,
                          colframe=red!55!white,
                          fonttitle=\bfseries,title=Definition}


% -------------------------------------------------- %
% ------------------- COMMANDS --------------------- %
% -------------------------------------------------- %
% Brackets, braces, etc. 
\newcommand{\abs}[1]{\lvert #1 \rvert}
\newcommand{\bigabs}[1]{\Bigl \lvert #1 \Bigr \rvert}
\newcommand{\bigbracket}[1]{\Bigl [ #1 \Bigr ]}
\newcommand{\bigparen}[1]{\Bigl ( #1 \Bigr )}
\newcommand{\ceil}[1]{\lceil #1 \rceil}
\newcommand{\floor}[1]{\lfloor #1 \rfloor}
\newcommand{\norm}[1]{\| #1 \|}
\newcommand{\bignorm}[1]{\Bigl \| #1 \Bigr \| #1}
\newcommand{\inner}[1]{\langle #1 \rangle}
\newcommand{\set}[1]{{ #1 }}


% -------------------------------------------------- %
% -------------------- SETUP ----------------------- %
% -------------------------------------------------- %
\title{\Huge{Homework 1}}
\author{\large{Mitch Harrison}}
\date{\today}   
\begin{document}
\setlength{\parskip}{1\baselineskip}
\setlength{\parindent}{15pt}
\maketitle
\newpage


% -------------------------------------------------- %
% --------------------- BODY ----------------------- %
% -------------------------------------------------- %
\section*{Question 1}
We are given that $\hat \theta_{2} = \hat \theta + W$, $W$ is a zero-mean
random variable (i.e. $\mathbb{E}(W) = 0$), and that $\hat \theta$ is an
unbiased estimator for $\theta$ (i.e. $\mathbb{E}(\hat \theta) = \theta$).
\begin{align*}
    \mathbb{E}(\hat \theta_{2}) &= \mathbb{E}(\hat \theta+W) \\
                                &= \mathbb{E}(\hat \theta)+\mathbb{E}(W) \\
                                &= \mathbb{E}(\hat \theta) + 0 \\
                                &= \theta
\end{align*}
Thus we have shown that $\mathbb{E}(\hat \theta_{2}) = \theta$. Therefore,
$\boxed{\hat \theta_{2} \text{ is an unbiased estimator for }\theta}$

\section*{Question 2}
\subsubsection{a)}
The PDF of $\hat \theta_{1}$ is

\subsubsection{b)}
\subsubsection{c)}
\subsubsection{d)}
\subsubsection{e)}
\subsubsection{f)}
\subsubsection{g)}
\subsubsection{h)}
\subsubsection{i)}


\end{document}

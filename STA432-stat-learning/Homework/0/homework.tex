\documentclass[titlepage, 12pt, leqno]{article}

% -------------------------------------------------- %
% -------------------- PACKAGES -------------------- %
% -------------------------------------------------- %
\usepackage{import}
\usepackage{pdfpages}
\usepackage{mathtools}
\usepackage{transparent}
\usepackage{enumitem}
\usepackage{xcolor}
\usepackage{tcolorbox}
\usepackage{amsmath}
\usepackage{amssymb}
\usepackage{parskip}
\usepackage{bbm}
\usepackage{algorithm}
\usepackage{algpseudocode}
\usepackage{algpseudocodex}
\usepackage[margin = 1in]{geometry}
\tcbuselibrary{breakable}
\tcbset{breakable = true}


% -------------------------------------------------- %
% -------------- CUSTOM ENVIRONMENTS --------------- %
% -------------------------------------------------- %
\newtcolorbox{note}{colback=black!5!white,
                          colframe=black!55!white,
                          fonttitle=\bfseries,title=Note}

\newtcolorbox{ex}{colback=blue!5!white,
                          colframe=blue!55!white,
                          fonttitle=\bfseries,title=Example}

\newtcolorbox{definition}{colback=red!5!white,
                          colframe=red!55!white,
                          fonttitle=\bfseries,title=Definition}


% -------------------------------------------------- %
% ------------------- COMMANDS --------------------- %
% -------------------------------------------------- %
% Brackets, braces, etc. 
\newcommand{\abs}[1]{\lvert #1 \rvert}
\newcommand{\bigabs}[1]{\Bigl \lvert #1 \Bigr \rvert}
\newcommand{\bigbracket}[1]{\Bigl [ #1 \Bigr ]}
\newcommand{\bigparen}[1]{\Bigl ( #1 \Bigr )}
\newcommand{\ceil}[1]{\lceil #1 \rceil}
\newcommand{\floor}[1]{\lfloor #1 \rfloor}
\newcommand{\norm}[1]{\| #1 \|}
\newcommand{\bignorm}[1]{\Bigl \| #1 \Bigr \| #1}
\newcommand{\inner}[1]{\langle #1 \rangle}
\newcommand{\set}[1]{{ #1 }}


% -------------------------------------------------- %
% -------------------- SETUP ----------------------- %
% -------------------------------------------------- %
\title{\Huge{Homework 0}}
\author{\large{Mitch Harrison}}
\date{\today}   
\begin{document}
\setlength{\parskip}{1\baselineskip}
\setlength{\parindent}{15pt}
\maketitle
\newpage


% -------------------------------------------------- %
% --------------------- BODY ----------------------- %
% -------------------------------------------------- %
\section*{Question 1}
This is a Gaussian integral. It evaluates to $\sqrt{\pi}$.

\section*{Question 2}
\begin{align*}
    \int x \cdot exp\{-x^{2}\} &= -\frac{1}{2}e^{-x^{2}} \\
    \int_{-\infty}^{\infty}xe^{-x^{2}} &= 0 - 0 \\
\Aboxed{\int_{-\infty}^{\infty}xe^{-x^{2}} &= 0}
\end{align*}

\section*{Question 3}
Assuming a sufficiently large sample size, sample independence, and a finite
mean and variance, the Central Limit Theorem states that the sample mean of
any random sample will be approximately normally distributed.

\section*{Question 4}
Uncorrelated random variables could be dependent. If we let $X$ be any value
between -10 and 10, and let $Y = |X|$, 
\begin{align*}
    Cov(X, Y) &= \mathbb{E}(XY) - \mathbb{E}(X)\mathbb{E}(Y) \\
              &= 0 - 0 \\
    Cov(X,Y) &= 0.
\end{align*}
With $Cov(X,Y) = 0$, $X$ and $Y$ are uncorrelated. However, because knowledge of
 $X$ gives direct insights about $Y$, they are \textit{dependent}.

\section*{Question 5}
Two independent random variables could be correlated or uncorrelated. This gives
rise to a spurious correlation. A brief Google search shows that the correlation
between US spending on space exploration and number of suicides by hanging is
 $r \approx 0.998$.

\section*{Question 6}
Since expectation is a linear operator, we can expand both terms,
\begin{align*}
    \mathbb{E}[(X+Y)^{2}] &= \mathbb{E}[X^{2} + 2XY + Y^{2}] \\
                          &= \mathbb{E}(X^{2}) + 2\mathbb{E}(XY) + 
                          \mathbb{E}(Y^{2}),
\end{align*}
and
\begin{align*}
    \mathbb{E}[(X+Y)]^{2} &= [\mathbb{E}(X) + \mathbb{E}(Y)]^{2} \\
                          &= \mathbb{E}(X)^{2} + 2\mathbb{E}(X)\mathbb{E}(Y) + 
                          \mathbb{E}(Y)^{2}.
\end{align*}
Thus, assuming independence and therefor $\mathbb{E}(XY) = \mathbb{E}(X)
\mathbb{E}(Y)$,
\begin{align*}
    \mathbb{E}[(X+Y)^{2}] - \mathbb{E}[(X+Y)]^{2} &= \mathbb{E}(X^{2}) + 
    2\mathbb{E}(XY)  + \mathbb{E}(Y^{2}) - \mathbb{E}(X)^{2} - 2\mathbb{E}(XY)
    - \mathbb{E}(Y)^{2} \\
    &= \mathbb{E}(X^{2}) - \mathbb{E}(X)^{2} + \mathbb{E}(Y^{2}) -
    \mathbb{E}(Y)^{2}
\end{align*}
The variance of $X$ (which is the same for $Y$ or any other random 
variable) is given by,
\[
Var(X) = \mathbb{E}(X^{2}) - [\mathbb{E}(X)]^{2}.
\]
Thus, assuming independence

\section*{Question 7}
We are given that $(X+Y)/2 = 1/2$, therefore $X + Y = 1$. That means that either
 $X$ or $Y$ are 0 and the other is 1. However, since 
 $X,Y\overset{\mathrm{iid}}{\sim}Bernoulli(p)$,
\[
\mathbb{E}\left(X \middle| \frac{X+Y}{2} = \frac{1}{2}\right) = \frac{1}{2}.
\]

\section*{Question 8}
The variance for a continuous uniform distribution is given by the formula
\[
\frac{1}{12}(b-a)^{2}.
\]
For the bounds $[0,1]$, that gives a variance of $1/12$. Using that same
formula for the bounds $[-3,3]$ we arrive at a variance of 3.

\section*{Question 9}
\[
\mathbb{P}(A|B) = \frac{\mathbb{P}(A \cap B)}{\mathbb{P}(B)}
\]

\section*{Question 10}
The OLS estimator for the coefficient vector $\beta$ is given by
\[
\hat \beta_{OLS} = (X^{T}X)^{-1}X^{T}Y,
\]
which is unbiased estimator of $\beta$. Thus,
\[
\mathbb{E}(\hat \beta_{OLS}) = \beta.
\]

\end{document}

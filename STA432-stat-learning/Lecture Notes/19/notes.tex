\documentclass[titlepage, 12pt, leqno]{article}

% -------------------------------------------------- %
% -------------------- PACKAGES -------------------- %
% -------------------------------------------------- %
\usepackage{import}
\usepackage{pdfpages}
\usepackage{mathtools}
\usepackage{transparent}
\usepackage{enumitem}
\usepackage{xcolor}
\usepackage{tcolorbox}
\usepackage{amsmath}
\usepackage{amssymb}
\usepackage{parskip}
\usepackage{bbm}
\usepackage{algorithm}
\usepackage{algpseudocode}
\usepackage{algpseudocodex}
\usepackage[margin = 1in]{geometry}
\tcbuselibrary{breakable}
\tcbset{breakable = true}


% -------------------------------------------------- %
% -------------- CUSTOM ENVIRONMENTS --------------- %
% -------------------------------------------------- %
\newtcolorbox{note}{colback=black!5!white,
                          colframe=black!55!white,
                          fonttitle=\bfseries,title=Note}

\newtcolorbox{ex}{colback=blue!5!white,
                          colframe=blue!55!white,
                          fonttitle=\bfseries,title=Example}

\newtcolorbox{definition}{colback=red!5!white,
                          colframe=red!55!white,
                          fonttitle=\bfseries,title=Definition}


% -------------------------------------------------- %
% ------------------- COMMANDS --------------------- %
% -------------------------------------------------- %
% Brackets, braces, etc. 
\newcommand{\abs}[1]{\lvert #1 \rvert}
\newcommand{\bigabs}[1]{\Bigl \lvert #1 \Bigr \rvert}
\newcommand{\bigbracket}[1]{\Bigl [ #1 \Bigr ]}
\newcommand{\bigparen}[1]{\Bigl ( #1 \Bigr )}
\newcommand{\ceil}[1]{\lceil #1 \rceil}
\newcommand{\floor}[1]{\lfloor #1 \rfloor}
\newcommand{\norm}[1]{\| #1 \|}
\newcommand{\bignorm}[1]{\Bigl \| #1 \Bigr \| #1}
\newcommand{\inner}[1]{\langle #1 \rangle}
\newcommand{\set}[1]{{ #1 }}


% -------------------------------------------------- %
% -------------------- SETUP ----------------------- %
% -------------------------------------------------- %
\title{\Huge{Lecture 19 - Exam Post-Mordem and UMP Tests}}
\author{\large{Mitch Harrison}}
\date{\today}   
\begin{document}
\setlength{\parskip}{1\baselineskip}
\setlength{\parindent}{15pt}
\maketitle
\tableofcontents
\newpage


% -------------------------------------------------- %
% --------------------- BODY ----------------------- %
% -------------------------------------------------- %
\section{Midterm 2}

\subsection{Problem 3}
We had $X_{1}, \cdots , X_{n} \sim f(x|\alpha)$. We say that the density was
\[
f(x|\alpha) = \frac{1 +\alpha x}{2},
\]
and $\mathbb{E}[X] = \alpha/3$ and $x \in (-1,1)$. We were asked to find a
method of moments estimator $\hat \alpha$. Because we were given $E[X] = 
\alpha/3$, we know that $\hat \alpha = 3E[X] = 3\bar X$.

\subsection{Problem 2}
\begin{note}
    To prep for these problems types of problems, we can search for distributions
    that have MLEs, repeat the problem for each of those distributions, and we
    should be prepared for that portion of the final.
\end{note}

Over the course of the problem, we recognize Fishers approximation as
\[
\sqrt{n}(\hat \theta_{MLE} - \theta_{0}) \rightarrow N(0, 1/\mathcal{I}(
\theta),
\]
and $\ell = \frac{1}{n}\sum log(x_{i})$ and $\mathcal{I}(\theta) = 1/\sigma^{2}$.
We also found a confidence interval
\[
    \hat \theta_{MLE} \pm Z_{1-\alpha}\sqrt{\frac{\sigma^{2}}{n}}.
\]
Part \textbf{f} of the test asked us to find a most powerful test for
$H_{0}:\theta = \theta_{0}$ against $H_{1}:\theta = \theta_{1}$. The
NPL tells us that by choosing the right $k$, we arrive at a UMP test. For this,
we need to recall that if a test does not depend on $H_{1}$, it is uniformly
most powerful.

\pagebreak
\section{UMP Tests}
\begin{definition}
    A family of distributions indexed by $\theta$ is said to have a
    \textbf{monotone likelihood ratio} (MLR) in a statistic $T(X)$ if, for any
    $\theta_{1} > \theta_{0}$, the likelihood ratio $\Lambda(X)$ is 
    non-decreasing in $T(X)$.

    For example, the log-normal distribution with a known $\sigma^{2}$ has a MLR
    in $\sum log(y_{i})$ for observations $y_{1}, \cdots , y_{n}$.
\end{definition}

\begin{definition}
    The \textbf{Karlin-Reuben theorem} states that for a test in the form
    $H_{0}:\theta \le \theta_{0}$ vs $H_{1}:\theta > \theta_{0}$ is given by
    $T(X) > k$, where $k$ is chosen such that the test has level $\alpha$. This
    works only when a distribution has a MLR under some statistic $T(X)$.
\end{definition}

We have previously found UMP tests using the NPL and finding out if our
result depends on $H_{1}$. If not, the test is UMP. Now, we have a second 
method. Specifically, we can find a statistic such that we have an MLR, and
then apply Karlin-Reuben. These work for two different use cases: the former
for simple $H_{0}$ and $H_{1}$, and the latter for complex $H_{0}$ and $H_{1}$.

In R, when we run regression, we only care about whether or not the $\beta$ 
coefficients are 0 or not. That's the test that R runs. But it uses an estimated
variance to conduct this test, which we haven't done yet.

\subsection{R-style example}

Let our new \textbf{goal} be, for $X_{1}, \cdots X_{n}
\overset{\mathrm{iid}}{\sim}N(\mu, \sigma^{2})$, to test $H_{0}: \mu = \mu_{0}$
against $H_{1}: \mu \ne \mu_{0}$.

Our first step is to, assuming $\sigma^{2}$ is known, is to find a likelihood
ratio $\Lambda(X)$. Recall that our $\Lambda$ looks like
\[
\Lambda(X) = \frac{f(X|\theta_{0})}{f(X|\theta_{1})}.
\]
But how do we do this for our current test? Well, the denominator would be
the same since we already have the correct form for that. But we want to find
the value of $\theta_{1}$ that is the best guess given that $H_{1}$ is true.
Fortunately, the MLE is the best value for this.

\end{document}

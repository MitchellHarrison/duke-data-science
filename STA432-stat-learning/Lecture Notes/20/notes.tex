\documentclass[titlepage, 12pt, leqno]{article}

% -------------------------------------------------- %
% -------------------- PACKAGES -------------------- %
% -------------------------------------------------- %
\usepackage{import}
\usepackage{pdfpages}
\usepackage{mathtools}
\usepackage{transparent}
\usepackage{enumitem}
\usepackage{xcolor}
\usepackage{tcolorbox}
\usepackage{amsmath}
\usepackage{amssymb}
\usepackage{parskip}
\usepackage{bbm}
\usepackage{algorithm}
\usepackage{algpseudocode}
\usepackage{algpseudocodex}
\usepackage[margin = 1in]{geometry}
\tcbuselibrary{breakable}
\tcbset{breakable = true}


% -------------------------------------------------- %
% -------------- CUSTOM ENVIRONMENTS --------------- %
% -------------------------------------------------- %
\newtcolorbox{note}{colback=black!5!white,
                          colframe=black!55!white,
                          fonttitle=\bfseries,title=Note}

\newtcolorbox{ex}{colback=blue!5!white,
                          colframe=blue!55!white,
                          fonttitle=\bfseries,title=Example}

\newtcolorbox{definition}{colback=red!5!white,
                          colframe=red!55!white,
                          fonttitle=\bfseries,title=Definition}


% -------------------------------------------------- %
% ------------------- COMMANDS --------------------- %
% -------------------------------------------------- %
% Brackets, braces, etc. 
\newcommand{\abs}[1]{\lvert #1 \rvert}
\newcommand{\bigabs}[1]{\Bigl \lvert #1 \Bigr \rvert}
\newcommand{\bigbracket}[1]{\Bigl [ #1 \Bigr ]}
\newcommand{\bigparen}[1]{\Bigl ( #1 \Bigr )}
\newcommand{\ceil}[1]{\lceil #1 \rceil}
\newcommand{\floor}[1]{\lfloor #1 \rfloor}
\newcommand{\norm}[1]{\| #1 \|}
\newcommand{\bignorm}[1]{\Bigl \| #1 \Bigr \| #1}
\newcommand{\inner}[1]{\langle #1 \rangle}
\newcommand{\set}[1]{{ #1 }}


% -------------------------------------------------- %
% -------------------- SETUP ----------------------- %
% -------------------------------------------------- %
\title{\Huge{Lecture 19 - Complex Hypotheses}}
\author{\large{Mitch Harrison}}
\date{\today}   
\begin{document}
\setlength{\parskip}{1\baselineskip}
\setlength{\parindent}{15pt}
\maketitle
\tableofcontents
\newpage


% -------------------------------------------------- %
% --------------------- BODY ----------------------- %
% -------------------------------------------------- %
\section{Complex Hypotheses}

Let's verify the monotone likelihood ration for the normal distribution. Let
$X_{1}, \cdots ,X_{n} \overset{\mathrm{iid}}{\sim}N(\mu, \sigma^{2})$ and let
$\sigma^{2}$ be known. The statistic
\begin{align*}
\frac{f(x|\mu_{1})}{f(x|\mu_{0})} 
&=
\frac{ \prod_{i=1}^{\infty}\frac{1}{\sqrt{2\pi\sigma^{2}}}exp\{
-\frac{1}{2\sigma^{2}}(x_{i} - \mu_{1})^{2}\}}{
\prod_{i=1}^{\infty}\frac{1}{\sqrt{2\pi\sigma^{2}}}exp\{
-\frac{1}{2\sigma^{2}}(x_{i} - \mu_{0})^{2}\}}{
}  \\
&= exp\{\frac{1}{\sigma^{2}}\sum (x_{i}-\mu_{1})^{2} + \frac{1}{2\sigma^{2}}
    \sum (x_{i}-\mu_{0})\} \\
&=
exp\{\frac{\mu_{1} - \mu_{0}}{\sigma^{2}}(\sum x_{i}) - \frac{n(\mu_{1}^{2} -
\mu_{0}^{2})}{2\sigma^{2}}\}
\end{align*}
is the likelihood ratio. We see that this statistic $\Lambda(X)$ is 
non-decreasing in $T(X) = \sum X_{i}$. For something to be decreasing \textit{in}
something else (say $T$), means that as $T$ increases, so too does
$\Lambda(X)$. A UMP level-$\alpha$ test of $H_{0}:\mu = \mu_{0}$ against
$H_{1}:\mu > \mu_{0}$ rejects $\sum X_{i} > k$ where $k$ is selected such that
its level is $\alpha$. that is,
\[
\mathbb{P}(\sum X_{i} > k | \mu = \mu_{0}) = \alpha.
\]
Thus, under $H_{0}$,
\[
\frac{\sqrt{n}\frac{1}{n}\sum(x_{i} - \mu_{0})}{\sigma} \sim N(0,1).
\]
We reject if 
\[
\bar X > \mu_{0}+\frac{\sigma}{\sqrt{n}}Z_{1-\alpha},
\]]
where $Z_{1-\alpha}$ is the $1-\alpha$ quantile. Let's find the power under the alternative. The power function is
\[
\mathbb{P}\left(\frac{\sqrt{n}(\bar X - \mu_{0})}{\sigma} > Z_{1-\alpha} |
    \mu = \mu_{0}\right)
\]
And
\[
T = \frac{\sqrt{n}(\bar X - \mu_{0})}{\sigma} \sim 
Normal\left( \frac{\sqrt{n}(\mu_{1} - \mu_{0})}{\sigma}, 1\right).
\]
Thus the power for this one-sided test is
\[
1 - \Phi\left(Z_{1-\alpha}-\frac{\sqrt{n}(\mu_{1} - \mu_{0})}{\sigma}\right).
\]
\begin{note}
    Some distributions (e.g. the Cauchy distribution) do not have an MLR, and 
    thus these tests are not "freebies."
\end{note}

\pagebreak

\section{Generalized Likelihood Ratio Test}
Let our test be $H_{0}:\theta \in \Theta_{0}$ and $H_{1}:\theta \in \Theta_{1}$.
For some family, let $\Theta_{0} = \{\theta_{0}\}$ and $\Theta_{1} = \{\theta :
\theta > \theta_{0}\}$. This is a one-sided test. But for $\Theta_{1} =
\{\theta : \theta \ne \theta_{0}\}$. This is a two-sided test. But there are
also tests where we may want something with an extra parameter. For example,
$\Theta_{0} = \{\mu = \mu_{0}, \sigma^{2}k\}$ and $\Theta_{1} = \{\mu \ne
\mu_{0}, \sigma^{2}j\}$, where $k$ and $j$ could be anything. This is no longer
a likelihood ratio test scenario as we know it.

\end{document}

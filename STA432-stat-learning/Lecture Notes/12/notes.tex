\documentclass[titlepage, 12pt, leqno]{article}

% -------------------------------------------------- %
% -------------------- PACKAGES -------------------- %
% -------------------------------------------------- %
\usepackage{import}
\usepackage{pdfpages}
\usepackage{mathtools}
\usepackage{transparent}
\usepackage{enumitem}
\usepackage{xcolor}
\usepackage{tcolorbox}
\usepackage{amsmath}
\usepackage{amssymb}
\usepackage{parskip}
\usepackage{bbm}
\usepackage{algorithm}
\usepackage{algpseudocode}
\usepackage{algpseudocodex}
\usepackage[margin = 1in]{geometry}
\tcbuselibrary{breakable}
\tcbset{breakable = true}


% -------------------------------------------------- %
% -------------- CUSTOM ENVIRONMENTS --------------- %
% -------------------------------------------------- %
\newtcolorbox{note}{colback=black!5!white,
                          colframe=black!55!white,
                          fonttitle=\bfseries,title=Note}

\newtcolorbox{ex}{colback=blue!5!white,
                          colframe=blue!55!white,
                          fonttitle=\bfseries,title=Example}

\newtcolorbox{definition}{colback=red!5!white,
                          colframe=red!55!white,
                          fonttitle=\bfseries,title=Definition}


% -------------------------------------------------- %
% ------------------- COMMANDS --------------------- %
% -------------------------------------------------- %
% Brackets, braces, etc. 
\newcommand{\abs}[1]{\lvert #1 \rvert}
\newcommand{\bigabs}[1]{\Bigl \lvert #1 \Bigr \rvert}
\newcommand{\bigbracket}[1]{\Bigl [ #1 \Bigr ]}
\newcommand{\bigparen}[1]{\Bigl ( #1 \Bigr )}
\newcommand{\ceil}[1]{\lceil #1 \rceil}
\newcommand{\floor}[1]{\lfloor #1 \rfloor}
\newcommand{\norm}[1]{\| #1 \|}
\newcommand{\bignorm}[1]{\Bigl \| #1 \Bigr \| #1}
\newcommand{\inner}[1]{\langle #1 \rangle}
\newcommand{\set}[1]{{ #1 }}


% -------------------------------------------------- %
% -------------------- SETUP ----------------------- %
% -------------------------------------------------- %
\title{\Huge{Lecture 12 - Confidence Intervals}}
\author{\large{Mitch Harrison}}
\date{\today}   
\begin{document}
\setlength{\parskip}{1\baselineskip}
\setlength{\parindent}{15pt}
\maketitle
\tableofcontents
\newpage


% -------------------------------------------------- %
% --------------------- BODY ----------------------- %
% -------------------------------------------------- %
\section{Confidence Intervals}

We seek for formalize the statement "how confident are we that the answer lies
in some range?" Let $Z \sim N(0,1)$. Say we want to give a range of values that
captures $Z$ with 95\% probability. In the simple case of a $N(0,1)$ random
variable, we know common confidence intervals. For 95\%, the range is 
$(-1.96,1.96)$.

Let $\alpha \in (0,1)$ be the level of confidence we desire (e.g. 0.95 for 95\%).
Then, the probability that a random variable falls into a range is then
$1-\alpha$. The lower and upper bounds are denoted as $-Z^{*}$ and $Z^{*}$
respectively. Letting $F$ be the CDF and $F^{-1}$ be the inverse CDF (i.e. the
"quantile function"),
\[
    Z^{*} = F^{-1}\left(1-\frac{\alpha}{2}\right).
\]
\subsection{Example}
Say we survey 500 students, chosen at random, and count how many favor a bill
that is being considered in the state senate. Let $X$ be the number of votes
in favor of the bill. It will have the form
\[
    X \sim Binom(500, p).
\]
We are searching for the population proportion $p$, which represents the true
proportion of the entire student body that supports the bill.

Say we observe $X=270$. Can we give a range of values that captures $p$ with
95\% probability? This differs from our normal distribution example because we
aren't searching for the range of values for a random variable. We are instead
looking for a range of values for a parameter.

Let's say that while we were designing our survey, we pre-determined that when
we observe $X$, we would report the range
\[
\left(\frac{X}{n}-\frac{1}{\sqrt{n}}, \frac{X}{n} + \frac{1}{\sqrt{n}}\right).
\]
That is, our range goes $\frac{X}{n} + \frac{1}{\sqrt{n}}$ in both directions.
This might not be a totally crazy guess, but it is reasonable because it scales
with $\frac{1}{\sqrt{n}}$, which is also what the standard deviation scales by.

Given we observe $X=270$, by our initial plan, we would report
\[
\left(\frac{270}{500}-\frac{1}{\sqrt{500}}, \frac{270}{500} +
\frac{1}{\sqrt{500}}\right) = [0.695,0.595].
\]
Let's evaluate that guess. We seek to find the probability that our range will
contain $p$ 95\% of the time. That is,
\[
\mathbb{P}\left(
\frac{X}{n}-\frac{1}{\sqrt{n}} \le p \le \frac{X}{n} + \frac{1}{\sqrt{n}}
\right) = 95\%.
\]
This may be unintuitive because $X$ is random. Let's rewrite by rearranging
terms.
\[
\mathbb{P}\left(-\frac{1}{\sqrt{n}}\le \frac{X}{n}-p \le \frac{1}{\sqrt{n}}
    \right) = 95\%.
\]
Now, the random terms are all in the middle of the inequality instead of
bounding it. Rearranging to keep $X$ alone in the center,
\[
    \mathbb{P}(np-\sqrt{n} \le X \le np + \sqrt{n}) = 95\%.
\]
When we simulate in R with a true value $p=0.5$, we see that this plan gives us
a probability of 95.6\%. However, that only works because we know that in the
simulated case, $p=0.5$. Let's do the exact calculation.
\[
    F_{Binom}(np+\sqrt{n}) - F_{Binom}(np-\sqrt{n}) \\
\]
\begin{note}
    If a random variable $X$ has a CDF of $F(x)$, then
    \[
        \mathbb{P}(a\le X \le b) = F(b) - F(a).
    \]
\end{note}
Because of our simulation, we know that the probability coverage is 0.956 (i.e.
95.6\%). If we run the same simulation over and over again with different values
of $p$, we arrive at a plot. In our case, the minimum value is at $p=0.5$. Thus,
our worst-case performance is 95.6\%. Because our worst case is better than our
desired 95\% confidence interval, we can say that our plan is sound.

\pagebreak

\section{Rules}

\begin{definition}
Let our data be $X$ with a parameter $\theta$. If $L(X) \le U(X)$ are such that
$\mathbb{P}(L(X) \le \theta \le U(X)) \ge 1 - \alpha$ for any value of $\theta$,
then $[L(X),U(X)]$ is a $100(1-\alpha)\%$ \textbf{confidence interval rule} for
$\theta$.
\end{definition}
Where we were previously using the term "plan," we may now call these plans
\textit{rules}.

\subsection{Example}
Suppose our data $X_{1}, \cdots , X_{n}$ represents the effectiveness of an
intervention given to patients that are having trouble sleeping. Let
\[
X_{1}, \cdots , X_{n} \overset{\mathrm{iid}}{\sim} N(\theta,\sigma^{2}).
\]
We will treat $\sigma$ as known for the purpose of this illustration. Also,
$\theta$ is some number such that $\theta \in (-\infty, \infty)$. We want a
95\% CI rule for $\theta$. First let's reduce the problem to a sufficient 
statistic. Two possible examples are
\begin{align*}
    T &= X_{1} + \cdots  + X_{n}\\
    \bar X &= \frac{1}{n}(X_{1} + \cdots +X_{n}).
\end{align*}
Now that we have a sufficient statistic, we want to find a rule for $\theta$ 
using them. Take $\bar X$ for example. We seek a rule in the form
\[
    \bar X \pm \text{ something}.
\]
One possible answer for that "something" may be the standard deviation of
$\bar X$ itself, which is $\sigma/\sqrt{n}$. However, we know that the 
95\% CI for a normal is 1.96 standard deviations, so we will modify that
slightly to
\[
    \bar X \pm 1.96\frac{\sigma}{\sqrt{n}}.
\]
This will be our rule. Let $\theta=0$.
\begin{align*}
    \mathbb{P}_{N(\theta,\sigma^{2}}\left(\bar X - 1.96\frac{\sigma}{\sqrt{n}}
        \le 0 \le\bar X + 1.96\frac{\sigma}{\sqrt{n}}\right)
    &= \mathbb{P}_{N(\theta,\sigma^{2})}\left(-1.96 \le \frac{\bar X-0}{
        (\sigma/\sqrt{n}}\le 1.96\right).
\end{align*}
Because $\bar X \sim N(0,\sigma^{2}/n)$, we know that
\[
\frac{\bar X -0}{\sigma/\sqrt{n}} \sim .
\]
Thus, this entire probability comes out to 95\%. If we plug in any arbitrary
$\theta$, we see that this probability is always 95\%. Therefore, our
rule is sound.

\subsection{Binomial example}

Let $X \sim  Binom(n,p)$ and $p \in (0,1)$. Let $\hat p = X/n$. We can say
that, approximately,
\[
    \frac{X}{n} \sim N\left(p, \frac{p(1-p)}{n}\right).
\]
Thus, we see that a good estimate of $n$ could be
\[
\frac{X}{n} \pm 1.96 \sqrt{\frac{\frac{X}{n}\left(1-\frac{X}{n}\right)}{n}}.
\]


\end{document}

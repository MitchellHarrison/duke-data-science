\documentclass[titlepage, 12pt, leqno]{article}

% -------------------------------------------------- %
% -------------------- PACKAGES -------------------- %
% -------------------------------------------------- %
\usepackage{import}
\usepackage{pdfpages}
\usepackage{mathtools}
\usepackage{transparent}
\usepackage{enumitem}
\usepackage{xcolor}
\usepackage{tcolorbox}
\usepackage{amsmath}
\usepackage{amssymb}
\usepackage{parskip}
\usepackage{bbm}
\usepackage{algorithm}
\usepackage{algpseudocode}
\usepackage{algpseudocodex}
\usepackage[margin = 1in]{geometry}
\tcbuselibrary{breakable}
\tcbset{breakable = true}


% -------------------------------------------------- %
% -------------- CUSTOM ENVIRONMENTS --------------- %
% -------------------------------------------------- %
\newtcolorbox{note}{colback=black!5!white,
                          colframe=black!55!white,
                          fonttitle=\bfseries,title=Note}

\newtcolorbox{ex}{colback=blue!5!white,
                          colframe=blue!55!white,
                          fonttitle=\bfseries,title=Example}

\newtcolorbox{definition}{colback=red!5!white,
                          colframe=red!55!white,
                          fonttitle=\bfseries,title=Definition}


% -------------------------------------------------- %
% ------------------- COMMANDS --------------------- %
% -------------------------------------------------- %
% Brackets, braces, etc. 
\newcommand{\abs}[1]{\lvert #1 \rvert}
\newcommand{\bigabs}[1]{\Bigl \lvert #1 \Bigr \rvert}
\newcommand{\bigbracket}[1]{\Bigl [ #1 \Bigr ]}
\newcommand{\bigparen}[1]{\Bigl ( #1 \Bigr )}
\newcommand{\ceil}[1]{\lceil #1 \rceil}
\newcommand{\floor}[1]{\lfloor #1 \rfloor}
\newcommand{\norm}[1]{\| #1 \|}
\newcommand{\bignorm}[1]{\Bigl \| #1 \Bigr \| #1}
\newcommand{\inner}[1]{\langle #1 \rangle}
\newcommand{\set}[1]{{ #1 }}


% -------------------------------------------------- %
% -------------------- SETUP ----------------------- %
% -------------------------------------------------- %
\title{\Huge{Final Exam Review}}
\author{\large{Mitch Harrison}}
\date{\today}   
\begin{document}
\setlength{\parskip}{1\baselineskip}
\setlength{\parindent}{15pt}
\maketitle
\tableofcontents
\newpage


% -------------------------------------------------- %
% --------------------- BODY ----------------------- %
% -------------------------------------------------- %
\section{Review}

First, recall that for any random variable $\hat \mu$,
\[
    Var(\hat \mu) = \mathbb{E}(\hat \mu)^{2} = [\mathbb{E}(\hat \mu)]^{2}.
\]
If this variance is zero or constant, it is no longer a function of the data
since there is no randomness from the data.

\subsection{Exam 2, Problem 3}

This question mainly tasked us with concepts backed by a limited number of 
algebraic manipulations. We have $X_{i} \overset{\mathrm{iid}}{\sim}
Pois(\lambda_{i})$.

\subsubsection{Find MLE for $\lambda_{i}$}
The nature of the question means that there must be $n$ answers, one for each
$\lambda_{i}$. Because the MLE of a Poisson is $\bar X$ and the fact that each
$X$ has its own $\lambda$, then each $\hat \lambda_{i} = X_{i}$.

\subsubsection{Find $\Lambda(X)$}
He leaves the algebra to us and didn't give us the answer. Sick.

\subsubsection{Conditioning to get to a Binomial}
\begin{align*}
    \mathbb{P}(X_{i}|X_{1} + X_{2} = t)
    &= \frac{\mathbb{P}(X_{1} = k, X_{1} + X_{2}=t)}{
    \mathbb{P}(X_{1} + X_{2} = t)} \\
    &= \frac{\mathbb{P}(X_{1} = k)\mathbb{P}(X_{2}=t-k)}{
    \mathbb{P}(X_{1} +X_{2}=t)}
\end{align*}
Observe that $X \sim Poisson$, we know that $\mathbb{P}(X_{1} + X_{2})
\sim Poisson(\lambda_{1} + \lambda_{2})$ because the summations of a
Poisson is another Poisson with the sum of the parameters. Plugging in these
Poissons will give us some hellish formula that will have some cancelling,
resulting in the PMF of a binomial as requested in the problem statement.

\pagebreak
\section{General Prep Guidance}
A good way to practice would be to go through common distributions and
perform common calculations. For example, there is a table online somewhere
of all distributions that have an MLE, and it is worth finding those. Recall
that sometimes, the derivative cannot be found due to some constraint placed
on the distribution in our specific problem (we have seen this before).
Things that we are \textbf{likely to see on the exam} are:
\begin{enumerate}
    \item Likelihood
    \item Log-likelihood
    \item Derivative of log-likelihood
    \item MLE
    \item Second derivative
    \item Take expectation of second derivative for $\mathcal{I}(\theta)$
    \item $\sqrt{n}(\hat \theta - \theta) \rightarrow Normal(0, 1/\mathcal{I}
        (\theta)$
    \item Construct an asymptotic or approximate interval using
        \[
        \hat \theta \pm Z_{1-\alpha/2} \sqrt{\frac{1}{\mathcal{I}(\theta)}},
        \]
        and recall that this is only a valid interval if $\mathcal{I}$ does
        not depend on $\theta_{0}$, the true parameter value.
    \item Calculate $J_{n}(\theta)$ by finding
        \[
        J_{n}(\theta) = -\sum_{i=1}^{n}\ell''(\theta|X_{i}),
        \]
        and recall that $\mathcal{J}_{n}/n \rightarrow \mathcal{I}$ and 
        $J_{n}\approx \mathcal{I}$. 
    \item Use this to find an interval of the form
        \[
            \hat \theta \pm Z_{1-\alpha/2}\sqrt{\frac{1}{J_{n}(\theta|\theta=
            \hat \theta)}}.
        \]
\end{enumerate}

We will need to know \textbf{consistency}, how to find if we have consistency,
and how $\bar X$ is consistent and that we can use it to find MoM estimators.

We will also be asked about Bonferroni's correction, what it is, and how we use
it. But we will probably not be asked to do any proofs using multiple hypothesis
testing (although concepts are fair game).

There may also be questions surrounding the method of moments, both in concept
and finding them.

Some basic understanding how how distributions interact will be helpful. For
example, sums of Poissons are Poisson and sums of Normals are Normal, etc.

It would be helpful to recall Chebychev's inequality as well.

\pagebreak
\section{Conclusion}
The goal of this course was to (hopefully) explore the mathematical backing 
behind most of the statistical techniques that we have been using throughout
our academic career. With hypothesis testing, because most of Duke is Bayesian,
we covered the frequentist methods because that is how most people in industry
are thinking, as well as how things are usually reported in the press, etc.
Hypothesis tests become a little bit more nightmarish in multiple dimensions
(beyond the scope of this course), but is an active area of research.

\end{document}
